% Options for packages loaded elsewhere
\PassOptionsToPackage{unicode}{hyperref}
\PassOptionsToPackage{hyphens}{url}
%
\documentclass[
  letterpaper,
]{scrbook}
\usepackage{amsmath,amssymb}
\usepackage{lmodern}
\usepackage{iftex}
\ifPDFTeX
  \usepackage[T1]{fontenc}
  \usepackage[utf8]{inputenc}
  \usepackage{textcomp} % provide euro and other symbols
\else % if luatex or xetex
  \usepackage{unicode-math}
  \defaultfontfeatures{Scale=MatchLowercase}
  \defaultfontfeatures[\rmfamily]{Ligatures=TeX,Scale=1}
\fi
% Use upquote if available, for straight quotes in verbatim environments
\IfFileExists{upquote.sty}{\usepackage{upquote}}{}
\IfFileExists{microtype.sty}{% use microtype if available
  \usepackage[]{microtype}
  \UseMicrotypeSet[protrusion]{basicmath} % disable protrusion for tt fonts
}{}
\makeatletter
\@ifundefined{KOMAClassName}{% if non-KOMA class
  \IfFileExists{parskip.sty}{%
    \usepackage{parskip}
  }{% else
    \setlength{\parindent}{0pt}
    \setlength{\parskip}{6pt plus 2pt minus 1pt}}
}{% if KOMA class
  \KOMAoptions{parskip=half}}
\makeatother
\usepackage{xcolor}
\usepackage{longtable,booktabs,array}
\usepackage{calc} % for calculating minipage widths
% Correct order of tables after \paragraph or \subparagraph
\usepackage{etoolbox}
\makeatletter
\patchcmd\longtable{\par}{\if@noskipsec\mbox{}\fi\par}{}{}
\makeatother
% Allow footnotes in longtable head/foot
\IfFileExists{footnotehyper.sty}{\usepackage{footnotehyper}}{\usepackage{footnote}}
\makesavenoteenv{longtable}
\usepackage{graphicx}
\makeatletter
\def\maxwidth{\ifdim\Gin@nat@width>\linewidth\linewidth\else\Gin@nat@width\fi}
\def\maxheight{\ifdim\Gin@nat@height>\textheight\textheight\else\Gin@nat@height\fi}
\makeatother
% Scale images if necessary, so that they will not overflow the page
% margins by default, and it is still possible to overwrite the defaults
% using explicit options in \includegraphics[width, height, ...]{}
\setkeys{Gin}{width=\maxwidth,height=\maxheight,keepaspectratio}
% Set default figure placement to htbp
\makeatletter
\def\fps@figure{htbp}
\makeatother
\setlength{\emergencystretch}{3em} % prevent overfull lines
\providecommand{\tightlist}{%
  \setlength{\itemsep}{0pt}\setlength{\parskip}{0pt}}
\setcounter{secnumdepth}{5}
% Make \paragraph and \subparagraph free-standing
\ifx\paragraph\undefined\else
  \let\oldparagraph\paragraph
  \renewcommand{\paragraph}[1]{\oldparagraph{#1}\mbox{}}
\fi
\ifx\subparagraph\undefined\else
  \let\oldsubparagraph\subparagraph
  \renewcommand{\subparagraph}[1]{\oldsubparagraph{#1}\mbox{}}
\fi
\newlength{\cslhangindent}
\setlength{\cslhangindent}{1.5em}
\newlength{\csllabelwidth}
\setlength{\csllabelwidth}{3em}
\newlength{\cslentryspacingunit} % times entry-spacing
\setlength{\cslentryspacingunit}{\parskip}
\newenvironment{CSLReferences}[2] % #1 hanging-ident, #2 entry spacing
 {% don't indent paragraphs
  \setlength{\parindent}{0pt}
  % turn on hanging indent if param 1 is 1
  \ifodd #1
  \let\oldpar\par
  \def\par{\hangindent=\cslhangindent\oldpar}
  \fi
  % set entry spacing
  \setlength{\parskip}{#2\cslentryspacingunit}
 }%
 {}
\usepackage{calc}
\newcommand{\CSLBlock}[1]{#1\hfill\break}
\newcommand{\CSLLeftMargin}[1]{\parbox[t]{\csllabelwidth}{#1}}
\newcommand{\CSLRightInline}[1]{\parbox[t]{\linewidth - \csllabelwidth}{#1}\break}
\newcommand{\CSLIndent}[1]{\hspace{\cslhangindent}#1}
\RedeclareSectionCommand[beforeskip=-2.5ex plus -1ex minus -.2ex,afterskip=0.3ex plus -.2ex]{section}
\usepackage[top=0.5in, left=0.5in, right=0.5in, bottom=1in]{geometry}
\bibliographystyle{catalog}
\usepackage{makeidx}
\makeindex
\renewcommand\toprule[2]\relax
\renewcommand\bottomrule[2]\relax
\usepackage{fancyhdr}
\lfoot{\textit{Coe College (2022-2023)}}
\makeatletter
\makeatother
\makeatletter
\@ifpackageloaded{bookmark}{}{\usepackage{bookmark}}
\makeatother
\makeatletter
\@ifpackageloaded{caption}{}{\usepackage{caption}}
\AtBeginDocument{%
\ifdefined\contentsname
  \renewcommand*\contentsname{Table of contents}
\else
  \newcommand\contentsname{Table of contents}
\fi
\ifdefined\listfigurename
  \renewcommand*\listfigurename{List of Figures}
\else
  \newcommand\listfigurename{List of Figures}
\fi
\ifdefined\listtablename
  \renewcommand*\listtablename{List of Tables}
\else
  \newcommand\listtablename{List of Tables}
\fi
\ifdefined\figurename
  \renewcommand*\figurename{Figure}
\else
  \newcommand\figurename{Figure}
\fi
\ifdefined\tablename
  \renewcommand*\tablename{Table}
\else
  \newcommand\tablename{Table}
\fi
}
\@ifpackageloaded{float}{}{\usepackage{float}}
\floatstyle{ruled}
\@ifundefined{c@chapter}{\newfloat{codelisting}{h}{lop}}{\newfloat{codelisting}{h}{lop}[chapter]}
\floatname{codelisting}{Listing}
\newcommand*\listoflistings{\listof{codelisting}{List of Listings}}
\makeatother
\makeatletter
\@ifpackageloaded{caption}{}{\usepackage{caption}}
\@ifpackageloaded{subcaption}{}{\usepackage{subcaption}}
\makeatother
\makeatletter
\@ifpackageloaded{tcolorbox}{}{\usepackage[skins,breakable]{tcolorbox}}
\makeatother
\makeatletter
\@ifundefined{shadecolor}{\definecolor{shadecolor}{rgb}{.97, .97, .97}}
\makeatother
\makeatletter
\makeatother
\makeatletter
\makeatother
\ifLuaTeX
  \usepackage{selnolig}  % disable illegal ligatures
\fi
\IfFileExists{bookmark.sty}{\usepackage{bookmark}}{\usepackage{hyperref}}
\IfFileExists{xurl.sty}{\usepackage{xurl}}{} % add URL line breaks if available
\urlstyle{same} % disable monospaced font for URLs
\hypersetup{
  pdftitle={Course Catalog 2022-2023},
  hidelinks,
  pdfcreator={LaTeX via pandoc}}

\title{Course Catalog 2022-2023}
\author{}
\date{2022-08-01}

\begin{document}
\frontmatter
\maketitle

\ifdefined\Shaded\renewenvironment{Shaded}{\begin{tcolorbox}[breakable, frame hidden, boxrule=0pt, sharp corners, interior hidden, borderline west={3pt}{0pt}{shadecolor}, enhanced]}{\end{tcolorbox}}\fi

\renewcommand*\contentsname{Table of contents}
{
\setcounter{tocdepth}{1}
\tableofcontents
}
\mainmatter
\bookmarksetup{startatroot}

\hypertarget{preface}{%
\chapter*{PREFACE}\label{preface}}

\markboth{PREFACE}{PREFACE}

\hypertarget{non-discrimination}{%
\section*{Non-Discrimination}\label{non-discrimination}}

\markright{Non-Discrimination}

Coe College does not discriminate on the basis of race, color,
ethnicity, age, religion, national origin, sexual orientation, gender
identity, sex, marital status, disability, or status as a U.S. Veteran.
All students have equal access to the facilities, financial aid, and
programs of the College.

\hypertarget{higher-education-opportunity-act-heoa}{%
\section*{Higher Education Opportunity Act
(HEOA)}\label{higher-education-opportunity-act-heoa}}

\markright{Higher Education Opportunity Act (HEOA)}

The College complies with Readmission Requirements for Service Members
as outlined in the Higher Education Opportunity Act section 487. This
applies to active duty in the Armed Forces, whether voluntary or
involuntary, including service as a member of the National Guard or
Reserve, for a period of more than 30 days under a call or order to
active duty.

The HEOA provides that a prompt readmission of a previously enrolled or
admitted student may not be denied to a service member of the uniformed
services for reasons relating to that service.~ In addition, a student
who is readmitted under this section must be readmitted with the same
academic status as the student had when they attended the college.

\hypertarget{equal-opportunity-in-employment}{%
\section*{Equal Opportunity in
Employment}\label{equal-opportunity-in-employment}}

\markright{Equal Opportunity in Employment}

Coe College is an equal opportunity employer in the recruitment and
hiring of faculty and staff.

\hypertarget{family-educational-rights-and-privacy-act-ferpa}{%
\section*{Family Educational Rights and Privacy Act
(FERPA)}\label{family-educational-rights-and-privacy-act-ferpa}}

\markright{Family Educational Rights and Privacy Act (FERPA)}

The provisions of the Family Educational Rights and Privacy Act (FERPA)
prohibit the College from releasing grades or other information about
academic standing to parents unless the student has released such
information in writing. Further information concerning Coe College
procedures in compliance with FERPA is available in the Office of the
Registrar and included on p.~68 of this catalog.

\hypertarget{solomon-amendment-of-1997}{%
\section*{Solomon Amendment of 1997}\label{solomon-amendment-of-1997}}

\markright{Solomon Amendment of 1997}

Pursuant to the regulations of the Solomon Amendment of 1997, Coe
College is required to make student recruiting information available to
military recruiters who request it.

\hypertarget{reservation-of-the-right-to-modify}{%
\section*{Reservation of the Right to
Modify}\label{reservation-of-the-right-to-modify}}

\markright{Reservation of the Right to Modify}

The provisions of this catalog are to be considered directive in
character and not as an irrevocable contract between the student and the
College. The College reserves the right to make changes that seem
necessary or desirable, including course and program cancellations.
Responsibility for understanding and meeting graduation requirements as
stated in the Coe College Catalog rests entirely with the student.
Faculty advisors and the Registrar will assist in every way possible.

\bookmarksetup{startatroot}

\hypertarget{mission-statement-of-the-college}{%
\chapter*{MISSION STATEMENT OF THE
COLLEGE}\label{mission-statement-of-the-college}}
\addcontentsline{toc}{chapter}{MISSION STATEMENT OF THE COLLEGE}

\markboth{MISSION STATEMENT OF THE COLLEGE}{MISSION STATEMENT OF THE
COLLEGE}

\begin{figure}

{\centering \includegraphics{catalog_sections/graphics/coe_bell.png}

}

\end{figure}

\index{Mission Statement}

edit!

Coe College is a national, residential liberal arts college offering a
broad array of programs in the arts, sciences and professions. Our
mission is to prepare students for meaningful lives and fulfilling
careers in a diverse, interconnected world. Coe's success will be judged
by the success of our graduates.

\emph{Coe College admits students without regard to sex, race, creed,
color, handicap, sexual orientation, national, or ethnic origin. All
students have equal access to the facilities, financial aid, and
programs of the College.}

\bookmarksetup{startatroot}

\hypertarget{fast-facts}{%
\chapter*{FAST FACTS}\label{fast-facts}}
\addcontentsline{toc}{chapter}{FAST FACTS}

\markboth{FAST FACTS}{FAST FACTS}

\textbf{COE COLLEGE} is a private, four-year co-educational liberal arts
college that was founded in 1851 and is historically affiliated with the
Presbyterian Church (U.S.A.), yet is ecumenical in practice and outlook.

\hypertarget{location}{%
\subsection*{LOCATION}\label{location}}
\addcontentsline{toc}{subsection}{LOCATION}

Coe is located just 225 miles west of Chicago in Iowa's second largest
city, Cedar Rapids (metropolitan population 175,000). The city is easily
accessible by the interstate highway system, bus services and several
airlines.

\hypertarget{campus}{%
\subsection*{CAMPUS}\label{campus}}
\addcontentsline{toc}{subsection}{CAMPUS}

Situated on 65 acres in the center of the metropolitan area, the campus
is urban but enclosed, with 30 buildings occupying an attractive
landscape.

\hypertarget{enrollment}{%
\subsection*{ENROLLMENT}\label{enrollment}}
\addcontentsline{toc}{subsection}{ENROLLMENT}

The student body of approximately 1,400 students represents most states
and around 15 foreign countries. All students are required to live on
campus unless they are residents of Cedar Rapids.

\hypertarget{library}{%
\subsection*{LIBRARY}\label{library}}
\addcontentsline{toc}{subsection}{LIBRARY}

Stewart Memorial Library is in the very center of the main campus. It
contains over 500,000 volumes and 16,000 pieces of media, and subscribes
to 3,500 print and online periodical subscriptions providing access to
over 100 databases. Coe's library offers students a variety of research
assistance, study areas, a small theater, preview room, a media editing
room, and a 3D printer and laser cutter. The Library houses the Learning
Commons and the College's permanent art collection.

\hypertarget{accreditation}{%
\subsection*{ACCREDITATION}\label{accreditation}}
\addcontentsline{toc}{subsection}{ACCREDITATION}

Coe College is accredited by the Higher Learning Commission
(hlcomission.org), an institutional accreditation agency recognized by
the U.S. Department of Education. Coe College's Bachelor of Music
program is accredited by the National Association of Schools of Music;
its education program is accredited by the Iowa Department of Education,
and its Bachelor of Science in Nursing is accredited by the Commission
on Collegiate Nursing Education
(http://aacn.nche.edu/ccne-accreditation).~ Coe College's chemistry
program is certified by the American Chemical Society.~ Copies of
accrediting and approval statements are available online at
https://www.coe.edu/why-coe/consumer-information.

\hypertarget{memberships}{%
\subsection*{MEMBERSHIPS}\label{memberships}}
\addcontentsline{toc}{subsection}{MEMBERSHIPS}

Coe is a charter member of the Associated Colleges of the Midwest, whose
other members are: Beloit, Carleton, Colorado, Cornell, Grinnell, Knox,
Lake Forest, Lawrence, Luther, Macalester, Monmouth, Ripon, and
St.~Olaf. Other memberships include: Phi Beta Kappa, Phi Kappa Phi, the
National Association of Independent Colleges and Universities, the
American Association of Colleges and Universities, Council on
Undergraduate Research, the Association of Presbyterian Colleges and
Universities, the American Rivers Conference, and the National
Collegiate Athletic Association.

\hypertarget{academic-program}{%
\subsection*{ACADEMIC PROGRAM}\label{academic-program}}
\addcontentsline{toc}{subsection}{ACADEMIC PROGRAM}

Academic areas of study are offered in managerial or public accounting,
African American studies, art, art history, Asian studies, biology,
business administration, chemistry, communication studies, computer
science, creative writing, data science, economics, elementary
education, English, film studies, French \& francophone studies,
interdisciplinary French \& francophone studies, general science,
history, international business, international economics, international
studies, kinesiology, literature, mathematics, music (B.A. or B.M.),
nursing (B.S.N.), philosophy, physics, political science, psychology,
religion, social \& criminal justice, sociology, Spanish, Spanish
studies, theatre arts, and writing. Interdisciplinary and/or collateral
majors are also available in biochemistry, environmental science,
environmental studies, molecular biology, neuroscience, organizational
science, and public relations. Coe also offers certificate programs in
primary and secondary education.

\hypertarget{extra-curricular-activities}{%
\subsection*{EXTRA-CURRICULAR
ACTIVITIES}\label{extra-curricular-activities}}
\addcontentsline{toc}{subsection}{EXTRA-CURRICULAR ACTIVITIES}

Students can participate in 11 men's and 11 women's NCAA Division III
varsity athletic teams, 5 coed varsity athletic teams, 8 club sports, 10
nationally affiliated men's and women's social fraternities, and 65
student organizations including but not limited to student government,
newspaper, intramural sports, departmental clubs, and residence hall
activities. Various vocal and instrumental ensembles are available for
course credit.

\hypertarget{financial-assistance}{%
\subsection*{FINANCIAL ASSISTANCE}\label{financial-assistance}}
\addcontentsline{toc}{subsection}{FINANCIAL ASSISTANCE}

Coe College is committed to assisting those families in need of
financial assistance. The average aid package for incoming students
enrolled during the 2021--2022 academic year totaled more than \$48,000.
The total cost of full-time tuition, room, board, and activity fee for
the 2021--2022 academic year is \$59,324.

\bookmarksetup{startatroot}

\hypertarget{history-of-coe-college}{%
\chapter*{HISTORY OF COE COLLEGE}\label{history-of-coe-college}}
\addcontentsline{toc}{chapter}{HISTORY OF COE COLLEGE}

\markboth{HISTORY OF COE COLLEGE}{HISTORY OF COE COLLEGE}

Coe College claims the shortest name of any American institution of
higher education, but the school has actually carried five titles
through its history. When the Rev.~Williston Jones founded the college
in 1851, he called it The School for the Prophets. Cedar Rapids' first
resident minister opened the parlor of his home to a group of young men
with the goal of educating them for the ministry to serve churches in
the Midwest. Two years later, while Jones was canvassing churches in the
East for money to send three of his students to Eastern seminaries, a
Catskills farmer named Daniel Coe stepped forward with a pledge of
\$1,500 and urged Jones to start his own college in the frontier town of
Cedar Rapids. Legend has it that the \$1,500 raised by Coe was brought
west from New York, sewn into the petticoat of a lady visitor traveling
by stagecoach to Iowa. Coe gave this generous gift with the stipulation
that the proposed institute should be ``made available for the education
of females as well as males.'' Accordingly, Coe was coeducational from
its founding.

With Jones' blessing, the Cedar Rapids Collegiate Institute was
incorporated in 1853 by a group of Cedar Rapids leaders chaired by Judge
George Greene. They used Daniel Coe's money to purchase two downtown
lots for the school and 80 acres of farmland on what was then the edge
of town. The farm evolved into today's campus. In 1868, in a failed
attempt to secure the Lewis Parsons estate, the trustees renamed the
school Parsons Seminary. After a period of severe financial
difficulties, the institution was reestablished in honor of its original
benefactor as the Coe Collegiate Institute in 1875.

T.M Sinclair, founder of the Sinclair Meat Packing Company, played the
key financial role in the final step toward the firm establishment of
Coe College. Sinclair liquidated all the debt from Parsons Seminary and
the Cedar Rapids Collegiate Institute. The Sinclair gift made it
practical for the property of the Coe Collegiate Institute---including
the original land paid for by Daniel Coe---to be transferred to Coe
College with the Iowa Presbyterian Synod to assume major responsibility
for the institution. Coe College has operated continuously since its
incorporation under that name on Feb.~2, 1881. From the first, the
College was committed to intellectual excellence. It has continued in
this tradition ever since.

The compact campus on the east edge of Cedar Rapids grew with many
building projects in its early years, including Old Main (1868),
Williston Hall (1881), Marshall Hall (1900), the first gymnasium (1904),
and the first T.M. Sinclair Memorial Chapel (1911). In 1907, Coe earned
accreditation from the North Central Association of Colleges and
Universities. Over the decades, Coe's reputation as a superior liberal
arts college has continued to grow. One recognition of this came in
1949, when Coe was granted a Phi Beta Kappa chapter, a distinction
reserved for about 10\% of American colleges and universities.

Central to the educational philosophy of Coe College is the belief that
a liberal arts education is the best preparation for life. Students have
the opportunity to experience a variety of subjects outside their
respective programs of study. Coe offers more than 40 areas of study
that cover a range of fields. The College awards the following
undergraduate degrees: Bachelor of Arts (B.A.), Bachelor of Music
(B.M.), and Bachelor of Science in Nursing (B.S.N.).

There are a number of factors that contribute to Coe College's strong
academic quality. The key to Coe's tradition of excellence in academic
quality relates directly to small class sizes and the interest shown by
professors to make learning a personalized experience. At Coe, the
average class size is 16, and the student-faculty ratio is 11:1. Classes
are taught by our involved and committed faculty, 91\% of whom hold the
highest degree in their field. This means classes are taught by
experienced professionals who have in-depth knowledge of their subjects.
To provide students with a well-rounded experience and solid preparation
for the future, Coe offers, along with quality instruction from superb
faculty, an abundance of out-of-class opportunities which include
student-faculty collaborative research, honors projects, service
learning, and internships. Within one year of graduation, according to
the annual survey results from the Center for Creativity and Careers,
98\% of Coe graduates are engaged in post-graduate activity such as
employment, graduate school, military, or travel/adventure.

With the addition of the east campus, Coe has nearly doubled in size
since 1989. New facilities on the east side of College Drive include
Athletic Recreation Center (2017), Clark Racquet Center and athletic
fields (1989), Clark Alumni House (1993), Nassif House (1999), and four
student apartment buildings (Morris House and Schlarbaum House in 2000,
Brandt House and Spivey House in 2002). McCabe Hall (2005), named in
honor of former Coe President Joseph E. McCabe, houses the offices of
the president, provost, advancement and alumni relations, and The Center
for Creativity, Careers and Community (C3) making way for the remodeling
of Coe's oldest building, Stuart Hall, and the first significant
addition of classroom space since Peterson Hall was built in the 1960s.
In 2012 and 2013, Peterson Hall of Science was completely renovated to
support Coe's science programs. To further enhance the campus
environment, Coe completed the largest capital project in its history in
2017. Make Your Move -- the Campaign for Eby and Hickok -- included \$24
million in essential enhancements, including an Athletic and Recreation
Complex project as well as the renovation and expansion of Hickok Hall,
one of the College's main academic buildings. The result is vastly
improved academic, recreational, wellness and competition facilities to
benefit future generations of students.

\part{THE EDUCATIONAL PROGRAM}

Coe College grants the undergraduate degrees of Bachelor of Arts (B.A.),
Bachelor of Music (B.M.), and Bachelor of Science in Nursing (B.S.N.).
The Bachelor of Arts degree (see p.~19) is earned upon fulfillment of
the conditions described below, including a major chosen from over 40
fields of study or an approved interdisciplinary major within the arts
and sciences. Students interested in music may earn either a B.A. or a
B.M. The B.M. degree requirements for students who wish to pursue music
as a profession or to prepare to teach music, can be found on p.~169.
The requirements for a B.S.N. degree can be found on p.~183.

\hypertarget{academic-calendar}{%
\chapter*{2022--2023 ACADEMIC CALENDAR}\label{academic-calendar}}
\addcontentsline{toc}{chapter}{2022--2023 ACADEMIC CALENDAR}

\markboth{2022--2023 ACADEMIC CALENDAR}{2022--2023 ACADEMIC CALENDAR}

\hypertarget{fall-term-2022}{%
\subsection*{FALL TERM 2022}\label{fall-term-2022}}
\addcontentsline{toc}{subsection}{FALL TERM 2022}

\begin{longtable}[]{@{}
  >{\raggedright\arraybackslash}p{(\columnwidth - 2\tabcolsep) * \real{0.3333}}
  >{\raggedleft\arraybackslash}p{(\columnwidth - 2\tabcolsep) * \real{0.6667}}@{}}
\toprule\noalign{}
\endhead
\bottomrule\noalign{}
\endlastfoot
Tues, August 23 & Open Registration \\
Wed, August 24 & Classes Begin \\
Tues, August 30 & Last Day to Add or Drop a Full-Term, or 1st Half-Term
Course without a W \\
Fri, September 2 & Census Date, Attendance Due on My.Coe \\
Mon, September 5 & No Classes (Holiday) Labor Day \\
Fri, September 9 & Date of Record \\
Thurs, September 22 & Last Day to Withdraw \&/ or Change Method of
Grading for First-Half Courses \\
Mon--Tues, October 10--11 & Fall Term Break \\
Thurs--Sat, October 13--15 & Homecoming \\
Mon, October 17 & Mid-Term Progress Report Due on My.Coe (11:59 PM) \\
Mon, October 17 & Begin Second-Half Term Courses \\
Fri, October 21 & May Term deposit and registration due \\
Fri, October 21 & Last Day to Add or Drop a Second-Half Term Course
without a W \\
Mon, October 24-Fri, November 11 & Advising Season \\
Fri, October 28 & Last Day to Withdraw \&/or Change Method of Grading
for Full-Term Courses \\
Mon--Fri, November 14--18 & Registration for Spring Term 2023 \\
Tues, November 15 & Last Day to Withdraw \&/or Change Method of Grading
for Second-Half Courses \\
Mon--Fri, November 21--25 & No Classes (Holiday) Thanksgiving Recess \\
Thurs, December 1 & Open Online Registration until Last Day of Finals \\
Fri, December 2 & Last Day of Fall Term Classes \\
Sat, December 3 & Reading Day \\
Mon--Thurs, December 5--8 & Final Exams \\
Wed, December 14 & Final Grades and Attendance are Due from the Faculty
on My.Coe (11:59 PM) \\
\end{longtable}

\hypertarget{spring-term-2023}{%
\subsection*{SPRING TERM 2023}\label{spring-term-2023}}
\addcontentsline{toc}{subsection}{SPRING TERM 2023}

\begin{longtable}[]{@{}
  >{\raggedright\arraybackslash}p{(\columnwidth - 2\tabcolsep) * \real{0.3333}}
  >{\raggedleft\arraybackslash}p{(\columnwidth - 2\tabcolsep) * \real{0.6667}}@{}}
\toprule\noalign{}
\endhead
\bottomrule\noalign{}
\endlastfoot
Wed, January 11 & Classes Begin \\
Mon, January 16 & No Day Classes (Holiday) Martin Luther King Jr.~Day \\
Wed, January 18 & Last Day to Add or Drop a Full-Term or 1st Half-Term
Course without a W \\
Fri, January 20 & Census Date, Attendance Due on My.Coe \\
Thurs, January 26 & Date of Record Mon, February 6 \\
Mon, February 6 & Last Day to Withdraw \&/or Change Method of Grading
for First-Half Courses \\
Mon, March 6 & Begin Second-Half Term Courses \\
Tues, March 7 & Mid-Term Progress Report Due Online (11:59 PM) \\
Fri, March 10 & Last Day to Add or Drop a Second-Half Term Course
without a W \\
Mon--Fri, March 13--17 & Spring Term Recess (Starting after Class on
Friday, March 10) \\
Mon--Fri, March 20--April 7 & Advising Season Thurs, March 30 \\
Thurs, March 30 & Last Day to Withdraw \&/or Change Method of Grading
for Full-Term Courses \\
Mon--Fri, April 10--14 & Registration for Fall Term 2023 \\
Mon, April 10 & Summer Registration Opens \\
Tues, April 11 & Last Day to Withdraw \&/or Change Method of Grading for
Second-Half Courses \\
Wed, April 12 & Student Research Symposium ** No Day Classes ** Evening
Classes Will Meet \\
Fri, April 28 & Last Day of Spring Term Classes \\
Sat, April 29 & Reading Day \\
Mon--Thurs, May 1--4 & Final Exams \\
Sat, May 6 & Honors Convocation / Baccalaureate \\
Sun, May 7 & Commencement \\
Tues, May 9 & Final Grades and Attendance are Due for Non-Graduating
Students on My.Coe (11:59 PM) \\
\end{longtable}

\hypertarget{may-term-2023}{%
\subsection*{MAY TERM 2023}\label{may-term-2023}}
\addcontentsline{toc}{subsection}{MAY TERM 2023}

\begin{longtable}[]{@{}
  >{\raggedright\arraybackslash}p{(\columnwidth - 2\tabcolsep) * \real{0.3333}}
  >{\raggedleft\arraybackslash}p{(\columnwidth - 2\tabcolsep) * \real{0.6667}}@{}}
\toprule\noalign{}
\endhead
\bottomrule\noalign{}
\endlastfoot
Wed, May 10 & Classes Begin \\
Fri, May 12 & Last Day to Add or Drop a Course Without a W \\
Mon, May 15 & Attendance Due Online; Last Day to Change Method of
Grading for May Term \\
Wed, May 17 & Last Day to Withdraw from May Term Courses \\
Fri--Sat, May 19--20 & Meeting of the Board of Trustees \\
Mon, May 29 & No Classes (Holiday) Memorial Day \\
Fri, June 2 & Last Day of May Term Classes \\
Sat, June 3 & Residence Halls Close \\
Sun, June 4 & Final Grades and Attendance for May Term Due on My.Coe
(11:59 PM) \\
\end{longtable}

\hypertarget{summer-term-2023}{%
\subsection*{SUMMER TERM 2023*}\label{summer-term-2023}}
\addcontentsline{toc}{subsection}{SUMMER TERM 2023*}

\begin{longtable}[]{@{}
  >{\raggedright\arraybackslash}p{(\columnwidth - 2\tabcolsep) * \real{0.3333}}
  >{\raggedleft\arraybackslash}p{(\columnwidth - 2\tabcolsep) * \real{0.6667}}@{}}
\toprule\noalign{}
\endhead
\bottomrule\noalign{}
\endlastfoot
Mon, June 5 & Classes Begin \\
Fri, June 9 & Last Day to Add or Drop a Full-Term Course Without a W \\
Mon, June 19 & No Classes (Holiday) Juneteenth \\
Fri, June 23 & Last Day to Change Method of Grading \&/or Withdraw from
Block A Courses \\
Tues, July 4 & No Classes (Holiday) Independence Day \\
Fri, July 7 & Last Day of Block A Courses \\
Mon, July 10 & Classes Begin: Block B Courses \\
Tues, July 11 & Mid-Term Progress Report Due Online (11:59 PM); Block A
Final Grades Due \\
Fri, July 14 & Last Day to Add or Drop a Course Without a W for Block B
Courses \\
Mon, July 17 & Last Day to Change Method of Grading \&/or Withdraw from
Full Term Courses \\
Tues, July 25 & Last Day to Change Method of Grading \&/or Withdraw from
Block B Courses \\
Fri, August 11 & Last Day of Term Classes: Block B and Full-Term
Courses \\
Tues, August 15 & Final Grades and Attendance Due (Block B and
Full-Term) on My.Coe (11:59 PM) \\
\end{longtable}

*Courses taught at the Wilderness Field Station are subject to the broad
dates of the summer, but will provide a specific add/drop and withdraw
calendar to students at the time of application.

\hypertarget{the-academic-calendar}{%
\chapter*{THE ACADEMIC CALENDAR}\label{the-academic-calendar}}
\addcontentsline{toc}{chapter}{THE ACADEMIC CALENDAR}

\markboth{THE ACADEMIC CALENDAR}{THE ACADEMIC CALENDAR}

The academic year consists of three terms (see Academic Calendar,
p.~15). Students normally take four course credits in the Fall Term and
four course credits in the Spring Term. Thus, eight course credits are
completed in an academic year. During optional May Term, students may
enroll for up to one course credit in one of the limited selection of
courses. Summer term is limited in scope and is not considered a regular
term. (The maximum course load is described in Course Load, p.~53).

\hypertarget{may-term-optional}{%
\section*{MAY TERM (OPTIONAL)}\label{may-term-optional}}
\addcontentsline{toc}{section}{MAY TERM (OPTIONAL)}

\markright{MAY TERM (OPTIONAL)}

Students may enroll for up to one course credit during May Term in one
of the limited selection of courses. All May Term courses require
consent of instructor prior to registration. May Term courses are
designed to meet at least two of the following shared learning outcomes:

\begin{itemize}
\tightlist
\item
  Evaluate and engage with complex interdependent systems and
  demonstrate understanding across diverse contexts.
\item
  Critically describe and break apart issues or problems through
  systematic analysis and illustrate logic for conclusions.
\item
  Engage with experiential learning practices such as learning by doing,
  while utilizing abilities to think critically, problem solve and make
  connections between knowledge gained in the classroom and experience
  beyond.
\end{itemize}

May Term courses are expected to have the same amount of contact time
and academic rigor per course credit as courses which meet over a Fall
or Spring Term.~ For every 1 course credit of May term students are
expected to complete 140 hours of work. Such contact time includes class
meetings, lectures by the instructor, supervised course related
activities and independent out of class activities.~ Off-campus May Term
courses at an off-campus location provide certain educational benefits
through site visits, guest lectures, etc., that also contribute to the
contact time for the course.

\hypertarget{coe-plan}{%
\section*{COE PLAN}\label{coe-plan}}
\addcontentsline{toc}{section}{COE PLAN}

\markright{COE PLAN}

Coe College's requirements for graduation, commonly known as the Coe
Plan were developed with the following outcomes in mind:

\begin{itemize}
\tightlist
\item
  Creation of a bridge from high school to Coe College that helps
  students understand the importance of a liberal arts education, the
  ways to develop the skills needed by any learner, and the
  opportunities they will have by going to Coe College.
\item
  Development of required curriculum that exposes the students to ways
  of learning in various contexts, big ideas in a myriad of disciplines,
  ways of being and understanding of cultures around the world, and
  processes to develop the skills needed by any learner.
\item
  Creation of a bridge from Coe College to life after Coe.
\end{itemize}

These outcomes are met through the College's First-Year Experience,
General Education program, Writing Emphasis courses, and the College's
Practicum experiences and areas of study, described in this section of
the Catalog.

\hypertarget{graduation-requirements}{%
\section*{GRADUATION REQUIREMENTS}\label{graduation-requirements}}
\addcontentsline{toc}{section}{GRADUATION REQUIREMENTS}

\markright{GRADUATION REQUIREMENTS}

All students who graduate from Coe College must complete at least one
major and earn at least 32 course credits (cc) with grades leading to a
cumulative grade point average (GPA) of 2.0 or higher. (The course is
the unit of academic credit.) Courses are one credit unless otherwise
indicated. Students are expected to complete 180 hours of work to earn
one course credit, although class times vary from course to course.
Other institutions may convert Coe credit to their system by considering
one course credit to be 6 quarter hours, or 4 semester hours.) No more
than a total of two course credits from courses which are less than 0.5
credit can be used to meet the 32-credit graduation requirement. No more
than eight course credits earned of Advanced Placement or International
Baccalaureate credit can be used to satisfy this requirement.

Students must meet one of the following requirements:

\begin{itemize}
\tightlist
\item
  Complete at least the final academic year of required courses
  registered through Coe.
\item
  Earn a total of 16 course credits or the equivalent at Coe. The last
  eight course credits needed for graduation must include at least four
  earned at Coe. Approved off-campus study programs and internships can
  be used to fulfill this requirement.
\end{itemize}

A student may be simultaneously awarded two degrees (B.A., B.M., B.S.N.)
after satisfactorily completing 40 course credits and the requirements
for both degrees. However, a simultaneous Bachelor of Music plus a
Bachelor of Arts with a music major is not permitted.

In addition, students must fulfill the requirements of the First-Year
Experience, General Education, Writing Emphasis, and Practicum.

To participate in Commencement exercises, students must submit a
completed Intent to Graduate form to the Office of the Registrar,
preferably three terms prior to Commencement.

\hypertarget{second-baccalaureate-degree}{%
\subsection*{Second Baccalaureate
Degree}\label{second-baccalaureate-degree}}
\addcontentsline{toc}{subsection}{Second Baccalaureate Degree}

A student who holds a baccalaureate degree from another institution may
earn a second baccalaureate degree at Coe, if the following criteria are
met:

\begin{itemize}
\tightlist
\item
  The first degree must be from a regionally accredited institution as
  recognized by the US Department of Education, or another appropriate
  accrediting body.
\item
  The first degree must be completed (not in progress) before beginning
  the second degree at Coe.
\end{itemize}

Students accepted at Coe to pursue a second degree are granted a maximum
of 24 course credits in transfer credit towards the 32 course credits
required for graduation. To graduate, at least eight course credits must
be earned at Coe College and all requirements for the major area of
study must be met with at least 40\% of the major course credits taken
at Coe. Students must earn a cumulative GPA of at least 2.0 as well as
meet any GPA or grade requirements in their area of study.

Second baccalaureate students are exempt from the following
requirements: first-year experience, general education, writing
emphasis, and practicum. They are not eligible to graduate with Latin
Honors or for induction in Phi Beta Kappa or Phi Kappa Phi.

\hypertarget{transfer-student-information}{%
\subsection*{Transfer Student
Information}\label{transfer-student-information}}
\addcontentsline{toc}{subsection}{Transfer Student Information}

To honor its mission and to preserve its academic integrity as a liberal
arts institution, the College accepts a course in transfer if that
course meets the spirit of the College's mission and is from a
regionally accredited institution. This section includes information, in
addition to that included in the section, Graduation Requirements (see
p.~16), germane to students who are transferring to Coe College from
another college or university.

Courses transferred to Coe can, as approved by the Registrar, fulfill
some graduation requirements. From institutions on a semester hour
system (at Coe, 1 course credit = 4 semester hours), only courses with
three or more semester hours can be used to fulfill any major or general
education requirements. From institutions on other than a semester hour
system, only courses equivalent to at least 0.75 course credits can be
used to fulfill any major or general education requirement. In some
cases, in consultation with the Registrar, multiple courses within the
same field may be used to fulfill one requirement.

Transfer credits earned after high-school graduation and before Coe
matriculation count towards the eight term, full-time residence
requirement (see p.~230). Full-time enrollment may include participation
in Coe College exchange programs, ACM off-campus study programs, and
other approved off-campus study programs.

All students must complete at Coe at least 40\% of the total course
credits required for each declared major or minor or three course
credits, whichever is greater. In addition to completing at least one
major area of study, transfer students must abide with the following to
complete the requirements for graduation:

\begin{itemize}
\tightlist
\item
  First-Year Experience. Transfer students are not required to fulfill
  the requirements of the First-Year Experience, if they have completed
  at least one full-time college term since graduation from high school.
\item
  Writing Emphasis. (See Writing Emphasis Courses, p.~21).
\item
  General Education. Requirements include Liberal Arts selections in the
  four divisional areas (Natural and Mathematical Sciences, Social
  Sciences, Humanities, Fine Arts) and Diverse Cultural Perspectives
  courses. Any courses accepted in transfer for at least 0.75 course
  credit that fit the criteria of the Liberal Arts and/or Diverse
  Cultural Perspectives core groups can be applied towards the general
  education requirements as determined by the Registrar. Advanced
  Placement and International Baccalaureate courses may not be used to
  meet any part of the General Education requirements.
\item
  Academic Practicum. Transfer students are required to fulfill this
  requirement.
\end{itemize}

\hypertarget{areas-of-study}{%
\chapter*{AREAS OF STUDY}\label{areas-of-study}}
\addcontentsline{toc}{chapter}{AREAS OF STUDY}

\markboth{AREAS OF STUDY}{AREAS OF STUDY}

The three undergraduate degrees have areas of study associated with
them. The Bachelor of Science in Nursing's area of study is nursing; the
Bachelor of Music's areas of study are performance, composition, and
education. The Bachelor of Arts' areas of study, commonly referred to as
majors, are listed below.

Students should declare an area of study by the end of their sophomore
year. All students must earn at least a 2.00 GPA in courses required to
complete their areas of study, as well as meet specific requirements set
forth for the areaof study.

\begin{itemize}
\tightlist
\item
  Accounting, Managerial
\item
  Accounting, Public
\item
  African American Studies
\item
  Art
\item
  Art History
\item
  Asian Studies
\item
  Biology
\item
  Business Administration
\item
  Chemistry
\item
  Communication Studies
\item
  Computer Science
\item
  Creative Writing
\item
  Data Science
\item
  Economics
\item
  Elementary Education
\item
  English
\item
  Film Studies
\item
  French \& Francophone Studies
\item
  General Science
\item
  History
\item
  Interdisciplinary
\end{itemize}

\begin{itemize}
\tightlist
\item
  French \& Francophone Studies
\item
  Interdisciplinary Studies*
\item
  International Business
\item
  International Economics
\item
  International Studies
\item
  Kinesiology
\item
  Literature
\item
  Mathematics
\item
  Music
\item
  Philosophy
\item
  Physics
\item
  Political Science
\item
  Psychology
\item
  Religion
\item
  Social \& Criminal Justice
\item
  Sociology
\item
  Spanish
\item
  Spanish Studies
\item
  Theatre Arts
\item
  Writing (Rhetoric)
\end{itemize}

* A coherent interdisciplinary sequence of courses devised by the
student, in consultation with faculty, suited to the student's
individual goals and approved by the Academic Policies Committee (see
p.~131). In addition to the areas of study/majors listed above, the
following \textbf{COLLATERAL MAJORS} are offered, which require a
student to satisfy the requirements of a major from the list above in
addition to the selected collateral major.

\begin{itemize}
\tightlist
\item
  Biochemistry
\item
  Molecular Biology
\item
  Public Relations
\item
  Environmental Science
\item
  Neuroscience
\item
  Environmental Studies
\item
  Organizational Science
\end{itemize}

\hypertarget{areas-of-study-minor-for-b.a.}{%
\subsection*{AREAS OF STUDY (MINOR) FOR
B.A.}\label{areas-of-study-minor-for-b.a.}}
\addcontentsline{toc}{subsection}{AREAS OF STUDY (MINOR) FOR B.A.}

\begin{itemize}
\tightlist
\item
  African American Studies
\item
  Anthropology
\item
  Art
\item
  Art History
\item
  Asian Studies
\item
  Chemistry
\item
  Classical Studies
\item
  Communication Studies
\item
  Computer Science
\item
  Creative Writing
\item
  Data Science
\item
  Economics
\item
  English
\item
  Film Studies
\item
  French \& Francophone Studies
\item
  Gender and Sexuality Studies
\item
  Health \& Society Studies
\item
  History
\item
  Interdisciplinary French \&
\item
  Francophone Studies
\item
  International Economics
\item
  Mathematics
\end{itemize}

\hypertarget{areas-of-study-majors-for-b.m.}{%
\subsection*{AREAS OF STUDY (MAJORS) FOR
B.M.}\label{areas-of-study-majors-for-b.m.}}
\addcontentsline{toc}{subsection}{AREAS OF STUDY (MAJORS) FOR B.M.}

\begin{itemize}
\tightlist
\item
  Keyboard or Instrumental Performance
\item
  Vocal Performance
\item
  Composition
\item
  Instrumental Music Education
\item
  Vocal Music Education
\end{itemize}

\hypertarget{area-of-study-major-for-b.s.n.}{%
\subsection*{AREA OF STUDY (MAJOR) FOR
B.S.N.}\label{area-of-study-major-for-b.s.n.}}
\addcontentsline{toc}{subsection}{AREA OF STUDY (MAJOR) FOR B.S.N.}

\begin{itemize}
\tightlist
\item
  Nursing
\end{itemize}

\hypertarget{practicum}{%
\chapter*{PRACTICUM}\label{practicum}}
\addcontentsline{toc}{chapter}{PRACTICUM}

\markboth{PRACTICUM}{PRACTICUM}

A practicum experience is required of all students for all undergraduate
degrees, except those earning second degrees.

Typically completed in the student's junior or senior year, all practica
are experiences that integrate academic components with career or other
life goals and are significant educational exercises outside the
classroom. A practicum experience can consist of an internship,
off-campus study, community-based project, honors project, or some other
kind of independent activity.

Depending upon the type selected, some practica are graded A--F, while
others are P/NP. Some practica are credit bearing, while others are not.
In some instances, the practicum must be approved by the student's major
department.

\begin{enumerate}
\def\labelenumi{\arabic{enumi}.}
\tightlist
\item
  Full-Term (16-week) Off-Campus Study
\item
  Wilderness Field Station Summer Courses
\item
  Crimson Fellows Thesis or Crimson Fellows Project, etc. as stated
\item
  Independent Project (in list of courses that follows starred courses *
  require department approval for practicum credit):
\item
  †Internship (see a complete listing of internships on p.~29)
\item
  †Community-Based Project (see course description on p.~70)
\end{enumerate}

†A maximum of two course credits earned through any combination of
Internships and Community-Based Projects may be included in the 32
course credits required for graduation.

\begin{itemize}
\tightlist
\item
  AAM-444 Independent Study
\item
  ANT-205 Archaeological Field School
\item
  ANT-444 Independent Study
\item
  ANT-474 Research Participation
\item
  ARH-444 Independent Study in Art History
\item
  ARH-474 Senior Seminar II \& Senior Project (WE)
\item
  ART-394 Directed Studies in Art
\item
  ART-444 Independent Study
\item
  ART-474 Senior Seminar II \& Senior Exhibition
\item
  AT-40\_ Clinical Athletic Training (successful completion of sequence
  of AT-20\_/ -30\_ and-40\_ - required to receive full credit)
\item
  BIO-115 Marine Biology
\item
  BIO-444 Independent Study
\item
  BIO-454 Research Participation
\item
  BIO-462 Advanced Biology Laboratory I
\item
  BUS-444 Independent Study
\item
  BUS-454 Research in Business
\item
  CHM-444 Independent Study
\item
  CHM-454 Undergraduate Summer Research
\item
  COM-394 Directed Studies in Communication Studies (WE)
\item
  COM-444 Independent Study in Communication Studies* (WE)
\item
  CRW-112 Advanced Literary Magazine Editing (two terms)
\item
  CRW-394 Directed Studies in Creative Writing (WE)
\item
  CRW-492 Manuscript Workshop (WE)
\item
  CS-444 Independent Study
\item
  CS-454 Research in Computer Science
\item
  DS-444 Independent Study in Data Science
\item
  DS-454 Research in Data Science
\item
  ECO-444 Independent Study
\item
  ECO-454 Research in Economics
\item
  EDU-215 Practicum in Education (WE)
\item
  EDU-481 Student Teaching in Art at the Secondary School (WE)
\item
  EDU-482 Student Teaching in Physical Education at the Secondary School
  (WE)
\item
  EDU-483 Student Teaching in Art at the Elementary School (WE)
\item
  EDU-485 Student Teaching in Physical Education at the Elementary
  School (WE)
\item
  EDU-489 Student Teaching in the Senior High School (WE)
\item
  EDU-490 Student Teaching in Middle School or Junior High School (WE)
\item
  EDU-491 Student Teaching in the Upper Elementary Grades: Grades 3-6
  (WE)
\end{itemize}

\part{THE EDUCATIONAL PROGRAM}

Coe College grants the undergraduate degrees of Bachelor of Arts (B.A.),
Bachelor of Music (B.M.), and Bachelor of Science in Nursing (B.S.N.).
The Bachelor of Arts degree (see p.~19) is earned upon fulfillment of
the conditions described below, including a major chosen from over 40
fields of study or an approved interdisciplinary major within the arts
and sciences. Students interested in music may earn either a B.A. or a
B.M. The B.M. degree requirements for students who wish to pursue music
as a profession or to prepare to teach music, can be found on p.~169.
The requirements for a B.S.N. degree can be found on p.~183.

\hypertarget{academic-calendar-1}{%
\chapter*{2022--2023 ACADEMIC CALENDAR}\label{academic-calendar-1}}
\addcontentsline{toc}{chapter}{2022--2023 ACADEMIC CALENDAR}

\markboth{2022--2023 ACADEMIC CALENDAR}{2022--2023 ACADEMIC CALENDAR}

\hypertarget{fall-term-2022-1}{%
\subsection*{FALL TERM 2022}\label{fall-term-2022-1}}
\addcontentsline{toc}{subsection}{FALL TERM 2022}

\begin{longtable}[]{@{}
  >{\raggedright\arraybackslash}p{(\columnwidth - 2\tabcolsep) * \real{0.3333}}
  >{\raggedleft\arraybackslash}p{(\columnwidth - 2\tabcolsep) * \real{0.6667}}@{}}
\toprule\noalign{}
\endhead
\bottomrule\noalign{}
\endlastfoot
Tues, August 23 & Open Registration \\
Wed, August 24 & Classes Begin \\
Tues, August 30 & Last Day to Add or Drop a Full-Term, or 1st Half-Term
Course without a W \\
Fri, September 2 & Census Date, Attendance Due on My.Coe \\
Mon, September 5 & No Classes (Holiday) Labor Day \\
Fri, September 9 & Date of Record \\
Thurs, September 22 & Last Day to Withdraw \&/ or Change Method of
Grading for First-Half Courses \\
Mon--Tues, October 10--11 & Fall Term Break \\
Thurs--Sat, October 13--15 & Homecoming \\
Mon, October 17 & Mid-Term Progress Report Due on My.Coe (11:59 PM) \\
Mon, October 17 & Begin Second-Half Term Courses \\
Fri, October 21 & May Term deposit and registration due \\
Fri, October 21 & Last Day to Add or Drop a Second-Half Term Course
without a W \\
Mon, October 24-Fri, November 11 & Advising Season \\
Fri, October 28 & Last Day to Withdraw \&/or Change Method of Grading
for Full-Term Courses \\
Mon--Fri, November 14--18 & Registration for Spring Term 2023 \\
Tues, November 15 & Last Day to Withdraw \&/or Change Method of Grading
for Second-Half Courses \\
Mon--Fri, November 21--25 & No Classes (Holiday) Thanksgiving Recess \\
Thurs, December 1 & Open Online Registration until Last Day of Finals \\
Fri, December 2 & Last Day of Fall Term Classes \\
Sat, December 3 & Reading Day \\
Mon--Thurs, December 5--8 & Final Exams \\
Wed, December 14 & Final Grades and Attendance are Due from the Faculty
on My.Coe (11:59 PM) \\
\end{longtable}

\hypertarget{spring-term-2023-1}{%
\subsection*{SPRING TERM 2023}\label{spring-term-2023-1}}
\addcontentsline{toc}{subsection}{SPRING TERM 2023}

\begin{longtable}[]{@{}
  >{\raggedright\arraybackslash}p{(\columnwidth - 2\tabcolsep) * \real{0.3333}}
  >{\raggedleft\arraybackslash}p{(\columnwidth - 2\tabcolsep) * \real{0.6667}}@{}}
\toprule\noalign{}
\endhead
\bottomrule\noalign{}
\endlastfoot
Wed, January 11 & Classes Begin \\
Mon, January 16 & No Day Classes (Holiday) Martin Luther King Jr.~Day \\
Wed, January 18 & Last Day to Add or Drop a Full-Term or 1st Half-Term
Course without a W \\
Fri, January 20 & Census Date, Attendance Due on My.Coe \\
Thurs, January 26 & Date of Record Mon, February 6 \\
Mon, February 6 & Last Day to Withdraw \&/or Change Method of Grading
for First-Half Courses \\
Mon, March 6 & Begin Second-Half Term Courses \\
Tues, March 7 & Mid-Term Progress Report Due Online (11:59 PM) \\
Fri, March 10 & Last Day to Add or Drop a Second-Half Term Course
without a W \\
Mon--Fri, March 13--17 & Spring Term Recess (Starting after Class on
Friday, March 10) \\
Mon--Fri, March 20--April 7 & Advising Season Thurs, March 30 \\
Thurs, March 30 & Last Day to Withdraw \&/or Change Method of Grading
for Full-Term Courses \\
Mon--Fri, April 10--14 & Registration for Fall Term 2023 \\
Mon, April 10 & Summer Registration Opens \\
Tues, April 11 & Last Day to Withdraw \&/or Change Method of Grading for
Second-Half Courses \\
Wed, April 12 & Student Research Symposium ** No Day Classes ** Evening
Classes Will Meet \\
Fri, April 28 & Last Day of Spring Term Classes \\
Sat, April 29 & Reading Day \\
Mon--Thurs, May 1--4 & Final Exams \\
Sat, May 6 & Honors Convocation / Baccalaureate \\
Sun, May 7 & Commencement \\
Tues, May 9 & Final Grades and Attendance are Due for Non-Graduating
Students on My.Coe (11:59 PM) \\
\end{longtable}

\hypertarget{may-term-2023-1}{%
\subsection*{MAY TERM 2023}\label{may-term-2023-1}}
\addcontentsline{toc}{subsection}{MAY TERM 2023}

\begin{longtable}[]{@{}
  >{\raggedright\arraybackslash}p{(\columnwidth - 2\tabcolsep) * \real{0.3333}}
  >{\raggedleft\arraybackslash}p{(\columnwidth - 2\tabcolsep) * \real{0.6667}}@{}}
\toprule\noalign{}
\endhead
\bottomrule\noalign{}
\endlastfoot
Wed, May 10 & Classes Begin \\
Fri, May 12 & Last Day to Add or Drop a Course Without a W \\
Mon, May 15 & Attendance Due Online; Last Day to Change Method of
Grading for May Term \\
Wed, May 17 & Last Day to Withdraw from May Term Courses \\
Fri--Sat, May 19--20 & Meeting of the Board of Trustees \\
Mon, May 29 & No Classes (Holiday) Memorial Day \\
Fri, June 2 & Last Day of May Term Classes \\
Sat, June 3 & Residence Halls Close \\
Sun, June 4 & Final Grades and Attendance for May Term Due on My.Coe
(11:59 PM) \\
\end{longtable}

\hypertarget{summer-term-2023-1}{%
\subsection*{SUMMER TERM 2023*}\label{summer-term-2023-1}}
\addcontentsline{toc}{subsection}{SUMMER TERM 2023*}

\begin{longtable}[]{@{}
  >{\raggedright\arraybackslash}p{(\columnwidth - 2\tabcolsep) * \real{0.3333}}
  >{\raggedleft\arraybackslash}p{(\columnwidth - 2\tabcolsep) * \real{0.6667}}@{}}
\toprule\noalign{}
\endhead
\bottomrule\noalign{}
\endlastfoot
Mon, June 5 & Classes Begin \\
Fri, June 9 & Last Day to Add or Drop a Full-Term Course Without a W \\
Mon, June 19 & No Classes (Holiday) Juneteenth \\
Fri, June 23 & Last Day to Change Method of Grading \&/or Withdraw from
Block A Courses \\
Tues, July 4 & No Classes (Holiday) Independence Day \\
Fri, July 7 & Last Day of Block A Courses \\
Mon, July 10 & Classes Begin: Block B Courses \\
Tues, July 11 & Mid-Term Progress Report Due Online (11:59 PM); Block A
Final Grades Due \\
Fri, July 14 & Last Day to Add or Drop a Course Without a W for Block B
Courses \\
Mon, July 17 & Last Day to Change Method of Grading \&/or Withdraw from
Full Term Courses \\
Tues, July 25 & Last Day to Change Method of Grading \&/or Withdraw from
Block B Courses \\
Fri, August 11 & Last Day of Term Classes: Block B and Full-Term
Courses \\
Tues, August 15 & Final Grades and Attendance Due (Block B and
Full-Term) on My.Coe (11:59 PM) \\
\end{longtable}

*Courses taught at the Wilderness Field Station are subject to the broad
dates of the summer, but will provide a specific add/drop and withdraw
calendar to students at the time of application.

\hypertarget{here-is-a-list-of-courses}{%
\chapter{Here is a list of courses:}\label{here-is-a-list-of-courses}}

\begin{itemize}
\tightlist
\item
  {``DS230 - Data Centric Programming''} (n.d.)
\item
  {``DS230 - Data Centric Programming''} (n.d.)
\item
  (\textbf{abcd?})
\item
  \textbf{\protect\hyperlink{courses-in-data-science}{DS-230}}
\item
  \textbf{DS-230 Introduction to Data Science}

  \begin{itemize}
  \item
    Provides a programming experience with applications that focus on
    data handling tasks. Students examine programming techniques to
    acquire and manage data from a variety of sources and formats; use
    relational databases to store and query data; and explore techniques
    to work with semi-structured and unstructured data sets.
    Prerequisite: Introduction to Programming (CS-125) or consent of
    instructor
  \end{itemize}
\item
  \textbf{DS-260}

  \begin{itemize}
  \item
    Studies intermediate data analytic techniques and concepts to
    visualize quantitative data. This course expands the mathematical
    background of students, with topics from statistical analysis and
    linear algebra. Students will learn advanced visualization
    techniques, with particular emphasis on creating graphics and
    animations using visualization libraries. Prerequisite: Data-Centric
    Computing (DS-230)
  \end{itemize}
\end{itemize}

\hypertarget{data-science}{%
\section{Data Science}\label{data-science}}

Here is a description of the major.

And the major requirments: - abcd - abcd - abcd

\hypertarget{courses-in-data-science}{%
\subsection{Courses in Data Science}\label{courses-in-data-science}}

\begin{itemize}
\item
  \textbf{DS-230 Introduction to Data Science}

  Provides a programming experience with applications that focus on data
  handling tasks. Students examine programming techniques to acquire and
  manage data from a variety of sources and formats; use relational
  databases to store and query data; and explore techniques to work with
  semi-structured and unstructured data sets. Prerequisite: Introduction
  to Programming (CS-125) or consent of instructor
\item
  \textbf{DS-260 Data Analysis and Visualization}

  Studies intermediate data analytic techniques and concepts to
  visualize quantitative data. This course expands the mathematical
  background of students, with topics from statistical analysis and
  linear algebra. Students will learn advanced visualization techniques,
  with particular emphasis on creating graphics and animations using
  visualization libraries. Prerequisite: Data-Centric Computing (DS-230)
\end{itemize}

\hypertarget{data-science-1}{%
\chapter{Data Science}\label{data-science-1}}

Here is a description of the major.

And the major requirments: - abcd - abcd - abcd

\hypertarget{courses-in-data-science-1}{%
\subsection{Courses in Data Science}\label{courses-in-data-science-1}}

\begin{itemize}
\item
  \textbf{DS-230 Introduction to Data Science}

  Provides a programming experience with applications that focus on data
  handling tasks. Students examine programming techniques to acquire and
  manage data from a variety of sources and formats; use relational
  databases to store and query data; and explore techniques to work with
  semi-structured and unstructured data sets. Prerequisite: Introduction
  to Programming (CS-125) or consent of instructor
\item
  \textbf{DS-260 Data Analysis and Visualization}

  Studies intermediate data analytic techniques and concepts to
  visualize quantitative data. This course expands the mathematical
  background of students, with topics from statistical analysis and
  linear algebra. Students will learn advanced visualization techniques,
  with particular emphasis on creating graphics and animations using
  visualization libraries. Prerequisite: Data-Centric Computing (DS-230)
\end{itemize}

\hypertarget{french-francophone-studies}{%
\chapter{French \& Francophone
Studies}\label{french-francophone-studies}}

Janca-Aji

The French \& Francophone Studies program is an intercultural and
interdisciplinary program featuring courses in language, cultural
history, literature and cinema, translation and interpretation, and
pre-professional studies. Students are strongly encouraged to pursue
opportunities for immersive and experiential learning through study
abroad, May Term courses, service learning, and community-based projects
and to explore ways to incorporate French in and with other major(s).

\hypertarget{french-francophone-studies-major}{%
\subsection{French \& Francophone Studies
Major}\label{french-francophone-studies-major}}

A grade of ``C'' (2.0) or higher must be earned in all courses counted
toward a major in French \& Francophone Studies. Students complete eight
credits of 300- to 400-level courses in French. FRE-315 Oral and Written
Communication Skills (WE) is required. Up to three credits may be earned
by successfully completing a study abroad program in France or a
francophone country that is approved by the College and the department.
Up to one credit may be earned from a list of approved courses taught in
English. One credit from a course taught in French must be taken in the
senior year. FRE-499 Exit Exam and Interview is required during the
final term before graduation.

\hypertarget{french-francophone-studies-minor}{%
\subsection{French \& Francophone Studies
Minor}\label{french-francophone-studies-minor}}

A grade of ``C'' (2.0) or higher must be earned in all courses counted
toward a minor in French \& Francophone Studies. Students complete a
minimum of four credits of 300- to 400-level courses in French. FRE-315
Oral and Written Communication Skills (WE) is required. Up to one credit
may be earned from a list of approved courses taught in English. FRE-499
Exit Exam and Interview is required during the final term before
graduation.

\hypertarget{interdisciplinary-french-francophone-studies-major}{%
\subsection{Interdisciplinary French \& Francophone Studies
Major}\label{interdisciplinary-french-francophone-studies-major}}

A grade of ``C'' (2.0) or higher must be earned in all courses counted
toward a major in Interdisciplinary French \& Francophone Studies.
Students complete 1) \textbf{four} credits of courses taught in French
at any level, including FRE-315 Oral and Written Communication Skills
(WE), 2) a departmentally approved term-long study abroad experience in
France or a Francophone country, and 3) \textbf{four} credits from
courses, taught in either French or English, from the list of approved
courses which include at least two different prefixes and demonstrate
thematic coherence. Courses not on this list may count for credit with
approval of the program coordinator. FRE-499 Exit Exam and Interview is
required during the final term before graduation.

\hypertarget{interdisciplinary-french-francophone-studies-minor}{%
\subsection{Interdisciplinary French \& Francophone Studies
Minor}\label{interdisciplinary-french-francophone-studies-minor}}

A grade of ``C'' (2.0) or higher must be earned in all courses counted
toward a minor in Interdisciplinary French \& Francophone Studies.
Students complete 1) \textbf{four} credits of courses taught in French
at any level, including FRE-315 Oral and Written Communication Skills
(WE), and 2) \textbf{three} credits from courses, taught in either
French or English, from the list of approved courses which include at
least two different prefixes and demonstrate thematic coherence. Courses
not on this list may count for credit with approval of the program
coordinator. FRE-499 Exit Exam and Interview is required during the
final term before graduation.

\hypertarget{courses-taught-in-english-that-can-be-used-for-credit-in-french-francophone-studies}{%
\subsection{Courses Taught in English that can be used for credit in
French \& Francophone
Studies}\label{courses-taught-in-english-that-can-be-used-for-credit-in-french-francophone-studies}}

\begin{verbatim}
?var:c.arh201.long


?var:c.arh218.long


?var:c.arh231.long


?var:c.arh307.long


?var:c.com236.long


?var:c.eng146.long


?var:c.fre145.long


?var:c.fre146.long


?var:c.fre148.long


?var:c.fre158.long


?var:c.his238.long


?var:c.his248.long


?var:c.his272.long


?var:c.his288.long


?var:c.his355.long


?var:c.his365.long


?var:c.his372.long


?var:c.phl230.long


?var:c.phl240.long


?var:c.phl255.long


?var:c.phl305.long


?var:c.phl345.long


?var:c.pol298.long


?var:c.rel148.long


?var:c.phl178.long
\end{verbatim}

\hypertarget{courses-in-french}{%
\subsection{Courses in French}\label{courses-in-french}}

\begin{itemize}
\item
  \textbf{\textbf{?var:c.fre115.long}}

  \textbf{?var:c.fre115.desc}
\item
  \textbf{\textbf{?var:c.fre125.long}}

  \textbf{?var:c.fre125.desc}
\item
  \textbf{\textbf{?var:c.fre145.long}}

  \textbf{?var:c.fre145.desc}
\item
  \textbf{\textbf{?var:c.fre146.long}}

  \textbf{?var:c.fre146.desc}
\item
  \textbf{\textbf{?var:c.fre148.long}}

  \textbf{?var:c.fre148.desc}
\item
  \textbf{\textbf{?var:c.fre158.long}}

  \textbf{?var:c.fre158.desc}
\item
  \textbf{\textbf{?var:c.fre199.long}}

  \textbf{?var:c.fre199.desc}
\item
  \textbf{\textbf{?var:c.fre235.long}}

  \textbf{?var:c.fre235.desc}
\item
  \textbf{\textbf{?var:c.fre315.long}}

  \textbf{?var:c.fre315.desc}
\item
  \textbf{\textbf{?var:c.fre335.long}}

  \textbf{?var:c.fre335.desc}
\item
  \textbf{\textbf{?var:c.fre339.long}}

  \textbf{?var:c.fre339.desc}
\item
  \textbf{\textbf{?var:c.fre340.long}}

  \textbf{?var:c.fre340.desc}
\item
  \textbf{\textbf{?var:c.fre345.long}}

  \textbf{?var:c.fre345.desc}
\item
  \textbf{\textbf{?var:c.fre394.long}}

  \textbf{?var:c.fre394.desc}
\item
  \textbf{\textbf{?var:c.fre400.long}}

  \textbf{?var:c.fre400.desc}
\item
  \textbf{\textbf{?var:c.fre444.long}}

  \textbf{?var:c.fre444.desc}
\item
  \textbf{\textbf{?var:c.fre446.long}}

  \textbf{?var:c.fre446.desc}
\item
  \textbf{\textbf{?var:c.fre452.long}}

  \textbf{?var:c.fre452.desc}
\item
  \textbf{\textbf{?var:c.fre494.long}}

  \textbf{?var:c.fre494.desc}
\item
  \textbf{\textbf{?var:c.fre495.long}}

  \textbf{?var:c.fre495.desc}
\item
  \textbf{\textbf{?var:c.fre499.long}}

  \textbf{?var:c.fre499.desc}
\end{itemize}

\bookmarksetup{startatroot}

\hypertarget{references}{%
\chapter*{References}\label{references}}
\addcontentsline{toc}{chapter}{References}

\markboth{References}{References}

\hypertarget{refs}{}
\begin{CSLReferences}{0}{0}
\end{CSLReferences}

\backmatter
\printindex

\end{document}
