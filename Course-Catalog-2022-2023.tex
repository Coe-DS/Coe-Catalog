% Options for packages loaded elsewhere
\PassOptionsToPackage{unicode}{hyperref}
\PassOptionsToPackage{hyphens}{url}
%
\documentclass[
  letterpaper,
]{scrbook}
\usepackage{amsmath,amssymb}
\usepackage{lmodern}
\usepackage{iftex}
\ifPDFTeX
  \usepackage[T1]{fontenc}
  \usepackage[utf8]{inputenc}
  \usepackage{textcomp} % provide euro and other symbols
\else % if luatex or xetex
  \usepackage{unicode-math}
  \defaultfontfeatures{Scale=MatchLowercase}
  \defaultfontfeatures[\rmfamily]{Ligatures=TeX,Scale=1}
\fi
% Use upquote if available, for straight quotes in verbatim environments
\IfFileExists{upquote.sty}{\usepackage{upquote}}{}
\IfFileExists{microtype.sty}{% use microtype if available
  \usepackage[]{microtype}
  \UseMicrotypeSet[protrusion]{basicmath} % disable protrusion for tt fonts
}{}
\makeatletter
\@ifundefined{KOMAClassName}{% if non-KOMA class
  \IfFileExists{parskip.sty}{%
    \usepackage{parskip}
  }{% else
    \setlength{\parindent}{0pt}
    \setlength{\parskip}{6pt plus 2pt minus 1pt}}
}{% if KOMA class
  \KOMAoptions{parskip=half}}
\makeatother
\usepackage{xcolor}
\usepackage{longtable,booktabs,array}
\usepackage{calc} % for calculating minipage widths
% Correct order of tables after \paragraph or \subparagraph
\usepackage{etoolbox}
\makeatletter
\patchcmd\longtable{\par}{\if@noskipsec\mbox{}\fi\par}{}{}
\makeatother
% Allow footnotes in longtable head/foot
\IfFileExists{footnotehyper.sty}{\usepackage{footnotehyper}}{\usepackage{footnote}}
\makesavenoteenv{longtable}
\usepackage{graphicx}
\makeatletter
\def\maxwidth{\ifdim\Gin@nat@width>\linewidth\linewidth\else\Gin@nat@width\fi}
\def\maxheight{\ifdim\Gin@nat@height>\textheight\textheight\else\Gin@nat@height\fi}
\makeatother
% Scale images if necessary, so that they will not overflow the page
% margins by default, and it is still possible to overwrite the defaults
% using explicit options in \includegraphics[width, height, ...]{}
\setkeys{Gin}{width=\maxwidth,height=\maxheight,keepaspectratio}
% Set default figure placement to htbp
\makeatletter
\def\fps@figure{htbp}
\makeatother
\setlength{\emergencystretch}{3em} % prevent overfull lines
\providecommand{\tightlist}{%
  \setlength{\itemsep}{0pt}\setlength{\parskip}{0pt}}
\setcounter{secnumdepth}{5}
% Make \paragraph and \subparagraph free-standing
\ifx\paragraph\undefined\else
  \let\oldparagraph\paragraph
  \renewcommand{\paragraph}[1]{\oldparagraph{#1}\mbox{}}
\fi
\ifx\subparagraph\undefined\else
  \let\oldsubparagraph\subparagraph
  \renewcommand{\subparagraph}[1]{\oldsubparagraph{#1}\mbox{}}
\fi
\RedeclareSectionCommand[beforeskip=-2.5ex plus -1ex minus -.2ex,afterskip=0.3ex plus -.2ex]{section}
\usepackage[top=0.5in, left=0.5in, right=0.5in, bottom=1in]{geometry}
\bibliographystyle{catalog}
\usepackage{makeidx}
\makeindex
\renewcommand\toprule[2]\relax
\renewcommand\bottomrule[2]\relax
\usepackage{fancyhdr}
\lfoot{\textit{Coe College (2022-2023)}}
\renewcommand{\part}[1]{\addcontentsline{toc}{part}{#1}}
\makeatletter
\@ifpackageloaded{bookmark}{}{\usepackage{bookmark}}
\makeatother
\makeatletter
\@ifpackageloaded{caption}{}{\usepackage{caption}}
\AtBeginDocument{%
\ifdefined\contentsname
  \renewcommand*\contentsname{Table of contents}
\else
  \newcommand\contentsname{Table of contents}
\fi
\ifdefined\listfigurename
  \renewcommand*\listfigurename{List of Figures}
\else
  \newcommand\listfigurename{List of Figures}
\fi
\ifdefined\listtablename
  \renewcommand*\listtablename{List of Tables}
\else
  \newcommand\listtablename{List of Tables}
\fi
\ifdefined\figurename
  \renewcommand*\figurename{Figure}
\else
  \newcommand\figurename{Figure}
\fi
\ifdefined\tablename
  \renewcommand*\tablename{Table}
\else
  \newcommand\tablename{Table}
\fi
}
\@ifpackageloaded{float}{}{\usepackage{float}}
\floatstyle{ruled}
\@ifundefined{c@chapter}{\newfloat{codelisting}{h}{lop}}{\newfloat{codelisting}{h}{lop}[chapter]}
\floatname{codelisting}{Listing}
\newcommand*\listoflistings{\listof{codelisting}{List of Listings}}
\makeatother
\makeatletter
\makeatother
\makeatletter
\@ifpackageloaded{caption}{}{\usepackage{caption}}
\@ifpackageloaded{subcaption}{}{\usepackage{subcaption}}
\makeatother
\ifLuaTeX
  \usepackage{selnolig}  % disable illegal ligatures
\fi
\IfFileExists{bookmark.sty}{\usepackage{bookmark}}{\usepackage{hyperref}}
\IfFileExists{xurl.sty}{\usepackage{xurl}}{} % add URL line breaks if available
\urlstyle{same} % disable monospaced font for URLs
\hypersetup{
  pdftitle={Course Catalog 2022-2023},
  hidelinks,
  pdfcreator={LaTeX via pandoc}}

\title{Course Catalog 2022-2023}
\author{}
\date{2022-08-01}

\begin{document}
\frontmatter
\maketitle

\renewcommand*\contentsname{Table of contents}
{
\setcounter{tocdepth}{1}
\tableofcontents
}
\mainmatter
\bookmarksetup{startatroot}

\chapter*{PREFACE}\label{preface}

\markboth{PREFACE}{PREFACE}

\section*{Non-Discrimination}\label{non-discrimination}

\markright{Non-Discrimination}

Coe College does not discriminate on the basis of race, color,
ethnicity, age, religion, national origin, sexual orientation, gender
identity, sex, marital status, disability, or status as a U.S. Veteran.
All students have equal access to the facilities, financial aid, and
programs of the College.

\section*{Higher Education Opportunity Act
(HEOA)}\label{higher-education-opportunity-act-heoa}

\markright{Higher Education Opportunity Act (HEOA)}

The College complies with Readmission Requirements for Service Members
as outlined in the Higher Education Opportunity Act section 487. This
applies to active duty in the Armed Forces, whether voluntary or
involuntary, including service as a member of the National Guard or
Reserve, for a period of more than 30 days under a call or order to
active duty.

The HEOA provides that a prompt readmission of a previously enrolled or
admitted student may not be denied to a service member of the uniformed
services for reasons relating to that service.~ In addition, a student
who is readmitted under this section must be readmitted with the same
academic status as the student had when they attended the college.

\section*{Equal Opportunity in
Employment}\label{equal-opportunity-in-employment}

\markright{Equal Opportunity in Employment}

Coe College is an equal opportunity employer in the recruitment and
hiring of faculty and staff.

\section*{Family Educational Rights and Privacy Act
(FERPA)}\label{family-educational-rights-and-privacy-act-ferpa}

\markright{Family Educational Rights and Privacy Act (FERPA)}

The provisions of the Family Educational Rights and Privacy Act (FERPA)
prohibit the College from releasing grades or other information about
academic standing to parents unless the student has released such
information in writing. Further information concerning Coe College
procedures in compliance with FERPA is available in the Office of the
Registrar and included on p.~68 of this catalog.

\section*{Solomon Amendment of 1997}\label{solomon-amendment-of-1997}

\markright{Solomon Amendment of 1997}

Pursuant to the regulations of the Solomon Amendment of 1997, Coe
College is required to make student recruiting information available to
military recruiters who request it.

\section*{Reservation of the Right to
Modify}\label{reservation-of-the-right-to-modify}

\markright{Reservation of the Right to Modify}

The provisions of this catalog are to be considered directive in
character and not as an irrevocable contract between the student and the
College. The College reserves the right to make changes that seem
necessary or desirable, including course and program cancellations.
Responsibility for understanding and meeting graduation requirements as
stated in the Coe College Catalog rests entirely with the student.
Faculty advisors and the Registrar will assist in every way possible.

\part{INTRODUCTORY RESOURCES}

\chapter{MISSION STATEMENT OF THE
COLLEGE}\label{sec-mission-statement-of-college}

\begin{center}
\includegraphics{catalog_sections/graphics/coe_bell.png}
\end{center}

\index{Mission Statement}

Coe College is a national, residential liberal arts college offering a
broad array of programs in the arts, sciences and professions. Our
mission is to prepare students for meaningful lives and fulfilling
careers in a diverse, interconnected world. Coe's success will be judged
by the success of our graduates.

\emph{Coe College admits students without regard to sex, race, creed,
color, handicap, sexual orientation, national, or ethnic origin. All
students have equal access to the facilities, financial aid, and
programs of the College.}

\chapter{FAST FACTS}\label{sec-fast-facts}

\textbf{COE COLLEGE} is a private, four-year co-educational liberal arts
college that was founded in 1851 and is historically affiliated with the
Presbyterian Church (U.S.A.), yet is ecumenical in practice and outlook.

\subsection{LOCATION}\label{location}

Coe is located just 225 miles west of Chicago in Iowa's second largest
city, Cedar Rapids (metropolitan population 175,000). The city is easily
accessible by the interstate highway system, bus services and several
airlines.

\subsection{CAMPUS}\label{campus}

Situated on 65 acres in the center of the metropolitan area, the campus
is urban but enclosed, with 30 buildings occupying an attractive
landscape.

\subsection{ENROLLMENT}\label{enrollment}

The student body of approximately 1,400 students represents most states
and around 15 foreign countries. All students are required to live on
campus unless they are residents of Cedar Rapids.

\subsection{LIBRARY}\label{library}

Stewart Memorial Library is in the very center of the main campus. It
contains over 500,000 volumes and 16,000 pieces of media, and subscribes
to 3,500 print and online periodical subscriptions providing access to
over 100 databases. Coe's library offers students a variety of research
assistance, study areas, a small theater, preview room, a media editing
room, and a 3D printer and laser cutter. The Library houses the Learning
Commons and the College's permanent art collection.

\subsection{ACCREDITATION}\label{accreditation}

Coe College is accredited by the Higher Learning Commission
(hlcomission.org), an institutional accreditation agency recognized by
the U.S. Department of Education. Coe College's Bachelor of Music
program is accredited by the National Association of Schools of Music;
its education program is accredited by the Iowa Department of Education,
and its Bachelor of Science in Nursing is accredited by the Commission
on Collegiate Nursing Education
(http://aacn.nche.edu/ccne-accreditation).~ Coe College's chemistry
program is certified by the American Chemical Society.~ Copies of
accrediting and approval statements are available online at
https://www.coe.edu/why-coe/consumer-information.

\subsection{MEMBERSHIPS}\label{memberships}

Coe is a charter member of the Associated Colleges of the Midwest, whose
other members are: Beloit, Carleton, Colorado, Cornell, Grinnell, Knox,
Lake Forest, Lawrence, Luther, Macalester, Monmouth, Ripon, and
St.~Olaf. Other memberships include: Phi Beta Kappa, Phi Kappa Phi, the
National Association of Independent Colleges and Universities, the
American Association of Colleges and Universities, Council on
Undergraduate Research, the Association of Presbyterian Colleges and
Universities, the American Rivers Conference, and the National
Collegiate Athletic Association.

\subsection{ACADEMIC PROGRAM}\label{academic-program}

Academic areas of study are offered in managerial or public accounting,
African American studies, art, art history, Asian studies, biology,
business administration, chemistry, communication studies, computer
science, creative writing, data science, economics, elementary
education, English, film studies, French \& francophone studies,
interdisciplinary French \& francophone studies, general science,
history, international business, international economics, international
studies, kinesiology, literature, mathematics, music (B.A. or B.M.),
nursing (B.S.N.), philosophy, physics, political science, psychology,
religion, social \& criminal justice, sociology, Spanish, Spanish
studies, theatre arts, and writing. Interdisciplinary and/or collateral
majors are also available in biochemistry, environmental science,
environmental studies, molecular biology, neuroscience, organizational
science, and public relations. Coe also offers certificate programs in
primary and secondary education.

\subsection{EXTRA-CURRICULAR
ACTIVITIES}\label{extra-curricular-activities}

Students can participate in 11 men's and 11 women's NCAA Division III
varsity athletic teams, 5 coed varsity athletic teams, 8 club sports, 10
nationally affiliated men's and women's social fraternities, and 65
student organizations including but not limited to student government,
newspaper, intramural sports, departmental clubs, and residence hall
activities. Various vocal and instrumental ensembles are available for
course credit.

\subsection{FINANCIAL ASSISTANCE}\label{financial-assistance}

Coe College is committed to assisting those families in need of
financial assistance. The average aid package for incoming students
enrolled during the 2021--2022 academic year totaled more than \$48,000.
The total cost of full-time tuition, room, board, and activity fee for
the 2021--2022 academic year is \$59,324.

\subsection{HISTORY OF COE COLLEGE}\label{sec-history-of-coe-college}

Coe College claims the shortest name of any American institution of
higher education, but the school has actually carried five titles
through its history. When the Rev.~Williston Jones founded the college
in 1851, he called it The School for the Prophets. Cedar Rapids' first
resident minister opened the parlor of his home to a group of young men
with the goal of educating them for the ministry to serve churches in
the Midwest. Two years later, while Jones was canvassing churches in the
East for money to send three of his students to Eastern seminaries, a
Catskills farmer named Daniel Coe stepped forward with a pledge of
\$1,500 and urged Jones to start his own college in the frontier town of
Cedar Rapids. Legend has it that the \$1,500 raised by Coe was brought
west from New York, sewn into the petticoat of a lady visitor traveling
by stagecoach to Iowa. Coe gave this generous gift with the stipulation
that the proposed institute should be ``made available for the education
of females as well as males.'' Accordingly, Coe was coeducational from
its founding.

With Jones' blessing, the Cedar Rapids Collegiate Institute was
incorporated in 1853 by a group of Cedar Rapids leaders chaired by Judge
George Greene. They used Daniel Coe's money to purchase two downtown
lots for the school and 80 acres of farmland on what was then the edge
of town. The farm evolved into today's campus. In 1868, in a failed
attempt to secure the Lewis Parsons estate, the trustees renamed the
school Parsons Seminary. After a period of severe financial
difficulties, the institution was reestablished in honor of its original
benefactor as the Coe Collegiate Institute in 1875.

T.M Sinclair, founder of the Sinclair Meat Packing Company, played the
key financial role in the final step toward the firm establishment of
Coe College. Sinclair liquidated all the debt from Parsons Seminary and
the Cedar Rapids Collegiate Institute. The Sinclair gift made it
practical for the property of the Coe Collegiate Institute---including
the original land paid for by Daniel Coe---to be transferred to Coe
College with the Iowa Presbyterian Synod to assume major responsibility
for the institution. Coe College has operated continuously since its
incorporation under that name on Feb.~2, 1881. From the first, the
College was committed to intellectual excellence. It has continued in
this tradition ever since.

The compact campus on the east edge of Cedar Rapids grew with many
building projects in its early years, including Old Main (1868),
Williston Hall (1881), Marshall Hall (1900), the first gymnasium (1904),
and the first T.M. Sinclair Memorial Chapel (1911). In 1907, Coe earned
accreditation from the North Central Association of Colleges and
Universities. Over the decades, Coe's reputation as a superior liberal
arts college has continued to grow. One recognition of this came in
1949, when Coe was granted a Phi Beta Kappa chapter, a distinction
reserved for about 10\% of American colleges and universities.

Central to the educational philosophy of Coe College is the belief that
a liberal arts education is the best preparation for life. Students have
the opportunity to experience a variety of subjects outside their
respective programs of study. Coe offers more than 40 areas of study
that cover a range of fields. The College awards the following
undergraduate degrees: Bachelor of Arts (B.A.), Bachelor of Music
(B.M.), and Bachelor of Science in Nursing (B.S.N.).

There are a number of factors that contribute to Coe College's strong
academic quality. The key to Coe's tradition of excellence in academic
quality relates directly to small class sizes and the interest shown by
professors to make learning a personalized experience. At Coe, the
average class size is 16, and the student-faculty ratio is 11:1. Classes
are taught by our involved and committed faculty, 91\% of whom hold the
highest degree in their field. This means classes are taught by
experienced professionals who have in-depth knowledge of their subjects.
To provide students with a well-rounded experience and solid preparation
for the future, Coe offers, along with quality instruction from superb
faculty, an abundance of out-of-class opportunities which include
student-faculty collaborative research, honors projects, service
learning, and internships. Within one year of graduation, according to
the annual survey results from the Center for Creativity and Careers,
98\% of Coe graduates are engaged in post-graduate activity such as
employment, graduate school, military, or travel/adventure.

With the addition of the east campus, Coe has nearly doubled in size
since 1989. New facilities on the east side of College Drive include
Athletic Recreation Center (2017), Clark Racquet Center and athletic
fields (1989), Clark Alumni House (1993), Nassif House (1999), and four
student apartment buildings (Morris House and Schlarbaum House in 2000,
Brandt House and Spivey House in 2002). McCabe Hall (2005), named in
honor of former Coe President Joseph E. McCabe, houses the offices of
the president, provost, advancement and alumni relations, and The Center
for Creativity, Careers and Community (C3) making way for the remodeling
of Coe's oldest building, Stuart Hall, and the first significant
addition of classroom space since Peterson Hall was built in the 1960s.
In 2012 and 2013, Peterson Hall of Science was completely renovated to
support Coe's science programs. To further enhance the campus
environment, Coe completed the largest capital project in its history in
2017. Make Your Move -- the Campaign for Eby and Hickok -- included \$24
million in essential enhancements, including an Athletic and Recreation
Complex project as well as the renovation and expansion of Hickok Hall,
one of the College's main academic buildings. The result is vastly
improved academic, recreational, wellness and competition facilities to
benefit future generations of students.

\part{ACADEMIC RESOURCES}

\chapter{2022--2023 ACADEMIC CALENDAR}\label{sec-academic-calenar}

\subsection{FALL TERM 2022}\label{fall-term-2022}

\begin{longtable}[]{@{}
  >{\raggedright\arraybackslash}p{(\columnwidth - 2\tabcolsep) * \real{0.3333}}
  >{\raggedleft\arraybackslash}p{(\columnwidth - 2\tabcolsep) * \real{0.6667}}@{}}
\toprule\noalign{}
\endhead
\bottomrule\noalign{}
\endlastfoot
Tues, August 23 & Open Registration \\
Wed, August 24 & Classes Begin \\
Tues, August 30 & Last Day to Add or Drop a Full-Term, or 1st Half-Term
Course without a W \\
Fri, September 2 & Census Date, Attendance Due on My.Coe \\
Mon, September 5 & No Classes (Holiday) Labor Day \\
Fri, September 9 & Date of Record \\
Thurs, September 22 & Last Day to Withdraw \&/ or Change Method of
Grading for First-Half Courses \\
Mon--Tues, October 10--11 & Fall Term Break \\
Thurs--Sat, October 13--15 & Homecoming \\
Mon, October 17 & Mid-Term Progress Report Due on My.Coe (11:59 PM) \\
Mon, October 17 & Begin Second-Half Term Courses \\
Fri, October 21 & May Term deposit and registration due \\
Fri, October 21 & Last Day to Add or Drop a Second-Half Term Course
without a W \\
Mon, October 24-Fri, November 11 & Advising Season \\
Fri, October 28 & Last Day to Withdraw \&/or Change Method of Grading
for Full-Term Courses \\
Mon--Fri, November 14--18 & Registration for Spring Term 2023 \\
Tues, November 15 & Last Day to Withdraw \&/or Change Method of Grading
for Second-Half Courses \\
Mon--Fri, November 21--25 & No Classes (Holiday) Thanksgiving Recess \\
Thurs, December 1 & Open Online Registration until Last Day of Finals \\
Fri, December 2 & Last Day of Fall Term Classes \\
Sat, December 3 & Reading Day \\
Mon--Thurs, December 5--8 & Final Exams \\
Wed, December 14 & Final Grades and Attendance are Due from the Faculty
on My.Coe (11:59 PM) \\
\end{longtable}

\subsection{SPRING TERM 2023}\label{spring-term-2023}

\begin{longtable}[]{@{}
  >{\raggedright\arraybackslash}p{(\columnwidth - 2\tabcolsep) * \real{0.3333}}
  >{\raggedleft\arraybackslash}p{(\columnwidth - 2\tabcolsep) * \real{0.6667}}@{}}
\toprule\noalign{}
\endhead
\bottomrule\noalign{}
\endlastfoot
Wed, January 11 & Classes Begin \\
Mon, January 16 & No Day Classes (Holiday) Martin Luther King Jr.~Day \\
Wed, January 18 & Last Day to Add or Drop a Full-Term or 1st Half-Term
Course without a W \\
Fri, January 20 & Census Date, Attendance Due on My.Coe \\
Thurs, January 26 & Date of Record Mon, February 6 \\
Mon, February 6 & Last Day to Withdraw \&/or Change Method of Grading
for First-Half Courses \\
Mon, March 6 & Begin Second-Half Term Courses \\
Tues, March 7 & Mid-Term Progress Report Due Online (11:59 PM) \\
Fri, March 10 & Last Day to Add or Drop a Second-Half Term Course
without a W \\
Mon--Fri, March 13--17 & Spring Term Recess (Starting after Class on
Friday, March 10) \\
Mon--Fri, March 20--April 7 & Advising Season Thurs, March 30 \\
Thurs, March 30 & Last Day to Withdraw \&/or Change Method of Grading
for Full-Term Courses \\
Mon--Fri, April 10--14 & Registration for Fall Term 2023 \\
Mon, April 10 & Summer Registration Opens \\
Tues, April 11 & Last Day to Withdraw \&/or Change Method of Grading for
Second-Half Courses \\
Wed, April 12 & Student Research Symposium ** No Day Classes ** Evening
Classes Will Meet \\
Fri, April 28 & Last Day of Spring Term Classes \\
Sat, April 29 & Reading Day \\
Mon--Thurs, May 1--4 & Final Exams \\
Sat, May 6 & Honors Convocation / Baccalaureate \\
Sun, May 7 & Commencement \\
Tues, May 9 & Final Grades and Attendance are Due for Non-Graduating
Students on My.Coe (11:59 PM) \\
\end{longtable}

\subsection{MAY TERM 2023}\label{may-term-2023}

\begin{longtable}[]{@{}
  >{\raggedright\arraybackslash}p{(\columnwidth - 2\tabcolsep) * \real{0.3333}}
  >{\raggedleft\arraybackslash}p{(\columnwidth - 2\tabcolsep) * \real{0.6667}}@{}}
\toprule\noalign{}
\endhead
\bottomrule\noalign{}
\endlastfoot
Wed, May 10 & Classes Begin \\
Fri, May 12 & Last Day to Add or Drop a Course Without a W \\
Mon, May 15 & Attendance Due Online; Last Day to Change Method of
Grading for May Term \\
Wed, May 17 & Last Day to Withdraw from May Term Courses \\
Fri--Sat, May 19--20 & Meeting of the Board of Trustees \\
Mon, May 29 & No Classes (Holiday) Memorial Day \\
Fri, June 2 & Last Day of May Term Classes \\
Sat, June 3 & Residence Halls Close \\
Sun, June 4 & Final Grades and Attendance for May Term Due on My.Coe
(11:59 PM) \\
\end{longtable}

\subsection{SUMMER TERM 2023*}\label{summer-term-2023}

\begin{longtable}[]{@{}
  >{\raggedright\arraybackslash}p{(\columnwidth - 2\tabcolsep) * \real{0.3333}}
  >{\raggedleft\arraybackslash}p{(\columnwidth - 2\tabcolsep) * \real{0.6667}}@{}}
\toprule\noalign{}
\endhead
\bottomrule\noalign{}
\endlastfoot
Mon, June 5 & Classes Begin \\
Fri, June 9 & Last Day to Add or Drop a Full-Term Course Without a W \\
Mon, June 19 & No Classes (Holiday) Juneteenth \\
Fri, June 23 & Last Day to Change Method of Grading \&/or Withdraw from
Block A Courses \\
Tues, July 4 & No Classes (Holiday) Independence Day \\
Fri, July 7 & Last Day of Block A Courses \\
Mon, July 10 & Classes Begin: Block B Courses \\
Tues, July 11 & Mid-Term Progress Report Due Online (11:59 PM); Block A
Final Grades Due \\
Fri, July 14 & Last Day to Add or Drop a Course Without a W for Block B
Courses \\
Mon, July 17 & Last Day to Change Method of Grading \&/or Withdraw from
Full Term Courses \\
Tues, July 25 & Last Day to Change Method of Grading \&/or Withdraw from
Block B Courses \\
Fri, August 11 & Last Day of Term Classes: Block B and Full-Term
Courses \\
Tues, August 15 & Final Grades and Attendance Due (Block B and
Full-Term) on My.Coe (11:59 PM) \\
\end{longtable}

*Courses taught at the Wilderness Field Station are subject to the broad
dates of the summer, but will provide a specific add/drop and withdraw
calendar to students at the time of application.

\chapter{THE ACADEMIC CALENDAR}\label{the-academic-calendar}

The academic year consists of three terms (see Academic Calendar,
p.~10-11). Students normally take four course credits in the Fall Term
and four course credits in the Spring Term. Thus, eight course credits
are completed in an academic year. During optional May Term, students
may enroll for up to one course credit in one of the limited selection
of courses. Summer term is limited in scope and is not considered a
regular term. (The maximum course load is described in Course Load,
p.~41).

\chapter{MAY TERM (OPTIONAL)}\label{may-term-optional}

Students may enroll for up to one course credit during May Term in one
of the limited selection of courses. All May Term courses require
consent of instructor prior to registration. May Term courses are
designed to meet at least two of the following shared learning outcomes:

\begin{itemize}
\tightlist
\item
  Evaluate and engage with complex interdependent systems and
  demonstrate understanding across diverse contexts.
\item
  Critically describe and break apart issues or problems through
  systematic analysis and illustrate logic for conclusions.
\item
  Engage with experiential learning practices such as learning by doing,
  while utilizing abilities to think critically, problem solve and make
  connections between knowledge gained in the classroom and experience
  beyond.
\end{itemize}

May Term courses are expected to have the same amount of contact time
and academic rigor per course credit as courses which meet over a Fall
or Spring Term.~ For every 1 course credit of May term students are
expected to complete 140 hours of work. Such contact time includes class
meetings, lectures by the instructor, supervised course related
activities and independent out of class activities.~ Off-campus May Term
courses at an off-campus location provide certain educational benefits
through site visits, guest lectures, etc., that also contribute to the
contact time for the course.

\chapter{COE PLAN}\label{sec-coe-plan}

Coe College's requirements for graduation, commonly known as the Coe
Plan were developed with the following outcomes in mind:

\begin{itemize}
\tightlist
\item
  Creation of a bridge from high school to Coe College that helps
  students understand the importance of a liberal arts education, the
  ways to develop the skills needed by any learner, and the
  opportunities they will have by going to Coe College.
\item
  Development of required curriculum that exposes the students to ways
  of learning in various contexts, big ideas in a myriad of disciplines,
  ways of being and understanding of cultures around the world, and
  processes to develop the skills needed by any learner.
\item
  Creation of a bridge from Coe College to life after Coe.
\end{itemize}

These outcomes are met through the College's First-Year Experience,
General Education program, Writing Emphasis courses, and the College's
Practicum experiences and areas of study, described in this section of
the Catalog.

\chapter{GRADUATION REQUIREMENTS}\label{sec-graduation-requirements}

All students who graduate from Coe College must complete at least one
major and earn at least 32 course credits (cc) with grades leading to a
cumulative grade point average (GPA) of 2.0 or higher. (The course is
the unit of academic credit.) Courses are one credit unless otherwise
indicated. Students are expected to complete 180 hours of work to earn
one course credit, although class times vary from course to course.
Other institutions may convert Coe credit to their system by considering
one course credit to be 6 quarter hours, or 4 semester hours.) No more
than a total of two course credits from courses which are less than 0.5
credit can be used to meet the 32-credit graduation requirement. No more
than eight course credits earned of Advanced Placement or International
Baccalaureate credit can be used to satisfy this requirement.

Students must meet one of the following requirements:

\begin{itemize}
\tightlist
\item
  Complete at least the final academic year of required courses
  registered through Coe.
\item
  Earn a total of 16 course credits or the equivalent at Coe. The last
  eight course credits needed for graduation must include at least four
  earned at Coe. Approved off-campus study programs and internships can
  be used to fulfill this requirement.
\end{itemize}

A student may be simultaneously awarded two degrees (B.A., B.M., B.S.N.)
after satisfactorily completing 40 course credits and the requirements
for both degrees. However, a simultaneous Bachelor of Music plus a
Bachelor of Arts with a music major is not permitted.

In addition, students must fulfill the requirements of the First-Year
Experience, General Education, Writing Emphasis, and Practicum.

To participate in Commencement exercises, students must submit a
completed Intent to Graduate form to the Office of the Registrar,
preferably three terms prior to Commencement.

\section{Second Baccalaureate Degree}\label{second-baccalaureate-degree}

A student who holds a baccalaureate degree from another institution may
earn a second baccalaureate degree at Coe, if the following criteria are
met:

\begin{itemize}
\tightlist
\item
  The first degree must be from a regionally accredited institution as
  recognized by the US Department of Education, or another appropriate
  accrediting body.
\item
  The first degree must be completed (not in progress) before beginning
  the second degree at Coe.
\end{itemize}

Students accepted at Coe to pursue a second degree are granted a maximum
of 24 course credits in transfer credit towards the 32 course credits
required for graduation. To graduate, at least eight course credits must
be earned at Coe College and all requirements for the major area of
study must be met with at least 40\% of the major course credits taken
at Coe. Students must earn a cumulative GPA of at least 2.0 as well as
meet any GPA or grade requirements in their area of study.

Second baccalaureate students are exempt from the following
requirements: first-year experience, general education, writing
emphasis, and practicum. They are not eligible to graduate with Latin
Honors or for induction in Phi Beta Kappa or Phi Kappa Phi.

\section{Transfer Student
Information}\label{transfer-student-information}

To honor its mission and to preserve its academic integrity as a liberal
arts institution, the College accepts a course in transfer if that
course meets the spirit of the College's mission and is from a
regionally accredited institution. This section includes information, in
addition to that included in the section, Graduation Requirements (see
p.~13), germane to students who are transferring to Coe College from
another college or university.

Courses transferred to Coe can, as approved by the Registrar, fulfill
some graduation requirements. From institutions on a semester hour
system (at Coe, 1 course credit = 4 semester hours), only courses with
three or more semester hours can be used to fulfill any major or general
education requirements. From institutions on other than a semester hour
system, only courses equivalent to at least 0.75 course credits can be
used to fulfill any major or general education requirement. In some
cases, in consultation with the Registrar, multiple courses within the
same field may be used to fulfill one requirement.

Transfer credits earned after high-school graduation and before Coe
matriculation count towards the eight term, full-time residence
requirement (see p.~211). Full-time enrollment may include participation
in Coe College exchange programs, ACM off-campus study programs, and
other approved off-campus study programs.

All students must complete at Coe at least 40\% of the total course
credits required for each declared major or minor or three course
credits, whichever is greater. In addition to completing at least one
major area of study, transfer students must abide with the following to
complete the requirements for graduation:

\begin{itemize}
\tightlist
\item
  First-Year Experience. Transfer students are not required to fulfill
  the requirements of the First-Year Experience, if they have completed
  at least one full-time college term since graduation from high school.
\item
  Writing Emphasis. (See Writing Emphasis Courses, p.~16).
\item
  General Education. Requirements include Liberal Arts selections in the
  four divisional areas (Natural and Mathematical Sciences, Social
  Sciences, Humanities, Fine Arts) and Diverse Cultural Perspectives
  courses. Any courses accepted in transfer for at least 0.75 course
  credit that fit the criteria of the Liberal Arts and/or Diverse
  Cultural Perspectives core groups can be applied towards the general
  education requirements as determined by the Registrar. Advanced
  Placement and International Baccalaureate courses may not be used to
  meet any part of the General Education requirements.
\item
  Academic Practicum. Transfer students are required to fulfill this
  requirement.
\end{itemize}

\chapter{AREAS OF STUDY}\label{sec-areas-of-study}

The three undergraduate degrees have areas of study associated with
them. The Bachelor of Science in Nursing's area of study is nursing; the
Bachelor of Music's areas of study are performance, composition, and
education. The Bachelor of Arts' areas of study, commonly referred to as
majors, are listed below.

Students should declare an area of study by the end of their sophomore
year. All students must earn at least a 2.00 GPA in courses required to
complete their areas of study, as well as meet specific requirements set
forth for the areaof study.

\begin{itemize}
\tightlist
\item
  Accounting, Managerial
\item
  Accounting, Public
\item
  African American Studies
\item
  Art
\item
  Art History
\item
  Asian Studies
\item
  Biology
\item
  Business Administration
\item
  Chemistry
\item
  Communication Studies
\item
  Computer Science
\item
  Creative Writing
\item
  Data Science
\item
  Economics
\item
  Elementary Education
\item
  English
\item
  Film Studies
\item
  French \& Francophone Studies
\item
  General Science
\item
  History
\item
  Interdisciplinary
\end{itemize}

\begin{itemize}
\tightlist
\item
  French \& Francophone Studies
\item
  Interdisciplinary Studies*
\item
  International Business
\item
  International Economics
\item
  International Studies
\item
  Kinesiology
\item
  Literature
\item
  Mathematics
\item
  Music
\item
  Philosophy
\item
  Physics
\item
  Political Science
\item
  Psychology
\item
  Religion
\item
  Social \& Criminal Justice
\item
  Sociology
\item
  Spanish
\item
  Spanish Studies
\item
  Theatre Arts
\item
  Writing (Rhetoric)
\end{itemize}

* A coherent interdisciplinary sequence of courses devised by the
student, in consultation with faculty, suited to the student's
individual goals and approved by the Academic Policies Committee (see
p.~131). In addition to the areas of study/majors listed above, the
following \textbf{COLLATERAL MAJORS} are offered, which require a
student to satisfy the requirements of a major from the list above in
addition to the selected collateral major.

\begin{itemize}
\tightlist
\item
  Biochemistry
\item
  Molecular Biology
\item
  Public Relations
\item
  Environmental Science
\item
  Neuroscience
\item
  Environmental Studies
\item
  Organizational Science
\end{itemize}

\subsection{AREAS OF STUDY (MINOR) FOR
B.A.}\label{areas-of-study-minor-for-b.a.}

\begin{itemize}
\tightlist
\item
  African American Studies
\item
  Anthropology
\item
  Art
\item
  Art History
\item
  Asian Studies
\item
  Chemistry
\item
  Classical Studies
\item
  Communication Studies
\item
  Computer Science
\item
  Creative Writing
\item
  Data Science
\item
  Economics
\item
  English
\item
  Film Studies
\item
  French \& Francophone Studies
\item
  Gender and Sexuality Studies
\item
  Health \& Society Studies
\item
  History
\item
  Interdisciplinary French \&
\item
  Francophone Studies
\item
  International Economics
\item
  Mathematics
\end{itemize}

\subsection{AREAS OF STUDY (MAJORS) FOR
B.M.}\label{areas-of-study-majors-for-b.m.}

\begin{itemize}
\tightlist
\item
  Keyboard or Instrumental Performance
\item
  Vocal Performance
\item
  Composition
\item
  Instrumental Music Education
\item
  Vocal Music Education
\end{itemize}

\subsection{AREA OF STUDY (MAJOR) FOR
B.S.N.}\label{area-of-study-major-for-b.s.n.}

\begin{itemize}
\tightlist
\item
  Nursing
\end{itemize}

\chapter{ACADEMIC ADVISING}\label{sec-academic-advising}

The role of the academic advisor is to acquaint students with their
academic options at Coe and assist them in selecting courses that
reflect individual interests and abilities. Advisors also help students
create a four-year education plan that allows students to make
connections between disparate areas of study, and between academic,
co-curricular, and non-academic areas. Students may change advisors at
any time upon request to the Registrar.

First-Year Seminar instructors serve as the primary academic advisors
for first-year students in their respective sections. Students thus see
their advisors frequently during their first term at Coe and have the
opportunity to work closely with them in developing overall programs of
study and long-range goals. After the first term, students may decide to
choose departmental faculty for academic advising or they may continue
to be advised by their First-Year Seminar instructors. Students are free
to speak at any time with professors in their major departments to
answer specific questions regarding requirements and courses in those
departments.

\chapter{FIRST-YEAR EXPERIENCE}\label{sec-first-year-experience}

The student's First-Year Experience at Coe is a deliberate strategy to
engage first-year students across multiple dimensions of college life in
the first year. Components of the First-Year Experience include writing
exercises, various campus events, and the First-Year Seminar (FYS).

The First-Year Seminar is required for all students who have not yet
completed a full-time college term after high school graduation. During
the Fall Term, a variety of First-Year Seminars---topics courses
exploring issues from multiple perspectives---are offered. The seminars
emphasize critical thinking, writing, speaking and research skills.

All First-Year Seminars carry the writing emphasis designation.
First-Year Seminar courses cannot fulfill any distributional, cultural
perspective, or major requirements. Students who drop or fail their
First-Year Seminar are required to complete a replacement course
designated by the Registrar the following term. The completion of this
Spring Term course makes it possible for the student to fulfill the FYS
graduation requirement. For students who fail the Fall Term First-Year
Seminar, successful completion of the Spring Term course also allows the
grade in the FYS to be changed from ``F'' to ``NP'' on the transcript.

Students who begin their college enrollment in the Spring Term must also
complete a course designated as a replacement.

\chapter{WRITING EMPHASIS COURSES}\label{sec-writing-emphasis-courses}

\textbf{Writing Across the Curriculum: Statement of Guiding Principles}

As stated in the Coe College Mission Statement, our reason to exist as
an institution is to ready students intellectually, professionally, and
socially to lead productive and satisfying lives in the global society
of the 21st century. In accordance with this mission, our curriculum
requires that students undertake ``a series of intensive writing
experiences, spread across four years of study.'' This requirement is
known as Writing Across the Curriculum, the guiding principles of which
are detailed below.

As a philosophy, Writing Across the Curriculum asserts that writing is
most effectively learned in context: to varied audiences, with varied
purposes. In adopting this philosophy, Coe College has committed to
making writing-intensive experiences available to students in all
disciplines. Known as ``Writing Emphasis'' credits, these courses are
divergent in subject matter but aligned in their commitment to giving
students content-rich and context-specific writing experiences that
foster a critical flexibility in transferring knowledge about effective
writing to multiple, even unknown, contexts.

Therefore, students who complete the requisite number of writing
emphasis courses will graduate from Coe knowing that writing is both a
means and an end: a method for exploring ideas and deepening one's
knowledge as well as a tool for sharing that knowledge and expressing
one's point of view. Similarly, graduating students will know that the
practice of writing is a recursive process rooted in revision, which
refers both to the reconsideration of one's ideas and to the refinement
of prose, and this process unfolds over a lifetime. Understanding that
students will encounter innumerable writing situations in their lives,
many of which may be unfamiliar to them, they will graduate from Coe
having developed the confidence and self-efficacy necessary to adapt or
draw from their existing knowledge in order to navigate new contexts.

Although many courses at Coe may include writing activities, courses
endorsed by Writing Across the Curriculum are those in which:

\begin{itemize}
\item
  Students are given opportunities to practice revision, whether via
  multiple drafts of a single project or multiple iterations of the same
  type of assignment;
\item
  Students receive instructor feedback on written work to facilitate
  revision; and
\item
  Writing assignments are frequent enough that they are integral to the
  learning throughout the course, enabling students to develop in one or
  more of the following learning outcomes, as appropriate to discipline
  and course objectives:

  \begin{itemize}
  \item
    \textbf{How to write for specific purposes and audiences}

    \begin{itemize}
    \tightlist
    \item
      \emph{Including attention to the ways purposes and audiences shape
      form, mode, voice, method, organization, engagement with and
      citation of research, and/or creativity and imagination}
    \end{itemize}
  \item
    \textbf{How to engage in critical thinking}

    \begin{itemize}
    \tightlist
    \item
      \emph{Particularly how to conduct analysis, how to synthesize
      information, how to interpret and/or use evidence and data, and
      how to present one's ideas coherently and stylistically}
    \end{itemize}
  \item
    \textbf{How to practice, assess, and develop effective habits for
    writing}

    \begin{itemize}
    \tightlist
    \item
      \emph{Specifically, how to read deeply, how to begin and later
      practice selection in research, how to be a skilled reader of
      one's own and others' in-process writing, how to accept and
      interpret feedback, and how to revise writing, all of which are
      parts of the process of learning how to have and develop ideas}
    \end{itemize}
  \item
    \textbf{How to engage in writing as a method for learning and
    discovery}

    \begin{itemize}
    \tightlist
    \item
      \emph{For deepening knowledge, thinking through questions and
      problems, and reflecting on connections and growth}
    \end{itemize}
  \item
    \textbf{How to name and describe one's own knowledge for others}

    \begin{itemize}
    \tightlist
    \item
      \emph{For example, in the form of personal statements for
      application to graduate study, cover letters and resumes for
      entering and advancing in the workforce, or other forms of
      self-summary that facilitate transition from the baccalaureate
      environment through the broader public.}
    \end{itemize}
  \end{itemize}
\end{itemize}

Many writing emphasis sections are offered each year, and, in addition,
the College's rhetoric department offers several interdisciplinary
writing courses designed to guide students learning to write effectively
at the college level. (See course descriptions starting on p.~49).
General Education or major courses that are also designated as writing
emphasis courses may be used to satisfy both requirements.

In this catalog, writing emphasis courses are designated by a (WE) after
the course title, e.g.~``RHE-200 Rhetorical Theory \& Practice (WE).''
In other contexts, the course code may end with a ``W''
(e.g.~RHE-200-W). All sections of such a course, regardless of
instructor, will carry writing emphasis credit.

Fulfilling the Writing Emphasis Requirement:

Only designated WE courses in which a student earns a grade of ``C''
(2.0) or better count toward fulfillment of this requirement.

Only designated courses taken at Coe College count toward fulfillment of
this requirement.

Undergraduate degree-seeking* students

\begin{itemize}
\item
  who start at Coe College or who transfer fewer than 8 course credits
  earned after graduation from high school must complete at least 5
  writing emphasis courses. Of these 5, one will usually be the FYS. Of
  the remaining 4, it is recommended (but not required) that at least 1
  be taken outside the student's intended major(s).
\item
  who transfer at least 8 but fewer than 16 course credits to Coe,
  earned after graduation from high school, must complete at least 3
  writing emphasis courses.
\item
  who transfer 16 or more course credits to Coe, earned after graduation
  from high school, must complete at least 2 writing emphasis credits.
\end{itemize}

For all students, regardless of transfer credit, it is recommended (but
not required) that at least one writing emphasis course be taken in the
upper division within the student's intended major(s). ``Upper
division'' courses are defined in this case as those numbered 300 and
above; upper division writing emphasis courses could include both
scheduled classroom courses and arranged writing-based capstone projects
(such as a thesis). See Departmental Writing Plans for more information
on writing in the major(s).

*Students seeking a second undergraduate degree from Coe should see
Second Baccalaureate Degree p.~13.

\subsubsection{COURSE NUMBERING}\label{course-numbering}

\begin{itemize}
\tightlist
\item
  Courses numbered 100--199 are introductory to the subject.
\item
  Courses numbered 200--299 assume a capacity for the independent
  acquisition of material and generally build on the methods and subject
  matter of 100-level courses.
\item
  Courses numbered 300--399 are typically oriented toward a major or
  minor. They require a strong foundation of knowledge specific to the
  discipline.
\item
  Courses numbered 400--499 are courses in the major or minor that are
  designed to challenge students to integrate discipline-specific
  knowledge in advanced ways. These courses typically contain advanced
  disciplinary coursework, capstone projects, and/or independent
  research.
\end{itemize}

\chapter{GENERAL EDUCATION COURSES}\label{sec-general-education-courses}

Completion of the General Education Program, described here, is required
for all students earning B.A. or B.S.N. degree at Coe College, but not
required of students earning a B.M. degree. An appropriate transferred
course, determined by the Office of the Registrar, with a grade of ``C''
(2.0) or better may be accepted to meet an individual requirement.\\
Advanced Placement and International Baccalaureate courses may not be
used to meet any part of this requirement. No more than two courses with
the same prefix may be used to fulfill the General Education Core
Requirements.

\textbf{A) Liberal Arts Core}

\begin{enumerate}
\def\labelenumi{\arabic{enumi}.}
\tightlist
\item
  A total of at least one course credit in the \textbf{Fine Arts Core}
  (courses with a prefix of ARH, ART, FLM, MU, MUA, THE)
\item
  A total of at least two course credits in the \textbf{Humanities Core}
  (courses with a prefix of AAM, AMS, CLA, COM, CRW, ENG, FRE, GER, GRK,
  HIS, HUM, LTN, JPN, PHL, REL, RHE, or SPA)
\item
  A total of at least two course credits in the \textbf{Natural Sciences
  and Mathematics Core} (courses with a prefix of BIO, CHM, CS, DS, MTH,
  PHY, STA, one of which must be a lab science with a prefix of BIO,
  CHM, or PHY)
\item
  A total of at least two course credits in the \textbf{Social Sciences
  Core} (courses with a prefix of ANT, ECO, POL, PSY, or SOC)
\end{enumerate}

\textbf{B) Diverse Cultural Perspectives (DCP) Core}

A liberally educated person should have some knowledge of other cultures
and some tools to aid in seeing one's own culture from other
perspectives.

The Diverse Cultural Perspectives courses (Non-Western Perspectives
(course number ends in 6), United States Pluralism (course number ends
in 7), Diverse Western Perspectives (course number ends in 8) help
students to understand their own cultural identities and to develop
appreciation for the range of different cultures to be found in the
world, in the nation, and on campus. These courses explore other
cultures in their own terms and as they interact with American culture.
As a group, they encourage reflection on different ways in which
cultural identities are formed, expressed and contested. Students are
encouraged to combine these courses with the study of a world language
and with study abroad.

Completion of one of the following options (see p.~19-21 for
descriptions and approved courses):

\begin{itemize}
\tightlist
\item
  Non-World Language Option (all of the following):

  \begin{itemize}
  \tightlist
  \item
    Any DCP course (course number that ends in 6, 7, 8)
  \item
    Non-Western Perspectives (course number that ends in 6) (A
    semester-long study abroad experience can fulfill the Non-Western
    Perspective and DCP requirement.)
  \item
    United States Pluralism (course number that ends in 7) (A
    semester-long U.S. off-campus study experience can fulfill the U.S.
    Pluralism and DCP requirement.)
  \end{itemize}
\item
  Elementary World Language Option (all of the following):

  \begin{itemize}
  \tightlist
  \item
    Any DCP course (course number that ends in 6, 7, 8)
  \item
    Two elementary world language courses in the same language not
    previously studied (may also fulfill one of the Humanities Core
    courses)
  \end{itemize}
\item
  Intermediate World Language Option (all of the following):

  \begin{itemize}
  \tightlist
  \item
    Any DCP course (course number that ends in 6, 7, or 8)
  \item
    One intermediate (-215) or above World Language course (\emph{may
    also fulfill one of the Humanities Core course requirements})
  \end{itemize}
\end{itemize}

\section{Diverse Cultural Perspectives: Non-Western
Perspectives}\label{sec-diverse-cultural-perspectives-non-western}

The Non-Western Perspectives (NWP) group includes courses in which a
preponderance of the content analyzes human experience from the
perspectives of diverse peoples outside of Western culture. Such
cultures often are characterized by values and beliefs different from
those of the United States and Western Europe.

\begin{itemize}
\tightlist
\item
  ANT 116 Cultural Anthropology
\item
  ANT 286 Topics in Anthropology:NWP
\item
  ARH 106 World Art
\item
  ARH 296 Topics in Art History:Global Persp
\item
  ASC 106 May Term in Asia
\item
  ASC 176 China and Japan
\item
  ASC 186 Modern South Asia
\item
  ASC 196 Modern South East Asia
\item
  +++MISSING INFO: c.asc216.long +++
\item
  BUS 446 International Business Management
\item
  BUS 466 Adv Top Mrktg:non-west persp
\item
  BUS 476 Ad Top Mgmt:non-west persp
\item
  COM 236 Intercultural Communication
\item
  ECO 336 Divergent Economic Growth
\item
  ECO 436 Econ Development
\item
  ECO 446 International Econ
\item
  ENG 146 Intro Postcolonial Literature
\item
  ENG 206 Gender and Literature: NWP
\item
  FRE 146 French Literature Translation:NWP
\item
  FRE 446 Colonial \& Multicultural Narratives
\item
  FSA 146 Turkey: History \&Culture
\item
  GS 136 Gender in Non-Western World
\item
  HIS 136 East Asian Civilization
\item
  HIS 216 History of Modern Korea (WE)
\item
  HIS 246 History of Modern China
\item
  HIS 256 History of Modern Japan
\item
  HIS 276 The ``Discovery'' of America: Clash
\item
  HIS 286 Modern Middle East
\item
  HIS 306 Revolution, Social Struggle, Testim
\item
  HIS 316 Topics in History:Non-Western Persp
\item
  HIS 466 Seminar Modern East Asian History
\item
  IS 116 Intro to International Studies
\item
  IS 126 HumanRightsBurmeseMigrant
\item
  IS 136 May Term in Mongolia
\item
  JPN 106 Images Foreign Culture
\item
  MU 166 Topics in Music:Non-Western Perspct
\item
  PHL 206 Buddhist Thought
\item
  POL 266 Latin American Politics
\item
  POL 276 African Politics
\item
  POL 286 Asian Politics
\item
  POL 296 Topics Pol Sci: Non-West Persp
\item
  POL 386 International Development
\item
  REL 106 Eastern Religions
\item
  REL 116 Buddhism
\item
  REL 136 Religions of China
\item
  REL 196 Hinduism
\item
  REL 206 Buddhist Thought
\item
  REL 226 Religions of China:Daoism
\item
  REL 236 Zen Buddhism
\item
  REL 296 Topics in Religion NWP
\item
  REL 306 Comparative Religion
\item
  REL 336 Tibetan Buddhist Cultrue
\item
  +++MISSING INFO: c.rel386.long +++
\item
  RHE 146 Creative Nonfiction:Global Perspect
\item
  SOC 226 Gender and Globalizaton
\item
  SOC 236 Topics in Sociology NWP
\item
  SPA 336 Hispanic Life/Cult-Latin America
\item
  SPA 446 Latin Am. \& Spanish Short Stories
\item
  SPA 486 Topics in Hispanic Lit:Latin Amer
\item
  THE 486 Spc Top Theatre or Film: NWP
\item
  WSH 286 Topics in Washington, D.C:NWP
\end{itemize}

\section{Diverse Cultural Perspectives: United States
Pluralism}\label{sec-diverse-cultural-perspectives-united-states}

The United States Pluralism (USP) group includes courses in which a
preponderance of the content addresses one or more of the groups within
the United States whose values, beliefs, and experiences differ from or
oppose those of the majority culture. These courses increase students'
knowledge of the history of such groups; of the ways members of these
groups have experienced democracy and culture in America differently
because of factors like social class, race, gender, and religion; and of
reform movements like feminism and civil rights, through which such
groups have attempted to achieve social and economic equality.

\begin{itemize}
\tightlist
\item
  AAM 107 Intro to African American Studies
\item
  AAM 137 African American Literature\\
\item
  AAM 217 Sport and Black Culture\\
\item
  AAM 227 Blackness \& Identity in America
\item
  AAM 287 Topics in African American Studies
\item
  AAM 367 Topics in AfricanAmericanLiterature
\item
  AAM 387 Adv Topics in African American Stud
\item
  AAM 447 Drtd Learn in African American Stds
\item
  AAM 457 Drtd Learn in African American Stds
\item
  AAM 467 Seminar in African American Lit
\item
  ARH 107 Gender and Art\\
\item
  ARH 297 Topics in Art History: US Pluralism\\
\item
  ARH 307 Modern and Contemporary Art
\item
  BUS 387 Adv Top:Human Res Mgt
\item
  BUS 437 Strategic Compensation\\
\item
  BUS 457 Employment and Discrimination Law
\item
  BUS 467 Consumer Behavior
\item
  COM 157 Introduction to Media Analysis
\item
  COM 237 Interpersonal Communication
\item
  COM 337 Persuasion\\
\item
  COM 357 Sex, Race, \& Gender in Media
\item
  COM 437 Special Topic Applied Communication
\item
  COM 447 Special Topics Production
\item
  COM 457 Special Topics in Media Studies
\item
  COM 467 Special Topics in Public Discourse
\item
  ECO 237 Labor Economics\\
\item
  ECO 247 Health Economics\\
\item
  ECO 447 Urban Economics\\
\item
  ECO 457 US Econ History\\
\item
  EDU 117 Exceptional Learners\\
\item
  EDU 187 Human Relations\\
\item
  EDU 237 English Language Learners
\item
  EDU 247 Foundations of Reading\\
\item
  ENG 107 Exploring Literature:US Pluralism
\item
  ENG 117 Asian American Literature
\item
  ENG 127 Social Justice and Literature
\item
  ENG 137 African American Literature
\item
  ENG 157 Latinx/Chicanx Literature
\item
  ENG 207 Gender \& Lit:US Pluralism
\item
  ENG 327 Literature of American Renaissance
\item
  ENG 337 American Realism \& Naturalism
\item
  ENG 347 Study in Modern or Contemp Amer Lit
\item
  ENG 357 Studies in Latinx/ChicanxLiterature
\item
  ENG 367 Studies in African Am Literature
\item
  ENG 467 Seminar inLit:USPluralism
\item
  +++MISSING INFO: c.evs137.long +++
\item
  GS 107 Intro Gender \& Sexuality Studies
\item
  GS 127 Dress, Gender, and Identity
\item
  GS 247 Gender \& Sexuality StudiesSymposium
\item
  GS 327 Thry\&Mthds/Gender \&Sexuality Stdy
\item
  GS 387 Topics: Gender \& Sexuality Studies
\item
  HIS 217 American War in Vietnam
\item
  HIS 227 American Civil War\\
\item
  HIS 257 Native American History
\item
  HIS 297 Women in America\\
\item
  HIS 317 Topics in History:US Pluralism\\
\item
  HIS 347 African American History
\item
  HIS 387 American Colonial History
\item
  KIN 347 Adapted Physical Education
\item
  MU 157 Introduction to Jazz History
\item
  NUR 137 Human Sexuality\\
\item
  NUR 297 Parent Child Relationships\\
\item
  NUR 387 Alternative Therapies for Hlth/Heal
\item
  PHL 277 Philosophy of Gender \& Race
\item
  POL 207 Religion \& American Politics
\item
  POL 277 Women \& Poltics in US
\item
  PSY 137 Human Sexuality\\
\item
  REL 217 Religion in America\\
\item
  RHE 137 Creative Nonfiction U.S.-Pluralism
\item
  RHE 257 Environmental Rhetoric\\
\item
  RHE 377 Cultural Studies\\
\item
  SOC 107 Introductory Sociology\\
\item
  SOC 207 Sociology of the Family\\
\item
  SOC 217 Sociology of Religion
\item
  SOC 237 Topics in Sociology:U S Pluralism
\item
  SOC 247 Sociology of Race
\item
  SOC 417 Sociology of Sex \& Sexuality
\item
  SPA 457 US LatinX Literature
\end{itemize}

\section{Diverse Cultural Perspectives: Diverse Western
Perspectives}\label{sec-diverse-cultural-perspectives-western}

The Diverse Western Perspectives (DWP) group includes courses in which a
preponderance of the content addresses one or more subgroups of the
Western world outside of the United States and the ways in which they
experience Western culture. These courses increase students' knowledge
of the history of particular groups and the ways they have interacted
with Western values. They typically address issues of difference and
conflict between and within Western cultures by examining the influence
of factors such as class, race, gender, and religion.

\begin{itemize}
\tightlist
\item
  ANT 288 Topics Anthropology/Archaeology:DWP
\item
  ANT 488 Adv Top Anthro/Archaeo:DWP
\item
  ARH 128 Introduction to Art History
\item
  ARH 218 The World of Renaissance Art
\item
  ARH 248 Baroque, Rococo, and Neoclassicism
\item
  ARH 268 History of Architecture
\item
  ARH 298 Topics in Art History:Div West Pers
\item
  CLA 108 Images of Foreign Culture
\item
  ENG 108 Exp Lit:Diverse Western Perspective
\item
  ENG 208 Gender \& Lit:DWP
\item
  ENG 378 Studies in Transatlantic Literature
\item
  ENG 388 Romantic Literature
\item
  FRE 148 French Literature Translation:DWP
\item
  FRE 158 France \& Francophone World
\item
  HIS 208 The First World War (WE)
\item
  HIS 218 The Second World War (WE)
\item
  HIS 238 Modern France
\item
  HIS 248 The French Revolution
\item
  HIS 268 Latin America
\item
  HIS 288 Renaissance \& Reformation
\item
  HIS 308 Legacies of the Cold War inLatin Am
\item
  HIS 318 Topics in History :Div West Persp
\item
  HIS 328 Modern France
\item
  MU 458 Music History \& Literature III
\item
  NUR 268 Cult Diver \& Health
\item
  PHL 128 Morality \& Moral Controversies
\item
  PHL 138 Freedom, State, and Society
\item
  POL 108 Introduction to Politics
\item
  POL 248 Political Violence and the Violent
\item
  POL 258 World Politics
\item
  POL 298 European Politics
\item
  POL 398 Religion \& World Politics
\item
  PSY 208 Gender Psychology
\item
  REL 108 Western Religions
\item
  REL 128 Judaism
\item
  REL 138 Modern Judaism
\item
  REL 148 Islam
\item
  REL 178 Christianity
\item
  REL 278 Mysticism
\item
  REL 338 Modern Religious Thought
\item
  SOC 238 Topics in Soc Div West Perspectives
\item
  SOC 328 Urban Sociology
\item
  SOC 338 Political Sociology
\item
  SPA 148 Spanish Literature in Translation
\item
  SPA 258 Spanish Lang Learn in Spain
\item
  SPA 338 Hispanic Life/Culture:Europe
\item
  SPA 418 Gender \& Sexuality in Hispanic Wrld
\item
  SPA 428 Indigeneity, Blackness, \& EthnicLit
\item
  SPA 458 Travel Writing \& Transatlantic Lit
\item
  THE 118 Theatre \& Arts in Serbia
\item
  THE 228 History of Theatre and Drama I
\item
  THE 238 History of Theatre and Drama II
\item
  THE 288 History of Dress
\item
  THE 488 Special Topics in THE/ FLM
\end{itemize}

\section{Independent Studies}\label{sec-independent-studies}

The one-credit independent study or directed readings students are
expected to complete a minimum of 140 hours of academic work, including
meeting with faculty members and independent work between meetings.

\section{Internships}\label{sec-internships}

The internship is a work or volunteer experience in the context of an
independent academic investigation of site-related issues and personal
aptitudes, values, and goals. The one-credit internship includes a
minimum of 140 hours of on-site or remote experience and the required
documentation and/or academic journal or paper as determined by the
faculty member. The details of the academic component are determined by
prior arrangement with the faculty internship advisor.

Ordinarily internships are completed during the academic year as one of
the 32 credits for graduation, or over the summer for a credit-bearing
or non-credit bearing-practicum fulfillment activity. In unusual
circumstances where an internship presents an opportunity to extend the
educational component of the experience significantly, an internship may
earn two credits. Application for non-departmental, two-credit
internships requires consultation with the Internship Faculty Advisor
and approval of the Committee on Petitions; departmental two-credit
internships, when permitted, are overseen by the department.

A combined maximum of 2.0 course credits may be counted toward the 32
credits required for graduation through Internships or Community-Based
Project (see p.~49). Many internships are completed in the Cedar Rapids
area; however, it is permissible to complete an internship outside of
the area during the summer, or done remotely if authorized by the
Internship Faculty Advisor.

Students interested in internships should consult with their Career
Specialist in C3: Creativity, Careers, Community, as well as with the
appropriate academic department who will be overseeing the internship
experience.

Each internship must include the consent of an Internship Faculty
Advisor and completion of the internship request form housed on the
College's online platform for internships.

Courses with an INT prefix and those on this list count as internship
credit:

\begin{itemize}
\tightlist
\item
  AAM 494 Internship in African American Stds
\item
  ANT 494 Internship in Anthropolgy
\item
  ARH 494 Internship in Art History
\item
  ARH 494 Internship in Art History
\item
  ART 494 Internship in Art
\item
  AT 494 Internship in Athletic Training
\item
  BIO 494 Internship in Biology
\item
  BUS 494 Internship in Business
\item
  CHM 494 Internship in Chemistry
\item
  COM 494 Internship in Journalism/Communicat
\item
  CRW 494 Internship in Creative Writing
\item
  CS 494 Internship in Computer Science
\item
  DS 494 Internship in Data Science
\item
  EDU 494 Internship in Education
\item
  ENG 494 Internship in English
\item
  FLM 494 Internship in Film
\item
  FRE 494 Internship in French
\item
  HIS 494 Internship in History
\item
  HSS 494 Health Professions Externship
\item
  INT 115 May Term: Topics 2
\item
  INT 494 Internship
\item
  INT 499 Summer Internship
\item
  KIN 494 Internship in Kin, Health \& Rec
\item
  MTH 494 Internship in Mathematics
\item
  NUR 494 Internship in Nursing
\item
  NYT 394 Internship in New York City
\item
  PHL 494 Internship in Philosophy
\item
  PHY 494 Internship in Physics
\item
  POL 494 Internship in Political Science
\item
  PR 494 Internship in Public Relations
\item
  PSY 494 Internship in Psychology
\item
  REL 494 Internship in Religion
\item
  RHE 494 Internship in Writing
\item
  SOC 494 Internship in Sociology
\item
  SMT 494 Internship in Sports Management
\item
  SPA 494 Internship in Spanish
\item
  THE 494 Internship in Theatre Arts
\item
  WSH 494 Washington Experience
\end{itemize}

Students completing internships that are not department specific should
register for one of the INT-494 or INT-499 courses on p.~49 (see the
\textbf{\emph{Coe Student Accounts Handbook}} for fee).

\section{Practicum}\label{sec-practicum}

A practicum experience is required of all students for all undergraduate
degrees, except those earning second degrees.

Typically completed in the student's junior or senior year, all practica
are experiences that integrate academic components with career or other
life goals and are significant educational exercises outside the
classroom. A practicum experience can consist of an internship,
off-campus study, community-based project, honors project, or some other
kind of independent activity.

Depending upon the type selected, some practica are graded A--F, while
others are P/NP. Some practica are credit bearing, while others are not.
In some instances, the practicum must be approved by the student's major
department.

\begin{enumerate}
\def\labelenumi{\arabic{enumi}.}
\tightlist
\item
  Full-Term (16-week) Off-Campus Study
\item
  Wilderness Field Station Summer Courses
\item
  Crimson Fellows Thesis or Crimson Fellows Project, etc. as stated
\item
  Independent Project (in list of courses that follows starred courses *
  require department approval for practicum credit):

  \begin{enumerate}
  \def\labelenumii{\arabic{enumii}.}
  \setcounter{enumii}{4}
  \tightlist
  \item
    †Internship (see a complete listing of internships on p.~22)
  \item
    †Community-Based Project (see course description on p.~49)
  \end{enumerate}
\end{enumerate}

†A maximum of two course credits earned through any combination of
Internships and Community-Based Projects may be included in the 32
course credits required for graduation.

\begin{itemize}
\tightlist
\item
  AAM 444 Ind Study-Afr-Am St
\item
  ANT 205 Archaeological Field Schl
\item
  ANT 444 Independent Study: Anthropology
\item
  ANT 474 Research Participation:Anthropology
\item
  ARH 444 Independent Study: Art History
\item
  ARH 474 Senior Seminar II
\item
  ART 394 Directed Learning in Art
\item
  ART 444 Independent Study in Art
\item
  ART 474 Senior Seminar II \& Senior Exhibit
\item
  +++MISSING INFO: c.at40.long +++ (\emph{successful completion of
  sequence of AT-20/-30 and -40 required to receive full credit})
\item
  BIO 115 Marine Biology
\item
  BIO 444 Independent Study
\item
  BIO 454 Research Participation
\item
  BIO 462 Advanced Research Lab I
\item
  BUS 444 Ind Study-Bus Admin
\item
  BUS 454 Research in Business
\item
  CHM 444 Independent Study Chemistry
\item
  CHM 454 Undergraduate Summer Research
\item
  COM 394 Directed Learning in Communication
\item
  COM 444 Independent Study in Comm Studies
\item
  CRW 112 Advanced Literary Magazine Editing
\item
  CRW 394 Directed Studies in Creative Writin
\item
  CRW 492 Manuscript Workshop
\item
  CS 444 Ind Study-Comp Sci
\item
  CS 454 Research in Computer Science
\item
  DS 444 Independent Study in Data Science
\item
  DS 454 Research in Data Science
\item
  ECO 444 Ind Study-Economics
\item
  ECO 454 Research in Economics
\item
  EDU 215 Practicum in Education
\item
  EDU 481 Stu Tchg Sec: ART
\item
  EDU 482 Stu Tchg Sec: Phys Education
\item
  EDU 483 Std Teaching Elem: ART
\item
  EDU 485 Std Teaching Elem: Phys Ed
\item
  EDU 489 Student Teaching Sr HS
\item
  EDU 490 Student Teaching Jr HS
\item
  EDU 491 Student Teaching 4-6
\item
  EDU 492 Student Teaching K-3
\item
  EDU 444 Ind Study-Tchr Ed
\item
  ENG 394 Directed Learning in English
\item
  ENG 454 Honors Research
\item
  FLM 442 Independent Study in Film
\item
  FLM 464 Seminar in Film II: (0.5 cc) and FLM 474 Senior Seminar II in
  Film \& Senior (0.5 cc)
\item
  FRE 394 Directed Learning in French
\item
  FRE 444 Ind Study-French
\item
  HIS 444 Ind Study-History
\item
  KIN 444 Ind Study-KIN
\item
  MTH 444 Ind Study-Math
\item
  MTH 454 Research in Mathematics
\item
  MU 421 Student Teaching Elementary Music
\item
  MU 422 Student Teaching Secondary Music
\item
  MU 444 Ind Study-Music
\item
  MUA 490 Senior Recital
\item
  NUR 444 Ind Study-Nursing
\item
  NUR 455 Leadership \& Cont Issues in Nursing
\item
  PHL 394 Directed Learning in Philosophy
\item
  PHL 444 Ind Study-Philos
\item
  PHL 490 Philosophy Colloquium (0.0 cc)
\item
  PHY 255 Advanced Laboratory I
\item
  PHY 355 Advanced Laboratory II
\item
  PHY 444 Ind Study-Physics
\item
  POL 444 Ind Study-Pol Sci
\item
  PSY 354 Research Participation
\item
  PSY 444 Ind Study-Psychology
\item
  PSY 455 Advanced Experimental Psychology
\item
  REL 394 Directed Learning in Religion
\item
  REL 444 Ind Study-Relig
\item
  RHE 394 Drtd Learning in Writing \& Rhetoric
\item
  RHE 444 Independent Study in Writing (WE)
\item
  RHE 490 Publications Practicum
\item
  SCJ 444 Independent Study in Social and Cri
\item
  SOC 365 Research Participation I
\item
  SOC 444 Ind Study-Soc
\item
  SPA 394 Directed Learning: Spanish
\item
  SPA 444 Ind Study-Spanish
\item
  THE 442 Adv Proj-Design/Tech Production
\item
  THE 444 Ind Study-Theatre
\item
  THE 452 Advanced Projects in Acting
\item
  THE 462 Advanced Projects in Directing
\end{itemize}

\chapter{SPECIAL PROGRAMS AND
OPPORTUNITIES}\label{special-programs-and-opportunities}

\section{Clinical Laboratory Sciences/Medical
Technology}\label{clinical-laboratory-sciencesmedical-technology}

In cooperation with the St.~Luke's Methodist Medical Laboratories in
Cedar Rapids, or upon arrangement with other accredited laboratories and
the approval of the College, Coe offers a four-year course leading to a
Bachelor of Arts degree and registration as a Clinical Laboratory
Scientist/Medical Technologist.

The first three years are spent in residence at Coe, where candidates
must complete all of the requirements for the B.A. degree, including
general education requirements and an approved major. The minimum
requirements of the Clinical Laboratory Sciences/Medical Technology
program in biology and chemistry are five course credits in each field
and at least one course in mathematics. The fourth year is a full
calendar year spent at St.~Luke's or another accredited medical
laboratory approved by Coe.

The St.~Luke's Hospital Medical Laboratory is approved as a school of
clinical laboratory sciences/medical technology by the Committee on
Allied Health Education and Accreditation of the American Medical
Association. Candidates completing the course are eligible to take the
certification examinations of the American Society of Clinical
Pathologists and the National Certification Agency and, if approved, may
practice anywhere in the United States.

\section{Crimson Fellows Program}\label{crimson-fellows-program}

Steffens, Westberg (Program Directors).

To graduate as a Crimson Fellow, a student must earn at least a 3.3
cumulative grade point average for all courses taken at Coe College and
complete all of the following:

\begin{enumerate}
\def\labelenumi{\arabic{enumi}.}
\tightlist
\item
  CFP 104 Topics in Crimson Fellows Program (0.2 cc)
\item
  +++MISSING INFO: c.CFP205.long +++
\item
  +++MISSING INFO: c.CFP301.long +++ (0.5 cc)
\item
  +++MISSING INFO: c.CFP302.long +++ (0.5 cc)
\item
  +++MISSING INFO: c.CFP401.long +++ (0.5 cc)
\end{enumerate}

Applications to the Crimson Fellows Program are accepted and reviewed on
a rolling basis.

Completed applications are assessed on ACT/SAT score, the high school
transcript, teacher recommendations, and the student essay. Although
there are no minimum thresholds, the historical average ACT score has
been over 27 with an average high school GPA above 3.70. Emphasis will
be placed on the student essay.

\begin{itemize}
\tightlist
\item
  CFP 104 Topics in Crimson Fellows Program

  \begin{itemize}
  \tightlist
  \item
    Reading and discussion of one or more classic texts from across
    intellectual disciplines. Students are expected to demonstrate
    mastery of the material and actively engage in class discussions.
    S/U basis only. May be taken more than once for credit, provided the
    topics are substantially different. Prerequisite: consent of
    instructor. (0.2 course credit)
  \end{itemize}
\item
  CFP 145 CFP: Culture and Revolution

  \begin{itemize}
  \tightlist
  \item
    Designed to examine cultures that subsequently undergo revolutionary
    change. The first part is devoted to giving a sense of the culture
    that is destroyed by the revolution, and the second part to the
    forces that lead to the revolution. This course focuses on
    masterworks in history, political science, philosophy, and
    literature.
  \end{itemize}
\item
  CFP 155 Style and Transformation in the Art

  \begin{itemize}
  \tightlist
  \item
    Focuses on periods during which the arts undergo a major
    transformation. The class studies both the artistic ideas that are
    being changed and the characteristic styles that result from these
    changes. The scope of this course may include literature, painting,
    music, and architecture.
  \end{itemize}
\item
  CFP 175 Continuity \&Transition Non-West Soc

  \begin{itemize}
  \tightlist
  \item
    Focuses on the great traditions in non-western cultures. By
    selectively dealing with the major traditions present in such
    cultures, the class deals with the dynamics of continuity and
    transition, which are crucial to understanding non-western
    societies.
  \end{itemize}
\item
  CFP 184 Topics in Scientific Inquiry

  \begin{itemize}
  \tightlist
  \item
    Designed to lead students to an intimate understanding of how the
    scientific process works and how scientific thought develops.
    Historical readings and discussions develop students' understanding
    of the course topic sufficiently for them to focus on particular
    scientific questions. Experimental approaches to these questions are
    discussed and developed into research projects. Results are shared
    and integrated, providing group members with greater knowledge of
    the course topic and an acute awareness of the process and
    limitations of science.
  \end{itemize}
\item
  CFP 205 Introduction to Engaged Scholarship

  \begin{itemize}
  \tightlist
  \item
    Establishes the foundation of the Fellows experience. Fellows write
    essays on a series of connected subjects designed to encourage
    critical thinking and reflection. The course includes discussions on
    education reform ideas, action research, student-driven education,
    the passion of life-long learning, global citizenship, community
    advocacy and civic engagement. Grounded in critical pedagogy, the
    course introduces students to the ideas of engaged scholarship.
    Students take command of their own education, developing their sense
    of agency, selfdiscipline, initiative, and self-direction.
    Prerequisite: admission to the Crimson Fellows Program or consent of
    instructor. (Offered Spring Term)
  \end{itemize}
\item
  CFP 301 Communicating Across Disciplines I

  \begin{itemize}
  \tightlist
  \item
    Focuses on developing competencies in communication across multiple
    audiences and perspectives. Students engage content/material related
    to their understanding and development of a year-long research
    project. The central objective of the course is to enhance student
    collaboration and communication skills. Prerequisite: Introduction
    to Engaged Scholarship (CFP-100) or consent of instructor. (0.5
    course credit)
  \end{itemize}
\item
  CFP 302 Communicating Across Disciplines II

  \begin{itemize}
  \tightlist
  \item
    Students work in multi-disciplinary teams to address a topic of
    local and/or global social interest. Prerequisite: Communicating
    Across Disciplines I (CFP-301) or consent of instructor. (0.5 course
    credit)
  \end{itemize}
\item
  CFP 401 Action Research I

  \begin{itemize}
  \tightlist
  \item
    Students work in multi-disciplinary teams to conduct a year-long
    project that addresses a specific challenge/issue for a
    local/regional community partner (e.g.~sustainability, water
    quality, soil degradation, health care, partner violence, refugees,
    immigration, civil rights, etc.). Faculty provide specific content
    related to working with community groups/agencies at the onset of
    the course. Prerequisite: Communicating Across Disciplines II
    (CFP-302) or consent of instructor. (0.5 course credit)
  \end{itemize}
\end{itemize}

\subsection{Crimson Fellows Projects \& Graduating with
Distinction}\label{crimson-fellows-projects-graduating-with-distinction}

To graduate with Distinction, at the time of graduation a student must
have:

\begin{enumerate}
\def\labelenumi{\arabic{enumi}.}
\tightlist
\item
  earned at least a 3.2 cumulative grade point average for all courses
  taken at Coe College,
\item
  earned at least a 3.5 GPA (or higher if set higher in the program in
  which you are pursuing distinction) in the courses taken toward the
  major or minor in which distinction is sought,
\item
  earned at least 14 course credits of graded courses at Coe College,
\item
  satisfactorily completed a Crimson Fellows project in a major or
  minor,
\item
  completed the ``Graduating with Distinction Form'' found on My.Coe and
  submitted it to the Office of the Registrar no later than March 15 of
  the Spring Term before graduation,
\item
  submitted a thesis or project artifact, approved by the majority of
  the student's Crimson Fellows Program examining committee, to the
  Director of Library Services no later than Reading Day of Spring Term.
\end{enumerate}

\subsection{Latin Honors}\label{latin-honors}

Cum laude is awarded to all graduating seniors with a cumulative GPA of
3.60 or higher. Magna cum laude is awarded to graduating seniors with a
cumulative GPA of 3.80 or higher who have completed an honors project.
Summa cum laude is awarded to graduating seniors with a cumulative GPA
of 3.98 or higher who have completed an honors project.

\subsection{Dean's List}\label{deans-list}

Special recognition is given to students who show exceptional academic
performance during Fall and Spring Terms. The designation ``Dean's
List'' is awarded a student if, at the end of a given grade reporting
period, the student:

\begin{enumerate}
\def\labelenumi{\arabic{enumi}.}
\tightlist
\item
  was enrolled as a full-time, degree-seeking student;
\item
  earned at least a 3.5 GPA for the grading period, having no incomplete
  marks, no repeat courses, and at least three letter graded courses;
  and
\item
  ranked in the top ten percent of the student body for that grading
  period.
\end{enumerate}

\section{Cross-Registration with Mount Mercy
University}\label{cross-registration-with-mount-mercy-university}

This agreement:

\begin{enumerate}
\def\labelenumi{\arabic{enumi}.}
\tightlist
\item
  Covers only courses that are not offered at Coe College in the same
  term unless a time conflict exists that cannot be resolved.
\item
  Is permitted on a space-available basis two weeks after the regular
  registration at Mount Mercy.
\item
  Holds students subject to administrative rules of the host institution
  for the courses taken.
\item
  Requires the student to register at both institutions.
\item
  Requires that a student be full-time and degree seeking in the term of
  the request and for at least one previous term at Coe College.
\end{enumerate}

Coe students wishing to enroll at Mount Mercy University may not be on
academic probation (see p.~43) and may not have been dismissed from Coe
College. Prior to registering, students must submit a cross-registration
request form to the Registrar, who grants approval to students wishing
to register at Mount Mercy University. If the course is to be counted
toward a major or minor, the approval of the appropriate Coe department
chair is also required. Declarations of Pass/Not Pass options are made
at Coe according to Coe policies. Both course credit and the letter
grade given at Mount Mercy are recorded in the student's permanent
record, as well as the fact that the course was taught at Mount Mercy.
Under the agreement no additional fees are charged for cross
registration, although the sum of the credits registered at both
institutions are used to determine full-time status and/or the need to
petition to take 5.0 course credits or more.

A cross-registered student missing a class at a cooperating college
because of calendar differences shall not be penalized for missing the
class. The student, however, is responsible for making up any work
missed in the class. Before registering for a course at Mount Mercy
under this agreement, a student must complete the Mount Mercy--Coe
College Cross Registration Form to be processed to ensure that all
stipulations of the agreement are met.

\section{English As A Second Language
Program}\label{english-as-a-second-language-program}

Drexler (Director), Welsh.

English as a Second Language (ESL) programs at Coe College are designed
to help study abroad and undergraduate international students assimilate
into the College and local community through English language
instruction.

Coe College offers the following programs:

\begin{itemize}
\tightlist
\item
  Intensive English Language program (IELP). IELP accepts students whose
  TOEFL ITP score falls lower than 500. IELP students enroll in 18 hours
  of English language courses per week for one or two terms. Students
  who successfully complete IELP may matriculate to the College.
\item
  English Academic Bridge (ELAB) program. ELAB accepts students whose
  TOEFL ITP score falls between 500 and 520, the Admission requirement.
  ELAB students are admitted as undergraduates with the following course
  requirements for their first term at the College: one ESL course, one
  Coe-credit course designed for English language learners, First-Year
  Seminar, and their choice of another Coe-credit course.~
\item
  Short Intensive English Language program (SIELP). SIELP accepts study
  abroad students for seven weeks during the Spring Term. SIELP students
  enroll in 18 hours of English language course per week and live on
  campus.
\item
  Summer Academic Orientation program (SAOP). SAOP accepts students who
  intend to study abroad or become undergraduates in colleges and
  universities in the United States or Canada. SAOP students enroll in
  20 hours of ESL courses per week for the first week of August.
\end{itemize}

In addition to these programs, Coe international students who are not
enrolled in an ESL program may register for ESL courses and work with
ESL faculty on a one-on-one basis to support~their language needs.

\begin{itemize}
\tightlist
\item
  ESL 195 Reading Writing Workshop

  \begin{itemize}
  \tightlist
  \item
    Strengthens critical reading, academic writing, and reasoning skills
    by engaging with a variety of texts. Students identify, challenge,
    and write arguments by practicing summary, analysis, paraphrase, and
    response to published work as well as work of their peers. May be
    taken more than once with consent of the program director provided
    the topics are substantially different. Prerequisite: enrolled in
    ELAB or undergraduate program with appropriate TOEFL score or
    consent of instructor.
  \end{itemize}
\item
  ESL 180 Topics in Listening

  \begin{itemize}
  \tightlist
  \item
    Develops academic listening and note-taking skills with a focus on
    micro listening such as listening for numbers, word stress, thought
    groups, and accent differences as well as macro listening like
    listening for headings, transition words and phrases, and
    distinguishing main ideas from details. May be taken more than once
    with consent of the program director provided the topics are
    substantially different. Prerequisite: enrolled in ELAB or
    undergraduate program with appropriate TOEFL score or consent of
    instructor.
  \end{itemize}
\item
  ESL 185 Topics in Speaking

  \begin{itemize}
  \tightlist
  \item
    Develops speaking fluency through vocabulary building by using
    language in authentic contexts and practicing pronunciation by
    studying language prosody (intonation and rhythm) and morphemes
    (minimal parts of language). May be taken more than once with
    consent of the program director provided the topics are
    substantially different. Prerequisite: enrolled in ELAB or
    undergraduate program with appropriate TOEFL score or consent of
    instructor.
  \end{itemize}
\item
  ESL 190 Topics in Structure

  \begin{itemize}
  \tightlist
  \item
    Develops grammar with the goal of using accurate grammar in
    presentations and written work. Students study various grammar
    points, practice grammar interactively, and integrate learned
    grammar into presentations and written work on a wide spectrum of
    topics. May be taken more than once with consent of the program
    director provided the topics are substantially different.
  \end{itemize}
\item
  ESL 210 Seminar in Culture

  \begin{itemize}
  \tightlist
  \item
    Develops language skills and explores American culture through
    texts, art, film, and community engagement. May be taken more than
    once with consent of the program director provided the topics are
    substantially different. Prerequisite: enrolled in ELAB or
    undergraduate program with appropriate TOEFL score or consent of
    instructor.
  \end{itemize}
\end{itemize}

\section{Pre-Professional Programs}\label{pre-professional-programs}

\subsection{Pre-Law}\label{pre-law}

J. Christensen (Program Director).

The cooperative 3+3 program allows qualified undergraduates from Coe
College to earn both a bachelor's degree and a law degree in six years
through partnership with the University of Iowa College of Law.

If eligible, students admitted under the 3+3 program will receive three
years of financial aid and pay Coe College tuition for the first three
years. Qualified undergraduates from Coe must have satisfied all
graduation requirements with the exception of the 32-course credit
requirement by the conclusion of their junior year for admission into
the College of Law. Students are not eligible for campus housing or
extracurricular activities at Coe during their fourth year. During the
fourth year of the program, which is the first year of law school,
students will pay tuition only to Iowa Law and apply for financial aid
through the University of Iowa. Credits earned during the first year of
law school at Iowa, which would have been their senior year at Coe, will
also apply to their undergraduate degree to complete the final credit
requirements at Coe. At the end of their fourth year of study, students
in the program will receive their bachelor's degree from Coe College,
while also having a year of law school completed at the University of
Iowa College of Law.

There is no prescribed curriculum for students intending to enter law
school after graduation. Law schools report that their most successful
students are those who have acquired a broad academic background in the
liberal arts, developed a capacity for logical analysis, and mastered
the ability to write clearly.

Study in one or several of a variety of disciplines will prepare
students to undertake legal training. Students interested in law should
consult their department advisor and the pre-law program director.

\subsection{Health Professions}\label{health-professions}

Storer (Program Director).

Coe's program for those interested in the health professions, such as
medicine and dentistry, is a flexible one based on the requirements of
the health professions' schools. Students interested in a health
profession usually major in one of the sciences, but all majors offered
by the College are acceptable.

\subsection{Professional School Degree Completion
Plan}\label{professional-school-degree-completion-plan}

Students who complete three years of coursework at Coe (24 course
credits), including general education and area of study requirements,
and who enter a college of architecture, engineering, or a physical
therapy program, can receive a baccalaureate degree from Coe. Required
for satisfactory completion of this program are (1) approval of the
program by the Provost and Dean of the Faculty before transferring to
the professional institution and (2) one year of full-time acceptable
study there.

\subsection{Cooperative Degree Program With The University Of Iowa's
College Of Public
Health}\label{cooperative-degree-program-with-the-university-of-iowas-college-of-public-health}

This combined undergraduate and graduate 5-year program allows students
to earn a Bachelor of Arts degree from Coe College and a Master of
Public Health (MPH) degree from the University of Iowa. This program is
available to students electing any undergraduate major offered by Coe.
The first four years in the program are spent in residence at Coe
College. Interested students take one University of Iowa undergraduate
course in the spring of their second year (\emph{Fundamentals of Public
Health}). In their third year, students take the GRE and apply to the
Master's program in Public Health. If accepted, the student completes up
to four graduate-level MPH courses during the fourth year at Coe College
(\emph{Introduction to Biostatistics, Introduction to Health Promotion
and Disease Prevention, Global Environmental Health, and/or
Epidemiology}). UI Public Health courses are accepted in transfer to Coe
College as elective credit.

\chapter{OFF-CAMPUS STUDY}\label{off-campus-study}

The College endorses a wide variety of off-campus experiences for
students. Coe strongly believes that students can benefit from study in
Washington, New York, and the Wilderness Field Station, as well as in
programs around the world.

Numerous domestic and international study programs are available to Coe
students. The four programs sponsored by Coe College are Asia Term, New
York Term, Washington Term and Wilderness Field Station courses. In
addition, programs are offered by the Associated Colleges of the Midwest
(ACM) and others, including several exchange programs, by colleges,
universities, and educational agencies in America and abroad. Students
who wish to study off-campus on Coe's sponsored programs must apply to
the individual program's director. Student proposals to study on
exchange programs must be submitted to the Director of Off-Campus
Studies at least six weeks before the end of the term just prior to the
off-campus experience. Student proposals to study on all other
off-campus programs must be submitted to the Director of Off-Campus
Studies by the last day of classes of Fall Term of the academic year
prior to the program.

Credits earned on off-campus programs are applied toward graduation on
the same basis as credits earned on campus. Any academic credit earned
from programs not sponsored by Coe may be transferred back to the
College in accordance with the College's general policy on transfer
credit. Application is open to sophomores, juniors, and seniors who have
a minimum 2.25 GPA at both the time of application and of enrollment for
the off-campus study.

NOTE: \emph{International students may not receive Coe College financial
aid for off-campus study outside the U.S.}

\section{Domestic Programs}\label{domestic-programs}

\subsection{New York Term}\label{new-york-term}

Carson (Program Director)

The New York Term is open to all students who meet the basic
requirements for off-campus study. Offered in odd-numbered years in the
Spring Term, this program provides abundant opportunity for experiences
in the performing and visual arts, as well as internship opportunities
for students from any major.

The central course, Fine Arts in New York City (NYT-250), which includes
attendance at concerts, theatre, and dance productions as well as tours
to art exhibits and film screenings, consists of five 0.4 credit
courses: art, music, theatre, dance, and film. A faculty member for each
area grades the respective course. An internship (or other project
approved in advance by the College) completes the program.

There is an extra fee for New York Term. All Coe financial aid applies,
and students are eligible to apply for additional financial aid based on
the additional costs of the term. (See p.~72 for descriptions and course
offerings.)

\subsection{Washington Term}\label{washington-term}

B. Nesmith (Program Director).

The Washington Term is open to all students who meet the basic
requirements for off-campus study. Students accepted for Washington Term
spend Fall or Spring Term in the nation's capital. Washington provides
an unusual opportunity to study national politics and government and to
enjoy a variety of cultural activities in the fine arts. (See p.~75 for
descriptions and course offerings.)

\subsection{Wilderness Field Station}\label{wilderness-field-station}

Ellis (Program Director).

The Coe College Wilderness Field Station, located on remote Low Lake in
Minnesota's Superior National Forest, offers students a unique and
unparalleled opportunity for off-campus study. Courses take advantage of
the serene surroundings for field observation, wilderness study, and
outdoor learning. Biology courses are at the heart of the field station
and often include aquatic biology, animal behavior, ornithology, and
behavioral ecology of vertebrates. Students use the base camp's
laboratories, herbarium, and library to supplement their field work.
Non-science electives, such as nature writing and wilderness and the
law, are also offered. There is an independent study option as well.
Participants take one course during a four-week session.

The program runs from mid-June to mid-July, mid-July to mid-August, or
both. Each course is limited to eight students. Each course offered at
the field station is one course credit. Particular courses satisfy lab
science and other general education requirements and can be used as
major elective credits. Any course taken at the field station satisfies
the College's practicum requirement. The regular application deadline is
March 1st; the final deadline is April 15th, with rolling applications
after that date. For more information and application materials, visit
The Field Station Webpage:
www.coe.edu/academics/coe-difference-centers-and-programs/off-campus-study/wilderness-field-station.

\section{ACM \& International
Programs}\label{acm-international-programs}

Coe offers opportunities to study in locations around the world. Some
programs are for students wishing to broaden their liberal arts
perspectives, while others allow intensive research and study in a
specific academic area. Although some programs provide grades in the
courses, all grades transfer back to Coe as P/NP. For detailed
information and applications, students should contact directors for each
program or visit http://www.acm.edu/off\_campus\_study/index.html.

\subsection{OCC-350 --- Japan Study}\label{occ-350-japan-study}

Nordmann (Program Director).

Students study at Waseda University's School of International Liberal
Studies in Tokyo after a brief orientation providing intensive language
practice and cultural discussions. In addition to required language
study, electives may be chosen from a wide range of Asian studies
courses taught in English. A family living experience in Tokyo provides
an informal education in Japanese culture and is in many ways the
dominant feature of the program, offering total immersion in the
Japanese way of life. The program is recommended for a full year of
study, although students usually cannot receive Coe gift aid for more
than one term. The full-year program includes a month-long cultural
practicum or internship in another region of Japan, usually in February
or March. Administered by Earlham College, Japan Study is recognized by
both ACM and GLCA. Learn more at
http://www.acm.edu/programs/8/japan/index.html.

\subsection{OCC-360 --- Newberry Seminar: Research in the
Humanities}\label{occ-360-newberry-seminar-research-in-the-humanities}

Swenson Arnold (Program Director).

Students in the Newberry Seminar do advanced independent research in one
of the world's great research libraries. They join ACM and GLCA faculty
members in close reading and discussion centered on a common theme, and
then write a major paper on a topic of their choice, using the Newberry
Library's rich collections of primary documents. The Fall Seminar runs
for a full term; the Spring Seminars are month-long. Students live in
Chicago apartments and take advantage of the city's rich resources. The
Newberry Seminar is for students looking for an academic challenge, a
chance to do independent work, and possibly planning to attend graduate
school. Administered by ACM, the Newberry Seminar is also recognized by
GLCA. Learn more at http://www.acm.edu/programs/14/newberry/index.html.

\subsection{OCC-325 --- Field Museum Semester: Research in Natural
History}\label{occ-325-field-museum-semester-research-in-natural-history}

Hughes (Program Director).

An intensive research-and class-based experience for upper-level
students interested in natural history research with a background in
evolutionary biology, anthropology, or related discipline. The program
provides the opportunity for students to explore scientific research and
the Field Museum collections through a substantive internship, a course
taught by the visiting Faculty Director, and a seminar led by Field
Museum professional staff. See ACM website for more information.

\subsection{OCC-365 --- Oak Ridge Science
Semester}\label{occ-365-oak-ridge-science-semester}

St.~Clair (Program Director).

The Oak Ridge Science Semester is designed to enable qualified
undergraduates to study and conduct research in a prestigious and
challenging scientific environment. As members of a research team
working at the frontiers of knowledge, participants engage in long-range
investigations using the facilities of the Oak Ridge National Laboratory
(ORNL) near Knoxville, Tennessee. The majority of a student's time is
spent in research with an advisor specializing in biology, engineering,
mathematics, or the physical or social sciences. Students also
participate in an interdisciplinary seminar designed to broaden their
exposure to developments in their major field and related disciplines.
In addition, each student chooses an elective from a variety of advanced
courses. The academic program is enriched in informal ways by guest
speakers, departmental colloquia, and the special interests and
expertise of the ORNL staff. Administered by Denison University, Oak
Ridge Science Semester is recognized by both ACM and GLCA. Learn more at
http://www.acm.edu/programs/15/oakridge/index.html.

\section{Asia Term}\label{asia-term}

Nordmann (Program Director).

The Asia Term is open to all students who meet the basic requirements
for off-campus study. Usually offered in the Spring Term, this program
provides students an opportunity to experience a variety of Asian
cultures in such countries as Thailand, Vietnam, and Cambodia. At each
site, students study language, read works in English about the culture,
engage in service learning, and work with students at the host
universities to gain a functional understanding of how each culture
works. Students are accompanied to Asia by Coe faculty members.

Students take four credits of coursework, typically one credit of Asian
Tonal Languages, one credit of Asian studies, and two credits of
independent study. In some iterations of the program, students take an
elective course in art, English, history, sociology, education, or
another discipline, depending on the field of the instructor leading the
program, and one credit of independent study. (See p.~70 for
descriptions and course offerings.)

\section{Exchange programs}\label{exchange-programs}

Coe College sponsors a number of programs with cooperating foreign
universities, offering Coe sophomores, juniors and seniors each year the
opportunity to study in a foreign setting. Coe College accepts in return
junior-level students from the foreign institution. Applications of the
recommended students are sent to the host institution, with the host
reserving the right to admit or reject each student nominated.

Any student who applies for one of these programs must have completed at
least one year of continuous study at their home institution. Students
may apply to any appropriate academic program offered at the host
institution as full-time, non-degree seeking, or unclassified students.
Any academic credit earned at the host institution is transferred back
to the home institution in accordance with the rules of that
institution. The length of stay may not exceed one academic year. Upon
completion of the time period specified at the host institution, the
participating students must return to their home institution. Any
extension of stay must be approved by both cooperating institutions. The
exchange student must abide by all rules and regulations of the host
institution.

An exchange student must register and pay tuition and required fees at
his or her home institution. In return, the student receives a tuition
waiver at the host institution. The host institution helps arrange the
necessary visa documents and also provides appropriate advising and
other assistance to the incoming students from Coe College. Please see
individual program descriptions for information regarding housing costs.
The host institution assists in finding housing on the foreign
university campus; Coe College assists in finding residence housing for
students from the foreign university. At the end of the school year, the
host institution submits to the home institution and official
transcripts of grades and credits earned. Grades from exchange programs
transfer to Coe as P/NP.

Beyond tuition and fees, the participating student is responsible for
the following expenses: meal expenses; transportation to and from the
host institution; medical insurance and/or medical expenses; textbooks,
clothing, and personal expenses; passport and visa costs; and all other
debts incurred during the course of the year.

\subsection{OCC-205 --- Coe/Kongju National University (South
Korea)}\label{occ-205-coekongju-national-university-south-korea}

Nordmann (Program Director).

Course offerings in Business, Economics, and Asian studies. Students pay
the cost of living at the destination. Credits earned from the Kongju
exchange program are evaluated on a P/NP basis.

\subsection{OCC-210 --- Coe/Chiang Mai University
(Thailand)}\label{occ-210-coechiang-mai-university-thailand}

Chaimov (Program Director).

Coe students usually pursue an independent research project based on
prior study in Thailand, as CMU offers no courses in English. Students
pay cost of living at destination.

Credits earned from the Chiang Mai exchange program are evaluated on a
P/NP basis.

\subsection{OCC-213 --- Coe/Rangsit University
(Thailand)}\label{occ-213-coerangsit-university-thailand}

Chaimov (Program Director).

Rangsit University. Coe students are responsible for securing their own
accommodations. Rangsit University offers English language bachelor's
degree programs in communications, international business, and
international political economy. Students pay cost of living at
destination.

Credits earned from the Rangsit University exchange program are
evaluated on a P/NP basis.

\subsection{OCC-215 --- Coe/Mid Sweden University
(Sweden)}\label{occ-215-coemid-sweden-university-sweden}

Carstens (Program Director).

Courses in English are available in such areas as business, social
sciences, and environmental studies. Students pay cost of living at the
destination. Credits earned from the Mid Sweden exchange program are
evaluated on a P/NP basis.

\subsection{OCC-220 --- Coe/Nagoya-Gakuin University
(Japan)}\label{occ-220-coenagoya-gakuin-university-japan}

Nordmann (Program Director).

One year of Japanese language study is recommended for students applying
for this program. Exchange students from Coe pay room expenses at Coe
College. In return, students receive a room expense waiver. Other costs
of living are paid at Nagoya-Gakuin University. Credits earned from the
Nagoya-Gakuin exchange program are evaluated on a P/NP basis.

\subsection{OCC-225 --- Coe/Northern Ireland Scholars Program (Northern
Ireland,
UK)}\label{occ-225-coenorthern-ireland-scholars-program-northern-ireland-uk}

Farrell (Program Director).

Students with a high GPA may be selected to study at one of several
universities in Northern Ireland, including Queens University Belfast
and the University of Ulster. Applications for this consortial exchange
are due in December of the year before study. Students pay cost of
living at destination.

Credits earned from the Northern Ireland exchange program are evaluated
on a P/NP basis.

\subsection{OCC-230 --- Coe/University of Jaume I (Castello,
Spain)}\label{occ-230-coeuniversity-of-jaume-i-castello-spain}

Rodríguez Moreno (Program Director).

Courses in Spanish in a wide range of topics. Students pay cost of
living at destination. UJI requires Coe students to have completed two
Spanish courses at Coe. Spanish language courses are available for an
extra cost.

Credits earned from the Jaume I exchange program are evaluated on a P/NP
basis.

\subsection{OCC-235 --- Coe/Sookmyung University (South
Korea)}\label{occ-235-coesookmyung-university-south-korea}

Nordmann (Program Director).

Courses offered in English on areas including the arts, linguistics,
international studies, business, biology, and political science.
Students pay cost of living at destination. Credits earned from the
Sookmyung exchange program are evaluated on a P/NP basis.

\subsection{OCC-240 --- Coe/University of Landau
(Germany)}\label{occ-240-coeuniversity-of-landau-germany}

Chaimov (Program Director).

Courses in English are available in, art, English literature,
linguistics, other topics. Also, a wide range of subject areas taught in
German. Students pay cost of living at the destination.

Credits earned from the Landau exchange program are evaluated on a P/NP
basis.

\subsection{OCC-245 --- Coe/University of Quebec (Saguenay,
Canada)}\label{occ-245-coeuniversity-of-quebec-saguenay-canada}

Janca-Aji (Program Director).

Courses in French are offered in a wide range of topics for students who
pass a proficiency test in French. Students pay cost of living at
destination. Credits earned from the Quebec exchange program are
evaluated on a P/NP basis.

\subsection{OCC-250 --- Coe/Izmir Institute of Technology
(Turkey)}\label{occ-250-coeizmir-institute-of-technology-turkey}

Akgun (Program Director).

Coe may send a student of Chemistry and a student of Physics to study
those subjects in an English-language setting at a science university in
Turkey. Students are responsible for housing, food, and all other costs
of living.

Credits earned on the Izmir exchange program are evaluated on a P/NP
basis.

\subsection{OCC-255 --- Coe/Polytechnic University of Upper France
(France)}\label{occ-255-coepolytechnic-university-of-upper-france-france}

Janca-Aji (Program Director).

Students choose from courses in English on business, communications, and
marketing or a wide range of courses in French. Students pay costs of
living at the destination. Credits earned from the France exchange
program are evaluated on a P/NP basis.

\subsection{OCC-260 --- Coe/National University of Villa Maria
(Argentina)}\label{occ-260-coenational-university-of-villa-maria-argentina}

Students with a good command of Spanish can take courses in Spanish in a
wide range of areas, including literature, rural development, social
sciences, environmental studies, communication, and computer science.
Students pay costs of living to Coe before departure and must transfer
at the same time as an incoming student from UNVM.

Credits earned on the Argentina exchange program are evaluated on a P/NP
basis.

\subsection{OCC-265 --- Coe/Istanbul Altinbas University
(Turkey)}\label{occ-265-coeistanbul-altinbas-university-turkey}

Duru (Program Advisor)

Teaches entire majors in English in psychology, sociology, international
relations, political science, economics, business. Students pay cost of
living at the destination. Credits earned from the Altinbas exchange
program are evaluated on a P/NP basis.

\subsection{OCC-270 --- Coe/Ashesi University
(Ghana)}\label{occ-270-coeashesi-university-ghana}

Eichhorn (Program Director).

Courses in African studies (sociology, anthropology, political science,
history, arts), computer science, business. Coe students pay room and
board expenses at Coe College and receive a waiver of these expenses in
Ghana. They must exchange at the same time as an Ashesi student.

Credits earned on the Ashesi exchange program are evaluated on a P/NP
basis.

\subsection{OCC-275 --- Coe/University of Salford
(England)}\label{occ-275-coeuniversity-of-salford-england}

Kuennen (Program Director).

The University of Salford offers courses in business, contemporary
European history, psychology, and many other areas. Students pay cost of
living at the destination and must arrange their own housing.

Credits earned from the Salford exchange program are evaluated on a P/NP
basis.

\subsection{OCC-285 --- Coe/University of Neuchatel
(Switzerland)}\label{occ-285-coeuniversity-of-neuchatel-switzerland}

Janca-Aji (Program Director).

Coe students take courses at the Institute of French Language and
Civilization. Students pay costs of living at the destination. Credits
earned from the Neuchatel exchange program are evaluated on a P/NP
basis.

\subsection{OCC-290 --- Coe/Jinan University
(China)}\label{occ-290-coejinan-university-china}

Nordmann (Program Director).

Located in southern China, this international university offers courses
in English in international economics and business, journalism, computer
science, and Chinese studies as well as the study of Chinese language.
Students pay costs of living at the destination.

Credits earned from the Jinan exchange program are evaluated on a P/NP
basis.

\subsection{OCC-291 --- Coe/Istanbul Kultur University
(Turkey)}\label{occ-291-coeistanbul-kultur-university-turkey}

Duru (Program Advisor).

Offers English curriculum in psychology, business, economics,
international relations, and other areas. Students pay costs of living
at the destination. Credits earned from the Istanbul Kultur University
exchange program are evaluated on a P/NP basis.

\part{CAMPUS RESOURCES}

\chapter{CAMPUS RESOURCES}\label{campus-resources-1}

\subsection{Libraries}\label{libraries}

The College libraries---Stewart Memorial Library, located at the center
of the campus and Fisher Music Library in Marquis Hall---contain over
500,000 volumes and 16,000 pieces of media. Current subscriptions to
some 3,500 periodicals and serials are maintained in print or electronic
format, and over 200,000 electronic resources with books and journal
volumes added annually.

The collections ably support undergraduate education and are especially
strong in the areas of literature, history, and music. The Fisher Music
Library contains over 5,000 compact discs and records, 5,300 scores and
books, and is equipped with listening facilities. Media services to the
campus are provided through the Media Technologies Department in the
library. These services include a circulating collection of over 8,000
DVDs, two media-equipped auditorium styled classrooms, editing stations,
an innovation studio that houses a 3D printer and laser cutter, and a
variety of cameras, recording equipment available for use.

The main library houses the Learning Commons (see description below) and
the college archives. The library provides an outstanding research
collection consisting of both print and electronic books, journals, and
reference resources. The Reference Department assists students with
their research needs through one on one research assistance, evaluating
resources and websites, citation assistance and multimedia evaluation.
In addition to library orientations, research classes are offered on
specific course related topics. The library offers computer stations,
iPad and laptop checkouts, study areas for individual and group study,
including technology enhanced study rooms. These resources are greatly
augmented by providing access to over 100 scholarly databases and an
extensive webpage: coe.edu/library.

The George T. Henry College Archives includes a research room and a
climate-controlled vault located on the lower level of the Stewart
Memorial Library. It houses and preserves the institutional records of
Coe College and the papers of staff, students, and alumni of the
college. Archive staff is available to aid students, faculty, and
scholars in navigation and use of more than 900 linear feet of primary
source documents. The Archives maintains the papers of journalist,
author, and World War II broadcaster William L. Shirer, Coe class of
1925 author of \emph{The Rise and Fall of the Third Reich}; the literary
works and selected private papers of Iowa poet Paul Engle, Coe class of
1931; and the photographs of longtime Coe College photographer George T.
Henry.

\subsection{Learning Commons}\label{learning-commons}

The Coe Learning Commons in the Stewart Memorial Library integrates all
of the College's academic support resources in a single location at the
heart of campus delivered through peer education and by professional
staff. Services and resources include academic coaching, supplemental
advising, Writing Center, AAP-TRIO program, tutoring, accessibility
support and accommodations, academic technology, Office of Off-Campus
Study, and fellowship and graduate school advising.

\subsection{Art Collections}\label{art-collections}

Selections from the College's Permanent Collection of Art totaling over
800 works by 200 artists are displayed in and near many of the campus
buildings. Most visible are the large outdoor sculptures on the campus,
yet almost every building features selections from the Permanent
Collection.~ For example, the Ella Poe Burling collection of
nineteenth-century American and French art and antiques is exhibited in
the upper lobby of Voorhees Hall.

A large portion of the Permanent Collection can be found in Stewart
Memorial Library.~ Four special galleries contain works by renowned
American painters Grant Wood, Marvin Cone (Coe class of 1914), and
Conger Metcalf (Coe class of 1936). Six large farm murals by Grant Wood
constitute the heart of the Permanent Collection's Regionalist works.
These murals are supplemented by nine smaller yet significant works by
Wood, including \emph{Daughters of Revolution}, a charcoal, pastel, and
pencil on paper drawing of Wood's painting of the same name.

Another signature feature of the Regionalist collection is the work of
Marvin Cone. A 1914 Coe graduate, Cone later became a faculty member who
founded the College's Art Department. Cone personally selected many of
the paintings and drawings in the collection as representative of his
own artistic development, underscoring the historic role of art as a
core element in Coe's teaching mission.

In addition to the works by native Iowans Cone and Wood, Coe College has
acquired a distinguished collection of 70 works by Conger Metcalf, an
American modernist painter, as well as paintings, drawings, and prints
by notable artists such as Milton Avery, Mauricio Lasansky, Henri
Matisse, Pablo Picasso, and Andy Warhol.

More information about the Permanent Collection can be found on either
the Permanent Collection's website,
http://picovado.com/jrogers/\#-h2-introduction-h2-, or the library's
webpage, www.coe.edu/academics/stewart-memorial-library.

\subsection{Information Technology}\label{information-technology}

The Information Technology Office~provides a wide range of technology
support to students,~faculty and staff. This includes management of
computer labs, classroom technology, college-wide software licensing,
wi-fi/internet, printers, cable TV, sound equipment, My Coe (my.coe.edu)
and more. Coe's technology facilities include over 3,000 ethernet ports,
full ethernet and wi-fi internet services within all campus buildings,
250 public/lab computers available~for student use and full access to
G-Suite services. The Office of Information Technology is located in
Voorhees Hall. Assistance from the IT staff can be requested through a
Help Desk/Spiceworks ticket (link found on~my.coe.edu).~

\subsection{Public Events and
Artists-in-Residence}\label{public-events-and-artists-in-residence}

Guest lecturers and artists provide an essential dynamism to the
educational climate at Coe. Programs are free to Coe students. In
addition to hearing speakers of national and international note,
students may have the opportunity to talk with them during a carry tray
lunch, to attend special issues dinners with the speakers, or to have
them as guests in a class. Performing groups appearing as
artists-in-residence often stay two or three days on campus to work with
students.

\subsubsection{Marquis Lecture \& Performance
Series}\label{marquis-lecture-performance-series}

The Marquis Lecture \& Performance Series hosts performances and
presentations throughout the academic year. The Marquis Series is
endowed by a gift from Sarah Marquis, Coe class of 1918, in honor of her
father, John A. Marquis, who was president of Coe from 1909 to 1919.

\subsubsection{Coe College Contemporary Issues
Forum}\label{coe-college-contemporary-issues-forum}

The Coe College Contemporary Issues Forum brings to audiences of the
College community the presence and

views of distinguished professionals whose work has received national
recognition. The forum is normally presented during the month of
February.

\subsubsection{Phi Beta Kappa Visiting
Scholar}\label{phi-beta-kappa-visiting-scholar}

The Coe chapter of Phi Beta Kappa sponsors a Phi Beta Kappa Visiting
Scholar who, in addition to presenting a public lecture, meets with
individual classes as appropriate to the scholar's area of expertise.

\chapter{STUDENT LIFE}\label{student-life}

Student Life provides personalized support to students, enabling each to
gain the best possible undergraduate education adding substantially to
the educational program. Residence accommodations, along with living and
learning values, an attractive campus social life, a sound health
program, good recreational facilities, and a program of co-curricular
activities are among the opportunities offered Student Life.

\subsection{Campus Civility Statement}\label{campus-civility-statement}

This statement was written by students in order to address standards of
civility and respect within the Coe College community. This statement is
a living document and is intended to evolve over time.

\begin{itemize}
\tightlist
\item
  We, the members of the Coe College community, expect our campus
  climate to be safe, mutually supportive, academically encouraging,
  egalitarian, and tolerant of all its members.
\item
  We expect the academic experience to extend beyond the classroom into
  our living environment.
\item
  We expect a campus free of incidents that create a hostile living
  environment.
\item
  We expect a healthy and responsible attitude to accompany all social
  gatherings.
\item
  We expect that intoxication will not be an excuse for incidents that
  occur while under the influence.
\item
  We expect that diversity of opinion should be cultivated and
  encouraged as well as respected within our community.
\item
  We expect that everyone will have the right to be respected for his or
  her individuality.
\item
  We expect all campus community members to respect the rights of other
  persons regardless of their actual or perceived age, color, creed,
  disability, gender identity, national origin, race, religion, sex, or
  sexual orientation.
\end{itemize}

A community is made up of individuals who model these standards and hold
each other accountable. In order for the community to encompass the
goals outlined above, each individual must be responsible and
accountable for her or his own actions and words.

\subsection{Student Contribution to College
Policy}\label{student-contribution-to-college-policy}

Coe is proud of its traditions and its ability to change. While
cognizant of the past, the College is also sensitive to the changing
nature and needs of students. Coe students play an integral part in the
initiation of change by utilizing available channels and by the creative
development and use of new ones. The Student Senate is a frequent forum
for the resolution of student concerns.

\subsection{Student Senate}\label{student-senate}

Student Senate is the representative government of Coe students and
coordinates many co-curricular activities. Through the student activity
fee, the Senate funds the weekly newspaper, the Cosmos, and other
student publications. The Student Activities Committee (SAC) of the
Student Senate sponsors bands, comedians, multicultural programming, and
other kinds of entertainment and activities.

\subsection{Student Handbook}\label{student-handbook}

The \emph{College Policies and Student Handbook} outlines the College's
expectations for responsible behavior reflecting maturity, mutual
respect, and cooperation among all members of the Coe community. Student
Life develops policies for conduct procedures, residence hall living,
student organizations, and other areas of student affairs for approval
by the Board of Trustees. The College Policies and Student Handbook is
available online at
https://www.coe.edu/student-life/student-development/college-policies-student-handbook.

\subsection{Committee Participation}\label{committee-participation}

Students serve on various committees, which aid in making educational
policy at the college. Most committees (Academic Policies, Assessment,
Athletics, Campus Technology, Diversity, Enrollment, Financial Aid and
Academic Progress, Executive, Finance and Facilities, First-Year
Program, Internationalization, Marquis Series, Petitions, Sustainability
Council, Wellness, and Writing) include students appointed by the
Student Senate as voting members.

\section{Student Services}\label{student-services}

\subsection{Residence Life}\label{residence-life}

Coe is a residential college, in that the residence experience is an
integral part of the educational process. Students are expected to live
on campus for four years and take meals in the College hall (see
On-Campus and Off-Campus Resident Students, Section 27.1.4). The
residence halls and apartments vary in style, size, and personality. All
of the residential facilities have generous visitation policies, and
campus life functions around the concept of the living units.

Residence hall and apartment regulations are published in the
\emph{Student Reference Book}. As room charges do not include Winter
Break or Spring Break, when residential facilities and the dining hall
are closed, an additional fee is assessed to students staying on campus
during those times. Information regarding housing is sent to students
who have accepted admission to the College.

\subsection{Student Health Service}\label{student-health-service}

Coe's Student Health Service offers students medical care provided by a
part-time Advanced Registered Nurse Practitioner (ARNP) and a full-time
nurse. The Health Services staff can diagnose, manage, and treat certain
medical diagnoses, free to full-time Coe students. Referrals are
available to a family physician or specialist in the Cedar Rapids
community as needed. In the event that a student needs hospitalization,
St.~Luke's Hospital or Mercy Medical Center is utilized. The student is
responsible for all health costs beyond those provided by Health
Services. These include hospitalizations, emergency room visits,
physician visits, and prescription medications. Therefore, all students
are expected to carry medical insurance. Provisions for special diets or
other arrangements which deviate from Coe's policies require a
recommendation from the student's healthcare provider and are available
through SODEXO food services. All students are required to have a
current immunization record on file prior to registration for classes at
Coe. Students without immunizations on file will have their registration
held. Students who wish to utilize the Student Health Service should
have a physical on file, which is required of all students in order to
play collegiate sports.

\subsection{Personal Counseling}\label{personal-counseling}

Realizing that students have concerns in areas other than academic
matters and career options, Coe provides appropriate individual and
group counseling. St.~Luke's Family Counseling Center, located next to
the Coe campus, and the College work together to provide for students'
counseling needs. St.~Luke's Family Counseling Center provides
assessment, short-term counseling, and, when appropriate, referral to
community resources. Individual counseling is available from a number of
counselors on an appointment basis for students with personal, social,
and family concerns. In addition, students may seek counseling from a
pastoral and spiritual perspective from the College Chaplain. If
long-term counseling is needed, Coe's counselor and Chaplain work with
students to identify cost-effective solutions on and off campus.

\subsection{Religious Life}\label{religious-life}

Coe College believes that it is important to foster an environment that
accepts and respects the religious faith and beliefs of all its
students, staff, and faculty. The Coe community is diverse in its
religious makeup, and all members of the community are encouraged to
express and practice their particular religious traditions. To this end,
the College Chaplain seeks to provide and create an atmosphere that is
consistent with the religious heritage of the College and conducive to
the development of spiritual and moral values.

Opportunities for worship, interfaith dialogue, Bible study, retreat,
small groups, theological study, outreach, mission, meditation, and
fellowship are abundant. There are also several active religious student
organizations on campus and a specialized leadership program for
students considering vocational ministry. The Chaplain is available for
pastoral care, guided prayer, theological dialogue, pre-marital
counseling, and other spiritual needs.

\subsection{Diversity, Equity \&
Inclusion}\label{diversity-equity-inclusion}

The Office of Diversity, Equity \& Inclusion is comprised of the Dean of
Students, Coordinator of Multicultural Affairs, Director of
International Affairs, and the College Chaplain. This team works closely
with LBGTQAI+ students, multicultural students, international students,
and student allies, with the goal of fostering an environment where all
Kohawks will thrive in an inclusive learning environment. Students
interested in getting involved with Coe's diversity and inclusion
efforts are encouraged to stop by the Student Life Office in Upper Gage
and speak with a team member.

\subsection{Campus Activities}\label{campus-activities}

There is much to do on the Coe campus and in the Cedar Rapids community.
Programming of campus activities is designed to meet the educational and
recreational needs of the Coe community in a creative way. The diversity
of the student body is considered in the scheduling of recitals, plays,
exhibits, lectures, films, and concerts, as well as all school events
and relaxing evenings in Charlie's. The Director of Campus Life
coordinates the events organized by the Student Activities Committee
(SAC).

\subsection{Student Activity Groups}\label{student-activity-groups}

Students earn credit for their participation in Coe's music ensembles
(the Jazz and Concert Bands, the Symphony Orchestra, the Concert Choir,
Chorale, and the Choral Chamber Ensemble), several of which have
completed study/concert tours of Europe, Great Britain, and Asia in the
past. Coe drama and forensics activities have received local and
national recognition for their presentations. Special interest
organizations are also represented on campus, as are national social
fraternities and sororities and honor societies (Phi Beta Kappa, Phi
Kappa Phi, Mortar Board, Alpha Lambda Delta, and Alpha Sigma Lambda).

\subsection{Athletics}\label{athletics}

Coe College sponsors 11 men's and 11 women's athletic teams that compete
in the American Rivers Conference of NCAA Division III. Our goal is to
provide our student athletes with positive educational and athletic
experiences.

Through hard work, intense training, and positive interactions with
coaches, student athletes are provided opportunities to succeed. The
College recognizes that many of its students enjoy participating in
organized athletics or watching athletic contests as forms of
recreational campus life. Basketball, volleyball, softball, table
tennis, flag football, and wrestling are representative events in a
year-round intramural program for both men and women. All students are
eligible to participate.

\subsection{Recreational Facilities}\label{recreational-facilities}

Gage Memorial Union is the center of student interest and activity.
Offices of the Student Activities Committee and other student
organizations are there, plus the College dining hall, and mailroom.
Informal programs and lectures are given there, and it serves as the
College's ``open house'' for students, faculty, and visitors.
``Charlie's,'' located in the adjacent P.U.B., is home to a coffee shop
(serving Starbucks coffee), a convenience store, and grill. Charlie's is
a relaxing place to meet friends or take in one of the many
performances.

The brand-new Coe College Athletics and Recreation Center includes two
pristine courts for basketball and volleyball, a wrestling room that
boasts three oversized mats, strength and conditioning room with
brand-new equipment, and a fitness center that overlooks the campus.
This is where Coe's basketball, volleyball, and wrestling teams host
their home events.

Moray Eby Fieldhouse includes three recently-renovated courts for
basketball and volleyball. Eby also has a natatorium, athletic training
rooms, indoor baseball/softball batting cages, and a rock-climbing wall.
All outside playing fields and tennis courts are also available to Coe
students.

The Clark Racquet Center offers a state-of-the-art facility for runners,
racquet enthusiasts, and everyone wanting to stay in shape. An aerobic
room, dance studio, and classrooms complement the indoor track, tennis
courts, and racquetball and squash courts. Professional staff manage the
center and offer instruction to students.

\part{COLLEGE REGULATIONS}

\chapter{EFFECTIVE CATALOG}\label{effective-catalog}

Students ordinarily are graduated under the provisions of the catalog of
their matriculation date. A student's matriculation date is the day of
first enrollment following admission. If the student is readmitted, the
matriculation date becomes the day of first enrollment following
readmission. However, students will be expected to satisfy, to the
extent practical, the graduation requirements of the catalog in effect
at the anticipated date of graduation. Any necessary modification of
general degree requirements will be worked out by the Provost and Dean
of the Faculty, the Registrar, the appropriate department chair, and the
Academic Policies Committee. Modification of major or minor requirements
will be worked out by the department chair involved, the Registrar, and
the student's advisor. A student has the right of petition to the
Committee on Petitions.

\subsection{Grading}\label{grading}

\textbf{GRADES} \textbar{} \textbar{} \textbar{} \textbar{}
\textbar----\textbar-----------------------------------\textbar------------------------------------\textbar{}
\textbar{} A \textbar{} Excellent \textbar{} 4.0 grade points per course
credit \textbar{} \textbar{} A- \textbar{} \textbar{} 3.7 grade points
per course credit \textbar{} \textbar{} B+ \textbar{} \textbar{} 3.3
grade points per course credit \textbar{} \textbar{} B \textbar{} Above
average \textbar{} 3.0 grade points per course credit \textbar{}
\textbar{} B- \textbar{} \textbar{} 2.7 grade points per course credit
\textbar{} \textbar{} C+ \textbar{} \textbar{} 2.3 grade points per
course credit \textbar{} \textbar{} C \textbar{} Satisfactory, minimum
expectation \textbar{} 2.0 grade points per course credit \textbar{}
\textbar{} \textbar{} credit \textbar{} \textbar{} \textbar{} C-
\textbar{} \textbar{} 1.7 grade points per course credit \textbar{}
\textbar{} D+ \textbar{} \textbar{} 1.3 grade points per course credit
\textbar{} \textbar{} D \textbar{} Passing, below expectation \textbar{}
1.0 grade points per course credit \textbar{} \textbar{} D- \textbar{}
\textbar{} 0.7 grade points per course credit \textbar{}

\subsection{Status Marks}\label{status-marks}

\begin{itemize}
\tightlist
\item
  \textbf{W} Approved withdrawal from a course.
\item
  \textbf{X} Course extends beyond term. An ``X'' status grade is given
  in courses designated in the Catalog as having coursework that extends
  beyond the end of the term. Under no circumstances can an ``X'' grade
  remain unresolved for more than one calendar year.
\item
  \textbf{O} No mark had been reported by the instructor by the time
  academic reports were processed.
\item
  \textbf{I} Incomplete. An ``I'' status grade is reported only for
  students who are unable to complete the work in the course due to
  extenuating circumstances. The normal length of time for resolution of
  an incomplete is within four weeks of the next Fall or Spring term in
  which the student enrolls. If the instructor believes the resolution
  of the incomplete will take longer, the instructor will note the later
  deadline when reporting the incomplete status grade. An unresolved
  incomplete will automatically become a failing grade after the
  deadline, unless the instructor notes otherwise when reporting the
  incomplete. Under no circumstances can an ``I'' grade remain
  unresolved for more than one calendar year.
\item
  \textbf{R} The prefix ``R'' to a grade (i.e., RA, RB, RC, RD, RF)
  indicates a grade of repeated course. A student may repeat a course
  previously taken, and registration must indicate this repeat. Failure
  to register for a repeat course properly results in no recognition of
  the second attempt. Only the grade earned when the course is retaken
  is used in computing the GPA. Credit may be earned only once for a
  given course. Courses may be repeated only once, although, students
  may petition for a second repeat if extenuating circumstances exist.
\item
  \textbf{EQ} Equivalent credit recognized; no credit given. Does not
  increase courses attempted. For a student who has completed four
  course credits of student teaching and who has high school or life
  experience equivalent to a regular catalog course, the said course,
  upon recommendation of the Education chair and the chair of the
  department in which equivalent credit is being recognized, may be
  listed on the student's transcript with the appropriate credit and a
  status mark of ``EQ.''
\end{itemize}

\subsection{Audited Courses}\label{audited-courses}

Students may audit courses with the consent of the instructor. In doing
so, they attend class but are not required to take tests or submit
papers. Audited courses receive no credit and do not appear on
transcripts. Auditors pay a reduced tuition charge.

\subsection{First Course Grading For Non-Traditional
Students}\label{first-course-grading-for-non-traditional-students}

Individuals who have been away from the collegiate routine for several
years may register on an audit basis in their first course, participate
fully in the class (including tests, papers, etc.), and decide at the
conclusion of the course if they wish to pay the other half of the
tuition and receive credit for the course. This policy applies only to
the first course---not to succeeding courses.

\subsection{Pass / Not-Pass Graded
Courses}\label{pass-not-pass-graded-courses}

Courses are graded A--F unless otherwise indicated in the course
description that only P/NP is an option (e.g., student teaching,
internships). Courses graded P/NP count as credits towards graduation
but do not affect the student's GPA. In addition, students may opt to
take up to four A--F graded courses on a P/NP basis. When students opt
for P/NP grading, the Office of the Registrar converts any grade a
faculty member provides of D- or better to a P, and any grade of F to an
NP. In order to count for Writing Emphasis credit, a submitted grade of
C or higher must be earned.

The following regulations apply:

\begin{enumerate}
\def\labelenumi{\arabic{enumi}.}
\tightlist
\item
  Students are permitted to change the method of grading for a course
  from a letter grade to P/NP. This change is allowed from the start of
  the term of enrollment through the last day to withdraw from courses
  during that term. See Academic Calendar Section 3.0.1 for official
  dates.
\item
  Students may elect to take up to four graded courses during their
  undergraduate career on a P/NP basis. However:

  \begin{enumerate}
  \def\labelenumii{\alph{enumii}.}
  \tightlist
  \item
    Students cannot use a course for which they elected P/NP grading to
    satisfy the requirements for a major or minor.
  \item
    Students cannot elect P/NP grading while on academic probation,
    though provisionally-admitted First Year students are allowed the
    option of P/NP grading.
  \end{enumerate}
\item
  The Registrar will not inform the instructor of the student's request
  for P/NP grading and the instructor must provide the Registrar with an
  appropriate letter grade.
\item
  A student's request for P/NP grading may be nullified at any time
  before the end of the third business day after grades are due in the
  student's final term. A written notice must be sent to the Office of
  the Registrar to communicate the student's intent to replace the P or
  NP grade with the instructor's letter grade.
\item
  A student's request to change their method of grading to P/NP in lieu
  of a letter grade counts as one of the four P/NP elected courses
  during the student's undergraduate career, regardless of whether it is
  later nullified.
\item
  Ordinarily a student is limited to one course credit per term on an
  elected P/NP basis. A student wishing to exceed this limitation must
  present a convincing rationale or significant mitigating circumstances
  to the Committee on Petitions.
\end{enumerate}

\subsection{Midterm Grades}\label{midterm-grades}

Midterm grades are not official evaluations and are not part of the
official transcript. The goal of midterm grades is to help students take
responsibility for their academic progress. Faculty submit midterm
grades of A-F for students in all full-term courses. Midterm grades are
not intended to be a guarantee, promise, or contract regarding the final
grade a student will earn in the class. Instead they provide information
for students about their academic performance.

\chapter{CLASS DESIGNATION}\label{class-designation}

Class Designation is determined by the number of course credits earned
following the Fall and Spring Terms.

\begin{itemize}
\tightlist
\item
  First-year student: Fewer than 8.0 course credits earned\\
\item
  Sophomore: 8.0--15.9 course credits earned
\item
  Junior: 16.0--23.9 course credits earned
\item
  Senior: 24.0 or more course credits earned
\end{itemize}

\chapter{REGISTRATION}\label{registration}

Before being allowed to register, students must have been admitted for
study by the Office of Admission, must have settled their account with
the Business Office, and must meet with their Academic Advisor.
Registrations are canceled for failure to pay fees on time.

Students are encouraged to develop a four-year comprehensive educational
plan with the help of their advisor or with other members of the
faculty.

Students who will not graduate during the current academic year register
online during the Spring Term for the Fall Term of the following
academic year during times specified by the Office of the Registrar.
Registration for the Spring and May Terms takes place during the
preceding Fall Term. Once the online registration period is over, all
changes to registration must take place in the Office of the Registrar
using accurate, legible, and completed registration forms. Entering
first-year students and transfer students receive instructions with
their orientation materials, and readmitted students receive
instructions from the Office of the Registrar concerning their
registrations. Registrations are not accepted for a term without
successful petition after the end of the first week of classes of that
term.

\subsection{Changes in Registration}\label{changes-in-registration}

\begin{enumerate}
\def\labelenumi{\arabic{enumi}.}
\tightlist
\item
  Unless a course is registered by a student online during the
  registration period, they must intentionally file a registration form
  with the Office of the Registrar in order to be registered for a
  course. Registration forms for course additions must be approved by
  the student's advisor and instructor of the course.
\item
  Courses may be added or dropped during the first five weeks of the
  Fall, Spring, or Summer Term, including 7-week courses. Students may
  add a May Term course during the first three days of the term. If a
  student needs to add a course after these deadlines due to extenuating
  circumstances, the student must petition the Committee on Petitions by
  completing the appropriate forms.
\end{enumerate}

For courses spanning a fraction of a Term, the last date to add or drop
without a ``W'' will be calculated as a proportionate time frame
comparatively as full-term courses. See Academic Calendar (see p.~12)
for official dates for full-term and half-term courses. 3. A student may
withdraw from one or more courses with the following results: - If a
student withdraws from a course when 2/3 or less of the Term is
completed, a ``W'' (withdrawal) grade will be entered on the student's
permanent record. This grade will not affect the student's GPA. This is
the date listed in the academic calendar as the Last Day to Withdraw
from a course.\\
- For courses spanning a fraction of a Term, the last date to withdraw
with a ``W'' will be calculated using the 2/3 fraction. See Academic
Calendar (p.~15) for official dates for full-term and half-term courses.
- If a student officially withdraws from a course after 2/3 of the Term
is completed, a ``WF'' will be entered on the student's permanent
record. This will affect the student's GPA (see p.~53). - A withdrawn
course, regardless of the date withdrawn, counts as attempted credits on
the transcript. Therefore, it also counts when calculating academic
standing (see p.~61), and satisfactory academic progress (see p.~275)
for financial aid.

\subsection{Course Load}\label{course-load}

A student is full-time for the Fall or Spring Term when enrolled for
three or more course credits. A student enrolled for less than this
course load is a part-time student. For financial aid purposes,
half-time is defined as enrollment in no fewer than two course credits
in each of the Fall and Spring Terms; three-fourths is defined as
enrollment in no fewer than 2.5 course credits in each of the Fall and
Spring Terms. Students who have earned a cumulative GPA of 3.4 or above,
or have earned both 23 credits and a cumulative GPA of 2.5 or above, may
take above 4.99 credits without the need for a petition, but approval
must be secured from the Office of the Registrar. Other students must
submit a petition to the Committee on Petitions and obtain approval in
order to register for five or more course credits. No student shall be
permitted to register for over 5.99 credits in each of the Fall and
Spring Terms. Only petitions from those students who have completed at
least one term as a full-time student will be considered by the
Committee on Petitions. Students may not register for more than one
course credit during May Term.

\chapter{ACADEMIC INTEGRITY POLICY}\label{academic-integrity-policy}

At Coe College, we expect academic integrity of all members of our
community. Academic integrity assumes honesty about the nature of one's
work in all situations. Such honesty is at the heart of the educational
enterprise and is a precondition for intellectual growth. Academic
dishonesty is the willful attempt to misrepresent one's work, cheat,
plagiarize, or impede other students' academic progress. Academic
dishonesty interferes with the mission of the College and will be
treated with the utmost seriousness as a violation of community
standards.

\subsection{Forms of Academic
Dishonesty}\label{forms-of-academic-dishonesty}

\textbf{Cheating} is the attempt to deceive an evaluator by claiming
credit for work one has not done or by knowingly assisting such an
attempt. It includes (but is not limited to) the use of unauthorized
sources of information on in-class or take-home exams, or other
assignments; copying from other students on exams, assignments, or lab
reports; fabrication of data, research, quotations, or other
information; and taking credit for collaborations to which one has not
contributed.

\textbf{Plagiarism} is the use of someone else's words or ideas without
acknowledgement and, when intentional, is a form of academic dishonesty.
The unacknowledged use of words or ideas from any published or
unpublished sources, including Internet resources or other student
papers, constitutes plagiarism. Plagiarism may occur intentionally or
unintentionally through the omission of appropriate citations. Any ideas
or information the student adopts from a source, whether or not directly
quoted, must be acknowledged by specific reference in notes or the text.

\textbf{Any words or phrases} that are taken from a source must be
quoted and cited. Any paraphrase---the restatement of an idea in your
own words---must be cited.

The methods of citation and documentation vary from discipline to
discipline. Students are responsible for determining the appropriate
method for any given assignment or, in the absence of a clearly stated
protocol, using any accepted academic method. Guidelines can be found on
the library website and in the Writing Center.

\textbf{Other forms of academic dishonesty} include (but are not limited
to) deliberately impeding other students' work and misuse of common
academic property, in the libraries, labs, and elsewhere.

\subsection{Sanctions}\label{sanctions}

Instructors have responsibility for determining whether academic
dishonesty has occurred. Instructors shall proceed with sanctions
accordingly. Any act of academic dishonesty that results in one of the
sanctions below shall be detailed in a formal report filed with the
Provost and Dean of the Faculty.

Cases of unintentional plagiarism may be dealt with through educational
procedures such as further assignments requiring the student to practice
documentation and citation methods, or other means determined by the
instructor.

Acts of academic dishonesty will be subject to one or more of the
following sanctions:

\begin{enumerate}
\def\labelenumi{\arabic{enumi}.}
\tightlist
\item
  failure of the assignment, i.e.~exam, paper, lab report, etc.
\item
  failure of the class
\item
  suspension or expulsion
\end{enumerate}

An instructor may impose the first two of these penalties. Suspension or
expulsion may only be carried out by the Provost and Dean of the
Faculty.

Repeated acts of academic dishonesty will result in suspension or
expulsion. When academic dishonesty has been determined to have occurred
a second time, the Provost and Dean of the Faculty shall decide on the
student's status at the College.

\subsection{Procedure}\label{procedure}

When an instance of academic dishonesty is suspected, the instructor
will meet with the student to discuss the incident and will decide
which, if any, of the above sanctions is appropriate.

If warranted, the instructor will send a report to the Provost, with a
copy given to the student, which details the nature of the violation and
the steps taken to address it. The Provost will send a letter to the
student within ten business days of receipt of the faculty member's
report. The letter will reiterate the incident, describe the sanctions,
and inform the student of their right to appeal. The report and letter
from the Provost will remain on file in the Academic Affairs Office
until five years after the student's graduation or severance from Coe.
The Vice President for Student Life will be notified that a report has
been filed. Information in the file will be confidential, to be shared
only at the discretion of the Provost and Dean of the Faculty for a
legitimate educational or legal purpose.

\subsection{Appeals Process}\label{appeals-process}

The student may appeal the charge and/or the sanction within ten
business days of receiving the Provost's letter of notice by emailing
the Provost and Dean of the Faculty requesting an appeals hearing.
Students wishing to appeal are strongly encouraged to consult with the
Director of Academic Achievement, who has been designated by the College
to provide information and advocacy in these matters.

The Provost's Office will convene an Academic Integrity Appeals Board
within ten business days of receipt of the request for appeal to hear
the appeal. The Academic Integrity Appeals Board will be chaired by
either the chair or co-chair of the Academic Policies Committee.
Additional members of the board will include: one additional faculty
member on the Academic Policies Committee, the senior Student Life
officer, one faculty member from the Committee on Admission, Retention,
and Enrollment, and the Associate Dean for Student Academics. ~In any
hearing the Provost may replace the Associate Dean for Student Success,
or other member of the board to avoid conflicts of interest. The student
may choose to have a faculty or staff member present as an observer. The
student and the instructor will each appear as witnesses and each may
request that other pertinent witnesses appear.

A majority vote of the Appeals Board is necessary to uphold or overturn
a sanction. If a sanction is overturned, the Appeals Board may impose a
lesser sanction. The Appeals Board will submit a written finding which
will be sent to the student and the faculty member(s) involved in the
case and which will become part of the student's file. If the appeal's
outcome is to overturn the dishonesty charge, the report in the Academic
Affairs Integrity file will be removed.

\chapter{ACADEMIC STANDING}\label{academic-standing}

All students are expected to meet the College's standards for academic
performance (see chart below). These are in place to keep students on
track towards meeting the graduation requirements of the college (32
credits with a cumulative GPA of 2.0). Students who do not meet Coe's
academic standards may be put on probation or suspended from the
College.

To make sure that students are aware when their academic standing is in
jeopardy, a series of communications are issued throughout an academic
term including D/F notices, academic warnings, and notices of academic
probation. Unless otherwise stated, these communications will be sent to
the student by email and by a letter to the student's Coe mailbox.
Additionally, an email will be sent to the student's advisor.

\subsection{D/F Notices}\label{df-notices}

Academic notices are issued at midterm to students who earn a D, F, or
NP in one or more courses. Students receiving these notices are expected
to meet with their academic advisor to identify appropriate support.

\subsection{Academic Warnings}\label{academic-warnings}

Academic warnings will be issued to students who earn a term grade point
average below 2.00. Students receiving academic warnings are strongly
encouraged to re-evaluate their current approach to their studies and/or
their academic plans. They are expected to meet with their academic
advisor and/or a Learning Commons staff member.

\section{Academic Probation}\label{academic-probation}

The Academic Standing Committee reviews academic records of all students
at the conclusion of both the Fall and Spring Terms. A student is placed
on academic probation if the cumulative GPA falls below the minimum GPA
levels listed below:

\begin{longtable}[]{@{}ll@{}}
\toprule\noalign{}
\textbf{Total Course Credits Attempted} & \textbf{Minimum GPA Levels} \\
\midrule\noalign{}
\endhead
\bottomrule\noalign{}
\endlastfoot
0.0 -- 4.99 & l.50 \\
5.0 -- 7.9 & l.75 \\
8.0 -- 11.9 & 1.80 \\
12.0 -- 15.9 & 1.90 \\
16.0+ & 2.00 \\
\end{longtable}

To return to good standing and be removed from academic probation, a
student shall earn a cumulative GPA greater than the threshold for
academic probation as specified above. If the student's cumulative GPA
decreases at the end of any term on probation, the student is subject to
academic suspension. Students who are placed on academic probation have
a maximum of two (2) consecutive terms, excluding May and Summer, to
return to good standing before they are subject to academic suspension.

Students who are on academic probation: - Are expected to comply with
any requirements outlined by the Academic Standing Committee. - May not
elect P/NP grading in lieu of a letter grade. - May not be excused from
attending class to participate in extra-curricular activities.

The status of academic probation is noted on a student's transcript.

Students on probation who are not meeting the conditions of their
probation and not performing at a passing level in their courses may be
withdrawn from the College during the term on the recommendation of the
Academic Standing Committee. If the student had previously appealed a
suspension, the suspension is reinstated. Otherwise, the student will
exit the College with a leave of absence and is eligible to return on
probation the following term. Students may appeal this decision to the
Provost.

\section{Academic Suspension}\label{academic-suspension}

Students who do not meet the conditions of their academic probation or
who fall below retention thresholds (see below) will be considered for
academic suspension by the Academic Standing Committee, and, if
suspended, will be unable to take courses at Coe College for a period of
at least one academic year. (One academic year is defined as a Fall and
Spring Term, and does not count May or Summer Terms.) A registration
hold preventing registration will be placed on the student's account. A
student already registered for the following term will be dropped from
those courses.

\begin{longtable}[]{@{}ll@{}}
\toprule\noalign{}
\textbf{Total Course Credits Attempted} & \textbf{Minimum GPA Levels} \\
\midrule\noalign{}
\endhead
\bottomrule\noalign{}
\endlastfoot
0.0 -- 4.99 & 0.50 \\
5.0 -- 7.9 & l.00 \\
8.0 -- 11.9 & 1.50 \\
12.0 -- 15.9 & 1.70 \\
16.0 -- 19.9 & 1.80 \\
20.0 -- 23.9 & 1.90 \\
24 + & 2.00 \\
\end{longtable}

Students who are placed on suspension for academic reasons will receive
a letter notifying them of their suspension at their home residence, by
registered mail, as well as a letter in their Coe mailbox and an email
to their Coe email. Suspended students have the right to appeal for
immediate readmission if they can provide evidence of circumstances that
would warrant reconsideration by the deadline indicated in their
suspension letter.

To appeal for immediate readmission:

\begin{itemize}
\tightlist
\item
  A suspended student appealing for readmission must submit an appeals
  letter by mail, email, or delivery in person to the Office of the
  Provost. The letter must include an explanation of any circumstances
  that affected the student's academic performance. The letter should
  provide a detailed plan with specific actions that the student will
  take to improve their academic standing, and explain how the student
  will overcome the obstacles that affected their academic performance.
  Student appeals will not be considered if a written statement is not
  received by the deadline.
\item
  Although not mandatory, the suspended student appealing for
  readmission is urged to schedule to meet with the Academic Standing
  Committee on the date designated in the suspension letter. At this
  meeting the student will have the opportunity to respond to questions
  the Committee may have on the circumstances outlined in the student's
  letter requesting readmission.
\item
  A suspended student appealing for readmission is encouraged to request
  a letter of support for immediate readmission from a faculty advisor
  or other faculty or staff member at Coe, if the letter can shed light
  on the student's ability and motivation to do well in future academic
  endeavors. Similarly, the student is encouraged to consult with one or
  more faculty or staff members to prepare a strong case for
  readmission.
\end{itemize}

Students who have been suspended for academic reasons from the College
once, can return to the College either through a successful appeal for
immediate readmission or through readmission after leaving for at least
one academic year. A student who is suspended for academic reasons more
than once cannot appeal for immediate readmission and must leave the
College for at least one academic year.

\section{Interim Suspension}\label{interim-suspension}

Interim Suspension is a situation where it is determined that a
student's continued presence at the college constitutes an immediate
threat of harm to the student, other individuals, or to the stability
and continuance of normal College functions. The Provost and Dean of
Students or their designee may suspend a student pending disciplinary
proceedings. Such suspension may become effective immediately and
without prior notice.

Interim suspension shall be considered an excused absence until the
conclusion of formal hearings. The student will be offered the
opportunity to make up any academic work missed during the time in which
the interim suspension was imposed. It is the student's responsibility
to make specific arrangements with faculty members to complete academic
work. The Dean of Students will initiate communication with the
appropriate faculty.

\section{Readmission Following
Suspension}\label{readmission-following-suspension}

Coe College's academic suspension policy allows students, who are not
readmitted immediately through appeal, to apply for readmission after at
least one academic year has passed.

To be considered for readmission, the student must submit a letter of
appeal to the Office of the Provost. In the appeals letter, the student
must present evidence that demonstrates how the circumstances that led
to the student's academic suspension have been addressed, and how the
student plans to be successful in his or her academic future.
Readmission is determined by the Academic Standing Committee. The
Academic Standing Committee reserves the right to conduct its own
investigation, review the case, and make a final decision concerning the
student's reinstatement to the College. When appropriate, certain
academic stipulations may be applied. If a suspended student provides
evidence of successful course completion elsewhere and/or written
evidence of motivation and maturity necessary to be academically
successful at Coe, the student may be readmitted on probation at Coe.

\section{Exiting the College}\label{exiting-the-college}

The exit process at Coe College, whether through withdrawing or taking a
leave of absence, is initiated by the student. The exit procedure is
initiated in the Learning Commons (Stewart Memorial Library) and begins
with an interview with the Associate Dean of Student Academics or the
Director of the TRIO-Academic Achievement Program. At the interview, the
student is given an official exit form on which to secure signatures
from the following: the Student Financial Services Office, to verify a
balance due or a credit to be refunded, as well as Student Loan
information, to be aware of financial aid adjustments; the Library, to
ascertain that all materials have been returned; and the Resident
Director of the student's residence hall, to arrange for room checkout.
The completed exit form is returned to the Learning Commons which will
then notify other pertinent areas of the student's withdrawal.

If a student is unable to complete the official withdrawal process, the
intent to withdraw or take a leave of absence can be communicated to one
of the following offices: Registrar, Student Financial Services, Student
Life.

If a student withdraws from all courses during a period of enrollment
for which he or she received financial aid, the Student Financial
Services Office will determine how much, if any, of the student's
financial aid proceeds must be returned to the College, based on a
federally mandated refund formula (see Return of Title IV
Funds/Institutional Refund Policy, p.~268).

Special consideration is given to students who withdrew due to a call to
active duty. Coe's ``Military Call Up/Refund''~and ``Readmission of
Service Member''~policies are published on the Admission/Financial Aid
webpage. Copies of these policies can be requested from the Student
Financial Services Office.

\section{Leave of Absence}\label{leave-of-absence}

A student may find it necessary to interrupt a program of study at the
College. Under this condition, the student may apply for a leave of
absence. A leave of absence may be granted for a period not to exceed 12
months. Students may extend a leave of absence for up to a total of 36
months by contacting the Office of the Registrar via email at
o-registrar@coe.edu. Students that do not renew their Leave of Absence,
or return to Coe, will be officially withdrawn from the College.
Coursework completed while on leave from the College is subject to the
same conditions as work in transfer.

\section{Requesting a leave of
Absence}\label{requesting-a-leave-of-absence}

Students planning to leave of absence from the College should consult
their academic advisor and then they \emph{must declare} their intent to
one of the following designated offices: Student Financial Services,
Registrar, Residence Life, or the Learning Commons. Please refer to the
previous section on exiting the College for the steps that follow.~

\section{On Leave of Absence}\label{on-leave-of-absence}

The Coe community is committed to supporting students while they are
away from the College. Thus, the Associate Dean of Student Academics may
assign a designee who will be the point of contact to each student on a
Leave of Absence. This designee will stay in contact with the student
during their time away from the College, as appropriate, and will assist
with the return process once the student is ready to resume coursework.~

\section{Returning to Coe after a Leave of
Absence}\label{returning-to-coe-after-a-leave-of-absence}

Students planning on returning to the College do not need to apply for
readmission, but must submit a statement of intent to re-enroll to the
Office of the Registrar and the Associate Dean of Student Academics, or
designee, who will assist with the return process. Students are strongly
encouraged to schedule a consultation with the Associate Dean of Student
Academics, or designee, by December 1st for returning the Spring Term,
or August 1st to return the Fall Term. If a student wants to register
during the regular registration time (November for Spring Term or late
March / early April for Fall Term), they will need to submit their
statement of intent and contact the Associate Dean of Student Academics,
or designee, at least two weeks before Registration. The exact date
of~Registration can be found on the Academic Calendar on the Coe website
under the Academics tab.~

\section{Withdrawal from the College and
Readmission}\label{withdrawal-from-the-college-and-readmission}

Admission for work toward a degree terminates and the student is
considered withdrawn from the College if:

\begin{enumerate}
\def\labelenumi{\arabic{enumi}.}
\tightlist
\item
  A full-time student does not enroll at Coe for the next term
  (excluding May Term) and has not completed a Leave of Absence form.
  This does not apply to students in College-approved off-campus study
  programs.
\item
  A part-time student does not enroll for a course at Coe in a 12-month
  period and has not completed a Leave of Absence form.
\end{enumerate}

Students wishing to resume work toward a degree, once admission status
has terminated, must apply for readmission.

Students previously enrolled at Coe and readmitted after an absence of
two years or more may request that all previous work at Coe be
re-evaluated by the Registrar on the same basis as credits offered in
transfer. Re-evaluation means that only courses with grades of C or
better will be counted for credit toward graduation. All courses
affected by the re-evaluation and the grade earned for each course will
remain on the student's permanent record but will not factor into the
cumulative GPA or be counted toward graduation.

\chapter{TRANSCRIPT EVALUATION
POLICIES}\label{transcript-evaluation-policies}

Official transcripts for courses taken at Coe College can only be issued
by the Office of the Registrar and only after the office has received a
written request and payment from the student.

\section{General Policy On Transfer
Credit}\label{general-policy-on-transfer-credit}

To honor its mission and to preserve its academic integrity as a liberal
arts institution, the College accepts a course in transfer for the
equivalent earned credit (4 semester hours = 1 course credit), if that
course meets the spirit of the College's mission and is from a
regionally accredited institution. In cases where it is unclear whether
the course would be acceptable for transfer credit, the Registrar and
the appropriate department chair will consult. Other exceptions are
referred to the Committee on Petitions.

\section{Evaluation Of Credits In
Transfer}\label{evaluation-of-credits-in-transfer}

The Office of the Registrar is responsible for the evaluation of
transfer credit. Credits accepted in transfer do not affect the
cumulative GPA. Grades for the credits accepted are not recorded on
Coe's transcript. Thus, transfer credits increase only the total courses
attempted and the total course credits earned. Credit is not accepted
for coursework earning a grade below ``C'' (2.0 on a 4.0 scale).

\section{Junior or Community College
Credit}\label{junior-or-community-college-credit}

No more than 50\% of the course credits required for a degree at Coe
will be accepted in transfer from 2-year colleges. A maximum of 16
credits will be accepted as transfer credit from 2-year regionally
accredited institutions. Transfer students who complete a regionally
accredited A.A. degree program or a regionally accredited
college-parallel A.S. degree program will be accorded junior status (16
course credits) at Coe. Transfer students who complete an A.A., A.S., or
A.A.S. degree from a regionally accredited institution with which Coe
has a specific articulation agreement will be awarded credit consistent
with that agreement.

\section{European Credit Transfer System
(ECTS)}\label{european-credit-transfer-system-ects}

ECTS credits are a relative rather than an absolute measure of student
workload.~ They specify how much of a year's workload a course unit
represents at the institution or department allocating the credits.~ECTS
is thus based on a full student workload and not limited to contact
hours only. In ECTS, 60 credits represent the workload of a normal
undergraduate academic year of study and normally 30 credits for a
semester and 20 credits for a term. Thus, ECTS credits will~normally~be
transferred to Coe College at a rate of 7.5 ECTS credits: 1 Coe credit.

\section{Occasional Transfer Credit For Degree-Seeking
Students}\label{occasional-transfer-credit-for-degree-seeking-students}

Degree-seeking students sometimes wish to transfer credit from another
institution toward their degree at Coe. Such credit must be approved in
advance of completion of the course by the Registrar. Departments must
approve in advance any courses counting toward a major, a minor, or
teacher certification requirements. Credit from junior or community
colleges is not accepted for students who have junior or higher status
at Coe.

\section{Evaluation of Credits For
Graduation}\label{evaluation-of-credits-for-graduation}

The Office of the Registrar certifies the completion of general degree
(see p.~16) and general education requirements (see p.~23). Credits
toward a major, minor, endorsement, license, authorization, etc. are
approved by the appropriate department chair, administrative
coordinator, or by Academic Policies Committee for interdisciplinary
majors.

\section{Credit By Examination}\label{credit-by-examination}

A maximum of eight course credits in satisfaction of degree requirements
may be applied from credit earned through the Advanced Placement Program
and International Baccalaureate.

\section{Updating Coursework}\label{updating-coursework}

In the natural course of reviewing academic records, a student may be
required to repeat certain courses (or appropriate substitutes) taken
more than four years prior to the review to bring studies in those areas
up to date. Review cases may be brought to the Academic Policies
Committee by any member of the faculty, and this committee makes the
final decision.

\section{Advanced Placement (AP)}\label{advanced-placement-ap}

Coe College's Advanced Placement code is 6101.

Coe College grants college credit for approved Advanced Placement
examination scores of 4 or 5. AP credit cannot be used to fulfill any
general education requirement, including First-Year Seminar, Liberal
Arts Core, Diverse Cultural Perspectives, the writing emphasis
requirement, credit in transfer to reduce the number writing emphasis
courses required, or academic practicum. Each AP exam may earn 1.0
course credit up to a maximum of eight course credits towards
graduation. Credit is granted upon receipt of the results of the
examination, which must be received directly from the Educational
Testing Service. Further information regarding Advanced Placement
examination reporting to the College may be obtained in the Office of
the Registrar and on Coe's website. If an equivalent course is taken at
Coe College, the AP credit is removed from the transcript.

\section{International Baccalaureate
(IB)}\label{international-baccalaureate-ib}

Coe College credit may be awarded for International Baccalaureate work.
Students may earn 1.0 course credit for each higher-level examination
score of 5, 6, or 7 to a maximum of eight course credits. No credit is
granted for standard level examinations. Students may not receive
college credit for both AP and IB in areas of similar content. IB credit
cannot be used to fulfill any general education requirement, including
First-Year Seminar, Liberal Arts Core, Diverse Cultural Perspectives,
the writing emphasis requirement, credit in transfer to reduce the
number of writing emphasis courses required, or academic practicum.
Credit is granted upon receipt of the results of the examination, which
must be received directly from IB. Further information regarding
International Baccalaureate examination reporting to the College may be
obtained in the Office of the Registrar and on Coe's website. If an
equivalent course is taken at Coe College, the IB credit is removed from
the transcript.

\section{Class Attendance}\label{class-attendance}

Regular class attendance is expected, although the instructor of each
course sets the standard expected to be met by the students. The College
expects attendance on all scheduled days, including the first and last
day of a term, as well as the class days immediately preceding and
following College holidays. Students officially representing the College
are excused as necessary prior to the absence. Students on academic
probation are not excused from attending class to participate in
extra-curricular activities.

\section{Final Exams}\label{final-exams}

The final exam schedule is published by the Registrar. It is expected
that final exams will be given during the time scheduled for each
course. There may be extraordinary cases when an individual student has
a compelling reason for taking an exam at a time other than that
scheduled. In such cases the instructor may properly decide to let that
individual take the exam at another time. Having more than two exams
scheduled on one day would justify allowing a student to take a third
exam on another day. The instructor of the course with the highest
course number will move the exam, for that student only, to a mutually
convenient time.

\section{Participation In
Commencement}\label{participation-in-commencement}

Students can participate in Commencement when they have met all the
requirements for graduation for one of Coe's degrees, their financial
obligations to Coe College are met, and they have completed their Intent
to Graduate form and it has been approved. In addition, all students,
unless excused in writing by the Provost, must complete a survey
assessing their educational experience at Coe. All pending graduates
must participate in Commencement exercises unless excused in writing by
the Registrar..

\section{Pending Graduates}\label{pending-graduates}

Students who need to earn no more than 2.0 course credits to complete
all graduation requirements may be permitted to participate in
Commencement exercises as long as the remaining course credits are
scheduled to be completed by August 15 of that year. Pending graduates
will be listed in the Commencement booklet, but since Latin honors (see
Section 13.2.2) are bestowed only after all graduation requirements are
met, they cannot be listed in the Commencement booklet for Latin honors,
although any earned honors will be listed on their diploma and
transcript.

\chapter{STUDENT NOTIFICATION OF RIGHTS UNDER
FERPA}\label{student-notification-of-rights-under-ferpa}

The Family Educational Rights and Privacy Act (FERPA) affords students
certain rights with respect to their education records. They are:

\begin{itemize}
\tightlist
\item
  the right to inspect and review the student's education records within
  45 days of the day the College receives a request for access. Students
  should submit to the Office of the Registrar written requests that
  identify the record(s) they wish to inspect. The Registrar will make
  arrangements for access and notifies the student of the time and place
  where the records may be inspected. If the records are not maintained
  by the Office of the Registrar, the Registrar shall advise the student
  of the correct official to whom the request should be addressed.
\item
  the right to request the amendment of the student's education records
  that the student believes are inaccurate or misleading. Students may
  ask the College to amend records that they believe are inaccurate or
  misleading. They should write the College official responsible for the
  records, clearly identify the part of the record they want changed,
  and specify why it is inaccurate or misleading. FERPA was not intended
  to provide a process to be used to question substantive judgments
  which are correctly recorded. The rights of challenge are not intended
  to allow students to contest, for example, a grade in a course because
  they felt a higher grade should have been assigned. If the College
  decides not to amend the record as requested by the student, the
  College will notify the student of the decision and advise the student
  of his or her right to a hearing regarding the request for amendment.
  Additional information regarding the hearing procedures will be
  provided to the student when notified of the right to a hearing.
\item
  the right to consent to disclosures of personally identifiable
  information contained in the student's education records, except to
  the extent that FERPA authorized disclosure without consent. One
  exception which permits disclosure without consent is disclosure to
  school officials with legitimate educational interests. A school
  official is a person employed by the College in an administrative,
  supervisory, academic or research, or support staff position
  (including law enforcement unit personnel and health staff); a person
  or company with whom the College has contracted (such as an attorney,
  auditor, collection agent, or official of the National Student
  Clearinghouse); or a student serving on an official committee, such as
  an admission, petitions, retention, honors recognition, disciplinary,
  or grievance committee, or assisting another school official in
  performing his or her tasks. A school official has a legitimate
  educational interest if the official needs to review an education
  record in order to fulfill his or her professional responsibility.
\item
  the right to file a complaint with the U.S. Department of Education
  concerning alleged failures by the College to comply with the
  requirements of FERPA.
\end{itemize}

A student is a person who attends or has attended Coe College, as
determined by matriculation and enrollment by the first date of an
academic term. Coe College obtains written permission from the student
before releasing any information from a student's educational record in
most cases. However, as the law allows, on a case-by-case basis,
appropriate parts of a student record may be disclosed, without consent
of the student, to the following parties:

\begin{itemize}
\tightlist
\item
  college employees who have a legitimate need to know.
\item
  persons who need to know in cases of health and safety emergencies.
\item
  accrediting organizations to carry out accrediting functions.
\item
  appropriate parties in connection with financial aid to a student.
\item
  federal, state, and local governmental officials for purposes
  authorized by law.
\item
  individuals who have lawfully obtained court orders or subpoenas.
\item
  organizations conducting educational studies for the College.
\item
  courts during litigation between the College and the student or
  parent.
\item
  victim of crime of violence after final results of a disciplinary
  hearing.
\item
  public after disciplinary proceedings determine student committed
  crime of violence.
\end{itemize}

In many situations, complaints relative to FERPA can be resolved with
the College on an informal basis by contacting the Registrar, in the
lower level of Voorhees Hall.

\textbf{To file a FERPA complaint} with the U.S. Department of
Education, contact the office that administers FERPA at: Family Policy
Compliance Office, U.S. Dept. of Education, 400 Maryland Avenue SW,
Washington, DC 20202-4605.

\part{COURSES OF INSTRUCTION}

\chapter{ACADEMIC PROGRAMS}\label{academic-programs}

\subsection{Asia Term}\label{asia-term-1}

ASC-195 Asian Tonal Languages An introduction to Thai, Vietnamese, and
other tonal Asian languages. Emphasis is on basic communication as well
as the distinguishing features of languages that use tones as part of
their linguistic system. ASC-196 Modern South East Asia An introduction
to several Asian cultures such as Thai, Cambodian, and Vietnamese. This
course varies depending on the field of the supervising Coe faculty
member. ASC-444 Independent Study A student-designed study of some
feature of Asian culture, arranged in consultation with the supervising
Coe faculty member. May be taken more than once for a maximum of 2.0
credits. Prerequisite: consent of instructor.

\subsection{Community-Based Project}\label{community-based-project}

CBP-325 Community-Based Project Supervised work on service projects
proposed by external constituencies such as non-profit institutions,
community agencies, and government organizations. Students learn about
the issues, problems, and techniques associated with developing,
organizing, and participating in projects that address and solve
real-world problems, as well as provide services and benefits to
community and project sponsors. Authorization for the community-based
project is determined by the supervising faculty member. May be taken
more than once. A minimum of 140 hours of work is required. P/NP basis
only.

\subsection{First-Year Seminar}\label{first-year-seminar}

FS-110 First-Year Seminar Required of all first-year students. The
First-Year Seminar introduces students to college-level study with
emphasis on critical thinking, writing, speaking, and research skills.
Faculty offer first-year-only topics courses exploring issues from
multiple perspectives within or across disciplines. Students in all
sections prepare portfolios of their written work and attend a variety
of cultural events on and off campus. Seminar instructors also serve as
the primary academic advisors for first-year students in their
respective sections. First-Year Seminars are writing emphasis and cannot
fulfill any distributional, cultural perspective, or major requirements.

\subsection{World Language}\label{world-language}

FSA-100 Foreign Study Abroad Study abroad during May Term supervised by
a Coe College faculty member, with site visits to places of historical
and cultural interest. When appropriate, may count as a course in the
major. May be taken more than once if offered in different locations.
Prerequisite: consent of instructor. (Offered May Term only) FSA-146
Turkey: History and Culture Study abroad course on the history and
culture of Turkey. An exploration of a unique secular-Muslim society.
Particular historical focus is on Istanbul and ancient cities on Aegean
coast. Prerequisite: consent of instructor. (Offered May Term only)

\subsection{Internships}\label{internships}

INT-494 Interdisciplinary Internship Supervised work or volunteer
experience related to a student's career interests. A minimum of 140
hours on-site or remote work experience is required. Authorization and
evaluation of the course credit for the internship is determined by the
department through which the student is completing the internship. P/NP
basis only. With departmental approval, credit may be applied to a major
only with consent of department chair. Prerequisites: consent of the
Internship Faculty Advisor. INT-499 Summer Internship -- Non-Credit
Bearing\\
Is a supervised summer work or volunteer experience related to a
student's career interests. The internship is not credit-bearing;
however, it can meet the practicum requirement. A minimum of 140 hours
on-site experience is required. The internship must be approved by the
Faculty Internship Advisor prior to registration. Students are required
to abide by the same guidelines as students completing credit-bearing
internships. Satisfactory completion determined by the supervising
faculty member. P/NP basis only. May be taken more than once.
Prerequisites: consent of the department in which the student is
completing the internship and completion of the Internship request form
housed on the College's online platform for internships.

\subsection{New York Term}\label{new-york-term-1}

See description, p.~40 New York Term is offered every other Spring Term
in odd-numbered years. There is an extra fee for New York Term. All Coe
financial aid applies, and students are eligible to apply for additional
financial aid based on the additional costs of the term. NYT-250 Fine
Arts in New York City Consists of five 0.4-credit courses: art, film,
music, theatre, and dance. Students attend approximately 35 concerts,
plays, and dance performances and make frequent visits to museums,
galleries, and artists' studios. Each of these events is accompanied by
discussion and seminar sessions with members of the resident New York
Term faculty. (0.4 course credit for each course. Total of 2.0 course
credits upon completion of the five courses.) NYT-394 Internship in New
York City Investigates a student's career interests through work or
volunteer experience. The internship is supervised by a faculty member
of the relevant department, in consultation with Coe's Center for
Creativity and Careers. P/NP basis only. Normally earns 2.0 credits, but
may be taken for 1.0 credits when combined with NYT-444. NYT-444
Independent Study A plan of study designed by the student in
consultation with the student's faculty advisor, and supervised by the
on-campus faculty advisor. Subject must be particularly appropriate for
study in New York City. Prerequisite: consent of instructor.
(Corequisite: NYT-394 for 1.0 credit.)

\section{Occasional Courses}\label{occasional-courses}

Additional courses, not found in the Catalog, may be offered
occasionally and serve one or more of the following purposes:

\begin{enumerate}
\def\labelenumi{\arabic{enumi}.}
\tightlist
\item
  to provide the opportunity for research, creative, or other scholarly
  activity for an instructor jointly with interested students;
\item
  to explore and develop intellectual pursuits which are attractive to
  members of the faculty;
\item
  to respond to student requests for courses which are distinctive,
  unusual, or meet specific needs;
\item
  to bridge between two or more disciplines or curricular categories;
\item
  to serve as one means of developing and testing a possible permanent
  course.
\end{enumerate}

The courses listed in the following section are ones approved by the
College but without plans of being offered in the next few years. The
approved list of courses includes, but is not limited to, the following:

\begin{itemize}
\tightlist
\item
  CHM-104 Introduction to Forensic Science

  \begin{itemize}
  \tightlist
  \item
    An introduction to all aspects of forensic science from obtaining
    specimens to identifying the criminal with accurate forensic tests.
    The course teaches students a basic understanding of the laboratory
    tests and processes of forensic science.
  \end{itemize}
\item
  CHM-105 Food Chemistry

  \begin{itemize}
  \tightlist
  \item
    An introductory course that introduces chemical concepts in the
    context of cooking. Topics include the makeup, shape, and behavior
    of the four major classes of food molecules, effects of chemical
    structure on physical properties, the role of vitamins and
    nutrients, and basic principles of energy. Three class meetings per
    week.
  \end{itemize}
\item
  CHM-442 Materials Chemistry

  \begin{itemize}
  \tightlist
  \item
    Study of the structure and properties of modern materials, including
    glasses, polymers, metals, semiconductors, and superconductors.
    Mechanical, thermal, optical, magnetic, and electrical properties
    are examined and related to structure. Prerequisite:
    Electromagnetism (PHY-265) or Physical Chemistry I (CHM-341) or
    consent of instructor. Corequisite: Materials Physics and Chemistry
    Laboratory (CHM-442L).
  \end{itemize}
\item
  CHM-442L Materials Chemistry Laboratory

  \begin{itemize}
  \tightlist
  \item
    Measurement of structure/property characteristics of materials using
    a variety of instrumentation. Materials studied include glasses,
    polymers, metals, semiconductors, and superconductors. Prerequisite:
    previous or concurrent registration in Materials Physics and
    Chemistry (CHM-442). Corequisite: Materials Physics and Chemistry
    Laboratory (CHM-442L). (0.2 course credit)
  \end{itemize}
\item
  CHM-471 Advanced Chemistry Laboratory II

  \begin{itemize}
  \tightlist
  \item
    Spectroscopic investigations of chemical systems and applications of
    chemical instrumentation for analysis based on current chemical
    literature. One class period and two laboratories per week.
    Prerequisite: Physical Chemistry I (CHM-341).
  \end{itemize}
\item
  EDU-405 Understanding Early Adolescence

  \begin{itemize}
  \tightlist
  \item
    An extension of the study of human development, focusing on the
    growth and development of the middle school age child. Special
    attention is given to the emotional, physical, and cognitive
    characteristics and needs of middle school age children for teachers
    in grades five through eight.
  \end{itemize}
\item
  EDU-415 Middle School Curriculum and Instruction

  \begin{itemize}
  \tightlist
  \item
    Introduction to the organization, structure, and sequence of
    learning experiences for middle grade students. Addresses such
    issues as curriculum integration, teaching teams, pedagogical
    practices for middle school, and developmental appropriateness
    across the range of school subjects. Prerequisite: Practicum in
    Education (WE) (EDU-215).
  \end{itemize}
\item
  EDU-430 Middle School Social Studies

  \begin{itemize}
  \tightlist
  \item
    Overview of the middle school social studies curriculum. A content
    course with a primary focus on geography and its relation to U.S.
    and world history. Prerequisite: Practicum in Education (WE)
    (EDU-215). (0.5 course credit)
  \end{itemize}
\item
  EDU-440 Middle School Mathematics

  \begin{itemize}
  \tightlist
  \item
    Overview of the middle school mathematics curriculum. A content
    course with a primary focus on algebra, problem solving, and number
    theory. Prerequisite: Practicum in Education (WE) (EDU-215). (0.5
    course credit)
  \end{itemize}
\item
  INT-115 May Term in Southern Africa

  \begin{itemize}
  \tightlist
  \item
    Provides opportunities for students to interact with a wide variety
    of communities in southern Africa, all of which are currently
    stressed by impacts of the HIV/AIDS pandemic and climate change.
    Students learn how communities provide health care delivery,
    nutritional support and access to clean water, which are needed to
    sustain the quality of human life in the region. Experiences provide
    hands-on opportunities for students to assist community change in
    these areas. Prerequisite: consent of instructor. (Offered May Term
    only)
  \end{itemize}
\item
  NUR-255 Topics in Health Care

  \begin{itemize}
  \tightlist
  \item
    Offers selected topics on specific health care and/or nursing
    issues, problems, interventions, and theories. Content varies as
    determined by the instructor. May be taken more than once for
    credit, provided the topics are substantially different.
    Prerequisite: sophomore standing. (Offered on an occasional basis)
  \end{itemize}
\item
  PHY-111 Musical Acoustics

  \begin{itemize}
  \tightlist
  \item
    An exploration of the physical principles involved in the
    production, propagation, and perception of musical sounds. Topics
    include simple vibrating systems, properties of waves, and Fourier
    analysis. The primary emphasis is on musical instruments, including
    the voice, but some consideration is also given to room acoustics
    and human perception of sound. Previous musical experience is
    helpful, but not necessary. This course satisfies the non-lab
    science course requirement.\\
  \end{itemize}
\item
  PHY-112/-112L Holography and Optics \& Laboratory

  \begin{itemize}
  \tightlist
  \item
    The making and understanding of holograms are used as the focus for
    a basic physics course in waves and optics. Includes one two-hour,
    (0.0 course credit) lab per week. This course satisfies the general
    education laboratory science requirement.
  \end{itemize}
\item
  PHY-325/-325L Digital Electronics \& Laboratory

  \begin{itemize}
  \tightlist
  \item
    Integrated circuit devices and their applications: the basic logic
    gates, counters, displays, flip-flops, multiplexers, memories. Some
    acquaintance with DC circuit concepts and with the binary number
    system desirable. Includes one two-hour, (0.0 course credit) lab
    weekly. This course satisfies the general education laboratory
    science requirement.
  \end{itemize}
\item
  REL-240 Intertestamental Literature

  \begin{itemize}
  \tightlist
  \item
    A survey of literature composed by Jews during the Hellenistic and
    early Roman periods essential for understanding the emergence and
    development of Rabbinic Judaism, early Christianity and Islam
    (Apocrypha and Pseudepigrapha, Josephus, Philo and the Dead Sea
    Scrolls). Prerequisite: Introduction to Hebrew Bible (WE) (REL-105)
    or consent of instructor.
  \end{itemize}
\item
  REL-278 Mysticism

  \begin{itemize}
  \tightlist
  \item
    A survey of mystical literature in the world's religious traditions.
    This course also addresses the question of the nature of mystical
    experience as well as that of the relation between the mystical
    element of religion and religion as a whole.
  \end{itemize}
\item
  REL-306 Comparative Religion

  \begin{itemize}
  \tightlist
  \item
    A comparative study of the recurring themes and patterns found in
    various religions, past and present. Particular attention is paid to
    the meaning of religious ritual and myth, and the nature of
    religious experience. This course surveys several of the currently
    most influential theories regarding the nature of religion.
    Prerequisite: Eastern Religions (REL-106), Western Religions (WE)
    (REL-108), or consent of instructor.
  \end{itemize}
\item
  SOC-499 Career Related Independent Investigation

  \begin{itemize}
  \tightlist
  \item
    Investigation of a career opportunity through field placement and
    directed reading. This course does not satisfy any of the
    requirements for a major or minor in sociology. Prerequisite:
    declared major in sociology, second term sophomore standing, or
    consent of department chair.
  \end{itemize}
\end{itemize}

\subsection{Skills Development}\label{skills-development}

SKD-115 Summer Bridge Engages students in a one-week course that takes
place prior to Fall Orientation. Students participate in two
mini-classes taught by college faculty. Each mini-class is followed by
small group discussions regarding course content, learning strategies,
college expectations, etc. Additional workshop sessions are held
throughout Summer Bridge on topics such as: financial literacy, campus
resources, goal setting and academic planning. Summer Bridge also offers
opportunities to make social connections through a variety of informal
and planned activities. P/NP basis only. (0.3 course credit) SKD-120
Concepts of Individualized Learning Provides a weekly engagement with an
academic coach to identify interests and explore strengths, applying
this knowledge to the development of a personal academic plan. Topics
include: learning strategies, self-regulation, personal and professional
growth, goal setting, campus engagement and self-reflection. P/NP basis
only. (0.0 course credit) SKD-125 College Foundations Familiarizes
students with the skills and methods of study that lead to competence in
college coursework. Through self-assessment and reflection, students
determine strategies that increase satisfaction and success in the
college environment. P/NP basis only. (0.3 course credit)\\
SKD-130 Personal Finance for College Students A study of managing
finances and making financial decisions that college students encounter.
Areas of study for this project-oriented course include student loans,
credit cards, savings and investments, cars, living on campus versus
apartment living, savings needed for life immediately following
graduation, and travel. (0.5 course credit) INT-100 Professionalism and
Self Presentation Introduces students to the fundamentals of job-seeking
strategies and professional expectations. Students are guided through
development of professional materials, networking techniques, and
interviewing skills. This course does not satisfy Coe's practicum
requirement.

\subsection{Washington Term}\label{washington-term-1}

WSH-284 Topics in Washington, D.C. Experiential learning, study,
writing, and discussion dealing with various subjects related to the
nation's capital. Examples of recent and proposed topics include Art and
Architecture, Campaigns and Elections, Congressional Relations, and
Politics and Communications. WSH-286 Topics in Washington, D.C.:
Non-Western Perspectives Same as Topics in Washington, D.C. (WSH-284)
except the course focuses on topics related to non-Western cultures.
Examples of recent and proposed topics include Globalization and the
U.S.; and People, Politics and Cultures of the Middle East.\\
WSH-464 Washington Term Internship Seminar Group discussion of
internship experiences. Students are exposed to various research
methodologies, readings and guest speakers for understanding Washington
politics. The goal of the course is to expose students to
generalizations about politics and how their internships are either
confirming or challenging those generalizations. WSH-494 The Washington
Experience Internship with an organization related to national or
international politics in Washington, D.C., supervised by one of the
resident staff of the Washington Term. Students establish learning goals
and prepare a portfolio that documents their learning and places it in
the larger context of the literature on American or international
politics. P/NP basis only. One course credit may be counted toward a
major in political science with consent of department, if credit has not
already been received for Internship in Political Science (POL-494).
(2.0 course credits)

\chapter{DEPARTMENTAL PROGRAMS}\label{departmental-programs}

\section{ACCOUNTING, MANAGERIAL}\label{accounting-managerial}

See \textbf{?@sec-accounting}

\section{ACCOUNTING, PUBLIC}\label{accounting-public}

See \textbf{?@sec-accounting}

\section{African American Studies}\label{african-american-studies}

Rodgers (Administrative Coordinator).

The African American Studies major offers students the opportunity to
study African American culture through an interdisciplinary approach.
Students take courses in areas such as literature, history, and
sociology in order to gain a more complete understanding of the major
figures and movements that helped define the culture.

\subsection{African American Studies
Major}\label{african-american-studies-major}

A major in African American studies requires a minimum cumulative 2.0
GPA in all courses counted toward the major.

\begin{enumerate}
\def\labelenumi{\arabic{enumi}.}
\tightlist
\item
  AAM 107 Intro to African American Studies
\item
  +++MISSING INFO: c.aam/eng137.long +++
\item
  HIS 347 African American History
\item
  One of the following:
\end{enumerate}

\begin{itemize}
\tightlist
\item
  HIS 145 History of United States to 1865
\item
  HIS 155 History of United States since 1865
\item
  HIS 227 American Civil War
\end{itemize}

\begin{enumerate}
\def\labelenumi{\arabic{enumi}.}
\setcounter{enumi}{4}
\tightlist
\item
  Five additional courses, at least three of which are numbered 200 or
  above, chosen from the following: -\textbf{(a)} Elective courses:

  \begin{itemize}
  \tightlist
  \item
    AAM 217 Sport and Black Culture
  \item
    AAM 287 Topics in African American Studies
  \item
    AAM 367 Topics in AfricanAmericanLiterature
  \item
    AAM 387 Adv Topics in African American Stud
  \item
    +++MISSING INFO: c.aam447/457.long +++
  \item
    AAM 467 Seminar in African American Lit
  \item
    AAM 494 Internship in African American Stds
  \item
    COM 236 Intercultural Communication
  \item
    COM 357 Sex, Race, \& Gender in Media
  \item
    EDU 187 Human Relations
  \item
    HIS 145 History of United States to 1865
  \item
    HIS 155 History of United States since 1865
  \item
    HIS 227 American Civil War
  \item
    MU 157 Introduction to Jazz History
  \item
    SOC 247 Sociology of Race
  \end{itemize}
\end{enumerate}

-\textbf{(b)} Elective courses (when topic is appropriate): These
courses can count toward the major or minor as determined by the African
American Studies administrative coordinator. - ARH 297 Topics in Art
History: US Pluralism - COM 157 Introduction to Media Analysis - COM 161
Visual Rhetoric - COM 361 Communication \& Social Change - ENG 107
Exploring Literature:US Pluralism - ENG 207 Gender \& Lit:US Pluralism -
ENG 347 Study in Modern or Contemp Amer Lit - ENG 394 Directed Learning
in English - HIS 297 Women in America - HIS 444 Ind Study-History - HIS
472 Seminar American History I - HIS 473 Seminar American History II -
SOC 237 Topics in Sociology:U S Pluralism - SOC 464 Capstone Seminar in
Sociology - THE 488 Special Topics in THE/ FLM

\begin{enumerate}
\def\labelenumi{\arabic{enumi}.}
\setcounter{enumi}{5}
\tightlist
\item
  AAM 444 Ind Study-Afr-Am St (completed during the senior year)
\end{enumerate}

\subsection{African American Studies
Minor}\label{african-american-studies-minor}

\begin{enumerate}
\def\labelenumi{\arabic{enumi}.}
\tightlist
\item
  AAM 107 Intro to African American Studies\\
\item
  +++MISSING INFO: c.aam/eng137.long +++
\item
  HIS 347 African American History\\
\item
  One of the following:
\end{enumerate}

\begin{itemize}
\tightlist
\item
  HIS 145 History of United States to 1865
\item
  HIS 155 History of United States since 1865
\item
  HIS 227 American Civil War
\end{itemize}

\begin{enumerate}
\def\labelenumi{\arabic{enumi}.}
\setcounter{enumi}{4}
\tightlist
\item
  Two additional courses from either list \textbf{5a} or, when
  appropriate, \textbf{5b} as listed above for the major
\end{enumerate}

\subsection{COURSES IN AFRICAN AMERICAN
STUDIES}\label{courses-in-african-american-studies}

\begin{itemize}
\tightlist
\item
  \textbf{AAM 107 Intro to African American Studies} What is African
  American culture, and what accounts for its cultural distinctions?
  This course introduces students to the study of African American
  culture and the field of African American Studies. through an
  interdiciplinary approach (literature, history, music, art and film)
  students will examine central themes and key debates pertinent to
  African American culture and history from its beginning to present.
\item
  \textbf{AAM 137 African American Literature} Reading and discussion of
  the writings of African Americans, with emphasis on the 20th century.
  May include some relevant writings on African Americans by other
  groups. Study of the artistic values and of the social and cultural
  significance of these writings. May be taken more than once, with
  consent of African American Studies administrative coordinator,
  provided the topics are substantially different.
\item
  \textbf{AAM 217 Sport and Black Culture} Examines through cultural
  analysis the complex relationships between sport and Black culture.~
  This course addresses the way sport has evolved from being merely a
  physical activity to a cultural expression in Black communities.~ This
  course emphasizes the historical patterns and current conditions of
  Blacks; participation in sport through various articles, videos, and
  books.~ The course also examines how many Black people have used sport
  as a means of resistance, survival, and social mobility. Students
  learn to analyze cultural expression, to understand race and its
  continuing impact in American life, and to understand how various
  sports pursuits by Black athletes are invested with multiple
  meanings.~Major topics and themes covered include: the concept of
  race, Black culture, the historical presence of Black athletes in
  sport, their current impact, and their dominance in certain sports.
\item
  \textbf{AAM 227 Blackness \& Identity in America} Gathers a wide range
  of scholarship about race and identity to explore what race is, why it
  matters, racial dynamics in organizations, and how best to address
  them. Students will explore questions of identity, privilege,
  ethnicity, gender, and class. In this course, ``race'' is reviewed as
  a shorthand for the interconnected complexity of race, ethnicity,
  culture, and color, and will be carefully analyzed to distinguish
  among such terms and ideas. Students will critically analyze the
  social construct and the popular understandings of race and identity
  as reinforced through cultural institutions. Students will understand
  and evaluate the foundational concepts and theories of race and
  identity and synthesize their knowledge through research,
  presentation, and writing.
\item
  \textbf{AAM 287 Topics in African American Studies} Examines an
  important theme or subject specific to African American experiences
  and culture. Content varies and is determined by the instructor.
  Students learn to understand African American experiences in context,
  to analyze texts and events from multiple disciplinary perspectives,
  and to write in clear, analytical prose. May be taken more than once
  for credit, provided the topics are substantially different.
\item
  \textbf{AAM 367 Topics in AfricanAmericanLiterature} None
\item
  \textbf{AAM 387 Adv Topics in African American Stud} Examines an
  important theme or subject specific to African American experiences
  and culture. Content varies and is determined by the instructor.
  Students learn to understand African American experiences in context,
  to analyze texts and events from multiple disciplinary perspectives,
  and to write in clear, analytical prose. May be taken more than once
  for credit, provided the topics are substantially different.
  Prerequisite: Introduction to African American Studies (AAM-107) or
  consent of instructor.
\item
  \textbf{AAM 444 Ind Study-Afr-Am St} Study of individually chosen
  research topics in African American studies under the direction of a
  faculty member in the area. May be taken for an X status grade with
  consent of instructor prior to registration. Prerequisites:
  Introduction to African American Studies (AAM-107) or African American
  Literature (AAM-137); consent of African American Studies
  administrative coordinator and submission of a written proposal for a
  project.
\item
  \textbf{+++MISSING INFO: c.aam447/457.long +++} +++MISSING INFO:
  c.aam447/457.desc +++
\item
  \textbf{AAM 467 Seminar in African American Lit} Intensive study of
  selected works and subjects in African American literature. May be
  taken more than once, with consent of African American Studies
  administrative coordinator, provided the topics are substantially
  different. Prerequisites: junior standing and The Art of Literary
  Research (ENG-301). May be taken more than once.
\item
  \textbf{AAM 494 Internship in African American Stds} None
\end{itemize}

\section{ANTHROPOLOGY (Minor Only)}\label{anthropology-minor-only}

Fairbanks, Ziskowski

\subsection{Anthropology Minor}\label{anthropology-minor}

\begin{enumerate}
\def\labelenumi{\arabic{enumi}.}
\tightlist
\item
  \textbf{One} of the following:
\end{enumerate}

\begin{itemize}
\tightlist
\item
  ANT 109 Intro to Archaeology:Method \& Thry
\item
  ANT 116 Cultural Anthropology
\end{itemize}

\begin{enumerate}
\def\labelenumi{\arabic{enumi}.}
\setcounter{enumi}{1}
\tightlist
\item
  ANT 215 Ethnographic Methods
\item
  \textbf{One} of the following:
\end{enumerate}

\begin{itemize}
\tightlist
\item
  ANT 284 Topics in Anthropology/Archaeology
\item
  ANT 286 Topics in Anthropology:NWP
\item
  ANT 288 Topics Anthropology/Archaeology:DWP
\end{itemize}

\begin{enumerate}
\def\labelenumi{\arabic{enumi}.}
\setcounter{enumi}{3}
\tightlist
\item
  ANT 450 Anthropological Theory
\item
  \textbf{One} of the following:
\end{enumerate}

\begin{itemize}
\tightlist
\item
  ANT 484 AdvTop: Anthropology or Archaeology
\item
  ANT 486 Advanced Topics in Anthropology:NWP
\item
  ANT 488 Adv Top Anthro/Archaeo:DWP
\end{itemize}

\begin{enumerate}
\def\labelenumi{\arabic{enumi}.}
\setcounter{enumi}{5}
\tightlist
\item
  One additional anthropology course Students must take at least one
  course focused on archaeology and one focused on anthropology. The
  following courses do \textbf{not} satisfy any of the requirements for
  a minor in anthropology:
\end{enumerate}

\begin{itemize}
\tightlist
\item
  ANT 444 Independent Study: Anthropology
\item
  ANT 474 Research Participation:Anthropology
\item
  ANT 494 Internship in Anthropolgy
\end{itemize}

\subsection{Courses in Anthropology}\label{courses-in-anthropology}

\begin{itemize}
\tightlist
\item
  \textbf{ANT 109 Intro to Archaeology:Method \& Thry} An introduction
  to the theoretical approaches and field methodologies of archaeology.
  The goal of this class is to familiarize the student with the history
  and theoretical frameworks of archaeology, in addition to the
  scientific methods with which material culture is collected,
  investigated, and evaluated. Major topics and themes covered in this
  class include: archaeological theory, excavation and survey, artifact
  analysis, death, social systems, economy, religion, and ethical
  practices in cultural heritage management.
\item
  \textbf{ANT 115 Ancient Greece:Hist as Archaeology} Explores the
  various ways in which archaeology can inform our understanding of
  ancient Greek history by visiting the monuments, museums, and
  archaeological sites in modern Greece. The course emphasizes the
  combination of the historical textual evidence and the ancient
  physical evidence to enrich our understanding of Greek social,
  political, and cultural history. THe course also involves an
  examination of the evolution of ancient institutions and practices
  that range from the rise of the Athenian democracy, to the first
  theatrical productions of comedy and tragedy, to the establishment of
  the ancient Olympics. Archaeological theory, archaeiligical practice,
  and historical process constitute the nexus around which the course
  revolves. (Offered May Term only)
\item
  \textbf{ANT 116 Cultural Anthropology} An introduction to cultural
  anthropology, presenting its place within the broader discipline of
  anthropology and outlining its characteristic methodological and
  theoretical approaches to the study of human life. The course
  emphasizes the diversity of approaches to common human experiences in
  a variety of cultural contexts.
\item
  \textbf{ANT 125 Art \& Archaeology Classical World} A survey of the
  art, architecture, and archaeological remains of the Greek and Roman
  civilizations from early Aegean Bronze Age cultures to the fall of the
  Roman empire.
\item
  \textbf{ANT 215 Ethnographic Methods} An introduction to ethnographic
  research methods, presenting a number of tools for collecting and
  analyzing ethnographic data. The course emphasizes the relationship
  between research questions and the methodological tools used by
  anthropologists to study those questions. Prerequisite: Cultural
  Anthropology (ANT-116) or consent of instructor.
\item
  \textbf{ANT 284 Topics in Anthropology/Archaeology} A focused
  examination of an anthropological theme, theory or research method.
  Content varies and is determined by the instructor. May be taken more
  than once for credit, provided the topics are substantially different.
\item
  \textbf{ANT 286 Topics in Anthropology:NWP} A focused examination of
  an anthropological theme, theory or research method. Content varies
  and is determined by the instructor. May be taken more than once for
  credit, provided the topics are substantially different.
\item
  \textbf{ANT 288 Topics Anthropology/Archaeology:DWP} A focused
  examination of an archaeological theme, theory or research method.
  Content varies and is determined by the instructor. May be taken more
  than once for credit, provided the topics are substantially different.
\item
  \textbf{ANT 415 Ancient Greek Pottery Studies} Considers Greek
  ceramics from both scientific and iconological perspectives. Students
  study issues of production, construction and distribution and then
  look at Greek pottery iconographically, focusing on the major
  stylistic periods but emphasizing regional variations through Greece.
  Discussions of connoisseurship and themes prevalent in Greek
  vase-painting such as death, myth, and gender round out the course's
  survey of evidence. Prerequisite: consent of instructor.
\item
  \textbf{ANT 444 Independent Study: Anthropology} Independent study
  under faculty guidance of a research problem chosen by the student.
  May be taken for an X status grade with consent of instructor prior to
  registration. This course does not satisfy any of the requirements for
  a minor in anthropology. Prerequisites: Previous or concurrent
  enrollment in Anthropology Theory (ANT-450) and consent of department
  chair. (Offered by arrangement)
\item
  \textbf{ANT 450 Anthropological Theory} An examination of the
  historical development of anthropological theory, emphasizing
  American, British and French traditions from the 19th century through
  the present. Prerequisite: Cultural Anthropology (ANT-116) or consent
  of instructor.
\item
  \textbf{ANT 474 Research Participation:Anthropology} Individual or
  group investigation with a faculty member on a research topic or
  topics of mutual interest. The student must obtain approval for a
  specific project and make necessary arrangements prior to the term of
  registration for the course. May be taken for an X status grade with
  consent of instructor prior to registration. This course does not
  satisfy any of the requirements for a minor in anthropology.
  Prerequisites: Cultural Anthropology (ANT-116) and consent of the
  instructor. (Offered by arrangement)
\item
  \textbf{ANT 484 AdvTop: Anthropology or Archaeology} Same as ANT-284,
  except at an advanced level. Prerequisite: Introduction to Archaeology
  (ANT-109) or Cultural Anthropology (ANT-116) or consent of instructor.
\item
  \textbf{ANT 486 Advanced Topics in Anthropology:NWP} Same as ANT-286,
  except at an advanced level. Prerequisite: Cultural Anthropology
  (ANT-116) or consent of instructor.
\item
  \textbf{ANT 488 Adv Top Anthro/Archaeo:DWP} Same as ANT-228, except at
  an advanced level. Prerequisite: Introduction to Archaeology (ANT-109)
  or Cultural Anthropology (ANT-116) or consent of instructor.
\item
  \textbf{ANT 494 Internship in Anthropolgy} Placement with a
  career-related organization. A minimum of 140 hours on-site experience
  is required. P/NP basis only. This course does not satisfy any of the
  requirements for a minor in anthropology. Prerequisites: declared
  minor in anthropology, junior standing and consent of department
  chair. (Offered by arrangement)
\end{itemize}

\section{ART AND ART HISTORY}\label{art-and-art-history}

Cohen, Goodson, Knight-Lueth, Rogers (Chair), Thompson.

A major in art or art history is designed to provide a foundation in
theory and practice. Students focus on making and meaning, and gain an
understanding of visual and contextual history and analysis. In studying
the formal, technical and conceptual aspects of art and art history,
students learn to communicate effectively to broad and diverse
audiences.

An art major may also complete an art history minor, but only two
courses may count toward both the major and the minor.

\subsection{Art Major}\label{art-major}

A grade of ``C'' (2.0) or higher must be earned in all courses counted
toward a major in art. 1. ARH-128 Introduction to Art History 2. Two
100-level ART- courses 3. Four of the following: - ART 201 Focus Course:
Creative Process - ART 202 Focus Course: Portfolio Development - ART 203
Focus: ContemporaryArtisticPractice - ART 211 Focus:ArtistStatements -
ART 212 Focus Course: Artist Websites - ART 213 Focus Course: Critique -
ART 291 Focus Course: Topics in Studio Art

The Focus Course requirement is considered a coherent set of experiences
that can be tailored to student needs. The completion of four Focus
Courses is considered as a single course credit with respect to
Graduation Requirements (see Catalog, p.~16).

\begin{enumerate}
\def\labelenumi{\arabic{enumi}.}
\setcounter{enumi}{3}
\tightlist
\item
  Two 300-level ART- Courses
\item
  One additional 200-level Art History course
\item
  One of the following:
\end{enumerate}

\begin{itemize}
\tightlist
\item
  ARH 307 Modern and Contemporary Art
\item
  ARH 310 Contemporary Art, Theory \&Criticism
\end{itemize}

\begin{enumerate}
\def\labelenumi{\arabic{enumi}.}
\setcounter{enumi}{6}
\item
  Junior Review A studio portfolio presentation for art majors that is
  assessed by the art and art history faculty. Students must complete 20
  course credits to be eligible for the junior review, and it must be
  completed prior to enrolling in ART-464 Senior Seminar I.
\item
  ART 464 Senior Seminar I
\item
  ART 474 Senior Seminar II \& Senior Exhibit
\end{enumerate}

In addition to the course requirements, a student must participate in a
Junior Review with the art and art history department's faculty. This
must be completed no later than April of the junior year. The material
submitted at the Review will be related to the senior graduation
requirement, which is a demonstration of proficiency through a public
exhibition of artwork, including portfolio of slides, exhibit
announcement, and résumé.

\textbf{NOTE}: Scheduling may be planned to allow at least one term of
study off-campus at one of the approved foreign or domestic programs.
With departmental approval, courses from these programs may count as 100
or 200 level courses.

\subsection{Art Minor}\label{art-minor}

A grade of ``C'' (2.0) or higher must be earned in all courses counted
toward a minor in art. 1. ARH 128 Introduction to Art History 2. Two
100-level ART- courses 3. Two 300-level ART- courses 4. One 200-level
ARH- course

Further information for all these requirements and programs is available
from the Art and Art History Department.

\subsection{Courses in Art}\label{courses-in-art}

\textbf{100-level courses} (except ART-130, see below) are open to any
student and may offer the opportunity to explore two or more areas of
art making in a studio environment. ART-130 Art in the Elementary
Classroom does not satisfy any of the requirements for a major or minor
in Art or Art History. It also does not count as a 100-level art course
needed for some courses as a prerequisite.

\begin{itemize}
\tightlist
\item
  \textbf{ART101 Art Appreciation} None
\item
  \textbf{ART 102 Sculpture: Material Investigations} Focuses on how
  material choices inform meaning. Students will use a variety of
  traditional and non-traditional materials to make sculptural art
  objects.
\item
  \textbf{ART 103 Ceramic Sculpture} Provides instruction on the
  creation of hand-built ceramic sculptures. Projects will utilize
  pinch, coil building, soft and stiff slab construction, and additive
  and subtractive processes.
\item
  \textbf{ART 105 Narrative Ceramics} Studies the history and evolution
  of ceramics as a storytelling medium. Projects will include
  tile-making, coil-building, surface decoration, and sculptural
  techniques as a means to explore how artists have used clay to create
  compelling narratives.
\item
  \textbf{ART 115 Drawing} Introduces a variety of drawing media,
  including graphite, charcoal, brush and ink, pastels, and collage.
  Students are challenged to observe the visual world around them and to
  respond to that observation with interpretive choices in mark-making
  and composition to produce visual expression. Media may include
  graphite, charcoal, brush and ink, pastels, and collage.
\item
  \textbf{ART 125 Painting} Introduces paint media to represent,
  amplify, and interpret the world. Students are also encouraged to
  discover the possibilities of color, shape, texture and mark-making.
\item
  \textbf{ART 130 Art in the Elementary Classroom} Provides an overview
  of the role of art in the elementary curriculum. Students learn to
  incorporate art activities into other content areas and gain an
  understanding of the objective of elementary classes taught by
  elementary art specialist teachers. This course does not satisfy the
  College's general education fine arts core group requirement nor does
  it satisfy major or minor requirements or prerequisites for Art or Art
  History. Prerequisite: admission to the Education Program or approval
  of the education department. (0.5 course credit)
\item
  \textbf{ART 131 Scultpure: Do, Undo, Redo} Practices the inherent
  creative aspects of making, unmaking and remaking in response to these
  various states of transformation. Artworks are made using clay and
  other sculptural materials while documenting the exploratory processes
  through drawing, photography and more.
\item
  \textbf{ART 135 Ceramics: Form, Function \& Meaning} Focuses on the
  creation of wheel thrown and handbuilt objects within the context of a
  social and cultural lens. Explores the relationship between making and
  meaning in the 21st century. Projects focus on fuctional work and its
  uses.
\item
  \textbf{ART 145 Digital Studio} Explores contemporary digital imaging
  and design. Students produce a series of studio projects ranging from
  digital photography, collage, gif animation, vector graphics, and
  two-dimensional design. Includes introductory-level instruction of
  digital cameras and Adobe Creative Suite: Lightroom, Photoshop,
  Illustrator.
\item
  \textbf{ART 150 Time Based Media} Explores the aesthetic and
  experiential qualities of time. The course includes readings,
  discussions, and screenings of historical and contemporary works plus
  hands-on studio projects using video, sound, performance,
  installation, and more. Includes introductory-level instruction in
  digital cameras and Adobe Lightroom and Premiere Pro. Through media
  production, students cultivate a range of technical skills plus a
  critical understanding of media culture.
\item
  \textbf{ART 151 Layers of Meaning} Pursues meaning through
  multiplicity of imagery. Some work may be done in collage, but other
  materials and techniques, such as drawing, painting, and digital art,
  are used to juxtapose and layer imagery and meaning.
\item
  \textbf{ART 155 Photography: Light Writing} Examines the ways that a
  photographic image can be viewed and interpreted. The course includes
  readings and research on the masters of photography. Technical skills
  include historic black and white analog photography including manual
  camera controls, film and print processing.
\item
  \textbf{ART 175 Printmaking} Focuses on traditional and contemporary
  printmaking techniques, introducing students to the fundamentals of
  materials including paper, ink, presses, and image processing. This
  course explores the development of technical, compositional, and
  conceptual skills through the evolution of printmaking's history of
  multiplicity.
\item
  \textbf{ART 191 Topics in Studio Art} Focuses on a specific theme or
  topic. Topics vary. May be taken more than once for credit, provided
  the topics are substantially different.
\end{itemize}

\textbf{Focus Courses} Focus Courses are seven-week courses designed to
prepare the student for a serious artistic practice.

ART-201 Focus Course: Creative Process Applies the tools and methods of
creative processes. An abbreviated course offered seven weeks of a term
designed to prepare the student for a serious artistic practice.
Students propose, iterate, and produce an object or set of objects in an
exploration of their own creative process. Prerequisite: two ART-
courses (each 1.0 course credit) or consent of instructor. (0.25 course
credit) ART-202 Focus Course: Portfolio Development Prepares students
for developing a portfolio that reflects the breadth and skills of their
own artistic practice and production. Prerequisite: two ART- courses
(each 1.0 course credit) or consent of instructor. (0.25 course credit)
ART-203 Focus Course: Contemporary Artistic Practice Highlights the art,
writings, routines and habits of contemporary artists and practitioners.
An abbreviated course offered seven weeks of a term designed to prepare
the student for a serious artistic practice. Prerequisite: two ART-
courses (each 1.0 course credit) or consent of instructor. (0.25 course
credit) ART-211 Focus Course: Artist Statements/Artist Talks Guides
students through the construction of artist statements and artist talks
to present their work to a wider public. An abbreviated course offered
seven weeks of a term designed to prepare the student for a serious
artistic practice. Prerequisite: two ART- courses (each 1.0 course
credit) or consent of instructor. (0.25 course credit) ART-212 Focus
Course: Artist Websites Guides students through best practices in the
process of developing and maintaining a professional website. An
abbreviated course offered seven weeks of a term designed to prepare the
student for a serious artistic practice. Prerequisite: two ART- courses
(each 1.0 course credit) or consent of instructor. (0.25 course credit)
ART-213 Focus Course: Critique Offers students the opportunity to
practice and improve critique skills. An abbreviated course offered
seven weeks of a term designed to prepare the student for a serious
artistic practice. Prerequisite: two ART- courses (each 1.0 course
credit) or consent of instructor. (0.25 course credit) ART-291 Focus
Course: Topics in Studio Art Focuses on a specific theme or topic. An
abbreviated course offered seven weeks of a term designed to prepare the
student for a serious artistic practice. Topics vary. May be taken more
than once for credit, provided the topics are substantially different.
Prerequisite: two ART- courses (each 1.0 course credit) or consent of
instructor. (0.25 course credit)

\textbf{300-level courses} offer students the opportunity to expand
their technical and conceptual skills at a more advanced level. Courses
numbered ART 300--349 are offered without prerequisite and open to any
student. Courses numbered ART 350--399 have one or more prerequisites.

ART-301 Socially Engaged Art Invites collaboration with individuals,
communities, and institutions in the creation of participatory art. The
genre explores social forms such as dinner parties, conversations, and
projects that intervene or intersect with real-world systems. Coursework
is done collaboratively and independently on projects that critically
engage with contemporary issues and explore a variety of contexts. A
variety of media will be used as a means to define interests and inform
social interventions. ART-313 Color and Design Examines the theory and
practice of color, with emphasis on the use of color as a compositional
element. ART-315 Installation Art Creates art environments that offer a
unique experience for the viewer. Projects will include research,
planning, drafting, and exhibiting artistic creations in spaces across
campus and throughout the local community. ART- 325 Contemporary
Photographic Genres Creates, researches, and analyzes contemporary
digital photography. Readings and writings support the weekly production
of imagery. Technical skills include camera functions, Adobe Photoshop,
and Adobe Lightroom. Cameras and software provided. ART-328 Art and
Industry Explores the relationship between art, mass production and the
steady, habitual effort of making. Topics of inquiry include the Arts
and Crafts Movement, the birth of industry and technology, and how 20th
and 21st-century artists have responded to the frenzy of mass production
and consumerism. Artworks are created using varying methods of mass
production such as tile making, mold making and surface decoration
techniques as a way to explore repetitive modes of making. ART-330
Methods of Teaching Art K--12 Includes discussion, lecture, and studio
work on campus, as well as experience in public school classes off
campus. This course does not satisfy the College's general education
fine arts core group requirement nor does it satisfy major or minor
requirements or prerequisites for Art of Art History. Prerequisite: art
major or minor and admission to the Education Program. ART-331 Open
Studio Provides an opportunity for independent work in the studio with a
midterm and final critique. P/NP basis only. May be taken more than
once. This course does not satisfy any of the requirements for a major
or minor in art. Prerequisites: Ceramics: Form, Function, \& Meaning
(ART-135) or consent of instructor. ART-352 Mark Making Offers
instruction on materials and techniques that involve mark making. These
may include painting and drawing on various surfaces, or digital
painting. Through guided projects and independent work, students explore
the connections between form and expression, with the aim of developing
a visual language that is uniquely their own. Prerequisite: any
100-level ART- course. ART 360 Advanced 3D Focuses on the conceptual,
aesthetic, and technical skills necessary to create more advanced and
sophisticated artworks. Designed to build upon the foundations provided
in beginning-level Ceramics and Sculpture classes. Prerequisite: any
100-level ART course. ART-361 Documentary Explores theory and practice
of documentary filmmaking through readings, research projects,
screenings, and in-class discussions. Students produce studio projects
ranging from short-form documentaries, podcasts, still photography
portfolios, and more. Prerequisite: Digital Studio (ART-145), Time-Based
Media (ART-150), or with permission by instructor. ART-363 Graphic
Design Studio Focuses on graphic design and the communication arts
industry. Projects use traditional and digital tools, materials and
procedures with a focus on finding creative visual solutions to
communication problems. Prerequisite: Digital Studio (ART-145), or
Workshop: Digital Toolbox (WKS-204), or Workshop: Vector Graphics
(WKS-213), or consent of instructor. ART-364 The Human Form Focuses on
the observation and interpretation of the human form. Working from life,
students develop skills in capturing the gesture and form of the body
and use those skills to create interpretive and expressive artworks,
principally using drawing materials, but also exploring with paint and
other materials. Prerequisite any 100-level ART- course. ART-370 Video
Art and Production Explores the theoretical and technical foundations of
video as a visual art medium. Students produce a series of short video
projects that are presented during formal critiques, enabling students
to cultivate meaningful dialogue about their work. Through production,
students develop the technical and professional experience needed to
enter the media industry. Prerequisite: Digital Studio (ART-145),
Time-Based Media (ART-150), or with permission by instructor. ART-371
Typography and Design Investigates the history, theory and practice of
letterforms and typography in graphics, advertising, design and visual
communication. Projects address principles of typography, letter
structure, typeface selection, fundamentals of computer typesetting, and
typographic layout. Prerequisite: Digital Studio (ART-145), or Workshop:
Digital Toolbox (WKS-204), or Workshop: Vector Graphics (WKS-213), or
consent of instructor. ART-373 Screen Printing Utilizes silkscreen
printing as a medium that can be integrated with photography, digital
imagery, and three-dimensional objects. Course offers an exploration of
the formal elements of design with an emphasis on the use of color.
Coursework includes studio production, lectures, demonstrations and
critiques. Prerequisite any 100-level ART- course. ART-374 Multiples in
Printmaking Utilizes intaglio and lithography as a medium to explore
artistic vision, personal imagery and design. Using the process of
intaglio, students investigate the states of development of an image by
printing multiple variations of the plate. Using the process of
lithography, students learn how to create a limited edition of prints.
Prerequisite any 100-level studio ART- course. ART-391 Advanced Topics
in Studio Art See also Art History (ARH-391), p.~92 Focuses on a
specific advanced studio art theme or topic. Topics vary. May be taken
more than once for credit, provided the topics are substantially
different. Prerequisite: any 100- or 200-level ART- course or consent of
instructor. ART-394 Directed Studies in Art Investigates a topic in
studio art selected by the student and instructor to fit the student's
particular interests and educational needs. May be taken more than once
for credit. Prerequisites: junior standing and consent of department
chair.

\textbf{400-level courses}

ART-444 Independent Study Focuses on a topic for independent work on a
selected project under the direction of a faculty member of the
department. Prerequisite: consent of instructor. ART-464 Senior Seminar
I Prepares students for advanced research in studio art. Emphasis is on
preparation of work toward the senior exhibit. Only art majors are
admitted to this course. Materials fee (where applicable) should be
discussed with instructor. Prerequisite: declared major in art and
successful completion of Junior Review. ART-474 Senior Seminar II \&
Senior Exhibition Prepares students for advanced research in studio art.
Emphasis is on preparation of work toward the senior exhibit. Only art
majors are admitted to this course. Materials fee (where applicable)
should be discussed with instructor. Prerequisite: successful completion
of Senior Seminar I (ART-464). ART-494 Internship in Art Investigates an
area of interest related to the major, through voluntary field placement
supervised by a faculty member of the art and art history department. A
minimum of 140 hours on-site experience is required. P/NP basis only.
Prerequisites: declared major in art, junior standing, and consent of
department chair.

\section{Art History}\label{art-history}

\subsection{Art History Major}\label{art-history-major}

A grade of ``C'' (2.0) or higher must be earned in all courses counted
toward a major in art history. An Art History major may also complete an
Art minor, but only two courses may count toward both the major and the
minor.

\begin{enumerate}
\def\labelenumi{\arabic{enumi}.}
\tightlist
\item
  ARH 128 Introduction to Art History\\
\item
  Two 200-level ARH courses\\
\item
  Two additional courses approved by the department, chosen from the
  following:

  \begin{itemize}
  \tightlist
  \item
    Courses in Art History (ARH-\_\_\_)\\
  \item
    ANT 125 Art \& Archaeology Classical World\\
  \item
    HIS 318 Topics in History :Div West Persp\\
  \item
    COM 161 Visual Rhetoric\\
  \item
    FLM 225 Film History\\
  \item
    COM 357 Sex, Race, \& Gender in Media\\
  \end{itemize}
\item
  One of the following

  \begin{itemize}
  \tightlist
  \item
    ARH 307 Modern and Contemporary Art\\
  \item
    ARH 310 Contemporary Art, Theory \&Criticism\\
  \end{itemize}
\item
  ARH 464 Senior Seminar I\\
\item
  ARH 474 Senior Seminar II\\
\item
  One 100- ART course\\
\item
  One 300-level ART course
\end{enumerate}

note: Scheduling may be planned to allow at least one term of study
off-campus at one of the approved foreign or domestic programs. With
departmental approval, up to three courses from these programs may count
toward either the two 200-level courses in item 2 or the two additional
courses in item 3.

Students interested in art history or visual culture at the graduate
level should complete Intermediate French I (FRE-215).

\subsection{Art History Minor}\label{art-history-minor}

A grade of ``C'' (2.0) or higher must be earned in all courses counted
toward a minor in art history.

\begin{enumerate}
\def\labelenumi{\arabic{enumi}.}
\tightlist
\item
  ARH 128 Introduction to Art History\\
\item
  Three additional art history courses, one of which must be ARH-200 or
  above. Can also include one of the following:

  \begin{itemize}
  \tightlist
  \item
    ANT 125 Art \& Archaeology Classical World\\
  \item
    HIS 318 Topics in History :Div West Persp\\
  \item
    COM 161 Visual Rhetoric\\
  \item
    FLM 225 Film History\\
  \item
    COM 357 Sex, Race, \& Gender in Media\\
  \end{itemize}
\item
  One of the following

  \begin{itemize}
  \tightlist
  \item
    ARH 307 Modern and Contemporary Art
  \item
    ARH 310 Contemporary Art, Theory \&Criticism\\
  \end{itemize}
\item
  One 100- or 300-level ART- course
\end{enumerate}

\subsection{COURSES IN ART HISTORY}\label{courses-in-art-history}

\begin{itemize}
\tightlist
\item
  ARH 106 World Art\\
  Traces key themes in art from a global perpective, focusing on the
  ways that cultures and civilizations across time have visually
  expressed social, religious and political values. Cross-cultural
  themes may include: religion and spirituality, word and image,
  violence and death, power and propaganda, gender and society, parks
  and memorials, and ritual and body decoration.\\
\item
  ARH 107 Gender and Art\\
  Explores of the ways in which visual culture reflects and projects
  cultural biases and issues related to gender from prehistory to the
  modern era. Analyzes how gender identities can be shaped by politics,
  religion, and culture, as well as the effect of an artist's sex and/or
  sexual preferences on subject choices, media, and market values.\\
\item
  ARH 128 Introduction to Art History\\
  Examines Western art and architecture from prehistory to the later
  19th century, with emphasis on the ways in which visual culture both
  reflects and shapes societies and civilizations. Explores how works of
  art create and sustain meaning for their original audiences, and how
  some objects or visual solutions transcend their historical moment and
  surface throughout time as familiar cultural icons or references.\\
\item
  ARH 191 Topics in Art History\\
  See also +++MISSING INFO: c.art91.short +++\\
  Focuses on a selected topic or theme in art history or visual culture.
  Topics vary. May be taken more than once for credit, provided the
  topics are substantially different.\\
\item
  ARH 201 Art of the Middle Ages\\
  Explores the art and architecture of the medieval world both
  chronologically and thematically. The course examines issues such as
  patronage of the arts, pilgrimage, the cult of saints, the arts as a
  medium of cultural exchange, and the role of the artist in the Middle
  Ages. Prerequisite: Introduction to Art History (ARH-128) or consent
  of instructor.\\
\item
  ARH 218 The World of Renaissance Art\\
  Explores the visual culture of Europe from the 14th through the 16th
  centuries, focusing on topics such as competition, display, devotion,
  portraits, the printing revolution, death, and gender issues.
  Prerequisite: Introduction to Art History (ART-128) or consent of
  instructor.\\
\item
  ARH 231 Romanticism, Realism, Impressionism\\
  Focuses on the sweeping transformations in the creation, production
  and consumption of visual culture in the 19th century. The rapidly
  changing aesthetics of the dawning modern era generated passionate
  debates about the creation and reception of art during this period.
  This course uses these debates as a series of touchstones for
  understanding the visual and social landscape of the times.
  Prerequisite: Introduction to Art History (ARH-128) or consent of
  instructor.\\
\item
  ARH 248 Baroque, Rococo, and Neoclassicism\\
  Focuses on a thematic exploration of the major art movements in the 17
  th and 18th centuries, emphasizing the pendulum swings of artistic
  practice, exploration, and institutional hierarchies. Inclusion of
  cultural phenomena---the Grand Tour, the Enlightenment, revolutions,
  and the establishment of academies---factor heavily in this course.
  Prerequisite: Introduction to Art History (ARH 128) or consent of
  instructor.\\
\item
  ARH 268 History of Architecture\\
  Investigates major monuments of architectural history from prehistory
  to the present day, with an emphasis on formal and conceptual
  concepts. Key figures, theories, innovations, and functions (both
  original and altered throughout time) are also discussed.
  Prerequisite: Introduction to Art History (ARH 128) or consent of
  instructor.\\
\item
  ARH 296 Topics in Art History:Global Persp\\
  Focuses on a selected topic or theme in art history. Topics vary. May
  be taken more than once for credit, provided the topics are
  substantially different. different. Prerequisite: Introduction to Art
  History (ARH-128) or consent of instructor.\\
\item
  ARH 297 Topics in Art History: US Pluralism\\
  Examines a selected topic or theme in art history. Topics vary. May be
  taken more than once for credit, provided the topics are substantially
  different. Prerequisite: Introduction to Art History (ARH-128) or
  consent of instructor.\\
\item
  ARH 298 Topics in Art History:Div West Pers\\
  Studies a selected topic or theme in art history. Topics vary, and may
  include: Art and Cultural Property; Memory, Environment and Landscape.
  May be taken more than once for credit, provided the topics are
  substantially different. Prerequisite: Introduction to Art History
  (ARH-128) or consent of instructor.\\
\item
  ARH 307 Modern and Contemporary Art\\
  Traces the development of major artistic movements in the 20th
  centrury to the more contemporary trends of the 21st century. Topics
  include: Post-Impressionism, Cubism, Futurism, Surrealism, German
  Expressionism, Dada, Pop Art, Minimalism, Conceptual Art,
  Postmodernism, installation, new media, performance, and digital
  production and distribution. Prerequisite: Introduction to Art History
  (ARH-128) and a 200-level ARH- course or consent of instructor.\\
\item
  ARH 310 Contemporary Art, Theory \&Criticism\\
  Investigates issues in contemporary art. Focuses on art of the late
  20th and early 21st centuries, considering stylistic, historical and
  theoretical developments. Prerequisite: Introduction to Art History
  (ARH-128) and a 200-level ARH- course or consent of instructor.\\
\item
  ARH 391 Advanced Topics in Art History\\
  See also \\
  Focuses on an advanced study of a selected topic or theme in art
  history. Topics vary. May be taken more than once for credit, provided
  the topics are substantially different. Prerequisites: Introduction to
  Art History (ARH-128) and a 200-level ARH- course or consent of
  instructor.\\
\item
  ARH 394 Directed Learning in Art History\\
  Investigates topics in art history selected by the student and
  instructor to fit the student's particular interests and educational
  needs. May be taken more than once for credit. Prerequisites: junior
  standing and consent of department chair.\\
\item
  ARH 444 Independent Study: Art History\\
  Focuses on independent work on a selected project under the direction
  of a faculty member of the Art and Art History department.
  Prerequisite: consent of instructor.\\
\item
  ARH 464 Senior Seminar I\\
  Introduces the research methods, theories, and curatorial practices
  affiliated with the discipline of art history. Additionally, students
  begin crafting their capstone projects in art history, which will be
  completed in Seminar II. Prerequisites: declared major in Art History,
  Introduction to Art History (ARH-128), one 200-level ARH course, and
  junior standing.\\
\item
  ARH 474 Senior Seminar II\\
  Completes the capstone research project (i.e.~research papers,
  exhibitions, or hybrid projects that involve making art and
  contextualizing visual culture). Prerequisite: Successful completion
  of Seminar in Art History I.\\
\item
  ARH 494 Internship in Art History\\
  Investigates an area of interest related to the major, through
  voluntary field placement supervised by a faculty member of the Art
  and Art History department. A minimum of 140 hours on-site experience
  is required. P/NP basis only. Prerequisites: declared major in art
  history, junior standing, and consent of department chair.
\end{itemize}

\section{Dance (Courses Only)}\label{sec-dance}

Rezabek, Wolverton.

\subsection{Courses in Dance}\label{courses-in-dance}

\begin{itemize}
\tightlist
\item
  \textbf{DAN 101 Dance - Jazz I} A beginning-level course designed to
  introduce the student to basic jazz dance techniques and skills.
  Emphasis on alignment and precise execution of jazz movements. Class
  includes functional kinesiology and injury prevention techniques, and
  presentations place jazz dance in socio-historical context. (0.2
  course credit)
\item
  \textbf{DAN 102 Dance - Jazz II} An intermediate-level course designed
  to increase the student's knowledge and skill in jazz dance
  techniques. May include Hatchett, Giordano, Luigi, and contemporary
  styles. Class continues functional kinesiology, and presentations
  place jazz dance in socio-historical context. Prerequisite: Jazz I
  (DAN-101). (0.2 course credit)
\item
  \textbf{DAN 111 Dance-Modern I} A beginning-level class focusing on
  fundamental modern dance techniques. Emphasis on placement of the
  spine and quality of movement. Features the techniques of Doris
  Humphrey, Jose Limon, and Erick Hawkins. Presentations discuss
  modernisms and postmodernism and place modern dance in
  socio-historical context. (0.2 course credit)
\item
  \textbf{DAN 112 Dance-Modern II} An intermediate-level class offering
  more complex modern dance styles and skills. May include Graham,
  Horton, and Cunningham techniques. Presentations continue the
  discussion of modernism and postmodernism and place modern dance in
  socio-historical context. Prerequisite: Modern I (DAN-111). (0.2
  course credit)
\item
  \textbf{DAN 131 Dance-Ballet I} Classical ballet is based on
  traditional positions and movements of the body emphasizing harmonious
  lines in space. It is the technical basis of many forms of dance. This
  course offers a working vocabulary of basic ballet movement skills and
  terminology. Presentations discuss the history and theory of ballet.
  (0.2 course credit)
\item
  \textbf{DAN 132 Dance-Ballet II} This course continues Ballet I,
  offering more complex ballet movement skills and terminology.
  Prerequisite: Ballet I (DAN-131). (0.2 course credit)
\item
  \textbf{DAN 141 Dance - Tap I} Simple tap steps and combinations,
  including adequate background to survive an audition. The course
  progresses to more complicated movement sequences, with emphasis on
  speed and clarity. (0.2 course credit)
\item
  \textbf{DAN 142 Dance - Tap II} More complex steps, styles, and
  rhythms. Emphasis on speed, clarity, strength, and dexterity.
  Prerequisite: Tap I (DAN-141). (0.2 course credit)
\item
  \textbf{DAN 151 Dance-Choreography I} Introduction to the
  choreographic craft, concentrating on generating original movement
  through short studies focusing on body, space, and time. (0.2 course
  credit)
\item
  \textbf{DAN 152 Dance-Choreography II} Further exploration of
  choreographic techniques, culminating in a short performance of
  student's work. Prerequisite: Choreography I (DAN-151). (0.2 course
  credit)
\end{itemize}

\section{Data Science}\label{data-science}

Here is a description of the major.

And the major requirments: - abcd - abcd - abcd

\subsection{Courses in Data Science}\label{courses-in-data-science}

\begin{itemize}
\tightlist
\item
  \textbf{DS 230 Data-Centric Computing} Provides a programming
  experience with applications that focus on data handling tasks.
  Students examine programming techniques to acquire and manage data
  from a variety of sources and formats; use relational databases to
  store and query data; and explore techniques to work with
  semi-structured and unstructured data sets. Prerequisite: Introduction
  to Programming (CS-125) or consent of instructor\\
\item
  \textbf{DS 260 Data Analysis and Visualization} Studies intermediate
  data analytic techniques and concepts to visualize quantitative data.
  This course expands the mathematical background of students, with
  topics from statistical analysis and linear algebra. Students will
  learn advanced visualization techniques, with particular emphasis on
  creating graphics and animations using visualization libraries.
  Prerequisite: Data-Centric Computing (DS-230)
\end{itemize}

\section{Economics}\label{sec-economics}

See \textbf{?@sec-business-administration-and-economics}

\section{Education}\label{sec-education}

N. Hayes, Haynes-Moore, Kigin (Placement Coordinator) Kress (Chair),
Russell Art: Rogers; Music: Carson, Shanley; Physical Education:
Atwater. Content Specialists (Part-time): Bakas, Christofferson,
Dabroski, Gaylord Robertson, Hanson J., Hanson M., Hynek, Johnson,
Neilly, Zahn, Zrudsky Student Teaching Supervisors: Bartlett, Lanich,
Oberbroeckling

Coe College believes that the most effective preparation for teaching is
one that combines a liberal arts education with courses in the theory
and practice of teaching. The Education Department has the
responsibility for coordinating the efforts of the College to provide
such a program.\\
Basic college requirements and those for a major area give students a
well-rounded general education. Professional courses in education
provide a foundation in principles and practices of teaching. Students
gain practical experience in applying professional and general education
through field experiences and, if pursuing licensure, through student
teaching in area schools. Students who successfully complete Coe's
Teacher Education Program and student teaching are eligible to apply for
an Iowa Initial Teacher License. Teaching licensure is governed by State
of Iowa regulations. When changes in licensure requirements occur at the
state level, they take precedence over College policies. For the most
current policy information, students should consult the \emph{Guide to
Teacher Education}, available from the Education Department. The
Education Department maintains records regarding Iowa licensure
requirements. Students should consult a faculty member in the Education
Department to arrange their respective courses of study. The Teacher
Education Program at Coe College is approved by the Iowa Department of
Education. Copies of the annual report filed with the Iowa Department of
Education are available on request.

\subsection{Elementary Education
Major}\label{elementary-education-major}

Teachers in elementary schools function as generalists who must draw
upon a broad knowledge base from multiple disciplines. In recognition of
this, the elementary teacher licensure program at Coe College consists
of two components: content knowledge gained from the liberal arts
classes and pedagogy learned in EDU courses. This program of
undergraduate preparation to teach in elementary schools is intended to
promote exploration and a balanced education drawn from a variety of
fields.

\begin{enumerate}
\def\labelenumi{\arabic{enumi}.}
\item
  At least one course in each of the four fields of mathematics, natural
  sciences, social sciences, and humanities. \textbf{Iowa Distribution
  Requirements} - a ``C'' or better in the following courses:
  \textbf{One} mathematics course with a prefix of MTH or STA
  \textbf{One} American History course \textbf{One} lab science in BIO
  or PHY-114 Modern Astronomy \textbf{One} social science course with a
  prefix of ANT, ECO, POL, PSY, or SOC
\item
  A K--8 endorsement in Art, English/Language Arts, French, Spanish,
  Health, History, Mathematics, Music, Physical Education, Reading,
  Science, Social Studies, or Speech Communication and Theater.
\item
  GPA of 2.7 or higher and a ``C'' or better in the following
  professional education courses: EDU 105 Foundations of Education EDU
  117 Exceptional Learners EDU 187 Human Relations EDU 195 Educ
  Psychology \& Development EDU 215 Practicum in Education EDU 219
  Educational Technology Lab EDU 237 English Language Learners
  \textbf{OR} EDU 270 Read \& Explore Childrens/Adol Lit EDU 275 Math
  Comprehension for Teaching EDU 300 Expressive Methods EDU 305 Methods
  of Elementary Science EDU 335 Methods of Elementary Mathematics EDU
  345 Methods of Elementary Language Arts EDU 355 Methods of Elementary
  Reading EDU 365 Methods of Elementary Social Studie KIN 112 Health
  Educ for Elementary Tchr
\end{enumerate}

\textbf{NOTE}: \emph{As is true for all majors, elementary education
students are responsible for completion of Coe's general education
requirements. Careful program planning may enable either greater breadth
within the liberal arts or a deeper concentration within a specialty
discipline. Programs containing such additional concentration may
require more than four years of study.}

\subsection{Secondary Education Minor}\label{secondary-education-minor}

\begin{enumerate}
\def\labelenumi{\arabic{enumi}.}
\tightlist
\item
  \emph{Iowa Distribution Requirements}: a ``C'' or better in at least
  one course of the four fields of mathematics, natural sciences, social
  sciences, and humanities.
\item
  A grade of ``C'' or better must be earned in all courses counted
  toward a major or minor in education and in subject areas in which
  students intend to teach.
\item
  Completion of a major in one or more teaching fields with a GPA of 2.7
  or higher. Teaching field(s) must be in subjects ordinarily taught in
  the secondary schools and for which Coe has approval by the Iowa
  Department of Education. Approved teaching fields include American
  Government, American History, Art, Basic Science, Biology, Business,
  Chemistry, Economics, English/Language Arts, French, Health, Math,
  Music, Physical Education, Physics, Psychology, Sociology, Spanish,
  Speech Communication and Theater, and World History.
\item
  GPA of 2.7 or higher and a ``C'' or better in the following
  professional education courses: EDU 105 Foundations of Education EDU
  117 Exceptional Learners EDU 187 Human Relations EDU 195 Educ
  Psychology \& Development EDU 215 Practicum in Education EDU 219
  Educational Technology Lab
\item
  \textbf{One or more} of the following Methods courses: ART 230 Art of
  Children \& Adolescents EDU 310 Meth Secondary Business Educ EDU 311
  Meth Secondary Language Arts EDU 312 Meth Secondary Social Studies EDU
  313 Methods Secondary Mathematics EDU 320 Methods of Secondary Science
  EDU 321 Methods of World Language K-12 KIN 415 Meth Secondary School
  PE \& Health MU 361 Choral Music Methods \textbf{AND} MU 362
  InstrMusMethChrlTeachr MU 363 Instrumental Music Methods \textbf{AND}
  MU 364 ChrlMusMethInstTeach
\end{enumerate}

\subsection{ADMISSION TO THE TEACHER EDUCATION
PROGRAM}\label{admission-to-the-teacher-education-program}

The Education Department offers a major in \textbf{elementary
education}, a minor in \textbf{secondary education}, and courses leading
to three types of teacher licensure: K-6 classroom (elementary
education); 5--12 content area (secondary education) and K-12
``specials'' content area (art, music, physical education). Students
pursuing any of these options must be admitted to the Teacher Education
Program before being allowed to enroll in a practicum course or methods
course. Admission to Coe College does not guarantee admission to the
Teacher Education Program. Information related to applying to the
Teacher Education Program is found in the \emph{Guide to Teacher
Education} available from the Education Department. State of Iowa
licensure requirements specify that a minimum of 80 hours of field
experience be completed after admission to the Teacher Education Program
and prior to student teaching. Teacher Education Program applications
are available online; contact your Education advisor. The Teacher
Education Committee has responsibility for review of the Teacher
Education Program, including admission of students into the program. In
reviewing applications for admission to the program, the committee
considers the following: 1. A ``C'' or better in two core EDU courses
(EDU 105 Foundations of Education strongly recommended) 2. GPA (a
minimum of 2.7 is required in education core courses, courses in the
major and overall) 3. A grade of ``C'' or higher must be earned in all
courses counted toward a major or minor in education and in subject
areas in which students intend to teach. 4. Performance in education
courses taken. 5. The essay included on the program application. 6.
Other relevant information as provided by the faculty of the Education
Department and comments from faculty in classes the student is enrolled
in or has recently taken.

The Teacher Education Committee reviews applications at the conclusion
of each semester. Students are notified in writing regarding the
committee's decision. Appeals may be made to the committee in writing.
The committee evaluates such appeals on an individual basis, using its
best judgment of the student's suitability to enter the teaching
profession. The process for further appeals is detailed in the
\emph{Guide to Teacher Education}.

\subsection{Graduates/Transfers from Other
Institutions}\label{graduatestransfers-from-other-institutions}

Graduates/transfers from other institutions should submit transcripts
and all other relevant materials for review to the Education department
chair. In most cases, applicants are required to take a minimum of four
EDU courses at Coe, in which at least a 2.7 GPA or higher and a ``C'' or
better is earned, prior to student teaching. Students who received the
highest degree more than 10 years prior to application to the Teacher
Education Program must complete two additional courses at Coe in the
major area, with a grade of ``C'' or better, in addition to the required
professional education courses.

\subsection{Elementary
Education/Non-Licensure}\label{elementary-educationnon-licensure}

The majority of students who major in elementary education intend to
become licensed K--6 classroom teachers and complete one term of
full-time student teaching after completing the major requirements.
Occasionally, however, a student may wish to work with young children in
settings other than those for which certification is required. Such
students may complete all of the required coursework for the elementary
education major without student teaching and are advised on
supplementary coursework to support their goals.

\subsection{REQUIREMENTS FOR STUDENTS PURSUING TEACHING
LICENSES}\label{requirements-for-students-pursuing-teaching-licenses}

To be recommended for licensure in any of the categories listed below, a
student must have a baccalaureate degree, a cumulative GPA of at least
2.7, and at least a 2.7 GPA in each teaching field. The department can
refuse to recommend for licensure a student who has been found to have
violated the College's Academic Integrity Policy.

\subsection{Iowa Distribution Requirements for All Students Seeking
Licensure}\label{iowa-distribution-requirements-for-all-students-seeking-licensure}

According to state regulations, all teachers in Iowa ``shall acquire a
core of liberal arts knowledge including, but not limited to, English
composition, mathematics, natural sciences, social sciences, and
humanities.'' While the state's distribution requirements are consistent
with Coe's general education requirements, they are not automatically
met by these requirements. Education students should consult with their
Education advisor to coordinate this mandate with the selection of
courses for general education.

\subsection{Elementary Licensure}\label{elementary-licensure}

\textbf{Requirements:} 1. Students licensed at the elementary level,
must have a 2.7 GPA or higher and a ``C'' or better in all courses in
their major and in endorsement areas. 2. \textbf{Iowa Distribution
Requirements} - a ``C'' or better in the following courses: \textbf{One}
mathematics course with a prefix of MTH or STA \textbf{One} American
History course \textbf{One} lab science in BIO or Modern Astronomy
\textbf{One} social science course 3. GPA of 2.7 or higher and a ``C''
or better in the following professional education courses: EDU 105
Foundations of Education EDU 117 Exceptional Learners EDU 187 Human
Relations EDU 195 Educ Psychology \& Development EDU 215 Practicum in
Education EDU 219 Educational Technology Lab EDU 237 English Language
Learners \textbf{OR} EDU 270 Read \& Explore Childrens/Adol Lit EDU 275
Math Comprehension for Teaching EDU 300 Expressive Methods EDU 305
Methods of Elementary Science EDU 335 Methods of Elementary Mathematics
EDU 345 Methods of Elementary Language Arts EDU 355 Methods of
Elementary Reading EDU 365 Methods of Elementary Social Studie KIN 112
Health Educ for Elementary Tchr 4. A K--6 Iowa teaching license must be
accompanied by at least one K--8 endorsement (state approved
specialization in a particular subject area) available at Coe. K--8
endorsements include: Art, English/Language Arts, French, Spanish,
Health, History, Mathematics, Music, Physical Education, Reading,
Science, Social Studies, or Speech Communication and Theater. Students
should consult their Education advisor for assistance in adding these
endorsements to their elementary teaching license. 5. Student Teaching
(4 course credits): EDU 492 Student Teaching K-3 EDU 491 Student
Teaching 4-6 6. Successful completion of Student Teaching Seminar -
Active engagement in and regular attendance in seminar - Dyslexia
training - Ethics training - Mock evaluations - CPR card (PE and
coaching candidates) - Successful completion and presentation of
e-portfolio 7. Demonstrate content-area knowledge and pedagogical
knowledge by successfully passing state-required Praxis II exams.

\textbf{NOTE:} \textbf{\emph{Students seeking elementary licensure must
have an advisor in the Education Department with whom they meet
regularly (at least twice per year) to ensure timely completion of all
graduation, general education, and licensure requirements.}}

\subsection{Secondary Licensure}\label{secondary-licensure}

Those interested in teaching at the secondary level must major in one or
more teaching fields, minor in education, and meet the state
requirements for a 5--12 endorsement (state-approved specialization in a
particular subject area). Teaching field(s) must be in subjects taught
in the secondary schools and for which Coe has approval by the Iowa
Department of Education.\\
\textbf{5--12 endorsements include}: American Government, American
History, Art, Biology, Business, Chemistry, Coaching (see Kinesiology),
Economics, English/Language Arts, French, Health, Math, Music, Physical
Education, Physics, Psychology, Science, Social Studies, Sociology,
Spanish, Speech Communication and Theater or World History.

\textbf{Requirements:} 1. Students licensed at the secondary level, must
have a 2.7 GPA or higher and a ``C'' or better in all courses in their
major and in endorsement areas. 2. Iowa Distribution Requirements: a
``C'' or better in at least one course in each of the four fields of
mathematics, natural sciences, social sciences, and humanities. 3. GPA
of 2.7 or higher and a ``C'' or better in the following professional
education courses: EDU 105 Foundations of Education EDU 117 Exceptional
Learners EDU 187 Human Relations EDU 195 Educ Psychology \& Development
EDU 215 Practicum in Education EDU 219 Educational Technology Lab 4.
\textbf{One or more} of the following Methods courses: EDU 310 Meth
Secondary Business Educ EDU 311 Meth Secondary Language Arts EDU 312
Meth Secondary Social Studies EDU 313 Methods Secondary Mathematics EDU
320 Methods of Secondary Science EDU 321 Methods of World Language K-12
5. Successful completion of Student Teaching: EDU 489 Student Teaching
Sr HS EDU 490 Student Teaching Jr HS 6. Successful completion of Student
Teaching Seminar - Active engagement in and regular attendance in
seminar - Dyslexia training - Ethics training - Mock evaluations - CPR
card (PE and coaching candidates) - Successful completion and
presentation of e-portfolio 7. Demonstrate content-area knowledge and
pedagogical knowledge by successfully passing state-required Praxis II
exams.

\textbf{NOTE:} \textbf{\emph{Not all Coe majors align perfectly with
State of Iowa requirements for 5--12 endorsements. Students seeking
secondary licensure should also have an advisor in the Education
Department with whom they meet regularly (at least once per year) to
ensure timely completion of all graduation, general education, and
licensure requirements.}}

\subsection{Art, Music and Physical Education (K--12
licensure)}\label{art-music-and-physical-education-k12-licensure}

Preparation for teaching at the elementary (K--8) and secondary (5--12)
levels in the subject areas of art, music, and physical education
includes state licensure requirements, in addition to work students
complete in their major.

\textbf{Requirements:} 1. Students licensed in art, music, or physical
education must earn a 2.7 GPA or higher and a ``C'' or better in all
courses in their major and in endorsement areas (state approved
specialization in a particular subject area). 2. Iowa Distribution
Requirements: A ``C'' or better in at least one course in each of the
four fields of mathematics, natural sciences, social sciences, and
humanities. 3. GPA of 2.7 or higher and a ``C'' or better in the
following professional educational courses: EDU 105 Foundations of
Education EDU 117 Exceptional Learners EDU 187 Human Relations EDU 195
Educ Psychology \& Development EDU 215 Practicum in Education EDU 219
Educational Technology Lab 4. \textbf{One or more} of the following
Methods courses: ART 230 Art of Children \& Adolescents KIN 315 Methods
Elementary Sch PE \& Health KIN 415 Meth Secondary School PE \& Health
MU 360 Elementary \& General Music Methods MU 361 Choral Music Methods
\textbf{AND} MU 362 InstrMusMethChrlTeachr MU 363 Instrumental Music
Methods \textbf{AND} MU 364 ChrlMusMethInstTeach 5. Successful
completion of Student Teaching: EDU 481 Stu Tchg Sec: ART EDU 482 Stu
Tchg Sec: Phys Education EDU 483 Std Teaching Elem: ART EDU 485 Std
Teaching Elem: Phys Ed MU 421 Student Teaching Elementary Music MU 422
Student Teaching Secondary Music 6. Successful completion of Student
Teaching Seminar - Active engagement in and regular attendance in
seminar - Dyslexia training - Ethics training - Mock evaluations - CPR
card (PE and coaching candidates) - Successful completion and
presentation of e-portfolio 7. Demonstrate content-area knowledge and
pedagogical knowledge by successfully passing state-required Praxis II
exams.

\textbf{NOTE:} \textbf{\emph{Not all Coe majors align perfectly with
State of Iowa requirements for licensure. Students seeking licensure
should also have an advisor in the Education Department with whom they
meet regularly (at least once per year) to ensure timely completion of
all graduation, general education, and licensure requirements.}}

\subsection{Student Teaching}\label{student-teaching}

Permission to student teach requires approval of the Teacher Education
Committee. Applications are due by February 1 for student teaching the
following Fall Term and by September 1 for student teaching the
following Spring Term. Applications are available from the Education
Office. When an application to student teach is considered, the
student's standing in the Teacher Education Program is reviewed. The
committee considers: 1. Recommendations of cooperating teacher(s) during
the student's field experience(s). 2. Consultation with student's major
department(s). 3. A grade of ``C'' or better must be earned in all
courses counted toward a major or minor in education and in all
endorsement areas. 4. GPA (a minimum of 2.7 is required in the
professional education courses, courses in the major, and overall). 5.
Review by Education Department faculty. Student teaching is the
culminating experience of the Teacher Education Program. It involves
observation and active participation in directing learning experiences
aligned with professional learning standards in a school classroom,
under the supervision of a cooperating teacher and the college
supervisor. A minimum of four course credits of student teaching is
required. Student teaching in all subjects is available for one to four
course credits for students earning both Elementary and Secondary
licensure. Normally, students complete two seven- to eight-week
placements, with each placement worth two credits. Successful completion
of the student teaching semester, including attendance at and
participation in the student teaching seminar, earns four credits.
Student teaching is a full-time obligation, must be completed in one
term, and is evaluated on a P/NP basis.

\textbf{Student Teaching Requirements} 1. One or more of the following:
EDU 481 Stu Tchg Sec: ART EDU 482 Stu Tchg Sec: Phys Education EDU 483
Std Teaching Elem: ART EDU 485 Std Teaching Elem: Phys Ed EDU 489
Student Teaching Sr HS EDU 490 Student Teaching Jr HS EDU 491 Student
Teaching 4-6 EDU 492 Student Teaching K-3 MU 421 Student Teaching
Elementary Music MU 422 Student Teaching Secondary Music 2. Successful
completion of Student Teaching Seminar - Active engagement in and
regular attendance in seminar - Dyslexia training - Ethics training -
Mock evaluations - CPR card (PE and coaching candidates) - Successful
completion and presentation of e-portfolio 3. Satisfactory performance
in the field as determined by the cooperating teacher and supervisor
evaluations.

\subsection{Ninth-Term/Fifth-Year
Programs}\label{ninth-termfifth-year-programs}

Students who want to include more coursework in their liberal arts and
sciences curriculum in conjunction with completing requirements for
elementary and/or secondary teacher licensure may apply for a
Ninth-Term/Fifth-Year tuition reduction. Both options are open to all
students who have been \textbf{\emph{admitted}} to the Teacher Education
Program and meet requirements for student teaching. Students
participating in the Ninth-Term/Fifth-Year Programs are eligible for a
60\% discount on full-time tuition, if the following criteria are met:
1. The student must be in good standing with the college (not on
academic probation); 2. The student must be registered as a full-time
student; 3. The student must have completed 32 course credits, sixteen
of which were taken at Coe; and 4. The student must not have previously
received the age discount. 5. The student submits the application by the
appropriate deadline: Fall Term: April 1st Spring Term: November 1st
Students who have satisfactorily completed four years of full-time
registration at Coe may also apply for this reduction. Students continue
to be eligible to apply for student loans and any available federal and
state government grants. This reduction can only be given for a maximum
of two terms and applies only to regular full-time tuition charges for
undergraduate campus-based programs. Applications for tuition reduction
are available from the Education Department. Exemptions from these
requirements may be granted in unusual circumstances with the approval
of the Provost and Dean of the Faculty and the Education department
chair. A written request for such consideration, detailing the
circumstances and rationale, should be made to the Education department
chair.

\subsection{Courses in Education}\label{courses-in-education}

\begin{itemize}
\tightlist
\item
  \textbf{EDU 105 Foundations of Education} Explores the many facets of
  schooliing in the United States. Students will examine historical,
  political, legal, social, and philosophical issues related to
  education and public schooling in the United States. A second course
  component consists of an introduction to curricular and instructional
  planning. A ten-hour field experience in local schools is required.\\
\item
  \textbf{EDU 109 Field Experience} Conducted in the public schools at
  either the elementary or secondary level. Students spend at least 60
  hours of supervised work in a school setting. Prerequisite:
  Foundations of Education (EDU-105) and consent of Education Department
  Chair. (0.5 course credit)\\
\item
  \textbf{EDU 117 Exceptional Learners} Addresses the nature and needs
  of students with exceptionalities in the general classroom. Topics
  include historical and legal foundations of special education;
  classroom teacher responsibilities under IDEA and Section 504 of the
  Vocational Rehabilitation Act, categories of exceptionality,
  appropriate instructional accommodations for students with
  exceptionalities, methods of instruction and assessment of students
  with special needs. The course addresses the different needs of all
  learners, including but not limited to gifted learners, learners with
  disabilities, English Language Learners, twice exceptional learners,
  and learners struggling with literacy.\\
\item
  \textbf{EDU 187 Human Relations} Examines the lifestyles, history, and
  contributions of various identifiable subgroups in our pluralistic
  society. Students explore and analyze issues related to such topics
  as: equity in the schools, multicultural education, sexism, racism,
  religious pluralism, sexuality, intercultural interaction, and sexual
  harassment. Coursework encourages students to build their capacities
  for recognizing, understanding, and respecting diversity of people and
  cultures in order to develop constructive interpersonal relationships
  and favorable learning experiences in the classroom. A 10-hour field
  experience is required.\\
\item
  \textbf{EDU 195 Educ Psychology \& Development} Introduces theories
  that address teaching and learning processes. Special attention is
  given to the following topics: the cognitive and emotional development
  of children and adolescents; learning and memory; intelligence and
  creativity; academic motivation; assessment; and classroom management.
  The course connects theories to practice by exploring the nature of
  formal learning environments that best serve the cognitive and
  emotional growth of students.\\
\item
  \textbf{EDU 215 Practicum in Education} Integrates theory and
  pedagogical practice. Students spend a minimum of 60 hours of
  practical experience in an elementary or secondary school classroom
  assisting in a range of instructional activities. College classroom
  experience complements the field experience and focuses on
  instructional planning, differentiation, assessment, classroom
  management and reflective teacing practices. Prerequisites: Admission
  to the Teacher Education Program (approved or conditional).\\
\item
  \textbf{EDU 219 Educational Technology Lab} Focuses on strategies for
  integrating educational technology in K-12 classrooms. Through
  hands-on lab work, course readings, reflective writing, and exposure
  to professionals in the field (both local and afar) help students gain
  the knowledge, skills, and attitudes needed to select, implement, and
  manage technology. The goal of the course is to help students plan,
  implement, and evaluate educational technology for teaching and
  learning. Course projects focus on the use of instructional technology
  tools to develop materials that support teaching and learning. P/NP
  basis only. Corequisite: Practicum in Education (EDU-215) or Practicum
  in Music Education (MU-205). (0.0 course credit)\\
\item
  \textbf{EDU 220 Professional Writing for Teachers} Practical
  experience creating professional documents for a variety of
  educational purposes and for a range of educational constituents. This
  course provides intensive practice in composing and designing
  documents central to teachers' work lives. The goal of the course is
  to help students develop confidence and skill as teaching
  professionals who excel at written communication. Prerequisiites:
  Foundations of Education (EDU 105) or consent of instructor.\\
\item
  \textbf{EDU 237 English Language Learners} Introduces the issues of
  language and literacy acquisition for English Language Learners. This
  class includes a focus on oral communication and K-12 literature. It
  is appropriate for teachers of non-native English students and
  international teaching of English. Prerequisite: sophomore standing or
  consent of instructor. (Offered Spring Term)\\
\item
  \textbf{EDU 241 Foundations of Reading} Explores past and current
  theories of language acquisition and development. Attention to the
  history of reading complements a focus on current research related to
  the psychological, socio-cultural, and linguistic foundations of
  reading and writing. This is the first course in the sequence leading
  to the reading endorsement. Prerequisite: admission to the Education
  Program or consent of Education Department Chair. (Offered Spring
  Term)\\
\item
  \textbf{EDU 260 Content Area Reading} Explores research-based
  instructional strategies for facilitating students' textual
  comprehension. Major topics include best practices in the teaching of
  vocabulary, meaning making, text structure, genre, and types of
  writing. Attention to instructional strategies especially helpful to
  English language learners, as well as technological tools for language
  and literacy instruction. This is the second course in the sequence
  leading to the reading endorsement. Prerequisites: admission to the
  Education Program and Foundations of Reading (EDU-247) or consent of
  Education Department Chair. (Offered Fall Term)\\
\item
  \textbf{EDU 270 Read \& Explore Childrens/Adol Lit} Provides students
  pursuing literature-related studies with models of how teachers use
  literature to support various levels of readers and content areas in
  the classroom. The course focuses on the history of children's
  literature, analyses of a variety of books for youths, creation of a
  functional bibliography, and using knowledge of child and adolescent
  development to assist in book selection for young readers. Includes
  study of the development of language skills, strategies to facilitate
  student learning of standard English, diversity issues, non-print
  materials, and technology tools. (Offered Fall Term)\\
\item
  \textbf{EDU 275 Math Comprehension for Teaching} Enhances
  understanding of the discipline of mathematics for elementary
  classroom teachers. Content is aligned with the NCTM curriculum
  standards of Number, Algebra, Geometry, Measurement, and Data Analysis
  and Probability. Substantial attention is given to discrete
  mathematics, reasoning and proof, active problem solving, technology,
  and connections within mathematics and to other disciplines studied in
  schools. The ability to communicate mathematically and provide
  justification or rationale for quantitative reasoning is fostered
  across mathematical applications. (Offered Fall Term)\\
\item
  \textbf{EDU 284 Topics in Education} Examines special topics in
  education. Content varies and is determined by the instructor. Example
  topics include: Assessment in Education, Educational Law, Leadership
  in Education, Social-Emotional Teaching and Learning, Current Events
  in Education. May be taken more than once provided the topics are
  substantially different. Prerequisite: Educational Foundations (EDU
  105) or consent of instructor. (.5 course credit)\\
\item
  \textbf{EDU 300 Expressive Methods} Integrates art, music, and
  physical education into the regular elementary classroom. Students
  will study of instructional methods for incorporating visual,
  aesthetic, auditory, and kinesthetic teaching and learning strategies
  into the classroom. This course promotes the integration of art,
  music, and physical education into the regular elementary classroom.
  Topics include research on learning styles, development of lessons and
  classroom management strategies that incorporate movement and creative
  expression, and alternative assessment as it relates to these
  strategies. Diversity issues, non-print materials, and technology
  tools are integrated. Prerequisite: Practicum in Education (EDU-215)
  or consent of Education Department Chair. (0.5 course credit) (Offered
  Fall Term)\\
\item
  \textbf{EDU 305 Methods of Elementary Science} Helps future teachers
  develop an integrated set of perspectives, attitudes, and skills,
  enabling them to give positive support to their students' natural
  inclination to be curious, manipulate, observe, and interpret.
  Multiiple instructional strategies and methods will be studied in an
  outside of class sessions. Integration of relevant children's
  literature is included. Prerequisite: Practicum in Education (EDU-215)
  or consent of Education Department Chair. (Offered Fall Term)\\
\item
  \textbf{EDU 310 Meth Secondary Business Educ} Prepares students for
  student teaching placement in a secondary business classroom. Students
  explore and examine approaches to teaching and learning. Among the
  topics studied are instructional strategies, lesson design, classroom
  management and learner assessment. Students will be exposed to
  strategies for teaching high school general business courses,
  including accounting, marketing, and economics. This course includes a
  30-hour field experience. Prerequisite: Practicum in Education
  (EDU-215) or consent of Education Department Chair.\\
\item
  \textbf{EDU 311 Meth Secondary Language Arts} Prepares students for
  student teaching in a secondary language arts classroom. Students
  explore and examine approaches to teaching and learning. Among the
  topics studied are instructional strategies, lesson design, and
  learner assessment. Students also experience working with texts such
  as YA literature, podcasts, novels, poetry, drama, etc.. The course
  includes a 30-hour field experience. Prerequisite: Practicum in
  Education (EDU-215) or consent of Ecducation Department Chair.\\
\item
  \textbf{EDU 312 Meth Secondary Social Studies} Prepares students for
  student teaching in a secondary social studies classroom. Students
  explore and examine approaches to teaching and learning. Among the
  topics studied are instructional strategies, lesson design, classroom
  management and learner assessment. This course includes a 30-hour
  field experience. Prerequisite: Practicum in Education (EDU-215) or
  consent of Education Department Chair. (Offered Fall Term)\\
\item
  \textbf{EDU 313 Methods Secondary Mathematics} Prepares students for
  student teaching in a secondary mathematics classroom. Students
  explore and examine approaches to teaching and learning. Among the
  topics studied are instructional strategies, lesson design, classroom
  management and learner assessment. This course includes a 30-hour
  field experience. Prerequisite: Practicum in Education (EDU-215) or
  consent of Education Department Chair.\\
\item
  \textbf{EDU 320 Methods of Secondary Science} Prepares students for
  student teaching in a secondary science classroom. Students explore
  and examine approaches to teaching and learning. Among the topics
  studied are instructional strategies, lesson design, classroom
  management and learner assessment. This course includes a 30-hour
  field experience. Prerequisite: Practicum in Education (EDU-215) or
  consent of Education Department Chair. (Offered Fall Term)\\
\item
  \textbf{EDU 321 Methods of World Language K-12} Prepares students for
  student teaching in a world language classroom. Students explore and
  examine approaches to teaching and learning. Among the topics studied
  are instructional strategies, lesson design, classroom management and
  learner assessment. This course includes a 30-hour field experience.
  Prerequisite: Practicum in Education (EDU-215) or consent of Education
  Department Chair.\\
\item
  \textbf{EDU 335 Methods of Elementary Mathematics} Examines basic
  concepts of curriculum and instruction for elementary school
  mathematics using the framework of the National Council of Teachers of
  Mathematics and the Iowa Core Curriculum, This course assists
  prospective teachers in developing effective instructional skills that
  foster problem solving abilities and the conceptual and procedural
  knowledge of mathematics. Instruction includes integration of
  math-related children's literature. A field experience of 30 contact
  hours is included. Prerequisites: Mathematics Comprehension for
  Teaching (EDU-275), Practicum in Education (EDU-215) or consent of
  Education Department Chair. (Offered Spring Term)\\
\item
  \textbf{EDU 345 Methods of Elementary Language Arts} Examines
  strategies for teaching the interrelated aspects of reading, writing,
  listening, speaking, spelling, and creative drama as they apply to the
  functional language of elementary school children. Assessing student
  literacy competence and planning for instruction are explored.
  Prerequisites: Concurrent registration in Teaching Reading (K-6)
  (EDU-355) and Practicum in Education (EDU-215) or consent of Education
  Department Chair. (Offered Spring Term)\\
\item
  \textbf{EDU 355 Methods of Elementary Reading} Examines the five
  components of reading-phonological awareness, phonics, fluency,
  comprehension and vocabulary. Instruction is provided in the
  strategies for mastering pre-reading, decoding, and comprehension
  skills. Methods of organizing, maintaining, and evaluating reading
  programs are addressed. Substantial study of children's literature is
  included. A field experience of 30 contact hours provides an
  opportunity to apply the content of the course. Prerequisites:
  Concurrent registration in Teaching Language Arts (K-6) (EDU-345).
  Practicum in Education (EDU-215) or consent of Education Department
  Chair. (Offered Spring Term)\\
\item
  \textbf{EDU 365 Methods of Elementary Social Studie} Examines the
  content and teaching strategies that collectively form the scope and
  sequence of elementary school social studies. Geography and ecological
  anthropology provide a framework for integrating social science
  disciplines with other subject matter fields, such as children's
  literature and science. Social issues are investigated on three
  levels: in relation to self, one's immediate environment, and the
  global ecosystem. Prerequisite: Practicum in Eduation (EDU-215) or
  consent of Education Department Chair. (Offered Fall Term)\\
\item
  \textbf{EDU 395 Diagnostic Reading \& Tutorial} Intensively studies
  diagnostic and screening instruments as well as multiple strategies to
  improve reading and writing skills. This course includes direct
  application of assessment and instructional strategies with individual
  children through a supervised tutorial outside of regular class time.
  Prerequisites: admission to the Education Program, Foundations of
  Reading (EDU-241), and Content Area Reading (EDU-260). (Offered Spring
  Term)\\
\item
  \textbf{EDU 444 Ind Study-Tchr Ed} Extensively studies selected
  problems in the teaching of elementary and secondary school subjects
  under the direction of a faculty member of the department. May be
  taken for an X status grade with consent of instructor prior to
  registration. Prerequisite: consent of Education Department Chair.\\
\item
  \textbf{EDU 454 Research Participation} Provides students engagement
  in educational research. Students should consult members of the
  department to determine projected programs. May be taken for an X
  status grade with consent of instructor prior to registration.
  Prerequisite: consent of Education Department Chair.\\
\item
  \textbf{EDU 494 Internship in Education} Explores a career area
  related to schools supervised by a faculty member of the department. A
  minimum of 140 hours on-site experience is required. A reflective
  paper analyzing the experience is required. May be completed during
  any term in which schools are in session. May include teaching,
  administrative, and/or counseling duties. S/U basis only. This course
  does not satisfy any of the requirements of the Education Program.
  Prerequisites: Foundations of Education (EDU-105) and consent of
  Education Department Chair.
\end{itemize}

\section{English}\label{sec-english}

Aspengren, Ferguson, Hausknecht, LeMay, McQueen, Shaw (Chair), Sodeman,
Valderrama

The study of English allows students to explore a wide range of literary
and cultural productions from varied parts of the world. Our courses
develop habits of mind that are fundamental to liberal education,
including the ability to read critically, to think creatively, and to
write clearly and with verve. In consultation with their advisors,
English majors are able to design personalized programs of study that
attend to the historical and geographical range of literatures in
English. Students considering graduate work in literature should consult
with English faculty to plan a rigorous course of study that would
typically include an honors thesis and coursework in a world language.

\subsection{English Major}\label{english-major}

A major in English requires a minimum cumulative 2.0 GPA in all courses
counted toward the major. 1. ENG 111 Introduction to Literary Studies 2.
ENG 281 Literatures in English to 1800 3. ENG 291 Literatures in English
after 1800 4. ENG 301 The Art of Literary Research 5. Three English
courses numbered 305 or above, one of which must be in British
literature before 1800. 6. Two additional English course credits, one of
which may be in Creative Writing or in French, or Spanish literature in
translation (FRE 145 French Literature in Translation, FRE 146 French
Literature Translation:NWP, FRE 148 French Literature Translation:DWP,
SPA 148 Spanish Literature in Translation) 7.ENG 464 Seminar in
Literature \emph{OR} ENG 467 Seminar inLit:USPluralism

\subsection{English Minor}\label{english-minor}

The minor in English consists of five course credits, including ENG 111
Introduction to Literary Studies and \emph{two} English courses numbered
300 or above. One course in Creative Writing (CRW-) or in French, or
Spanish literature in translation (FRE 145 French Literature in
Translation, FRE 146 French Literature Translation:NWP, FRE 148 French
Literature Translation:DWP, SPA 148 Spanish Literature in Translation)
may count toward the five course credits.

\subsection{Courses in Literature}\label{courses-in-literature}

\begin{itemize}
\tightlist
\item
  \textbf{ENG105 Composition I} None\\
\item
  \textbf{ENG 107 Exploring Literature:US Pluralism} Explores works by
  writers from one or more subgroups in American culture with the
  purposes of stimulating the appreciation of literary art and
  considering the various functions of literature in the contemporary
  world.\\
\item
  \textbf{ENG105 Composition I} None\\
\item
  \textbf{ENG 108 Exp Lit:Diverse Western Perspective} Explores works by
  writers from one or more subgroups in Western Civilization outside the
  United States with the purposes of stimulating the appreciation of
  literary art and considering the various functions of literature in
  the contemporary world.\\
\item
  \textbf{ENG 110 Ancient Mythology} Study of the myths of the ancient
  Greeks and Romans, while briefly touching on their Near Eastern
  predecessors as a way of investigating the character of myth and the
  purpose it plays in society. We examine these myths not only through
  the ancient texts, but by studying theoretical models and approaches
  to mythology.\\
\item
  \textbf{ENG 111 Introduction to Literary Studies} The Art of Reading
  and Writing (WE). Study of selected works of fiction, poetry, and
  drama with an emphasis on close reading and expository writing. This
  course introduces terms and skills necessary for further literary
  study.\\
\item
  \textbf{ENG 112 Environmental Humanities} Introduces fundamental
  issues, questions, and methods relating to humans and our environment.
  This course explores how human cultures may participate in the
  response to environmental challenges through an analysis of various
  literary works and other forms of cultural production.\\
\item
  \textbf{ENG 115 The Classical Tradition} Study of ancient Greek,
  Roman, or other classical literature, read in translation. This course
  focuses on the diverse genres of classical literature, including epic,
  lyric, and drama, while attending to its continuities with medieval
  continental literature.\\
\item
  \textbf{ENG 117 Asian American Literature} Reading and discussion of
  literature by Asian Americans and an introduction to its literary,
  cultural, and historical context.\\
\item
  \textbf{ENG 127 Social Justice and Literature} Study of literary works
  that represent and reimagine issues of human rights in the twentieth
  and twenty-first centuries.\\
\item
  \textbf{ENG 137 African American Literature} Reading and discussion of
  the writings of African Americans, with emphasis on the twentieth
  century. May include some relevant writings on African Americans by
  other groups. Study of the artistic values and of the social and
  cultural significance of these writings. May be taken more than once,
  with consent of African American Studies administrative coordinator,
  provided the topics are substantially different.\\
\item
  \textbf{ENG 146 Intro Postcolonial Literature} Study of
  twentieth-century prose and poetry arising out of the cultural,
  social, economic, and legal harms of imperialism and colonization.\\
\item
  \textbf{ENG 157 Latinx/Chicanx Literature} Explores the lived
  experiences and cultures of Latinx and Chicanx communities in the U.S.
  through fiction, poetry, and non-fiction by Latina/os and Chicana/os.
  This course examines cultural works from dominant U.S. sub-groups---
  such as Mexican-Americans---and from communities with roots in South
  America, Central America, and the Caribbean. It interprets these works
  in relation to cultural, historical, and sociopolitical contexts.\\
\item
  \textbf{ENG 175 Contemporary Literature} Study of American or British
  literature from 1945 to the present.\\
\item
  \textbf{ENG 206 Gender and Literature: NWP} Examination of a
  particular author, theme, region, or genre in the context of gender
  and sexuality studies. Course focuses on topics related to non-western
  perspectives, such as women in the global south and the role that
  African American literature plays within different regions of the
  world.\\
\item
  \textbf{ENG 207 Gender \& Lit:US Pluralism} Examination of a
  particular author, theme, region, or genre in the context of gender
  and sexuality studies. Course focuses on topics related to United
  States pluralism, such as American Women Writers or Gender and Race in
  American Literature.\\
\item
  \textbf{ENG 208 Gender \& Lit:DWP} Examination of a particular author,
  theme, region, or genre in the context of gender and sexuality
  studies. Course focuses on topics related to Western cultural
  diversity, such as Gender Identity in Literature, Renaissance Women
  Writers, Women's Autobiography.\\
\item
  \textbf{ENG 281 Literatures in English to 1800} Charts literary
  developments and transformations before 1800 in relation to changing
  historical conditions, from the history of books and writing to the
  wider histories of social, political, and cultural movements. An
  excursion into the literary, social, and cultural histories of the
  English speaking world, this course may include works by Bristish,
  American, or Anglophone writers.\\
\item
  \textbf{ENG 291 Literatures in English after 1800} Charts literary
  developments and transformations after 1800 in relation to changing
  historical conditions, from history of books and writing to the wider
  histories of social, political, and cultural movements. An excursion
  into the literary, social, and cultural histories of the English
  speaking world, this course may include works by British, American, or
  Anglophone writers.\\
\item
  \textbf{ENG 301 The Art of Literary Research} Practice in literary
  research, with particular attention to the varied critical and
  theoretical approaches necessary for advanced study. Prerequisite:
  Introduction to Literary Studies: The Art of Reading and Writing
  (ENG-101).\\
\item
  \textbf{ENG 312 Studies in Environmental Humanities} Studies selected
  works by major environmental writers, filmmakers, philosophers, and
  activists. Students study environmental texts in their ecological,
  historical, and cultural contexts and uses the interdisciplinary,
  justice-oriented methods of the environmental humanities to analyze
  environmental challenges and our cultural responses to them. Specific
  topics vary from year to year.\\
\item
  \textbf{ENG 327 Literature of American Renaissance} Study of
  literature from 1830 to the Civil War or of an author, topic, or genre
  of the period. Prerequisite: Introduction to Literary Studies: The Art
  of Reading and Writing (ENG-101) or consent of instructor.\\
\item
  \textbf{ENG 337 American Realism \& Naturalism} Study of literature
  from the Civil War to World War I, or of an author, topic, or genre of
  the period. Prerequisite: Introduction to Literary Studies: The Art of
  Reading and Writing (ENG-101) or consent of instructor.\\
\item
  \textbf{ENG 347 Study in Modern or Contemp Amer Lit} Study of novels,
  short fiction, and poetry by American writers of the twentieth and
  twenty-first centuries or of an author, topic, or genre of the period.
  Prerequisite: Introduction to Literary Studies: The Art of Reading and
  Writing (ENG-101) or consent of instructor.\\
\item
  \textbf{ENG 357 Studies in Latinx/ChicanxLiterature} Reviews selected
  works of literature by Latina/os and Chicana/os. Students explore
  through reading and discussion methods and concepts relating to human
  rights, coloniality, diaspora, Latinidad and/or linguistic philosophy
  to better understand the lived experiences, identities, and cultures
  of Latinx and Chicanx communities in the U.S. Specific topics vary
  from year to year.\\
\item
  \textbf{ENG 367 Studies in African Am Literature} See also African
  American Studies (AAM-367), Course focuses on specific literary genres
  or time periods and functions as an intermediate course between the
  introductory African American Literature (AAM/ENG-267) and the
  advanced English seminar (particularly Seminar in Literature: United
  States Pluralism (ENG-707)). Students read and discuss both primary
  and secondary sources in an effort to gain a fuller understanding and
  appreciation of the artistic values and of the social and cultural
  significance of these writings. Specific topics vary from year to
  year. Possible topics include African American Historical Fiction,
  African American Speculative Fiction, Literature of the Civil Rights
  Era, and the African American Bildungsroman. May be taken more than
  once, with consent of African American Studies administrative
  coordinator, provided the topics are substantially different.
  Prerequisite: Introduction to Literary Studies: The Art of Reading and
  Writing (ENG-101) or consent of instructor.\\
\item
  \textbf{ENG 375 Study in European Literature} Study of selected works
  in English or translation by major European writers. Prerequisite:
  Introduction to Literary Studies: The Art of Reading and Writing
  (ENG-101) or consent of instructor.\\
\item
  \textbf{ENG 378 Studies in Transatlantic Literature} Study of selected
  works in English from the Atlantic world. This course examines the
  diverse literatures, politics, and trade relations of the Atlantic
  world and considers how transatlantic relations alter the ways we read
  and understand national literatures. Prerequisite: Introduction to
  Literary Studies: The Art of Reading and Writing (ENG-101) or consent
  of instructor.\\
\item
  \textbf{ENG 380 The Age of Chaucer} Reading and discussion of
  literature of the later Middle Ages, with some emphasis on the work of
  Geoffrey Chaucer. This course may count toward the upper-level early
  British literature requirement for the English major. Prerequisite:
  Introduction to Literary Studies: The Art of Reading and Writing
  (ENG-101) or consent of instructor.\\
\item
  \textbf{ENG 382 Shakespeare:Com/Rom} Reading, viewing, and discussion
  of comedies and romances spanning Shakespeare's career. This course
  may count toward the upper-level early British literature requirement
  for the English major. Prerequisite: Introduction to Literary Studies:
  The Art of Reading and Writing (ENG-101) or consent of instructor.\\
\item
  \textbf{ENG 383 Shakespeare:Tragedies \& Histories} Reading, viewing,
  and discussion of history plays and tragedies, with some emphasis on
  the middle period of Shakespeare's career, including the major
  tragedies. This course may count toward the upper-level early British
  literature requirement for the English major. Prerequisite:
  Introduction to Literary Studies: The Art of Reading and Writing
  (ENG-101) or consent of instructor.\\
\item
  \textbf{ENG 384 British Renaissance Literature} Study of the
  development of English literature in the sixteenth and seventeenth
  centuries. Typically the focus is on either poetry or drama. This
  course may count toward the upper-level early British literature
  requirement for the English major. Prerequisite: Introduction to
  Literary Studies: The Art of Reading and Writing (ENG-101) or consent
  of instructor.\\
\item
  \textbf{ENG 385 Restoration \&18th Cntry British Lit} Study of major
  works from 1660 to the end of the eighteenth century or of an author,
  topic, or genre of the period. This course may count toward the
  upper-level early British literature requirement for the English
  major. Prerequisite: Introduction to Literary Studies: The Art of
  Reading and Writing (ENG-101) or consent of instructor.\\
\item
  \textbf{ENG 388 Romantic Literature} Study of major works from 1780 to
  1830, with emphasis on writings by Blake, Wollstonecraft, Wordsworth,
  Coleridge, Austen, Byron, Keats, P. Shelley, and M. Shelley. This
  course may count toward the upper-level early British literature
  requirement for the English major. Prerequisite: Introduction to
  Literary Studies: The Art of Reading and Writing (ENG-101) or consent
  of instructor.\\
\item
  \textbf{ENG 390 Victorian Literature} Study of poetry and prose of the
  Victorian era from 1832 through the fin-de-siècle, including such
  writers as Tennyson, Browning, Barrett Browning, Arnold, Mill, Ruskin,
  C. Rossetti, D. G. Rossetti, Swinburne, Hopkins, and Wilde.
  Prerequisite: Introduction to Literary Studies: The Art of Reading and
  Writing (ENG-101) or consent of instructor.\\
\item
  \textbf{ENG 392 19th Century British Novel} Study of major British
  novelists such as Austen, Scott, Dickens, C. Brontë, E. Brontë, Eliot,
  Collins, and Hardy. Prerequisite: Introduction to Literary Studies:
  The Art of Reading and Writing (ENG-101) or consent of instructor.\\
\item
  \textbf{ENG 393 Study in Cont/Mod British Fiction} Study of novels,
  short fiction, and poetry by British writers of the twentieth and
  twenty-first centuries, or of an author, topic, or genre of the
  period. Prerequisite: Introduction to Literary Studies: The Art of
  Reading and Writing (ENG-101) or consent of instructor.\\
\item
  \textbf{ENG 394 Directed Learning in English} Study of individually
  chosen topics in literature or execution of projects in writing under
  the direction of a faculty member of the department. Registration only
  after the instructor has approved a written proposal for the project.
  Prerequisite: three courses in literature.
\end{itemize}

\subsection{ADVANCED STUDY IN ENGLISH}\label{advanced-study-in-english}

Students registering for these courses must fulfill the
\textbf{prerequisites} first and should consult with English department
faculty if they have questions.

\begin{itemize}
\tightlist
\item
  \textbf{ENG 454 Honors Research} Development and completion of an
  honors thesis under the direction of a faculty member of the
  department. Normally taken in Fall Term of the senior year for an X
  status grade. Prerequisites: Seminar in Literature (ENG-464) or
  Seminar in Literature: United States Pluralism (ENG-467) and consent
  of instructor.\\
\item
  \textbf{ENG 464 Seminar in Literature} Advanced study of a specialized
  topic in literature. As a capstone experience, the seminar fosters
  student-driven inquiry and requires students to present their
  work-in-progress and to complete a culminating project (typically, an
  extensive research paper). To be taken at or near the end of the
  English major. May be counted toward the upper-level early British
  literature requirement for the English major when topic is
  appropriate. May be taken more than once for credit toward the English
  major. Prerequisites: junior standing, The Art of Literary Research
  (ENG-301), and an English course numbered 305 or above.
\item
  \textbf{ENG 467 Seminar inLit:USPluralism} Advanced study of a
  specialized topic in literature. As a capstone experience, the seminar
  fosters student-driven inquiry and requires students to present their
  work-in-progress and to complete a culminating project (typically, an
  extensive research paper). To be taken at or near the end of the
  English major. May be taken more than once for credit toward the
  English major. Prerequisites: junior standing, The Art of Literary
  Research (ENG-301), and an English course numbered 305 or above.
\item
  \textbf{ENG 494 Internship in English} Exploration of a career area
  related to English. Application and supervision through the Internship
  Specialist. A minimum of 140 hours on-site experience is required. S/U
  basis only. This course does not satisfy any of the requirements for a
  major in English, but with consent of rhetoric department chair, one
  credit from an appropriate internship may satisfy the requirements for
  a minor in writing. Prerequisites: junior standing and consent of
  department chair.
\end{itemize}

\section{Environmental Science (Collateral
Major)}\label{sec-environmental-science}

St.~Clair, Sanchini (Administrative Coordinators).

\subsection{Collateral Major in Environmental
Science}\label{collateral-major-in-environmental-science}

A major in environmental science requires a minimum cumulative 2.0 GPA
in all courses counted toward the major. Concurrent completion of a
primary major in biology or chemistry is required.\\
1. \textbf{One} of the following combinations: Biology majors: MTH 135
Calculus I BIO 445 Environmental Microbiology and +++MISSING INFO:
c.bio445L.long +++ \textbf{or} Chemistry majors: MTH 145 Calculus II PHY
185 General Physics I and +++MISSING INFO: c.phy185L.long +++ PHY 195
General Physics II and +++MISSING INFO: c.phy195L.long +++ BIO 445
Environmental Microbiology and +++MISSING INFO: c.bio445L.long +++ 2.
BIO 295 Spatial Ecology and +++MISSING INFO: c.bio295L.long +++ 3. CHM
211 Analytical Chemistry and +++MISSING INFO: c.bio211L.long +++ 4. CHM
221 Organic Chemistry I 5. \textbf{Two} of the following with the
associated laboratories: BIO 115 Marine Biology BIO 165 Ecology and
Biology of Birds BIO 175 Field Botany BIO 185 Entomology BIO 275 Aquatic
Ecology BIO 325 Microbiology BIO 385 Behavior/Ecology of Vertebrates
+++MISSING INFO: c.bio411.long +++ BIO 425 Ecology BIO 444 Independent
Study Any course taught at the Wilderness Field Station with BIO prefix.

\emph{Recommended:} Any course taught at the Wilderness Field Station
BIO105 Introductory Biology MTH 145 Calculus II RHE 257 Environmental
Rhetoric STA 100 Statistical Reasoning I-Foundations \textbf{and} STA
110 Stats IIA: Inferential Reasoning STA 100 Statistical Reasoning
I-Foundations \textbf{and} STA 130 Stats IIB: Experimental Design

\textbf{NOTE:} \emph{Students should select courses from the economics,
political science, and philosophy departments as part of their general
education program.}

\section{Environmental Studies (Collateral
Major)}\label{sec-environmental-studies}

St.~Clair (Administrative Coordinator).

The Environmental Studies Major is a collateral, interdisciplinary
program which requires students to study environmental issues using the
techniques and perspectives from a variety of academic disciplines. In
the sophomore or junior year, a student intending to complete an
environmental studies collateral submits a proposal to the Environmental
Studies administrative coordinator, outlining the plan of study for the
major and describing plans for independent research, internship, or
off-campus study. Students seeking this collateral major should submit
this document prior to enrollment in EVS-484 Topics in Environmental
Studies. In addition to the requirements listed below, before
undertaking a practicum experience, each student is strongly encouraged
to consult with the Environmental Studies administrative coordinator to
identify a practicum experience that supports his or her study of
environmental issues. Students choosing a collateral major in
environmental studies may not select the collateral major in
environmental science.

\subsection{Collateral Major in Environmental
Studies}\label{collateral-major-in-environmental-studies}

A major in environmental studies requires a minimum cumulative 2.0 GPA
in all courses counted toward the major. Concurrent completion of any of
the majors listed under
\hyperref[Areasux5cux2520Ofux5cux2520Study]{Areas of Study} of the
Catalog is required.

\begin{enumerate}
\def\labelenumi{\arabic{enumi}.}
\tightlist
\item
  BIO105 Introductory Biology
\item
  BIO 155 Organismal \& Ecological Biology and +++MISSING INFO:
  c.bio155L.long +++
\item
  ECO 175 Principles of Macroeconomics
\item
  ECO 195 Prin of Environmental Economics
\item
  \textbf{One} of the following: +++MISSING INFO: c.evs112.long +++ PHL
  205 Environmental Ethics
\item
  \textbf{One} of the following: BUS 190 Statistical Analysis BUS 340
  Applied Regression Analysis PSY 300 Stat Methods and Data Analysis SOC
  235 Methods of Sociological Research STA 100 Statistical Reasoning
  I-Foundations \textbf{and} STA 110 Stats IIA: Inferential Reasoning
  STA 100 Statistical Reasoning I-Foundations \textbf{and} STA 130 Stats
  IIB: Experimental Design STA 315 Mathematical Probability
\item
  EVS 484 Topics in Environmental Studies
\item
  \textbf{Four} courses to be chosen as indicated from the three lists
  below. Courses with a substantial focus on environmental content and
  relevant to a student's particular course of study may be substituted
  to fulfill this category, subject to prior approval by the
  Environmental Studies administrative coordinator.

  \begin{enumerate}
  \def\labelenumii{\alph{enumii}.}
  \tightlist
  \item
    \textbf{One or more} of the following \textbf{Natural Science}
    courses: BIO 165 Ecology and Biology of Birds BIO 175 Field Botany
    BIO 185 Entomology BIO 285 Animal Behavior (\textbf{NOTE:}
    \emph{Taught at the Wilderness Field Station}) BIO 295 Spatial
    Ecology and +++MISSING INFO: c.bio295L.long +++ CHM 103 Selected
    Concepts in Chemistry and +++MISSING INFO: c.chm103L.long +++
    \textbf{OR} CHM 121 General Chemistry I and +++MISSING INFO:
    c.chm121L.long +++
  \item
    \textbf{One or more} of the following \textbf{Social Science}
    courses: BUS 170 Bus Sustainability \& Environment ECO 155
    Econ/Ethics of Alternative Energy ECO 215 Prin Microeconomics
    +++MISSING INFO: c.evs112.long +++ (if not used to satisfy \#5)
    +++MISSING INFO: c.evs137.long +++ POL 115 American National Gov \&
    Pol POL 386 International Development SOC 328 Urban Sociology
  \item
    \textbf{One or more} of the following \textbf{Humanities} courses:
    PHL 205 Environmental Ethics (if not used to satisfy \#5) RHE 135
    Writers Colony (when topic is appropriate to major) RHE 257
    Environmental Rhetoric RHE 345 Writing Wilderness
  \end{enumerate}
\end{enumerate}

\subsection{Courses in Environmental Studies (Collateral
Major)}\label{courses-in-environmental-studies-collateral-major}

\begin{itemize}
\tightlist
\item
  \textbf{BUS 170 Bus Sustainability \& Environment} Examines the global
  trend of the increasing attention organizations are giving to
  environmental and sustainability issues, including evaluation of
  successful business practices being pursued by leading corporations.
  Consideration is given to why and how corporations are embracing such
  trends. Topics covered include eco-labeling, corporate environmental
  strategies, NGO-business partnerships, and emerging markets for
  environmental goods and services. Significant attention is given to
  global warming concerns and the emerging strategies for measuring and
  reducing the carbon footprint of business.
\item
  \textbf{ECO 195 Prin of Environmental Economics} An introduction to
  the economics of the environment and natural resource allocation. This
  course is focused on optimal resource allocation and the problems
  associated with externalities and public goods in the context of a
  market economy. Environmental issues and policy are analyzed using
  standard economic models. This course may be used to satisfy the
  requirements for a major in Environmental Studies but does not satisfy
  any of the requirements for a major or minor in economics.
  Prerequisite: Principles of Macroeconomics (ECO-175).
\item
  \textbf{+++MISSING INFO: c.evs112.long +++} +++MISSING INFO:
  c.evs112.desc +++
\item
  \textbf{+++MISSING INFO: c.evs137.long +++} +++MISSING INFO:
  c.evs137.desc +++
\item
  \textbf{EVS 484 Topics in Environmental Studies} An intensive
  examination of selected works and subjects dealing with environmental
  issues. Specific topics vary from year to year. Prerequisite: junior
  standing or permission of instructor. May be taken more than once for
  credit with permission of instructor.
\end{itemize}

\section{Film Studies}\label{sec-film-studies}

Cohen (Co-Coordinator), Lausch, Rogers (Co-Coordinator).

The film studies major is an interdisciplinary major that allows a
student to focus on the particular aspect of film studies that interests
her or him most -- for example, film writing, cinema studies, or digital
production.

\subsection{Film Studies Major}\label{film-studies-major}

A major in film requires a minimum cumulative 2.0 GPA in all courses
counted toward the major. A major in film studies requires ten courses
taken from each of the three disciplines represented in the major. 1.
ART 150 Time Based Media 2. FLM 105 Introduction to Film and +++MISSING
INFO: c.flm105L.long +++ 3. FLM 200 Film Analysis and +++MISSING INFO:
c.flm200L.long +++ 4. FLM 225 Film History and +++MISSING INFO:
c.flm225L.long +++ 5. FLM 464 Seminar in Film II: and FLM 474 Senior
Seminar II in Film \& Senior 6. \textbf{Five} courses chosen from the
following, in consultation with the Film Studies administrative
coordinator. At least two of the courses must be completed at the 300
level or above, and no more than two courses may be selected with the
same prefix. ART 145 Digital Studio ART 155 Photography: Light Writing
ART 325 Contemporary Photographic Genres ART 361 Documentary ART 370
Video Art and Production COM 151 Introduction to New Media Studies COM
157 Introduction to Media Analysis COM 161 Visual Rhetoric COM 337
Persuasion COM 341 Digital Storytelling COM 357 Sex, Race, \& Gender in
Media CRW 255 Playwriting Workshop I or THE 255 Playwriting Workshop I
CRW 350 Screenwriting FLM 250 Film Topics FLM 350 Advanced Topics in
Film FLM 442 Independent Study in Film FLM 494 Internship in Film MU 140
Film Music MU 195 Music Production RHE 137 Creative Nonfiction
U.S.-Pluralism RHE 146 Creative Nonfiction:Global Perspect RHE 200
Rhetorical Theory and Practice RHE 255 The Essay RHE 257 Environmental
Rhetoric THE 130 Technical Production I THE 140 Design for the Stage THE
220 Tech Theatre Lab THE 290 Directing I

\subsection{Film Studies Minor}\label{film-studies-minor}

A minor in film studies requires six courses chosen in consultation with
the Film Studies administrative coordinator. 1. ART 150 Time Based Media
2. FLM 105 Introduction to Film and +++MISSING INFO: c.flm105L.long +++
3. FLM 200 Film Analysis and +++MISSING INFO: c.flm200L.long +++ 4. FLM
225 Film History and +++MISSING INFO: c.flm225L.long +++ 5. \textbf{One}
of the following: ART 145 Digital Studio ART 155 Photography: Light
Writing ART 325 Contemporary Photographic Genres ART 361 Documentary ART
370 Video Art and Production COM 151 Introduction to New Media Studies
COM 157 Introduction to Media Analysis COM 161 Visual Rhetoric CRW 255
Playwriting Workshop I or THE 255 Playwriting Workshop I MU 140 Film
Music MU 195 Music Production RHE 137 Creative Nonfiction U.S.-Pluralism
RHE 146 Creative Nonfiction:Global Perspect RHE 200 Rhetorical Theory
and Practice RHE 255 The Essay RHE 257 Environmental Rhetoric THE 130
Technical Production I THE 140 Design for the Stage THE 220 Tech Theatre
Lab THE 290 Directing I 6. \textbf{One} of the following: CRW 350
Screenwriting COM 337 Persuasion COM 341 Digital Storytelling COM 357
Sex, Race, \& Gender in Media FLM 350 Advanced Topics in Film FLM 442
Independent Study in Film FLM 494 Internship in Film

\subsection{Courses in Film Studies}\label{courses-in-film-studies}

\begin{itemize}
\tightlist
\item
  \textbf{+++MISSING INFO: c.flm105/flm105L.long +++} +++MISSING INFO:
  c.flm105/flm105L.desc +++
\item
  \textbf{+++MISSING INFO: c.flm200/flm200L.long +++} +++MISSING INFO:
  c.flm200/flm200L.desc +++
\item
  \textbf{FLM 225 Film History} Familiarizes students with the history
  of film from the beginning to the present. Additional viewing time
  outside of class is required.
\item
  \textbf{FLM 250 Film Topics} Includes a brief introduction to film
  analysis, but focuses on a specific topic such as: adaptations, a
  genre, a period, an individual director, a studio. May be repeated,
  with consent of instructor, provided the topics are substantially
  different.
\item
  \textbf{FLM 350 Advanced Topics in Film} Advanced study of a selected
  topic or method in cinematic production. Example topics: Documentary,
  Interactive Multimedia, Motion Graphics, Advanced Editing, Producing,
  and Directing. May be taken more than once for credit, provided the
  topics are substantially different. Prerequisities: FLM-105
  Introduction to Film and ART-170 Time-Based Art I.
\item
  \textbf{FLM 442 Independent Study in Film} Independent Study in Film
  Independent work on a selected project under the direction of a
  faculty member of the department. Prerequisites: Previous or
  concurrent enrollment in Movement (ART 170), Introduction to Film (FLM
  105), Film Analysis (FLM 200), Film History (FLM 225). (Offered by
  arrangement)
\item
  \textbf{FLM 464 Seminar in Film II:} Emphasis is on preparation of
  work toward the senior capstone project. Only film majors are admitted
  to this course. Prerequisite: successful completion of Senior Seminar
  I (FLM-444).
\item
  \textbf{FLM 474 Senior Seminar II in Film \& Senior} Emphasis is on
  preparation of work toward the senior capstone project. Only fillm
  majors are admitted to this course. Prerequisite: successful
  completion of Senior Seminar I (FLM-454).
\item
  \textbf{FLM 494 Internship in Film} An internship with a focus on film
  production. A minimum of 140 hours on-site experience is required. S/U
  basis only. Prerequisite: junior standing and consent of the Film
  Studies administrative coordinator.
\end{itemize}

\section{French \& Francophone
Studies}\label{sec-french-and-francophone-studies}

Janca-Aji

The French \& Francophone Studies program is an intercultural and
interdisciplinary program featuring courses in language, cultural
history, literature and cinema, translation and interpretation, and
pre-professional studies. Students are strongly encouraged to pursue
opportunities for immersive and experiential learning through study
abroad, May Term courses, service learning, and community-based projects
and to explore ways to incorporate French in and with other major(s).

\subsection{French \& Francophone Studies
Major}\label{french-francophone-studies-major}

A grade of ``C'' (2.0) or higher must be earned in all courses counted
toward a major in French \& Francophone Studies. Students complete eight
credits of 300- to 400-level courses in French. FRE-315 Oral and Written
Communication Skills (WE) is required. Up to three credits may be earned
by successfully completing a study abroad program in France or a
francophone country that is approved by the College and the department.
Up to one credit may be earned from a list of approved courses taught in
English. One credit from a course taught in French must be taken in the
senior year. FRE-499 Exit Exam and Interview is required during the
final term before graduation.

\subsection{French \& Francophone Studies
Minor}\label{french-francophone-studies-minor}

A grade of ``C'' (2.0) or higher must be earned in all courses counted
toward a minor in French \& Francophone Studies. Students complete a
minimum of four credits of 300- to 400-level courses in French. FRE-315
Oral and Written Communication Skills (WE) is required. Up to one credit
may be earned from a list of approved courses taught in English. FRE-499
Exit Exam and Interview is required during the final term before
graduation.

\subsection{Interdisciplinary French \& Francophone Studies
Major}\label{interdisciplinary-french-francophone-studies-major}

A grade of ``C'' (2.0) or higher must be earned in all courses counted
toward a major in Interdisciplinary French \& Francophone Studies.
Students complete 1) \textbf{four} credits of courses taught in French
at any level, including FRE-315 Oral and Written Communication Skills
(WE), 2) a departmentally approved term-long study abroad experience in
France or a Francophone country, and 3) \textbf{four} credits from
courses, taught in either French or English, from the list of approved
courses which include at least two different prefixes and demonstrate
thematic coherence. Courses not on this list may count for credit with
approval of the program coordinator. FRE-499 Exit Exam and Interview is
required during the final term before graduation.

\subsection{Interdisciplinary French \& Francophone Studies
Minor}\label{interdisciplinary-french-francophone-studies-minor}

A grade of ``C'' (2.0) or higher must be earned in all courses counted
toward a minor in Interdisciplinary French \& Francophone Studies.
Students complete 1) \textbf{four} credits of courses taught in French
at any level, including FRE-315 Oral and Written Communication Skills
(WE), and 2) \textbf{three} credits from courses, taught in either
French or English, from the list of approved courses which include at
least two different prefixes and demonstrate thematic coherence. Courses
not on this list may count for credit with approval of the program
coordinator. FRE-499 Exit Exam and Interview is required during the
final term before graduation.

\subsection{Courses Taught in English that can be used for credit in
French \& Francophone
Studies}\label{courses-taught-in-english-that-can-be-used-for-credit-in-french-francophone-studies}

\begin{verbatim}
ARH 201 Art of the Middle Ages
ARH 218 The World of Renaissance Art
ARH 231 Romanticism, Realism, Impressionism
ARH 307 Modern and Contemporary Art
COM 236 Intercultural Communication
ENG 146 Intro Postcolonial Literature
FRE 145 French Literature in Translation
FRE 146 French Literature Translation:NWP
FRE 148 French Literature Translation:DWP
FRE 158 France & Francophone World
HIS 238 Modern France
HIS 248 The French Revolution
HIS 272 History of Medieval Europe
HIS 288 Renaissance & Reformation
HIS 355 19th Century Europe
HIS 365 20th Century Europe
HIS 372 Early Modern Europe
PHL 230 Medieval Philosophy
PHL 240 Early Modern Philosophy
PHL 255 Existentialism
PHL 305 Contemporary Continental Philosophy
PHL 345 Philosophy of Language
POL 298 European Politics
REL 148 Islam
REL 178 Christianity
\end{verbatim}

\subsection{Courses in French}\label{courses-in-french}

\begin{itemize}
\tightlist
\item
  \textbf{FRE 115 Elementary French I} Designed for students with no
  previous knowledge of French. In the first term, emphasis is placed on
  oral practice and exposure to aspects of contemporary French culture.
  Activities in class are designed to develop the student's proficiency
  in understanding, speaking, writing, and reading the French language.
  In the second term, these skills are developed further in the context
  of class discussion and short compositions. Note: FRE-115 is not open
  to students with one or more full years of French in secondary school
  without consent of the instructor in consultation with the student's
  advisor.
\item
  \textbf{FRE 125 Elementary French II} Designed for students with no
  previous knowledge of French. In the first term, emphasis is placed on
  oral practice and exposure to aspects of contemporary French culture.
  Activities in class are designed to develop the student's proficiency
  in understanding, speaking, writing, and reading the French language.
  In the second term, these skills are developed further in the context
  of class discussion and short compositions. Note: FRE-115 is not open
  to students with one or more full years of French in secondary school
  without consent of the instructor in consultation with the student's
  advisor.
\item
  \textbf{FRE 145 French Literature in Translation} Reading, in
  translation, of a selection of works centering on a theme, genre, time
  period, or author. Students preparing a French major or minor must
  write a paper in French. Taught in English.
\item
  \textbf{FRE 146 French Literature Translation:NWP} Same as French
  Literature in Translation (FRE-145) except selected works focus on
  francophone writers from Africa, Asia, or the Caribbean.
\item
  \textbf{FRE 148 French Literature Translation:DWP} Same as French
  Literature in Translation (FRE-145) except selected works focus on
  diverse Western perspectives: women's writing, gay and lesbian
  literature, or Quebeçois literature.
\item
  \textbf{FRE 158 France \& Francophone World} An interdisciplinary
  survey of topics and issues central to an understanding of
  contemporary France:social, cultural, political, and economic. Films
  and speakers may be included in the format of this course. Taught in
  English.
\item
  \textbf{+++MISSING INFO: c.fre199.long +++} +++MISSING INFO:
  c.fre199.desc +++
\item
  \textbf{FRE 235 Intermediate French} Reviews French grammar and
  vocabulary thoroughly in the context of an introduction to French and
  Francophone culture. Includes short readings and compositions, films,
  discussions, and immersion activities to improve language skills.
  Designed for students who have completed two semesters of French at
  Coe or with 2-4 years of high school French. Prerequisite: placement,
  Elementary French II (FRE-125) or consent of instructor.
\item
  \textbf{FRE 315 French Composition \& Conversation} Serves as the
  capstone of previous language courses, sharpens oral and written
  communication skills, and introduces students to the particular ways
  in which native speakers of French tend to express themselves through
  listening exercises, discussion, immersion activities, interviews, and
  compositions on cultural aspects of communication. This course is
  required of all students majoring or minoring in French. It is
  expected that this course be followed by Introduction to French
  Literature (FRE-335) or History of French: Language and Culture
  (FRE-3XX), in the following term. Prerequisite: Intermediate French II
  (FRE-225) or consent of instructor.
\item
  \textbf{FRE 335 Introduction to French Literature} Close reading and
  critical analysis of different genres of literary texts through
  discussion, performance, and imitation. Students learn to use literary
  theory, to write, revise, and edit longer papers, and to develop their
  own creative voices in French. Prerequisite: French Composition and
  Conversation (FRE-315) or consent of instructor.
\item
  \textbf{FRE 339 French for Health Care} Prepares students for using
  French in health and wellness contexts. Students learn essential
  vocabulary, improve written and oral communication skills, examine
  cultural differences, develop skills in translation and medical
  interpretation, and complete an independent project based on
  professional interests. May involve community engagement. Can be taken
  by arrangement. Prerequisite: Oral and Written Communication Skills
  (FRE-315) or consent of instructor.
\item
  \textbf{FRE 340 French \& Fracophone Cinema (WE)} Surveys some of the
  major genres, directors, and films of French and francophone cinema
  from the Lumière brothers to the present, as well as discussions of
  French film culture and cinema's relations to history, literature, and
  other forms of visual and media arts. Students produce their own short
  films in French. Prerequisite: Oral and Written Communication Skills
  (FRE-315) or consent of instructor.
\item
  \textbf{FRE 345 History of Fr: Language \& Culture} None
\item
  \textbf{FRE 394 Directed Learning in French} For students wishing to
  investigate a particular aspect of French literature unavailable
  through the regular sequence of courses offered. Periodic conferences
  and papers are required. May be taken more than once for credit.
  Prerequisite: Introduction to French Literature (FRE-335) or consent
  of instructor.
\item
  \textbf{FRE 400 Advanced Language Skills} Focuses on improving
  linguistic skills and deepening understanding of how French is used in
  a variety of contexts and media through intensive practice. Topics
  vary from term to term. May be taken more than once for credit for a
  maximum of 2.0 credits. Prerequisite: Oral and Written Communication
  Skills (FRE-315) or consent of instructor. (0.5 course credit)
\item
  \textbf{FRE 444 Ind Study-French} Independent investigation of a
  selected project in French under the direction of a faculty member of
  the department. May be taken for an X status grade with consent of
  instructor prior to registration. Prerequisite: consent of department
  chair.
\item
  \textbf{FRE 446 Colonial \& Multicultural Narratives} A study of the
  history of ``la francophonie'' in terms of France's colonial
  experiments, and their literary legacies, and current issues of
  immigration and multiculturalism. Includes novels and films from
  France, Cameroon, Senegal, Canada, Morocco, and Martinique.
  Prerequisite: Introduction to French Literature (FRE- 335) or History
  of French: Language and Culture (FRE-345).
\item
  \textbf{FRE 452 Adv Language Skills \& Trnsltn Wkshp} Focus on
  improving linguistic skills and deepening understanding of how French
  is used in a variety of contexts and media through intensive practice
  translating from and into French May be taken more than once for
  credit for a maximum of 3.0 credits. Prerequisite: French Composition
  and Conversation (FRE-315) or consent of instructor.
\item
  \textbf{FRE 494 Internship in French} Exploration of a career area
  related to French. Application and supervision through the Internship
  Specialist. A minimum of 140 hours on-site experience is required. S/U
  basis only. This course does not satisfy any of the requirements for a
  major in French. Prerequisites: junior standing and consent of
  department chair.
\item
  \textbf{FRE 495 Top in French/Francophone Lit \& Clt} Literature and
  culture course centering on a theme, region, time period, or genre.
  Approach and content vary from term to term as determined by the
  instructor. Topics include: Writing and Painting in 19th-century
  Paris, Contemporary Writing in French, Postcolonial Lives. May be
  taken more than once for credit for a maximum of 3.0 credits.
  Prerequisite: Introduction to French Literature (FRE- 335), or History
  of French: Language and Culture (FRE-3XX).
\item
  \textbf{FRE 499 Exit Exam and Interview} Evalutates the student's
  progress in French and experinece of the program. To be completed with
  faculty in French during the last semester before graduation. P/NP
  basis only. Prerequisite: consent of instructor. (0.0 course credit)
\end{itemize}

\section{General Science}\label{sec-general-science}

Singleton (Administrative Coordinator)

The General Science major is of particular value to students who desire
pre-professional preparation in medicine and related fields, as well as
those planning to teach in natural science areas at the secondary level.
Pre-engineering students frequently use this concentration in 3-2 plans.

\subsection{General Science Major}\label{general-science-major}

A major in general science requires a minimum cumulative 2.0 GPA in all
courses counted toward the major. Students choosing a collateral major
in the natural sciences or Neuroscience may not select the major in
General Science.

Students earning more than one major in the natural sciences or a major
and a minor in the natural sciences are not eligible for the General
Science major.

A major in general science requires a minimum of thirteen course credits
from the natural science areas of biology, chemistry, mathematics, and
physics. At least seven of these 13 course credits must be above the
introductory level and at least two course credits must be in biology.
1. \textbf{Six} course credits in one of the four natural science areas
2. \textbf{Five} course credits in a second natural science field (or
mathematics through MTH 265 Linear Algebra) 3. \textbf{Two} course
credits in a third natural science field Satisfactory completion of
comprehensive requirements in each of the major areas.

\section{Greek (Courses Only)}\label{sec-greek}

Langseth.

\subsection{Courses in Greek}\label{courses-in-greek}

\begin{itemize}
\tightlist
\item
  \textbf{CLA 155 Latin/Greek Origins Med Terminology} Examines the
  origins of contemporary medical terminology, in part by studying the
  development of a distinct technical vocabulary, with historical roots
  in the Greco-Roman, Arabic, and Modern-European worlds, which
  developed as physicians discovered distinct ways of communicating both
  with their patients and with each other.
\item
  \textbf{GRK 115 Basic Greek} An intensive examination and analysis of
  Greek grammar and syntax. Selected readings from works of the koine or
  Classical tradition. Combination of drill work, lecture, and
  discussion. Prerequisite: no prior instruction in Greek or up to two
  terms of secondary school Greek and consent of instructor.
\item
  \textbf{GRK 125 Selected Readings in Ancient Greek} Review of basic
  grammar and syntax and examination of more advanced grammar and
  syntax. In-depth readings from authors selected from the koine or
  Classical tradition. Combination of drill work, lecture, and
  discussion. Prerequisite: Basic Greek (GRK-115) or two or more terms
  of secondary school Greek and consent of instructor.
\item
  \textbf{+++MISSING INFO: c.grk284/384.long +++} +++MISSING INFO:
  c.grk284/384.desc +++
\end{itemize}

\section{Gender And Sexuality Studies (Minor
Only)}\label{sec-gender-and-sexuality-studies}

Janca-Aji (Administrative Coordinator).

The Gender and Sexuality Studies program is an interdisciplinary inquiry
into the ways in which gender and sexuality inform constructions of
identity, societies, and ideologies across race, ethnicities, class,
cultures, and historical periods. Because core courses and electives are
offered by faculty across the College, students are encouraged to meet
with the Gender and Sexuality Studies administrative coordinator for
informal advising as soon as they declare a minor.

\subsection{Gender and Sexuality Studies
Minor}\label{gender-and-sexuality-studies-minor}

A minor in gender studies requires a minimum cumulative 2.0 GPA in all
courses counted toward the minor. 1. GS 107 Intro Gender \& Sexuality
Studies 2. GS 327 Thry\&Mthds/Gender \&Sexuality Stdy 3. \textbf{Four}
electives from the lists below, with at least \textbf{three} from the
list of core courses. Other courses may count as electives when topics
are appropriate with the consent of the academic coordinator.
\textbf{Core Courses} ARH 107 Gender and Art COM 357 Sex, Race, \&
Gender in Media ENG 206 Gender and Literature: NWP ENG 207 Gender \&
Lit:US Pluralism ENG 208 Gender \& Lit:DWP GS 127 Dress, Gender, and
Identity GS 136 Gender in Non-Western World GS 247 Gender \& Sexuality
StudiesSymposium GS 387 Topics: Gender \& Sexuality Studies HIS 297
Women in America NUR 137 Human Sexuality or PSY 137 Human Sexuality PHL
277 Philosophy of Gender \& Race POL 277 Women \& Poltics in US PSY 208
Gender Psychology SOC 417 Sociology of Sex \& Sexuality **Component
Courses ANT 116 Cultural Anthropology COM 236 Intercultural
Communication COM 357 Sex, Race, \& Gender in Media COM 361
Communication \& Social Change EDU 187 Human Relations RHE 377 Cultural
Studies SCJ 350 Human Rights \& Comparative Justice SOC 207 Sociology of
the Family THE 288 History of Dress

\subsection{Courses in Gender and Sexuality
Studies}\label{courses-in-gender-and-sexuality-studies}

\begin{itemize}
\tightlist
\item
  \textbf{GS 107 Intro Gender \& Sexuality Studies} An examination of
  the ways in which societies shape our notions of gender, including,
  but not limited to, how class, race, and sexual orientation influence
  this process. Topics for lecture, discussion, and readings are
  selected by the instructor and are drawn from a variety of academic
  fields.
\item
  \textbf{GS 127 Dress, Gender, and Identity} Explores the issues of
  personal adornment and dress related to the projection of gender and
  identity. Examples from contemporary cultures around the globe are
  analyzed and compared following a variety of themes. Those themes
  include: revealing and concealing gender and/or identity, dress codes
  as social markers, body image and gender, and group identity through
  dress and adornment.
\item
  \textbf{GS 136 Gender in Non-Western World} A study of gender rituals,
  family practices, sexuality, gendered work, and other aspects of
  gender in non-western cultures.
\item
  \textbf{GS 247 Gender \& Sexuality StudiesSymposium} None
\item
  \textbf{GS 327 Thry\&Mthds/Gender \&Sexuality Stdy} Offers a thorough
  introduction to critical theories and methods in feminist and queer
  studies with particular emphasis on historical foundations, questions
  of power and discourse, intersectionalities, and global perspectives.
  A research project using feminist or queer theory is required.
  Prerequisite: Introduction to Gender and Sexuality Studies (GS-107)
  and completion of one elective that counts toward a Gender and
  Sexuality Studies major. (Offered Spring Term, alternate years)
\item
  \textbf{GS 387 Topics: Gender \& Sexuality Studies} Focuses on
  specific authors, events, or issues in feminist, queer, and
  masculinity studies. May be taken more than once, provided the topics
  are substantially different. Prerequisites: Introduction to Gender and
  Sexuality Studies (GS-107) or consent of instructor. NOTE: Students
  are encouraged to complete Theory and Methods in Gender and Sexuality
  (GS-327) before enrolling in GS-387. \textbf{NOTE:} \emph{Students are
  encouraged to complete GS 327 Thry\&Mthds/Gender \&Sexuality Stdy
  before enrolling in GS 387 Topics: Gender \& Sexuality Studies. }
\end{itemize}

\section{History}\label{sec-history}

Swenson Arnold, Buckaloo, Keenan (Chair), Nordmann, Ziskowski.

The history department allows students the opportunity to study history
broadly, while also focusing on specific interests. Students learn how
to express themselves, both orally and in writing, and, in the liberal
arts tradition, are taught to learn quickly, communicate clearly, and
build a rewarding career and life in their chosen field.

\subsection{History Major}\label{history-major}

A major in history requires a minimum cumulative 2.0 GPA in all courses
counted toward the major. 1. \textbf{Four} courses in one of these two
combinations: HIS 115 History of Europe to 1500 HIS 125 History of
Europe Since 1500 HIS 145 History of United States to 1865 \textbf{or}
HIS 155 History of United States since 1865 One additional U.S. History
course \textbf{OR} HIS 145 History of United States to 1865 HIS 155
History of United States since 1865 HIS 115 History of Europe to 1500
\textbf{or} HIS 125 History of Europe Since 1500 One additional European
History course 2. HIS 205 Historians Craft 3. \textbf{One} of the
following: HIS 136 East Asian Civilization HIS 246 History of Modern
China HIS 256 History of Modern Japan 4. \textbf{One} of the following:
HIS 465 Seminar in Ancient History HIS 466 Seminar Modern East Asian
History HIS 472 Seminar American History I HIS 473 Seminar American
History II HIS 474 Seminar Modern European History 5. \textbf{Three}
additional history courses \textbf{NOTE:} \emph{With departmental
approval, successful completion of the ACM Newberry Seminar: Research in
the Humanities Program off-campus study experience may be used to
satisfy \#4 above.}

\subsection{History Minor}\label{history-minor}

\textbf{One} of the following combinations: HIS 115 History of Europe to
1500 HIS 125 History of Europe Since 1500 \textbf{One} course in United
States history \textbf{One} course in Asian history \textbf{Two}
additional history courses approved by the department \textbf{OR} HIS
145 History of United States to 1865 HIS 155 History of United States
since 1865 \textbf{One} course in European history \textbf{One} course
in Asian history \textbf{Two} additional history courses approved by the
department

\subsection{Courses in History by Content
Area}\label{courses-in-history-by-content-area}

\emph{Asian History} HIS 136 East Asian Civilization HIS 216 History of
Modern Korea (WE) HIS 246 History of Modern China HIS 256 History of
Modern Japan HIS 316 Topics in History:Non-Western Persp

\emph{European History} HIS 115 History of Europe to 1500 HIS 125
History of Europe Since 1500 HIS 238 Modern France HIS 248 The French
Revolution HIS 272 History of Medieval Europe HIS 275 Ancient Greek
History HIS 285 History of Ancient Rome HIS 288 Renaissance \&
Reformation HIS 292 History of Modern England HIS 318 Topics in History
:Div West Persp HIS 355 19th Century Europe HIS 365 20th Century Europe
HIS 372 Early Modern Europe

\emph{U.S. History} HIS 145 History of United States to 1865 HIS 155
History of United States since 1865 HIS 217 American War in Vietnam HIS
227 American Civil War HIS 257 Native American History HIS 297 Women in
America HIS 317 Topics in History:US Pluralism HIS 325 Recent American
History I HIS 335 Recent American History II HIS 347 African American
History HIS 387 American Colonial History HIS 395 United States
Diplomatic History

\emph{Methods and Research} HIS 205 Historians Craft HIS 465 Seminar in
Ancient History HIS 466 Seminar Modern East Asian History HIS 472
Seminar American History I HIS 473 Seminar American History II HIS 474
Seminar Modern European History

\emph{Other History Courses} HIS 208 The First World War (WE) HIS 218
The Second World War (WE) HIS 268 Latin America HIS 276 The
``Discovery'' of America: Clash HIS 278 History of the Holocaust (WE)
HIS 286 Modern Middle East HIS 300 Public History HIS 306 Revolution,
Social Struggle, Testim HIS 308 Legacies of the Cold War inLatin Am HIS
328 Modern France HIS 444 Ind Study-History HIS 494 Internship in
History

\subsection{Courses in Health and Society
Studies}\label{courses-in-health-and-society-studies}

\begin{itemize}
\tightlist
\item
  \textbf{HIS 115 History of Europe to 1500} The development of Western
  civilization from the earliest times to 1500, with primary emphasis on
  the culture and thought of the Ancient, Medieval, and Renaissance
  eras.
\item
  \textbf{HIS 125 History of Europe Since 1500} The development of
  Western civilization from 1500 to modern times, with emphasis on the
  cultural and intellectual development of the West and such topics as
  the Reformation, the Enlightenment, the Industrial Revolution,
  Imperialism, and the rise of Totalitarianism.
\item
  \textbf{HIS 136 East Asian Civilization} A survey of East Asian
  civilization with emphasis on China from its origins to 1700.
  Particular attention is paid to cultural and political factors.
\item
  \textbf{HIS 145 History of United States to 1865} A survey of
  relations between indigenous, European, and African peoples in places
  that would become the United States from the colonial era through the
  Civil War. Topics include European colonization; indigenous
  resistance; racial slavery; the American Revolution and creation of
  the United States; westward territorial expansion; and the Civil War.
\item
  \textbf{HIS 155 History of United States since 1865} A survey of
  American history from the Reconstruction to the present. Topics
  include the changing American economy, the inclusion and exclusion of
  various Americans defined by race, class, and gender, and the
  emergence of the United States as a world power.
\item
  \textbf{HIS 205 Historians Craft} Introduction to the nature and craft
  of history. Emphases include use of sources, historiography,
  philosophy of history, and various forms of historical writing. The
  capstone assignment is a research paper. Prerequisite: sophomore
  standing.
\item
  \textbf{HIS 208 The First World War (WE)} Introduces students to the
  global history of the First World War and the start of the interwar
  period. The class focuses on the political, cultural, and social
  effects of the war through a study of historical documents, period
  literature, and film. Through class discussion, short compositions,
  response papers, and digital humanities work students will: critically
  analyze historical documents; identify and evaluate the significance
  of key actors, events, and ideas of the First World War; and
  synthesize evidence from to produce effective written and oral
  arguments.
\item
  \textbf{HIS 216 History of Modern Korea (WE)} Introduces students to
  modern Korean history by examining Korea's transition from a
  politically isolated rural state to, in the South, an industrialized
  country electing its own government and, in the North, a totalitarian
  state capable of producing nuclear weapons. Students will examine
  historical documents, literature and film and use class discussion and
  formal papers to analyze key events such as: Korea's colonization
  under Japan; the Korean War; South Korea's economic and democratic
  ``miracles;'' North Korean communism and nuclear brinksmanship; and
  South Korea's emergence as a major producer of exported entertainment
  like K-Pop and K-Drama.
\item
  \textbf{HIS 217 American War in Vietnam} Examines the American war in
  Vietnam from its earliest roots to its latest ramifications and from
  multiple perspectives. Emphases include the French Indochina War,
  American policy debate, the sources of American policy, Vietnamese
  perspectives, and the war in fiction and film.
\item
  \textbf{HIS 218 The Second World War (WE)} Introduces students to the
  global history of the Second World War, beginning with the interwar
  period and moving into the start of the Cold War. The course will
  focus on both the European and Pacific battle and home fronts,
  including the Holocaust, through a study of historical documents,
  period literature, and film. Through class discussion, short
  compositions, response papers, and digital humanities work students
  will: critically analyze historical documents; identify and evaluate
  the significance of key actors, events, and ideas of the Second World
  War; and synthesize evidence from to produce effective written and
  oral arguments.
\item
  \textbf{HIS 227 American Civil War} A study of events, issues, ideas,
  and forces leading to the American Civil War and the resulting
  Reconstruction. Topics emphasized include slavery, social, economic,
  and political events leading to the war, the political and military
  strategies of the war, the efforts of Americans on the homefront, and
  the trials and triumphs during Reconstruction.
\item
  \textbf{HIS 238 Modern France} An examination of the major political,
  cultural, and social developments in France from the end of the French
  Revolution to present day, with an emphasis on the changes and shifts
  in French national identity and global relations.
\item
  \textbf{HIS 246 History of Modern China} A history of modern China
  since 1700. Chinese states and society from the height of Qing
  culture, through the impact of the West and Japan, the rise of Chinese
  Republicanism and Marxism, the Liberation of 1949, the Great Leap
  Forward, the death of Mao Zedong, and rule of Deng Xiaoping, up to the
  present day.
\item
  \textbf{HIS 248 The French Revolution} An investigation of the origins
  and course of the French Revolution from the Ancien Regime to 1815.
  The course covers the Enlightenment, the collapse of the Ancien
  Regime, the opening of the Revolution, the Terror, and Napoleon.
\item
  \textbf{HIS 256 History of Modern Japan} A history of modern Japan
  since 1700. Historical analysis of Edo period culture, politics and
  society, rapid Meiji era changes, constitutionalism and imperialism,
  Japan's expansion in Asia, World War II, the post-war social change,
  and economic recovery and rise to international leadership, up to the
  present.
\item
  \textbf{HIS 257 Native American History} Examines the history of
  Native Americans and their relationships to Europe and Anglo-America
  from pre- Columbian times to the present. Emphases include
  understanding Native cultures, early Anglo-Indian relations, the
  western ``Indian wars,'' and the Red Power movement of the 1970s.
\item
  \textbf{HIS 268 Latin America} Survey of the social, political,
  economic, and cultural factors of Latin America from the colonial era
  to the present, with an emphasis on how these factors influence
  present day Latin America.
\item
  \textbf{HIS 272 History of Medieval Europe} A survey of Medieval
  Europe focusing on the intellectual, cultural, religious, artistic,
  and literary achievements of the High Middle Ages from roughly 1000
  A.D. to 1300 A.D.
\item
  \textbf{HIS 275 Ancient Greek History} An examination of the evolution
  of certain key institutions and traditions in the ancient Greek
  world---political, constitutional, military, social, and
  economic---with particular emphasis on the revolutionary changes
  experienced during the Classical Age, i.e., the fifth and fourth
  centuries B.C.
\item
  \textbf{HIS 276 The ``Discovery'' of America: Clash} Focuses on the
  biggest empires in the Americas (Maya, Aztec, and Inca) and the
  process of conquest and colonization after Christopher Columbus'
  arrival. The students will read accounts written during the 15th-17th
  centuries and consider how to critically engage them. The class aims
  to reflect not only on the stories of the conquistadores, but also on
  those of the conquered people through historical accounts that attempt
  to give a voice to the voiceless, analyze the challenges of facing a
  completely unknown culture, and the ethical implications of imposing
  your values and traditions on others. By the end of the class, the
  student will be able to provide an overview of the Spanish Conquest,
  its history, and controversies. Students also will develop skills to
  analyze primary and secondary sources and write small response papers
  as well as a small essay that demonstrates an understanding of the
  time period and the ability to support an argument.
\item
  \textbf{HIS 278 History of the Holocaust (WE)} Introduces students to
  the examination of the causes, experiences, and legacy of the
  Holocaust, studying viewpoints of victims, perpetrators, and
  bystanders. Students will study historical documents and literature as
  well as audiovisual sources in course discussions and written work as
  they cover social, cultural, and political history from pre-war
  Germany through World War II, and evaluate memory of the Holocaust in
  the time period since 1945. Through written work and course
  discussions, students will: identify and evaluate the significance of
  key actors, events, and ideas connected to the Holocaust; analyze
  primary and secondary sources; synthesize evidence to produce cogent
  written and oral arguments; and critically examine understanding of
  the Holocaust.
\item
  \textbf{HIS 285 History of Ancient Rome} An examination of the
  evolution of certain key institutions and traditions---political,
  constitutional, military, social, and economic---in the ancient Roman
  world, with emphasis on the revolutionary changes during the
  transition from the Republic to the Empire, i.e., the first centuries
  B.C. and A.D.
\item
  \textbf{HIS 286 Modern Middle East} A survey of recent history of the
  Middle East, from the 1800's to the present with an emphasis on the
  20th century. Course focuses on the Middle East and its global
  interactions.
\item
  \textbf{HIS 288 Renaissance \& Reformation} The European experience
  from the waning of the Middle Ages through the period of the religious
  wars, with the emphasis on art, the rise of nation states, overseas
  expansionism, the scientific revolution, and the Protestant and
  Catholic Reformations.
\item
  \textbf{HIS 292 History of Modern England} A survey of the major
  constitutional, political, and social developments in England from The
  Glorious Revolution to the present. Particular attention is paid to
  the growth of Empire, the Industrial Revolution, the rise of labor,
  and the effects of The Great War and World War II.
\item
  \textbf{HIS 297 Women in America} A survey of the role and power of a
  variety of women in America's history from colonial times to the
  present, with an emphasis on understanding the place of women today.
\item
  \textbf{HIS 300 Public History} Introduces students to the methods and
  practices of public history. Public history refers to the many ways
  history is utilized and applied outside traditional academic research
  and college classroom settings. Through readings, discussions, site
  visits, guest speakers, and writing assignments, students will learn
  about the many ways history is applied, interpreted, and used by
  museums, historical sites, non-historians, and other publics. They
  will be introduced to career opportunities in public history and will
  engage with the cultural and historical resources of Cedar Rapids.
  Students will analyze primary and secondary sources, refine written
  and verbal communication skills, and demonstrate an understanding of
  and ability to use historical research techniques. The course
  culminates in the production of a hands-on, research-based,
  collaborative class history project (e.g., exhibit, website,
  historical markers, etc.) that will serve a public beyond the
  classroom.
\item
  \textbf{HIS 306 Revolution, Social Struggle, Testim} Revolution,
  Social Struggle, and Testimonio in Latin America 20th Century
\end{itemize}

Introduces students to the testimonial literature in Latin America, a
genre capable of allowing the oppressed to bring forward their
perceptions, world view, and experience of a specific historical moment.
This class includes narratives from Guatemala, Nicaragua, El Salvador,
México, and Bolivia, among others. It will also include a theoretical
segment that will reflect on the advantages and shortcomings of this
genre. At the end of the class, students understand and describe major
social problems during specific time frames in the 20th century of the
nations studied, articulate the characteristics of this genre, and use a
testimonio to gain an insight into a sociohistorical event. Through
written assignments and discussions students will analyze the experience
of marginalized communities in relation to specific institutional
policies. The course ends with a testimonio project where students
explore the genre using personal experience, family history, or by
interviewing an acquaintance that can offer a special insight regarding
a social problem. - \textbf{HIS 308 Legacies of the Cold War inLatin Am}
Legacies of the Cold War in Argentina, Peru, Colombia, and Chile Is an
interdisciplinary class centered on the relation between the Cold War
and the internal violence that several South American nations lived
through during the second half of the 20th century. The influence of
anti-imperialist ideas throughout Latin America, the development of the
guerrilla movement and its revolutions, and the militarized
counter-revolution served as the political basis for moments of crisis
that each nation endured: The National Reorganization Process in
Argentina (1976-1982), the Civil War in Peru (1980-2000), the Military
Dictatorship in Chile (1973-1990), and the long-term war in Colombia
(1948-present). The classwork with historical and anthropological
accounts, testimonios, art, film, and theory to give a more diverse
image of experiences and implications of these events for different
sectors of the communities. At the end of the course, the student will
understand major Latin American historical trends in relation to the
Cold War, as well as the specificities of the conflict in each one of
these nations. They also will be able to analyze one of the factors that
play a role in the development of these crises through a small research
project that is developed by written assignments. - \textbf{HIS 316
Topics in History:Non-Western Persp} An intensive study of a selected
topic (or topics) in history related to non-Western cultures through
lectures or group discussion or directed learning or research and
writing. (Offered on an occasional basis) - \textbf{HIS 317 Topics in
History:US Pluralism} An intensive study of a selected topic (or topics)
in history related to United States Pluralism through lectures or group
discussion or directed learning, or research and writing. (Offered on an
occasional basis) - \textbf{HIS 318 Topics in History :Div West Persp}
An intensive study of a selected topic (or topics) in history related to
Diverse Western Perspectives through lectures or group discussion or
directed learning or research and writing. (Offered on an occasional
basis) - \textbf{HIS 325 Recent American History I} An examination of
American domestic development from the last quarter of the 19th century
to the present. Topics include immigration, workers in a changing
economy, the World Wars, the Cold War, civil rights, the changing role
of women in society, and the contested post-1960's move to the political
right. Prerequisite: History of the United States Since 1865 (HIS-155)
or consent of instructor. - \textbf{HIS 328 Modern France} An
examination of the major political, cultural, and social developments in
France from the end of the French Revolution to present day, with an
emphasis on the changes and shifts in French national identity and
global relations. - \textbf{HIS 335 Recent American History II} An
examination of American foreign relations from the 1890s to the present,
including the World Wars, the Cold War, and the post-9/11 world.
Prerequisite: History of the United States Since 1865 (HIS-155) or
consent of instructor. - \textbf{HIS 347 African American History} An
examination of changes and continuities in the lives of African
Americans from the colonial era to the present. Emphases include the
origins and evolving nature of slavery, race, and racism; development of
African-American culture, individual struggles for freedom and civil
rights; migrations; segregation; and large-scale movements for social
and political change. - \textbf{HIS 355 19th Century Europe} The
evolution of Europe from the French Revolution to World War I, with an
emphasis on such topics as Liberalism, the Industrial Revolution,
Nationalism, Marxian Socialism, Imperialism, and Great Power diplomacy.
- \textbf{HIS 365 20th Century Europe} A historical survey of modern
Europe from 1900 to the present. Emphases are on World War I, the
Russian Revolution, the rise of Fascism, Hitler, Nazi Germany, Stalin
and Soviet Communism, World War II and the Cold War, as well as the
emergence of the European Community (EC) and the collapse of Communism
and the Soviet Union. - \textbf{HIS 372 Early Modern Europe} The history
of Europe from 1603-1815 with emphasis on the English Revolutions, the
age of Absolute Monarchy, the Enlightenment, the decline of the Ancien
Regime, the French Revolution, the Napoleonic era, and concluding with
the Congress of Vienna. - \textbf{HIS 387 American Colonial History} The
history of the American colonies from their founding through the
American Revolution, focusing on the diverse perspectives of people
varying in religion, ethnicity, and gender, as well as economic and
political backgrounds. - \textbf{HIS 395 United States Diplomatic
History} Begins with an examination of the relations of the colonies to
international affairs. The course focuses attention on topics in
American diplomatic history which provide a background for an
understanding of the nation's present status as a world power. -
\textbf{HIS 444 Ind Study-History} Independent work on a selected
project under the direction of a faculty member of the department. May
be taken for an X status grade with consent of instructor prior to
registration. Prerequisite: consent of instructor. - \textbf{HIS 465
Seminar in Ancient History} An intensive reading and research seminar on
a selected topic in the history of the ancient world. Prerequisite: The
Historian's Craft (WE) (HIS-205) and junior standing. - \textbf{HIS 466
Seminar Modern East Asian History} An intensive reading and research
seminar on selected topics in the 19th- and 20th-century history of
China, Japan and Korea. Prerequisite: The Historian's Craft (HIS-205)
and junior standing. - \textbf{HIS 472 Seminar American History I} An
introduction to historical research in earlier American history (prior
to 1865). Critical inquiry is emphasized in detailed examination of
specific topics. Prerequisite: The Historian's Craft (HIS-205) and
junior standing. - \textbf{HIS 473 Seminar American History II} An
introduction to historical research in later American history since
1865. Critical inquiry is emphasized in detailed examination of specific
topics. Prerequisite: The Historian's Craft (HIS-205) and junior
standing. - \textbf{HIS 474 Seminar Modern European History} An
intensive reading and research seminar on selected topics in 19th- and
20th-century European political, diplomatic, and intellectual history.
Prerequisite: The Historian's Craft (HIS-205) and junior standing. -
\textbf{HIS 494 Internship in History} Part-time work experience for one
term in a history-related field supervised by a faculty member of the
department in cooperation with the Internship Specialist. A minimum of
140 hours on-site experience is required. S/U basis only. This course
does not satisfy any of the requirements for a major or minor in
history. Prerequisites: junior standing and consent of department chair.

\section{Health and Society Studies (Minor
Only)}\label{sec-health-and-society-studies}

Kelly, LeMay (Administrative Coordinators).

The Health and Society Studies program is an interdisciplinary cluster
of courses that allows students to examine the breadth of issues
impacting health and healthcare from the perspectives of natural
sciences, social sciences, and humanities. This Health and Society
Studies minor broadens students understanding of the impact of health
and healthcare delivery issues, personally, professionally, and within
our society. In addition, the minor will provide a strong
interdisciplinary foundation for students interested in graduate studies
related to health professions. Because required and elective courses are
offered by faculty across the College, students are encouraged to meet
with a Health and Society Studies administrative coordinator for
informal advising before choosing the Health and Society minor.

\subsection{Health and Society Studies
Minor}\label{health-and-society-studies-minor}

A minor in Health and Society Studies requires a minimum cumulative 2.0
GPA in all courses counted toward the minor. In addition, only (1) one
course from a student's major or collateral major can count towards the
Health and Society Studies Minor.

\textbf{Required Courses} 1. PHL 165 Bio-medical Ethics 2. SOC 107
Introductory Sociology 3. \textbf{One} of the following Biology courses:
a. BIO 103 Anatomy \& Physiology b. BIO 120 Biology, Health, \& Society
c.~BIO 155 Organismal \& Ecological Biology 4. \textbf{Three} elective
courses from the list below, only (1) one of which may be a 100-level
course. ANT 116 Cultural Anthropology CLA 155 Latin/Greek Origins Med
Terminology COM 236 Intercultural Communication COM 237 Interpersonal
Communication COM 332 Health Communication (WE) KIN 115
Fundamentals:Exercise \& Nutrition KIN 125 Public and Consumer Health
KIN 155 Substance Abuse PHL 128 Morality \& Moral Controversies ECO 221
Health Economics PHY 321 Health Physics PSY 205 Developmental Psychology
PSY 235 Abnormal Psychology PSY 325 Health Psychology PSY 350 Drugs \&
Behavior SPA 339 Spanish for Health Care HSS 494 Health Professions
Externship

\textbf{NOTE:} \emph{additional health-related courses may be approved
as an elective with administrative coordinator approval.}

\subsection{Courses in Health and Society
Studies}\label{courses-in-health-and-society-studies-1}

\begin{itemize}
\tightlist
\item
  \textbf{HSS 100 Exploration in Health \& Society} Allows students
  interested in health-related careers to explore various career options
  (e.g., medicine, dentistry, physical therapy, occupational therapy,
  nursing, psychological counseling); to assess and reflect on their own
  values, abilities, and motivations; and to develop some of the applied
  knowledge and skills necessary for entry into these occupations.
  Appropriate for first and second year students. (0.5 course credit)
\end{itemize}

\section{International Business}\label{sec-international-business}

Students majoring in business administration have two options: the
BUSINESS ADMINISTRATION major (see
\textbf{?@sec-business-administration-and-economics}) and the
INTERNATIONAL BUSINESS major. The international business major is
designed to prepare students for specific challenges related to
operating an organization in an international context. Because of
significant overlap in the business administration and the international
business requirements, only one of the two majors may be elected by a
student.

\subsection{International Business
Major}\label{international-business-major}

A major in international business requires a minimum cumulative 2.0 GPA
in all courses counted toward the major. 1. ACC 171 Principles of
Accounting I 2. ECO 175 Principles of Macroeconomics 3. ECO 215 Prin
Microeconomics 4. BUS 190 Statistical Analysis 5. \textbf{One} of the
following: BUS 250 Principles of Management BUS 300 Human Resource
Management 6. BUS 315 Business Law I 7. BUS 446 International Business
Management 8. \textbf{One} of the following: BUS 466 Adv Top
Mrktg:non-west persp ECO 336 Divergent Economic Growth ECO 436 Econ
Development ECO 446 International Econ 9. ECO 495 International Finance
10. Two courses in world language numbered 115 or above 11. One of the
following: A semester abroad A May Term abroad A capstone project in
international business, supervised by a member of the business faculty

\section{Interdisciplinary Studies}\label{sec-interdisciplinary-studies}

The interdisciplinary major is a rigorous academic program in which the
individual student assumes primary responsibility for designing her or
his own academic program. Since the interdisciplinary major is
structured by the student to serve individual needs, the primary courses
selected for the major invariably cross traditional departmental
boundaries. In all instances, the interdisciplinary major must display
internal topical coherence. Assisting the student in developing a
coherent interdisciplinary major will be the supervising faculty member,
the student's academic advisor (if the advisor is not the supervising
faculty member), faculty from those departments in which the student is
taking primary courses, and the Academic Policies Committee. This
committee is responsible for approving student-initiated majors and for
offering, where necessary, constructive advice on student proposals.
Further guidelines for interdisciplinary majors may be obtained from the
Office of the Registrar. The decision to undertake and to seek approval
of an interdisciplinary major should be made after the student has had
sufficient consultation with the appropriate persons within the College.
Normally, this decision is made in the sophomore year, but in no case is
an interdisciplinary major approved after the student has embarked upon
the final eight courses leading to graduation. Application for an
interdisciplinary major must contain the following: 1) a list of the
nine to 11 primary courses which constitute the major; 2) a list of
secondary or supportive courses which are tangentially related to the
major; and 3) a three- to four-page typewritten statement of the
rationale for the proposal, indicating, among other things, the internal
coherence of the major.

\subsection{Additional guidelines and
requirements:}\label{additional-guidelines-and-requirements}

\begin{enumerate}
\def\labelenumi{\arabic{enumi}.}
\tightlist
\item
  Courses for the major are selected from three or more academic
  disciplines. At least 40\% of the courses in a proposed major must be
  taken at Coe.
\item
  A student's program of study must include, among the primary and
  secondary courses listed on the proposal, at least five upper division
  classes within a single discipline or within two closely related
  disciplines. Students may consider an established, departmental minor
  when a minor appropriate to the proposed major is available. Students
  should not propose a major that simply recombines courses from majors
  and minors that will already be recognized on the student's
  transcript. An interdisciplinary major should be substantially
  distinct from the student's other majors and minors.
\item
  Students must complete an integrative senior project which
  demonstrates competence in bringing together at least two disciplines
  within the major. A student develops a project in consultation with
  the student's academic advisor and at least one other faculty
  consultant from an appropriate department other than that of the
  academic advisor. The senior project must be reviewed. The form of the
  review, which may be similar to the defense of an honors project, is
  determined by the project advisor in consultation with the student.
  The project must also be reviewed by at least one other faculty member
  from a department other than that of the faculty advisor. Students may
  undertake the project with or without academic credit. Registration
  for credit may take one of two forms. Students may register for
  regular catalog courses in individualized instruction (e.g., directed
  readings or directed writing classes) or they may apply for an
  independent study which requires approval by the Chair of the Academic
  Policies Committee (Provost).
\item
  An interdisciplinary major may include an internship.
\end{enumerate}

\subsection{Courses in Interdisciplinary
Studies}\label{courses-in-interdisciplinary-studies}

\begin{itemize}
\tightlist
\item
  \textbf{INT 494 Internship} An internship related to the student's
  field of interdisciplinary study supervised by the Internship
  Specialist. A minimum of 140 hours on-site experience is required. S/U
  basis only. Prerequisites: junior standing and approval of an
  interdisciplinary major.
\end{itemize}

\section{International Economics}\label{sec-international-economics}

Students majoring in economics have two options: the ECONOMICS major
(see Section~\ref{sec-economics}) and the INTERNATIONAL ECONOMICS major.
The international economics major is designed to allow students who are
interested in international studies to explore issues of development,
growth, and international finance, using the tools of economic analysis.
Because of significant overlap in the economics and the international
economics requirements, only one of the two majors may be elected by a
student. Students majoring in economics or international economics may
not minor in either economics or international economics.

\subsection{International Economics
Major}\label{international-economics-major}

\begin{enumerate}
\def\labelenumi{\arabic{enumi}.}
\tightlist
\item
  ECO 175 Principles of Macroeconomics
\item
  ECO 215 Prin Microeconomics
\item
  ECO 315 Intermediate Macroeconomic Theory
\item
  ECO 336 Divergent Economic Growth
\item
  ECO 345 Intermediate Price Theory
\item
  ECO 375 Econometrics
\item
  ECO 436 Econ Development
\item
  ECO 446 International Econ
\item
  ECO 495 International Finance
\item
  Two courses in world language numbered 115 or above
\item
  \textbf{One} of the following: A semester abroad A May Term abroad A
  capstone project in international economics, supervised by a member of
  the economics faculty
\end{enumerate}

\textbf{NOTE:} \emph{International economics majors intending to do
graduate work in international studies or intending to pursue a career
with a governmental or non-governmental agency are strongly encouraged
to consider a minor or major in political science or one of the
International Studies tracks to complement the international economics
major. Students are also encouraged to work closely with a member of the
department to select complementary general education courses. }

\subsection{International Economics
Minor}\label{international-economics-minor}

\begin{enumerate}
\def\labelenumi{\arabic{enumi}.}
\tightlist
\item
  ECO 175 Principles of Macroeconomics
\item
  ECO 215 Prin Microeconomics
\item
  ECO 336 Divergent Economic Growth
\item
  ECO 436 Econ Development
\item
  ECO 446 International Econ
\item
  ECO 495 International Finance
\end{enumerate}

\section{International Studies}\label{sec-international-studies}

Chaimov (Administrative Coordinator).

\subsection{International Studies
Major}\label{international-studies-major}

A major in international studies, in any track, requires a minimum
cumulative 2.0 GPA in all courses counted toward the major.

\subsection{International Studies Core Courses (required of students
completing a major in international
studies):}\label{international-studies-core-courses-required-of-students-completing-a-major-in-international-studies}

\begin{enumerate}
\def\labelenumi{\arabic{enumi}.}
\tightlist
\item
  IS 116 Intro to International Studies
\item
  ANT 116 Cultural Anthropology
\item
  POL 258 World Politics
\item
  ECO 175 Principles of Macroeconomics
\item
  \textbf{Two} world language courses, numbered 125 or above in Spanish
  or French, 115 or above in another language.
\item
  A term-long study abroad experience, subject to prior approval by the
  International Studies program committee.
\item
  IS 464 International Studies Colloquium
\end{enumerate}

\textbf{NOTE:} \emph{International Studies majors select one of the
following tracks: The Global South, International Relations, or Global
Cultural Studies. (Students who plan to pursue a graduate program that
requires a strong foundation in economics should consider the
International Economics major offered by the Business and Economics
Department.)}

\subsection{International Studies Major --- The Global South
track}\label{international-studies-major-the-global-south-track}

The Global South examines the characteristics of regions in what has
come to be called the ``Global South,'' especially Latin America,
Africa, and Asia. It investigates how the evolution of these regions
relates to social, political and economic systems. Students are
introduced to a variety of disciplinary and theoretical approaches to
the study of international development; graduates better understand the
history and legacy of colonialism, the conditions that create or resolve
poverty, and relations between the Global South and North. 1.
\textbf{Seven} international studies core courses (See above.) 2.

\subsection{Courses in Interdisciplinary
Studies}\label{courses-in-interdisciplinary-studies-1}

\begin{itemize}
\tightlist
\item
  \textbf{INT 494 Internship} An internship related to the student's
  field of interdisciplinary study supervised by the Internship
  Specialist. A minimum of 140 hours on-site experience is required. S/U
  basis only. Prerequisites: junior standing and approval of an
  interdisciplinary major.
\end{itemize}

\section{International Studies}\label{sec-international-studies}

Chaimov (Administrative Coordinator).

\subsection{International Studies
Major}\label{international-studies-major-1}

A major in international studies, in any track, requires a minimum
cumulative 2.0 GPA in all courses counted toward the major.

\subsection{International Studies Core Courses (required of students
completing a major in international
studies):}\label{international-studies-core-courses-required-of-students-completing-a-major-in-international-studies-1}

\begin{enumerate}
\def\labelenumi{\arabic{enumi}.}
\tightlist
\item
  IS 116 Intro to International Studies
\item
  ANT 116 Cultural Anthropology
\item
  POL 258 World Politics
\item
  ECO 175 Principles of Macroeconomics
\item
  \textbf{Two} world language courses, numbered 125 or above in Spanish
  or French, 115 or above in another language.
\item
  A term-long study abroad experience, subject to prior approval by the
  International Studies program committee.
\item
  IS 464 International Studies Colloquium
\end{enumerate}

\textbf{NOTE:} \emph{International Studies majors select one of the
following tracks: The Global South, International Relations, or Global
Cultural Studies. (Students who plan to pursue a graduate program that
requires a strong foundation in economics should consider the
International Economics major offered by the Business and Economics
Department.)}

\subsection{International Studies Major --- The Global South
track}\label{international-studies-major-the-global-south-track-1}

The Global South examines the characteristics of regions in what has
come to be called the ``Global South,'' especially Latin America,
Africa, and Asia. It investigates how the evolution of these regions
relates to social, political and economic systems. Students are
introduced to a variety of disciplinary and theoretical approaches to
the study of international development; graduates better understand the
history and legacy of colonialism, the conditions that create or resolve
poverty, and relations between the Global South and North. 1.
\textbf{Seven} international studies core courses (See above.) 2. ECO
215 Prin Microeconomics 3. \textbf{Two} courses to be chosen as
indicated from the two lists below. a. \textbf{One} of the following:
ECO 436 Econ Development \textbf{or} Subject to prior approval by the
International Studies administrative coordinator: ECO 336 Divergent
Economic Growth ECO 446 International Econ b. \textbf{One} of the
following: POL 386 International Development \textbf{or} Subject to
prior approval by the International Studies administrative
coordinator:\\
POL 310 International Organizations 4. \textbf{Two or more} additional
courses from either the courses under \#3 or any of the following,
subject to prior approval by the International Studies administrative
coordinator. (Not more than one of these two courses can be from
political science and not more than one of these two courses can be from
economics.). At least one course must be completed at the 300 level or
above. +++MISSING INFO: c.ant286/486.long +++ ASC 186 Modern South Asia
ENG 146 Intro Postcolonial Literature GS 136 Gender in Non-Western World
HIS 246 History of Modern China HIS 268 Latin America HIS 286 Modern
Middle East HIS 316 Topics in History:Non-Western Persp IS 316 Top in
Internl Stds:Non-West Persp POL 266 Latin American Politics POL 276
African Politics POL 286 Asian Politics

\subsection{International Studies Major --- International Relations
track}\label{international-studies-major-international-relations-track}

International Relations combines theoretical approaches to and empirical
knowledge of state and non-state actors, power, and international
structures.\\
1. \textbf{Seven} international studies core courses 2. POL 108
Introduction to Politics 3. POL 310 International Organizations 4.
\textbf{Three} courses to be chosen as indicated from the two lists
below. No more than two of these three courses can be completed in the
Political Science department. a. \textbf{One} of the following: ASC 186
Modern South Asia HIS 216 History of Modern Korea (WE) HIS 217 American
War in Vietnam HIS 246 History of Modern China HIS 256 History of Modern
Japan POL 266 Latin American Politics POL 276 African Politics POL 286
Asian Politics POL 296 Topics Pol Sci: Non-West Persp b. \textbf{Two} of
the following: +++MISSING INFO: c.ant286/486.long +++ ECO 446
International Econ ECO 495 International Finance HIS 286 Modern Middle
East HIS 335 Recent American History II HIS 395 United States Diplomatic
History IS 315 Topics in International Studies IS 316 Top in Internl
Stds:Non-West Persp POL 305 Terrorism POL 365 American Foreign Policy
POL 386 International Development POL 398 Religion \& World Politics

\subsection{International Studies Major --- Contemporary European
Studies}\label{international-studies-major-contemporary-european-studies}

The Contemporary European Studies track emphasizes the study of cultural
commonalities and differences in Europe since the mid-19th century.
Contemporary European Studies 1. Seven international studies core
courses 2. HIS 365 20th Century Europe 3. POL 298 European Politics 4.
\textbf{Three} of the following, two of which must be numbered 300 or
above. At least two academic disciplines must be represented. ARH 248
Baroque, Rococo, and Neoclassicism ARH 307 Modern and Contemporary Art
ENG 393 Study in Cont/Mod British Fiction FRE 158 France \& Francophone
World FRE 340 French \& Fracophone Cinema (WE) HIS 238 Modern France HIS
278 History of the Holocaust (WE) HIS 292 History of Modern England IS
315 Topics in International Studies MU 458 Music History \& Literature
III REL 138 Modern Judaism REL 148 Islam REL 178 Christianity THE 238
History of Theatre and Drama II A course approved by the International
Studies administrative coordinator

\subsection{Courses in International
Studies}\label{courses-in-international-studies}

\begin{itemize}
\tightlist
\item
  \textbf{IS 116 Intro to International Studies} A range of case studies
  introduces students to the nomenclature and analytical approaches of
  international studies. Focus areas include migration and human
  geography; global economics, trade, and development; global
  governance; media and culture across borders. Global health, the
  environment, and gender issues are also included.
\item
  \textbf{IS 126 HumanRightsBurmeseMigrant} Offers the opportunity
  during a May-Term residency at a school on the border of Thailand and
  Myanmar to gain an understanding of the lives of Burmese youth who
  live as educational migrants in Thailand. Through teacher/ pupil
  interactions and tutoring students learn about the conditions that
  drive Burmese children from their homeland to seek an education in a
  foreign country and about the challenges these children face far from
  home. Students also gain insight into the rewards and pitfalls of
  international humnitarian work. Prerequisite: consent of instructor.
  (Offered May Term only).
\item
  \textbf{IS 315 Topics in International Studies} A study of a selected
  topic or theme in international studies. Topics vary and may include
  interdisciplinary approaches to human migration, transitional justice,
  human rights. May be taken more than once for credit, provided the
  topics are substantially different. Prerequisite: Introduction to
  International Studies (IS-116) or consent of instructor.
\item
  \textbf{IS 316 Top in Internl Stds:Non-West Persp} Same as Topics in
  International Studies (IS-315) except selected works focus on
  Non-Western Perspectives.
\item
  \textbf{IS 464 International Studies Colloquium} Majors are required
  to discuss a set of readings to foster synthesis across the various
  tracks, submit 15-25 pages of finished writing on an issue in
  International Studies, and present their work orally to students and
  faculty. The Colloquium is usually taken during Spring Term of the
  senior year. Satisfactory completion of the Colloquium is required for
  graduation with a major in international studies. S/U basis only.
  Prerequisite: consent of instructor. (0.5 course credit)
\end{itemize}

\section{Japanese (Courses Only)}\label{sec-japanese}

Potter.

\subsection{Courses in Japanese}\label{courses-in-japanese}

\begin{itemize}
\tightlist
\item
  \textbf{JPN 106 Images Foreign Culture} A course dealing with the
  picture of human activity and values given in foreign short stores,
  novels, films, and other works of art, as well as in historical
  documents and the like. It celebrates and explores the special
  character of a national or ethnic identity. Topics vary from year to
  year. Taught in English.
\item
  \textbf{JPN 115 Elementary Japanese I} Beginning Japanese phonology,
  structure, and vocabulary. Study of hiragana and katakana syllabaries
  with introduction of some kanji.
\item
  \textbf{JPN 125 Elementary Japanese II} Continuing study of Japanese
  phonology, structure and vocabulary. Study of hiragana and katakana
  syllabaries with introduction of some kanji. Prerequisite: Elementary
  Japanese I (JPN-115) or consent of instructor.
\item
  \textbf{JPN 215 Intermediate Japanese I} Reading and discussion of
  selected Japanese texts, with continued work on grammatical
  structures. Increased emphasis on kanji. Prerequisite: Elementary
  Japanese II (JPN-165) or consent of instructor.
\item
  \textbf{JPN 225 Intermediate Japanese II} Reading and discussion of
  selected Japanese texts, with continued work on grammatical
  structures. Increased emphasis on kanji. Prerequisite: Intermediate
  Japanese I (JPN-215) or consent of instructor.
\item
  \textbf{JPN 305 Topics in Japanese Language} An advanced study of the
  Japanese language. May be taken more than once. Prerequisite:
  Intermediate Japanese II (JPN-225) or consent of instructor.
\end{itemize}

\section{Kinesiology}\label{sec-kinesiology}

Atwater, Brendes, Chandler, R. Christensen, Galbraith, Gee, Griffith,
LeFevre, Libby, Martin, Molinari, Parks, Prunty, Rice, Roberts, E.
Rodgers, Rydze (Chair), Snyder, Tiedt, Walter, Woodin.

The kinesiology department's program enables students to earn a major in
physical education for teacher certification, a coaching endorsement, a
coaching authorization, and a teacher health certification endorsement.
Students may also pursue a track that focuses on health and human
performance.

\subsection{Kinesiology Major}\label{kinesiology-major}

A major in Kinesiology requires a minimum cumulative 2.0 GPA in all
courses counted toward the major.

\subsection{Secondary Education Emphasis in Physical
Education}\label{secondary-education-emphasis-in-physical-education}

\begin{enumerate}
\def\labelenumi{\arabic{enumi}.}
\tightlist
\item
  BIO 155 Organismal \& Ecological Biology
\item
  \textbf{One} of the following: BIO 103 Anatomy \& Physiology BIO 215
  Human Anatomy \textbf{AND} BIO 225 Human Physiology
\item
  KIN 105 Foundations of Physical Movement
\item
  KIN 165 Master Activity Class for Teachers
\item
  KIN 175 Prevention \& Care Ath Inj
\item
  KIN 225 Motor Learning
\item
  KIN 347 Adapted Physical Education
\item
  KIN 365 Measurement/Eval/Prescription KIN
\item
  KIN 415 Meth Secondary School PE \& Health
\item
  KIN 440 Org-Admin of PE, Health \& Athl
\item
  KIN 442 Physiology of Exercise
\item
  KIN 495 Kinesiology
\end{enumerate}

\subsection{Kinesiology Major --- Fitness
Development}\label{kinesiology-major-fitness-development}

\begin{enumerate}
\def\labelenumi{\arabic{enumi}.}
\tightlist
\item
  BIO 155 Organismal \& Ecological Biology
\item
  \textbf{One} of the following: BIO 103 Anatomy \& Physiology BIO 215
  Human Anatomy \textbf{AND} BIO 225 Human Physiology
\item
  KIN 105 Foundations of Physical Movement
\item
  KIN 115 Fundamentals:Exercise \& Nutrition
\item
  KIN 175 Prevention \& Care Ath Inj
\item
  KIN 365 Measurement/Eval/Prescription KIN
\item
  KIN 440 Org-Admin of PE, Health \& Athl
\item
  KIN 442 Physiology of Exercise
\item
  \textbf{All} of the following courses in the \textbf{Strength and
  Conditioning Group}: KIN 103 PE or KIN 104 PE KIN 103 PE or KIN 104 PE
  KIN 185 Group Exercise KIN 385 Methods Strength Trng \& Condition KIN
  405 Program Design KIN 495 Kinesiology \textbf{or} \textbf{All} of the
  following courses in the \textbf{Health and Wellness Group}: KIN 125
  Public and Consumer Health KIN 135 Concepts of Individual Wellness KIN
  155 Substance Abuse
\end{enumerate}

\subsection{State of Iowa Coaching Endorsement (\#55,
K-12)}\label{state-of-iowa-coaching-endorsement-55-k-12}

\begin{enumerate}
\def\labelenumi{\arabic{enumi}.}
\tightlist
\item
  Teaching certification in an academic area
\item
  KIN 175 Prevention \& Care Ath Inj
\item
  KIN 201 Fundamentals of Coaching The Coaching Endorsement (\#55, K-12)
  may be granted to any student with teacher certification completing a
  major in kinesiology and by taking KIN 205 Theory of Coaching
  \textbf{OR} KIN 201 Fundamentals of Coaching.
\end{enumerate}

\subsection{State of Iowa Coaching
Authorization}\label{state-of-iowa-coaching-authorization}

\begin{enumerate}
\def\labelenumi{\arabic{enumi}.}
\tightlist
\item
  KIN 175 Prevention \& Care Ath Inj
\item
  KIN 201 Fundamentals of Coaching
\end{enumerate}

\subsection{State of Iowa Health Certification
Endorsement}\label{state-of-iowa-health-certification-endorsement}

\begin{enumerate}
\def\labelenumi{\arabic{enumi}.}
\tightlist
\item
  KIN 115 Fundamentals:Exercise \& Nutrition
\item
  KIN 125 Public and Consumer Health
\item
  KIN 135 Concepts of Individual Wellness
\item
  KIN 155 Substance Abuse
\item
  KIN 175 Prevention \& Care Ath Inj
\item
  \textbf{One} of the following: NUR 137 Human Sexuality NUR 297 Parent
  Child Relationships SOC 207 Sociology of the Family
\end{enumerate}

\subsection{Kinesiology Major --- Athletic
Training}\label{kinesiology-major-athletic-training}

\begin{enumerate}
\def\labelenumi{\arabic{enumi}.}
\tightlist
\item
  BIO 155 Organismal \& Ecological Biology
\item
  +++MISSING INFO: c.bio215/215L.long +++
\item
  BIO 225 Human Physiology
\item
  KIN 105 Foundations of Physical Movement
\item
  KIN 115 Fundamentals:Exercise \& Nutrition
\item
  KIN 175 Prevention \& Care Ath Inj
\item
  KIN 365 Measurement/Eval/Prescription KIN
\item
  KIN 442 Physiology of Exercise
\item
  KIN 495 Kinesiology \textbf{All} of the following courses in the
  Athletic Training Group KIN 110 Emergency Life Skills KIN 210 Athletic
  Injury Evaluation KIN 410 Athletic Injury Treatment \& Rehab
\end{enumerate}

\emph{Students interested in the Athletic Training Group track are
strongly encouraged to meet with the Kinesiology faculty for informal
advising as soon as they have chosen this major. Additional coursework
may be needed for those students pursuing a licensure in Athletic
Training. }

\subsection{Courses in Kinesiology}\label{courses-in-kinesiology}

\begin{itemize}
\tightlist
\item
  \textbf{+++MISSING INFO: c.kin100/102/103/104.long +++} +++MISSING
  INFO: c.kin100/102/103/104.desc +++
\item
  \textbf{KIN 105 Foundations of Physical Movement} The foundations,
  historical development, professional qualification, and opportunities
  in the field of physical education. Movement education theory is
  studied as it applies to all grade levels along with the study of
  growth, maturation, physical activity and performance in young
  adolescents.
\item
  \textbf{KIN 110 Emergency Life Skills} Incorporates the latest
  principles of emergency care in order to become effective initial
  responders. Broader discussion will focus on the efficacy of Good
  Samaritan laws, while at the same time addressing current changes to
  society as it relates to these laws. In addition, this course will
  prepare students to recognize and care for a variety of first aid
  emergencies such as burns, cuts, scrapes, sudden illnesses, head,
  neck, back injuries, heat and cold emergencies. Students will gain
  both the competency and proficiencies associated with learning how to
  respond to breathing and cardiac emergencies. Successful students will
  receive a certificate for Adult First Aid/CPR/AED. (0.5 credit)
\item
  \textbf{KIN 111 Phys Educ for Elementary Teacher} Emphasizes a survey
  of modern health and physical education practice in school and
  community, and the development of exercise in the elementary school
  through singing games and rhythms, folk and square dancing, games, and
  sports. (0.5 course credit)
\item
  \textbf{KIN 112 Health Educ for Elementary Tchr} Surveys
  health-related issues that directly affect the lives of young
  children. Topics include health, fitness, substance abuse, and
  physical and mental insult. Students learn to identify and respond to
  children who have been placed at risk. (0.5 course credit)
\item
  \textbf{KIN 115 Fundamentals:Exercise \& Nutrition} Study and
  evaluation of existing health patterns, which are tested to elicit
  positive behavior changes. Principles of exercise physiology and
  fitness, weight control and human nutrition, and problems associated
  with malnutrition and overnutrition are included to provide students
  with concepts for an enhanced lifestyle.
\item
  \textbf{KIN 125 Public and Consumer Health} Philosophy and practice of
  public and community health, including economic, sociological, and
  legal justification. Examination of health care products, services and
  consumer protection vehicles in today's marketplace. Information and
  guidelines enable individuals to select health care products and
  services intelligently.
\item
  \textbf{KIN 135 Concepts of Individual Wellness} Examination of
  personal health and positive lifestyle through the enhancement of
  physical, social, and mental/emotional wellness.
\item
  \textbf{KIN 155 Substance Abuse} Effects of drugs on the body,
  problems and risks of drug abuse, and drug education programs are
  examined.
\item
  \textbf{KIN 165 Master Activity Class for Teachers} Covers all phases
  of the teaching methodology. This class assesses the skill acquisition
  of the learner. Students model instructional procedure. After
  introduction of the activity or skill, students pre-test, teach, and
  assess each other in the activity or skill. Activities and skills
  include the following: Soccer/ Football/Volleyball, Basketball/
  Softball/Track and Field, Racquet Activities, Strength
  Conditioning/Fitness, Cycling/Bowling/Archery, Dance/Rhythms, and
  Aquatic Activities.
\item
  \textbf{KIN 175 Prevention \& Care Ath Inj} Principles of human
  biology, hygienic applications to the care of the body, and the
  effects of alcohol and substance abuse. Also covered is the nature of
  injuries frequently sustained in athletic participation and the
  control, handling, and care of injuries. This course does not satisfy
  any of the requirements for a major in athletic training. Credit is
  given for Prevention and Care of Athletic Injuries \& Laboratory
  (KIN-175) or Basic Athletic Training (AT-100), not both.
\item
  \textbf{KIN 185 Group Exercise} Explores the up-to-date,
  ever-changing, group aerobic activities available in the 21st century.
  Students discover the advantages and disadvantages, background,
  physical benefits, and techniques of performing a variety of group
  aerobic activities. Prerequisite: Foundations of Physical Movement
  (KIN-105). (0.5 course credit)
\item
  \textbf{KIN 201 Fundamentals of Coaching} Addresses the structure and
  function of the human body in relation to physical activity, theory
  and techniques of coaching interscholastic athletics. Topics include
  professional ethics and legal aspects of coaching as well as the study
  of human growth and development of children.
\item
  \textbf{KIN 202 Family Life Education} Examines family life and human
  relationships through the practice of equipping and empowering family
  members to develop knowledge and skills that enhance well-being. The
  course will examine human sexuality from a biological, psychological,
  and social perspective. Topics not limited to reproduction,
  development, communication and expression will be examined across the
  lifespan. Perspectives include diversity, variations from the
  majority, and the influence of gender, race, class, religion, sexual
  orientation, abilities, age, and culture on sexuality and sexual
  expression. Topics will also include the understanding of families by
  using the application of family theory and current research in order
  to understand family dynamics.
\item
  \textbf{KIN 205 Theory of Coaching} Sports treated from the standpoint
  of theory and practice. Topics vary from term to term and include, but
  are not limited to, football, basketball, baseball and softball,
  wrestling, track, swimming, and volleyball. May be taken more than
  once for credit, provided the topics are substantially different. (0.5
  course credit)
\item
  \textbf{KIN 210 Athletic Injury Evaluation} Introduces students to the
  procedures used in the examination of injuries involving both the
  upper and lower extremity. This course will develop a student's
  systematic ability to identify, evaluate, and monitor commonly
  experienced orthopedic injuries in athletics. Investigation into
  surface palpation, goniometry, manual muscle, and special testing will
  be developed. Prerequisite: Prevention and Care of Athletic Injuries
  \& Laboratory (KIN-175).
\item
  \textbf{KIN 215 Psychology of Coaching} An introduction to the area of
  sport psychology, which focuses on the underlying psychological and
  psychophysiological factors that influence performance in sports and
  physical activity. The following topics are emphasized: motivation,
  positive mental attitude (goal setting, self talk, mental imagery),
  anxiety/stress, self confidence, concentration, communication skills,
  sportsmanship, and psychological preparation for competition.
\item
  \textbf{KIN 225 Motor Learning} Designed to acquaint students with
  research findings, empirical evidence, and theoretical constructs
  regarding the learning and teaching of motor skills. Emphasis placed
  upon the state of the learner, the learning process, and the
  conditions for learning. Prerequisite: sophomore standing.
\item
  \textbf{KIN 315 Methods Elementary Sch PE \& Health} Analysis of the
  program of physical education and health for the elementary school.
  Selection of activities, teaching methods, program planning, equipment
  and facilities, class management, and evaluation is examined.
  Prerequisite: Practicum in Education (EDU-215) or consent of
  department chair.
\item
  \textbf{KIN 347 Adapted Physical Education} A program of activities
  adapted for individuals with physical disabilities. Development of a
  rehabilitative exercise program for correction of physical handicaps
  or deviations.
\item
  \textbf{KIN 365 Measurement/Eval/Prescription KIN} Study of various
  methods of measurement and evaluation of motor skills and motor
  performance in physical activity. Construction of skills tests, proper
  testing procedures, and basic statistics used in evaluating the
  results covered. Assessment of physical fitness components followed by
  prescribed activity to improve those components is studied.
\item
  \textbf{KIN 385 Methods Strength Trng \& Condition} Application of
  resistance and movement training techniques. Topics include methods of
  teaching progressions for resistance training, flexibility,
  speed/agility training, stretching, plyometrics, safety, successful
  routines, and exercise prescription for beginning to advanced
  trainees. Prerequisites: Organismal and Ecological Biology (BIO-155),
  either Human Anatomy (BIO-215) or Anatomy and Physiology (BIO-103),
  and junior standing. NOTE: Students are encouraged to complete
  Advanced Weight Training (PE-103/-104) and Movement Training
  (PE-103/-104) before enrolling in KIN-385. \textbf{NOTE:}
  \emph{Students are encouraged to complete Advanced Weight Training
  (PE-103/-104) and Movement Training (PE-103/-104) before enrolling in
  KIN-385.}
\item
  \textbf{KIN 405 Program Design} A theory-based course that discusses
  different strength and conditioning techniques used to program
  workouts for students, athletes, and members of the community. The
  course examines physiological factors, such as volume, intensity,
  rest, frequency, duration, and the acute and chronic effects of
  exercise. Students design and evaluate their own strength and
  conditioning programs based on the needs of the individual.
  Prerequisite: Methods of Teaching Strength Training and Conditioning
  (KIN-385).
\item
  \textbf{KIN 410 Athletic Injury Treatment \& Rehab} Introduces various
  forms of individualized exercise and rehabilitation programs used in
  an athletic injury setting. Didactic and laboratory components provide
  the background and practical application of the principles and
  techniques related to the rehabilitation of injuries. Students will be
  able to identify signs and symptoms of the healing process, describe
  the indications and contraindications of treatment, and instruct
  patients on proper technique and execution of rehabilitation
  exercises. Surgical and non-surgical orthopedic injury rehabilitation
  protocols discussed with a special emphasis given to progressions back
  to sport. Prerequisite: Prevention and Care of Athletic Injuries \&
  Laboratory (KIN-175) \& Kinesiology (KIN-495).
\item
  \textbf{KIN 415 Meth Secondary School PE \& Health} Analysis of the
  program of physical education for the secondary schools. Selection of
  activities, teaching methods, program planning, equipment and
  facilities, class management, and evaluation. Includes 30-hour field
  experience. Prerequisite: Practicum in Education (EDU-215) or consent
  of department chair.
\item
  \textbf{KIN 440 Org-Admin of PE, Health \& Athl} Objectives,
  principles, and methods of organization and administration of physical
  education, health education, recreation, and athletics in elementary
  and secondary schools, as well as colleges. Prerequisite: sophomore
  standing.
\item
  \textbf{KIN 442 Physiology of Exercise} The study and evaluation of
  the effects of exercise upon the biological control systems of the
  human body. Topics include bioenergetics, exercise metabolism,
  endocrine function during and cardiopulmonary response to exercise,
  neuromuscular function, acid-base regulation, temperature regulation,
  and the effect of endurance training on various organ systems.
  Prerequisites: Organismal and Ecological Biology (BIO-155), and either
  both Human Anatomy (BIO-215) and Human Physiology (BIO-225), or
  Anatomy and Physiology (BIO-103).
\item
  \textbf{KIN 444 Ind Study-KIN} Independent investigation of a selected
  project in Kinesiology under the direction of a faculty member of the
  department. May be taken for an X status grade with consent of
  instructor prior to registration. Prerequisites: consent of instructor
  and department chair.
\item
  \textbf{KIN 494 Internship in Kin, Health \& Rec} Investigation of
  kinesiology, health, or recreation through voluntary field placement
  supervised by a faculty member of the department. Not available to
  prospective teachers in physical education. A minimum of 140 hours
  on-site experience is required. S/U basis only. One credit may be
  counted toward a major in kinesiology. Prerequisites: declared major
  in kinesiology or interdisciplinary major and consent of department
  chair.
\item
  \textbf{KIN 495 Kinesiology} Application of the principles of
  structure and mechanics involved in human movement. Prerequisites:
  Organismal and Ecological Biology (BIO-155), and either both Human
  Anatomy (BIO-215) and Human Physiology (BIO-225), or Anatomy and
  Physiology (BIO-103).
\end{itemize}

\section{Literature}\label{sec-literature}

Sodeman (Administrative Coordinator).

The following requirements are designed to provide a framework within
which the student can shape a program to suit individual interests: 1.
ENG 301 The Art of Literary Research 2. \textbf{One} of the following:
LIT 464 Seminar in World Literature A course by arrangement chosen in
consultation with Literature Administrative Coordinator 3. \textbf{Two}
world language courses in the same language, intermediate or advanced,
taken at Coe College with permission of Literature Administrative
Coordinator.\\
4. \textbf{At least} one literature course taught in a world language 5.
\textbf{Six} courses to be chosen in consultation with the Literature
Administrative Coordinator as indicated from the two lists below. No
more than three from the English department can be counted in the six
courses. a) \textbf{At least} four courses from the following list:
courses in literature in translation courses in English and American
literature and in creative writing offered by the English department
literature courses taught in a world language linguistics courses
offered by the English or world language departments Independent Study
b) \textbf{One or more} courses from the following list of core courses
to bring the total number of elective courses to six: AAM 137 African
American Literature or ENG 137 African American Literature ENG 110
Ancient Mythology ENG 207 Gender \& Lit:US Pluralism

\subsection{Senior Seminar Course}\label{senior-seminar-course}

\begin{itemize}
\tightlist
\item
  \textbf{LIT 464 Seminar in World Literature} Required of all
  literature majors in the senior year. While the subject changes from
  year to year, the seminar explores how literature from different
  countries relates to other human activities and disciplines. Thus, the
  emphasis may be on a historical period (literature and politics in the
  Renaissance); on an interdisciplinary topic (literature and Freudian
  analysis); or on a cross-cultural mix (literature, games and play).
  When the number of senior majors is too small to justify offering the
  seminar, other arrangements are made for majors to satisfy this
  requirement: either independent study or participation in an English
  department seminar, with the stipulation that the term paper be on a
  topic in comparative literature.
\end{itemize}

\section{Latin (Courses Only)}\label{sec-latin}

Langseth.

\subsection{Courses in Latin}\label{courses-in-latin}

\begin{itemize}
\tightlist
\item
  \textbf{CLA 155 Latin/Greek Origins Med Terminology} Examines the
  origins of contemporary medical terminology, in part by studying the
  development of a distinct technical vocabulary, with historical roots
  in the Greco-Roman, Arabic, and Modern-European worlds, which
  developed as physicians discovered distinct ways of communicating both
  with their patients and with each other.
\item
  \textbf{LTN 115 Basic Latin} An intensive examination and analysis of
  Latin grammar and syntax. Selected readings from the great literary
  works of the Republic and Empire. A combination of lecture, drill
  work, and discussion. Prerequisite: no prior instruction in Latin or
  fewer than two terms of secondary school Latin and consent of
  instructor.
\item
  \textbf{LTN 125 Selected Readings in Latin} Review of basic grammar
  and syntax and examination of more advanced grammar and syntax.
  In-depth readings from selected authors. Combination of lecture, drill
  work, and class discussion. Prerequisite: Basic Latin (LTN-115) or two
  or more terms of secondary school Latin and consent of instructor.
\item
  \textbf{+++MISSING INFO: c.ltn284/384.long +++} +++MISSING INFO:
  c.ltn284/384.desc +++
\end{itemize}

\section{Molecular Biology (Collateral
Major)}\label{sec-molecular-biology}

Leonardo (Administrative Coordinator). This major might be considered by
students interested in finding technical positions in academia or
industry after graduation or those planning to pursue a graduate program
in molecular biology, cell biology, or microbiology.

\subsection{Collateral Major in Molecular
Biology}\label{collateral-major-in-molecular-biology}

A major in molecular biology requires a minimum cumulative 2.0 GPA in
all courses counted toward the major. Concurrent completion of a primary
major in biology is required (see \textbf{?@sec-biology}). Students
choosing a collateral major in Molecular Biology may not select the
collateral major in Biochemistry. 1. CHM 221 Organic Chemistry I 2. CHM
321 Organic Chemistry II 3. CHM 322 Organic Laboratory 4. +++MISSING
INFO: c.bio345/345L.long +++ 5. BIO 405 Current Topics in Molecular
Biology 6. One of the following: +++MISSING INFO: c.chm431/431L.long +++
+++MISSING INFO: c.chm432/432L.long +++ 7. One of the following
sequences: +++MISSING INFO: c.bio325/325L.long +++ +++MISSING INFO:
c.bio415/415L.long +++ +++MISSING INFO: c.bio435/435L.long +++
+++MISSING INFO: c.bio455/455L.long +++ Strongly recommended: One of the
following sequences: +++MISSING INFO: c.phy165/165L.long +++ and
+++MISSING INFO: c.phy175/175L.long +++ +++MISSING INFO:
c.phy185/185L.long +++ and +++MISSING INFO: c.phy195/195L.long +++

\section{Mathematical Sciences}\label{sec-mathematical-sciences}

Cross (Chair, Fall), Herron, Hostetler, Hughes, Miller, Stobb, White
(Chair, Spring).

The department of mathematical sciences offers a complete range of
courses, with majors and minors available in COMPUTER SCIENCE, DATA
SCIENCE and MATHEMATICS, as well as courses in statistics for additional
breadth. The department adheres to its belief that the mathematical
sciences and the habits of mind that they engender are components of a
fine liberal arts education.

\subsection{Mathematics Major}\label{mathematics-major}

A major in mathematics requires a minimum cumulative 2.0 GPA in all
courses counted toward the major. 1. MTH 135 Calculus I 2. MTH 145
Calculus II 3. MTH 215 Foundations of Advanced Mathematics 4. MTH 255
Calculus III 5. MTH 265 Linear Algebra 6. MTH 385 Modern Algebra I 7. CS
125 Computer Science I 8. \textbf{One} of the following: MTH 415 Real
Analysis I MTH 445 Complex Analysis 9. \textbf{Three} of the following:
STA 315 Mathematical Probability STA 325 Mathematical Statistics MTH 305
Advanced Geometry MTH 325 Differential Equations WE MTH 395 Modern
Algebra II MTH 415 Real Analysis I (if not used to satisfy \#8) MTH 425
Real Analysis II MTH 444 Ind Study-Math MTH 445 Complex Analysis (if not
used to satisfy \#8) MTH 455 Mathematical Modeling MTH 463 Set Theory \&
Toplogy MTH 484 Special Topics \textbf{NOTE:} \emph{A course in
statistics and a course in geometry are required for the Iowa teaching
endorsement in mathematics at either the elementary or secondary school
level. Students planning to teach should consult with faculty in the
education department. }

\subsection{Mathematics Minor}\label{mathematics-minor}

\begin{enumerate}
\def\labelenumi{\arabic{enumi}.}
\tightlist
\item
  MTH 135 Calculus I
\item
  MTH 145 Calculus II
\item
  MTH 215 Foundations of Advanced Mathematics
\item
  \textbf{Three} of the following: STA 315 Mathematical Probability STA
  325 Mathematical Statistics MTH 255 Calculus III MTH 265 Linear
  Algebra MTH 305 Advanced Geometry MTH 325 Differential Equations WE
  MTH 385 Modern Algebra I MTH 395 Modern Algebra II MTH 415 Real
  Analysis I MTH 425 Real Analysis II MTH 444 Ind Study-Math MTH 445
  Complex Analysis MTH 455 Mathematical Modeling MTH 463 Set Theory \&
  Toplogy MTH 484 Special Topics
\end{enumerate}

\subsection{Courses in Mathematics}\label{courses-in-mathematics}

\begin{itemize}
\tightlist
\item
  \textbf{MTH 105 Math for Social Justice} An introduction to
  contemporary mathematical thinking with emphasis on its connections to
  society. Logical thinking and the ability to read critically are
  interwoven with elementary mathematical skills. The course
  concentrates on discussions about mathematics---about its nature, its
  content, and its applications to a variety of topics, such as
  management science, network science, finance, data, statistics,
  probability, fairness, apportionment, voting theory, and social
  choice. This course is appropriate for a varied audience. This course
  does not satisfy any of the requirements for a major or minor in the
  mathematical sciences. Prerequisite: Some ability in arithmetic,
  geometry, and elementary algebra.
\item
  \textbf{MTH 135 Calculus I} An introduction to the concepts of limits,
  continuity, differentiation of elementary functions, applications,
  definite and indefinite integrals, and the Fundamental Theorem.
  Prerequisite: three years of secondary school mathematics, Algebra and
  Trigonometry (MTH-115), or consent of instructor.
\item
  \textbf{MTH 145 Calculus II} Further study of the techniques of
  differentiation and integration, the calculus of exponential,
  logarithmic and trigonometric functions, sequences, series, and
  applications. Prerequisite: Calculus I (MTH-135) or consent of
  instructor.
\item
  \textbf{MTH 215 Foundations of Advanced Mathematics} A survey of
  material common to all advanced study of mathematics, including
  elements of formal logic, axiomatic set theory, induction, relations,
  functions, cardinality, and various other topics in discrete
  mathematics. This course is specifically intended to serve both as a
  transition to upper-division mathematics courses and also as a survey
  of some areas of mathematics important for future teachers of
  mathematics and related fields. Prerequisite: Calculus I (MTH-135) or
  Computational Linear Algebra (MTH-165) or consent of instructor.
\item
  \textbf{MTH 255 Calculus III} Further study of curves, surfaces, power
  series, partial derivatives, iterated and multiple integrals, and an
  introduction to differential and integral vector calculus.
  Prerequisite: Calculus II (MTH-145) or consent of instructor.
\item
  \textbf{MTH 265 Linear Algebra} A study of the elementary concepts of
  vector spaces, including matrix algebra, basis and dimension, inner
  products, linear transformations. Prerequisites: Calculus II (MTH-145)
  and Foundations of Advanced Math (MTH-215) or consent of instructor.
\item
  \textbf{MTH 305 Advanced Geometry} A course designed to give the
  student an introduction to the modern approaches to geometry at an
  advanced level. Topics include foundations, Euclidean, projective, and
  non-Euclidean geometries. Prerequisites: Calculus II (MTH-145) and
  Foundations of Advanced Math (MTH-215), or consent of instructor.
\item
  \textbf{MTH 325 Differential Equations WE} The theory, solution,
  techniques, and applications of elementary types of ordinary
  differential equations. Prerequisite: Calculus II (MTH-145) or consent
  of instructor.
\item
  \textbf{MTH 385 Modern Algebra I} A rigorous introduction to advanced
  algebra. Topics include mappings, operations, groups, rings, fields,
  integral domains, and homomorphisms. Prerequisite: Computational
  Linear Algebra (MTH-165) and Foundations of Advanced Math (MTH-215),
  or consent of instructor.
\item
  \textbf{MTH 395 Modern Algebra II} A continuation of Modern Algebra I
  (MTH 385), including homomorphisms, permutation groups, symmetry,
  unique factorization domains, quotient rings, and field extensions.
  Prerequisite: Modern Algebra I (MTH-385). (Offered on an occasional
  basis)
\item
  \textbf{MTH 415 Real Analysis I} A rigorous introduction to selected
  topics in analysis. Topics selected from number systems, Euclidean
  spaces, metric spaces, limits, continuity, derivatives, and integrals.
  Prerequisites: Calculus II (MTH-145) and Foundations of Advanced Math
  (MTH-215).
\item
  \textbf{MTH 425 Real Analysis II} A continuation of Real Analysis I
  (MTH-415), including a study of such topics as Riemann Stieltjes and
  Lebesgue integration, series and series expansions. Prerequisite: Real
  Analysis I (MTH-415). (Offered by arrangement)
\item
  \textbf{MTH 444 Ind Study-Math} An opportunity for independent and
  intensive study in mathematics. May be taken for an X status grade
  with consent of instructor prior to registration. Prerequisite:
  appropriate background courses depending on the nature of the work
  planned and consent of supervising instructor. (Offered by
  arrangement)
\item
  \textbf{MTH 445 Complex Analysis} An introduction to the theory,
  techniques, and applications of functions of a complex variable.
  Topics include elementary and analytic functions, limits,
  differentiation, integration, series, mappings, and applications.
  Prerequisite: Calculus III (MTH-255) or consent of instructor.
\item
  \textbf{MTH 454 Research in Mathematics} Individual or group
  investigation with a mathematics faculty member on a research topic of
  mutual interest. The student must obtain approval for a specific
  project and make necessary arrangements prior to the term of
  registration for the course. This course is offered on an P/NP basis
  and does not satisfy any of the requirements for a major or minor in
  computer science. May be taken more than once for credit for a maximum
  of 2.0 credits. Prerequisites: Foundations of Advanced Mathematics
  (MTH-215) and consent of supervising instructor. (0.2-1.0 credit;
  Offered by arrangement)
\item
  \textbf{MTH 455 Mathematical Modeling} An introduction to the
  application of mathematical techniques used in the solution of
  real-world problems. These techniques include interpolation, ordinary
  differential equations, taylor series expansions, curve fitting,
  matrix inversion, numerical differentiation, and integration.
  Prerequisites: Introduction to Programming (CS-125); and Calculus II
  (MTH-145)
\item
  \textbf{MTH 463 Set Theory \& Toplogy} A rigorous introduction to
  abstract set theory and to metric and topological spaces, including a
  discussion of such topics as separation, connectedness, and
  compactness. Prerequisites: Calculus II (MTH-145) and Foundations of
  Advanced Math (MTH-215), or consent of instructor.
\item
  \textbf{MTH 484 Special Topics} An opportunity to study current and
  topical material unavailable through the regular catalog offerings.
  Prerequisites: Calculus II (MTH-145) and Foundations of Advanced Math
  (MTH-215), or consent of instructor.
\item
  \textbf{MTH 494 Internship in Mathematics} Investigation of a career
  area related to the student's interest in mathematics supervised by a
  faculty member of the department in cooperation with the Internship
  Specialist. A minimum of 140 hours on-site experience is required. S/U
  basis only. This course does not satisfy any of the requirements for a
  major or minor in mathematics. Prerequisites: junior standing and
  consent of supervising instructor. (Offered by arrangement)
\end{itemize}

\section{Music}\label{sec-music}

Benson, Brewer, Carson (Chair), Falk, Hanisch, Lawrence, Lovegood,
Shanley, Weiler, B. Wolgast, Zeidieh

Teaching Artists: Altfillisch, Bishop, Brumwell, Capistran, Farley,
Fleer, Hall, Harris, Holmes-Bendixen, Marrs, Morton, Nagel, Nothnagle,
Phelps, Reznicow, Rothrock, Schamberger, Sentman, Terrell, Wissenberg,
M. Wolgast

All Coe students are encouraged to participate in music as part of their
liberal arts education. Choral and instrumental ensembles, private
lessons in applied music, and academic courses are open to non-music
majors.

The \emph{Coe College Music Department Faculty/Student Handbook}
supplements the descriptions of courses and requirements for music
majors and outlines departmental policies and procedures. Copies of the
\emph{Handbook} are available in the Music Office, Marquis Hall 103 or
online at \texttt{www.coe.edu/academics/majors-areas-study/music}.

Students who major in music may select either the Bachelor of Arts
degree or the Bachelor of Music degree.

\subsection{BACHELOR OF MUSIC}\label{bachelor-of-music}

Candidates for the \textbf{BACHELOR OF MUSIC} degree must successfully
complete: 1. \textbf{Six} theory and history courses with a grade of
``C'' (2.0) or higher: MU 109 Theory of Music I/+++MISSING INFO:
c.mu109L.long +++ MU 209 Theory of Music II/+++MISSING INFO:
c.mu209L.long +++ MU 309 Theory of Music III/+++MISSING INFO:
c.mu309L.long +++ MU 255 Music History \& Literature I MU 355 Music
History \& Literature II MU 458 Music History \& Literature III 2. MU
285 Conducting I 3. First-Year Experience requirements (see
Chapter~\ref{sec-first-year-experience}) 4. The writing emphasis
requirement of the general education requirements for the B.A. degree
(see Chapter~\ref{sec-writing-emphasis-courses}) 5. The \emph{Keyboard
Fundamentals Examination} by the end of the sophomore year. Exceptions
must be approved by a majority vote of the music faculty. After
successful completion of this examination, credits may be accumulated
toward a secondary performance area in piano. \textbf{NOTE:} \emph{A
description of this examination is given in the Coe College Music
Department Faculty/Student Handbook at
\texttt{https://www.coe.edu/academics/majors-areas-study/music}} 6. The
Bachelor of Music Advanced Standing Assessment (BMASA), an audition
evaluation whereby the music faculty assesses the suitability of a
student for student teaching or for the presentation of a senior
recital. Students performing at the approved level are permitted to
register for lessons and classes at the Advanced Level, leading to
student teaching or the presentation of a senior recital. Students who
are not recommended for study at the Advanced Level are required to
select a degree program other than the Bachelor of Music program, or to
reapply in a later term.\\
7. The Senior Assessment Examination in the final term prior to
graduation. 8. \textbf{One} of the four areas of study applicable
towards the Bachelor of Music degree:

\begin{verbatim}
**Composition**
1. MU 195 Music Production
2. MU 409 Theory of Music IV/+++MISSING INFO: c.mu409L.long +++
3. MUA 413 Orchestration
4. MUA 423 Saxophone (Adv Std)
5. **Eight** terms of participation in the Applied Music Course, which must include four terms each of MUA-202C and MUA-303C
6. **One** course each from the social sciences, natural sciences, and humanities. (New York Term may be used to satisfy the humanities requirements, but the two required language courses described below may not.) 
7. **Two** courses in French in consecutive terms (may not be used to satisfy the humanities requirement of #6) 
8. **Two** additional non-music courses 
9. MUA 490 Senior Recital

**Keyboard and Instrumental Performance**
1. MU 409 Theory of Music IV/+++MISSING INFO: c.mu409L.long +++
2. **Four** of the following:
    MU 140 Film Music
    MU 151 Record Label
    MU 157 Introduction to Jazz History
    MU 195 Music Production
    MU 270 Musical Theatre Acting
    +++MISSING INFO: c.mu166/284.long +++
    MU 385 Conducting II
    MU 444 Ind Study-Music
    +++MISSING INFO: c.mua133V.long +++
    +++MISSING INFO: c.mua134V.long +++
    +++MISSING INFO: c.mua413V.long +++
    +++MISSING INFO: c.mua423V.long +++
    One course credit in MUA-courses beyond those required in the Applied Music Course.
3. **Eight** terms of participation in the Applied Music Course. 
4. **One** course each from the social sciences, natural sciences, and humanities. (New York Term may be used to satisfy the humanities requirements.) 
5. **Two** courses in French in consecutive terms (may not be used to satisfy the humanities requirement of #4)
6. MUA 490 Senior Recital

**Vocal Performance**
1. +++MISSING INFO: c.mua133V.long +++
2. +++MISSING INFO: c.mua134V.long +++
3. MU 409 Theory of Music IV/+++MISSING INFO: c.mu409L.long +++
4. **Four** of the following:
    MU 140 Film Music
    MU 151 Record Label
    MU 157 Introduction to Jazz History
    MU 195 Music Production
    MU 270 Musical Theatre Acting
    +++MISSING INFO: c.mu166/284.long +++
    MU 385 Conducting II
    MU 444 Ind Study-Music
    +++MISSING INFO: c.mua133V.long +++
    MUA 413 Orchestration
    MUA 423 Saxophone (Adv Std)
    One course credit in MUA-courses beyond those required in the Applied Music Course. 
5. **Eight** terms of participation in the Applied Music Course. 
6. **One** course each from the social sciences, natural sciences, and humanities. (New York Term may be used to satisfy the humanities requirements.) 
7. **Two** courses in French in consecutive terms (may not be used to satisfy the humanities requirement of #6) 
8. MUA 490 Senior Recital

**Music Education (Vocal or Instrumental)**
To complete the Vocal or Instrumental Music Education area of study in four years, a student must begin the sequence of courses during the first year. A student entering either area of study as a sophomore should anticipate spending a fifth year to complete all degree requirements. 
In addition to the following requirements, according to state regulations, all teachers in Iowa “shall acquire a core of liberal arts knowledge including, but not limited to, mathematics, natural sciences, social sciences, and humanities.”  The Mathematics and Natural Science course requirements may be met through coursework at Coe, comparable courses from an accredited college, or a score of 4 or higher on AP examinations in mathematics or science.  Social Science and Humanities core are met with the required EDU courses and MU Music History and Literature courses.
1. EDU 105 Foundations of Education
2. EDU 117 Exceptional Learners
3. EDU 187 Human Relations
4. EDU 195 Educ Psychology & Development
5. EDU 219 Educational Technology Lab
6. MU 205 Practicum in Music Education
7. MU 360 Elementary & General Music Methods
8. **One** of the following:
    MU 361 Choral Music Methods, MU 362 InstrMusMethChrlTeachr, and +++MISSING INFO: c.mua134.long +++ (for Vocal Music Education majors)
    MU 363 Instrumental Music Methods and MU 364 ChrlMusMethInstTeach (for Instrumental Music Education majors)
9. MU 385 Conducting II
10. MU 421 Student Teaching Elementary Music
11. MU 422 Student Teaching Secondary Music
12. **Seven** terms of participation in the Applied Music Course.
13. Fulfillment of the general licensure grade point requirement (minimum 2.7 cumulative and in the major) 
**NOTE:** *A recital is not required to complete the music education area of study but may be given with the approval of the studio instructor. *
\end{verbatim}

\subsection{BACHELOR OF ARTS: MUSIC
MAJOR}\label{bachelor-of-arts-music-major}

Candidates for the \textbf{BACHELOR OF ARTS} degree with a major in
music must successfully complete: 1. MU 109 Theory of Music I/+++MISSING
INFO: c.mu109L.long +++ 2. MU 209 Theory of Music II/+++MISSING INFO:
c.mu209L.long +++ 3. MU 309 Theory of Music III/+++MISSING INFO:
c.mu309L.long +++ 4. MU 458 Music History \& Literature III 5.
\textbf{Four} terms of participation in the Applied Music Course with
piano as the primary or secondary performing area. 6. \textbf{One} of
the following: MU 255 Music History \& Literature I MU 355 Music History
\& Literature II 7. \textbf{One} of the following (if not used to
satisfy \#6): +++MISSING INFO: c.mua131V.long +++ MU 140 Film Music MU
151 Record Label MU 157 Introduction to Jazz History MU 195 Music
Production MU 255 Music History \& Literature I MU 270 Musical Theatre
Acting +++MISSING INFO: c.mu166/284.long +++ MU 285 Conducting I MU 355
Music History \& Literature II MU 409 Theory of Music IV +++MISSING
INFO: c.mua133V.long +++ +++MISSING INFO: c.mua134V.long +++ 8. The
Senior Assessment Examination in the final term prior to graduation

\subsection{Jazz Emphasis}\label{jazz-emphasis}

Students completing a \textbf{BACHELOR OF ARTS} degree may choose to
supplement their music major with an Emphasis in Jazz. Concurrent
completion of a major in music is required. A minimum of 6.1 course
credits must be taken that do not count toward a major in music. 1. MU
157 Introduction to Jazz History 2. MU 409 Theory of Music IV/+++MISSING
INFO: c.mu409L.long +++ 3. \textbf{Five} terms of MUA 102 Jazz \&
Improvisation Lab 4. \textbf{Seven} terms of MUA 101 Jazz Ensemble 5.
\textbf{Two} terms of +++MISSING INFO: c.mua102K.long +++ 6.
\textbf{Two} terms of +++MISSING INFO: c.mua103K.long +++ 7. AAM 107
Intro to African American Studies

\subsection{Music Industry Emphasis}\label{music-industry-emphasis}

Students completing a \textbf{BACHELOR OF ARTS} degree may choose to
supplement their music major with an Emphasis in Music Industry.
Concurrent completion of a major in music is required. A minimum of 6.5
course credits must be taken that do not count toward a major in music.
1. MU 195 Music Production 2. MU 202 Arts Administration 3. \textbf{One}
of the following: PR 205 Public Relations RHE 265 Professional Writing
4. \textbf{One} of the following: BUS 250 Principles of Management BUS
330 Principles of Marketing 5. \textbf{Three} terms of MU 151 Record
Label 6. \textbf{Two} terms of MU 251 Record Label Seminar

\subsection{Musical Theatre Emphasis}\label{musical-theatre-emphasis}

Students completing a \textbf{BACHELOR OF ARTS} degree may choose to
supplement their music major with an Emphasis in Musical Theatre.
Concurrent completion of a major in music with voice as the primary
instrument is required. A minimum of 6.4 course credits must be taken
that do not count toward a major in music. 1. THE 150 Acting I 2. THE
170 Voice \& Diction 3. THE 270 Musical Theatre Acting 4. \textbf{Two}
terms of +++MISSING INFO: c.mua202V.long +++ with musical theatre
repertoire (may not be used to satisfy the requirement of \#1) 5.
\textbf{One} additional course credit in practical musical theatre
chosen from a combination of the following: +++MISSING INFO:
c.mua130V.long +++ +++MISSING INFO: c.mua131V.long +++ A musical theatre
internship approved by the music or theatre arts department 6.
\textbf{Six} seven-week dance courses (DAN-101 through DAN-142) (0.2 cc)

\subsection{Pre-Music Therapy
Emphasis}\label{pre-music-therapy-emphasis}

Students completing a \textbf{BACHELOR OF ARTS} degree may choose to
supplement their music major with an Emphasis in Pre-Music Therapy.
Concurrent completion of a major in music is required.\\
The Bachelor of Arts in Music, Pre-Music Therapy Emphasis, is designed
to prepare students to apply to a Music Therapy certification program.
It is not a free-standing major, and it does not include music therapy
courses. Rather, the courses listed below are intended to better prepare
Bachelor of Arts in Music students for future study in Music Therapy at
an academic program approved by the American Music Therapy Association.
Students must complete all requirements for a Bachelor of Arts in Music,
in addition to the courses listed below. These include 5 non-music
course credits and 2 additional music course credits for a minimum total
of 7 course credits that must be taken that do not count toward a major
in music.\\
Courses taken to satisfy requirements in the Bachelor of Arts in Music
may NOT also be used to satisfy the requirements of the Pre-Music
Therapy Emphasis below, but all courses MAY be used towards general
education requirements. 1. PSY 100 Introductory Psychology 2. BIO 103
Anatomy \& Physiology 3. BIO 155 Organismal \& Ecological Biology \&
+++MISSING INFO: c.bio155L.long +++ (concurrent with BIO-155) 4.
\textbf{One} of the following: EDU 117 Exceptional Learners EDU 187
Human Relations EDU 195 Educ Psychology \& Development PSY 235 Abnormal
Psychology PSY 315 Learning \& Behavior 5. \textbf{One} of the
following: THE 100 Introduction to Theatre THE 150 Acting I THE 160
Movement for the Stage THE 170 Voice \& Diction THE 270 Musical Theatre
Acting 6. \textbf{Two} credits (or more) from the following: MU 195
Music Production MU 285 Conducting I MU 409 Theory of Music IV
+++MISSING INFO: c.mua102K.long +++ Additional terms of ensemble
participation, beyond those required for the Bachelor of Arts in Music:
Concert Band (0.3 each), Jazz Band (0.2 each), Concert Choir (0.3 each),
Chorale (0.2 each), Orchestra (0.3 each), or Handbells (0.2 each)
Additional terms of hour lessons, beyond those required for the Bachelor
of Arts in Music, on your primary instrument (0.6 each):\\
+++MISSING INFO: c.mua104K.long +++ +++MISSING INFO: c.mua151S.long +++
+++MISSING INFO: c.mua300A.long +++

Strongly recommended: - MU 255 Music History \& Literature I \textbf{or}
MU 355 Music History \& Literature II (whichever is not taken to fulfill
the BA in Music requirement) - MU 285 Conducting I and MU 385 Conducting
II - Additional music performance courses (ensembles and lessons) -
Additional theatre, art, biology, business, education, psychology,
sociology, and statistics courses - Fulfill the academic practicum
requirement with a music therapy-related internship

\subsection{Music Minor}\label{music-minor}

The minor in music consists of six course credits. At least four of
these credits must be MU courses. MUA credits (lessons and ensembles)
must be taken as the Applied Music Course if they are to apply to the
music minor. Individual partial credit courses may not accumulate toward
credit for a minor in music.

\subsection{Departmental Notes}\label{departmental-notes}

\begin{itemize}
\tightlist
\item
  Private music lessons are available for credit only. Declared music
  majors who have achieved junior status (see
  \textbf{?@sec-class-designation}) and who perform in a major ensemble
  are not charged for lessons. Other Coe students are charged by term
  according to the total registered lesson credit (see
  \textbf{?@sec-financial-information}). Students who have not attended
  three lessons by the end of the third week of classes are dropped from
  music lessons and may be reinstated only by petition. After the third
  week of lessons, students will be charged for the full term, even if
  they choose to drop.
\item
  Attendance at the numerous musical events on campus throughout the
  year is an essential part of the student's musical training. Students
  taking lessons for 0.6 credit must attend a specific number of these
  events each term and the weekly Recital Hour. They must also perform
  in Recital Hour once each term, except for the first term of the first
  year of study.
\end{itemize}

\subsection{APPLIED MUSIC}\label{applied-music}

\subsection{The Applied Music Course}\label{the-applied-music-course}

Intensive studies in performance and repertoire through lessons,
ensembles, pedagogy, and techniques. The applied music course
requirements are: 1. An hour lesson each week, as specified in the
\emph{Coe College Music Department Faculty/Student Handbook} for each
major, along with participation and attendance at the weekly Recital
Hour and concerts as described above in \emph{Departmental Notes}. (0.6
course credit) 2. Participation in one of the following ensembles, as
assigned by audition: Concert Band, Concert Choir, Orchestra, Chorale.
3. Additional ensembles, lessons, or courses in pedagogy, techniques,
and literature (0.1-0.6 course credit) that are required for any of the
areas of study that lead to one of the music degrees. The \emph{Coe
College Music Department Faculty/Student Handbook} (see above) contains
descriptions and additional information about these ensembles and
courses. \textbf{Prerequisite:} an audition; consent of department
chair.

\subsection{APPLIED MUSIC CLASSES}\label{applied-music-classes}

Artist teachers in violin, viola, cello, string and electric bass,
piano, piano accompanying\textsuperscript{+},
harpsichord\textsuperscript{+}, organ\textsuperscript{+}, flute, oboe,
clarinet, saxophone, bassoon, trumpet, horn, trombone, low brass,
percussion, guitar, harp, composition\textsuperscript{+}, and voice are
available to all students who wish to study applied music.
\textbf{\textsuperscript{+}Prerequisite:} proficiency in piano and
consent of instructor.

Composition lessons and classes in keyboard, brass, woodwind, and string
literature are also available. Credit for applied music study is granted
after the student has satisfactorily passed an examination by a faculty
committee (jury exam) at the close of each term. Two hours daily
practice per weekly hour lesson are expected of all students taking
applied music.

\subsection{APPLIED BRASSES--LESSONS}\label{applied-brasseslessons}

\begin{verbatim}
+++MISSING INFO: c.mua101B.long +++
+++MISSING INFO: c.mua202B.long +++
+++MISSING INFO: c.mua303B.long +++
+++MISSING INFO: c.mua111B.long +++
+++MISSING INFO: c.mua212B.long +++
+++MISSING INFO: c.mua313B.long +++
+++MISSING INFO: c.mua121B.long +++
+++MISSING INFO: c.mua222B.long +++
+++MISSING INFO: c.mua323B.long +++
+++MISSING INFO: c.mua131B.long +++
+++MISSING INFO: c.mua232B.long +++
+++MISSING INFO: c.mua333B.long +++
\end{verbatim}

\subsection{APPLIED
COMPOSITION--LESSONS}\label{applied-compositionlessons}

\begin{verbatim}
+++MISSING INFO: c.mua101C.long +++
+++MISSING INFO: c.mua202C.long +++
+++MISSING INFO: c.mua303C.long +++
MUA 413 Orchestration
MUA 423 Saxophone (Adv Std)
\end{verbatim}

\subsection{APPLIED KEYBOARD--LESSONS}\label{applied-keyboardlessons}

\begin{verbatim}
+++MISSING INFO: c.mua100K.long +++
+++MISSING INFO: c.mua101K.long +++
+++MISSING INFO: c.mua102K.long +++
+++MISSING INFO: c.mua103K.long +++
+++MISSING INFO: c.mua202K.long +++
+++MISSING INFO: c.mua303K.long +++
+++MISSING INFO: c.mua304K.long +++
+++MISSING INFO: c.mua111K.long +++
+++MISSING INFO: c.mua212K.long +++
+++MISSING INFO: c.mua313K.long +++
+++MISSING INFO: c.mua121K.long +++
+++MISSING INFO: c.mua222K.long +++
+++MISSING INFO: c.mua323K.long +++
\end{verbatim}

\subsection{APPLIED
PERCUSSION--LESSONS}\label{applied-percussionlessons}

\begin{verbatim}
+++MISSING INFO: c.mua101P.long +++
+++MISSING INFO: c.mua202P.long +++
+++MISSING INFO: c.mua303P.long +++
\end{verbatim}

\subsection{APPLIED STRINGS--LESSONS}\label{applied-stringslessons}

\begin{verbatim}
+++MISSING INFO: c.mua101S.long +++
+++MISSING INFO: c.mua202S.long +++
+++MISSING INFO: c.mua303S.long +++
+++MISSING INFO: c.mua111S.long +++
+++MISSING INFO: c.mua212S.long +++
+++MISSING INFO: c.mua313S.long +++
+++MISSING INFO: c.mua121S.long +++
+++MISSING INFO: c.mua222S.long +++
+++MISSING INFO: c.mua323S.long +++
+++MISSING INFO: c.mua131S.long +++
+++MISSING INFO: c.mua232S.long +++
+++MISSING INFO: c.mua333S.long +++
+++MISSING INFO: c.mua141S.long +++
+++MISSING INFO: c.mua242S.long +++
+++MISSING INFO: c.mua343S.long +++
+++MISSING INFO: c.mua151S.long +++
+++MISSING INFO: c.mua252S.long +++
+++MISSING INFO: c.mua353S.long +++
\end{verbatim}

\subsection{APPLIED VOICE--LESSONS}\label{applied-voicelessons}

(Offerings vary each term and include sight singing and/or Alexander
Technique.) +++MISSING INFO: c.mua101V.long +++ +++MISSING INFO:
c.mua110V.long +++ +++MISSING INFO: c.mua202V.long +++ +++MISSING INFO:
c.mua303V.long +++ +++MISSING INFO: c.mua300A.long +++

\subsection{APPLIED WOODWINDS--LESSONS}\label{applied-woodwindslessons}

\begin{verbatim}
+++MISSING INFO: c.mua101W.long +++
+++MISSING INFO: c.mua202W.long +++
+++MISSING INFO: c.mua303W.long +++
+++MISSING INFO: c.mua111W.long +++
+++MISSING INFO: c.mua212W.long +++
+++MISSING INFO: c.mua313W.long +++
+++MISSING INFO: c.mua121W.long +++
+++MISSING INFO: c.mua222W.long +++
+++MISSING INFO: c.mua323W.long +++
+++MISSING INFO: c.mua131W.long +++
+++MISSING INFO: c.mua232W.long +++
+++MISSING INFO: c.mua333W.long +++
+++MISSING INFO: c.mua141W.long +++
+++MISSING INFO: c.mua242W.long +++
+++MISSING INFO: c.mua343W.long +++
\end{verbatim}

\subsection{APPLIED PEDAGOGY/ ENSEMBLE AND
LITERATURE}\label{applied-pedagogy-ensemble-and-literature}

\begin{verbatim}
+++MISSING INFO: c.mua133B.long +++
+++MISSING INFO: c.mua134B.long +++
+++MISSING INFO: c.mua133K.long +++
+++MISSING INFO: c.mua134K.long +++
+++MISSING INFO: c.mua133P.long +++
+++MISSING INFO: c.mua134P.long +++
+++MISSING INFO: c.mua133S.long +++
+++MISSING INFO: c.mua134S.long +++
+++MISSING INFO: c.mua133W.long +++
+++MISSING INFO: c.mua134W.long +++
+++MISSING INFO: c.mua134V.long +++
\end{verbatim}

\subsection{APPLIED TECHNIQUES}\label{applied-techniques}

\begin{verbatim}
MUA 143 Instrumental Techniques:Strings
MUA 153 Instr Tech:Single Reeds & Flute
MUA 163 Instr Tech: Double Reeds
MUA 173 Instr Tech: High Brass
MUA 183 Instr Tech: Low Brass
MUA 193 Instr Tech: Percussion
\end{verbatim}

\subsection{COURSES IN MUSIC BY
CATEGORY}\label{courses-in-music-by-category}

\subsection{ENSEMBLES}\label{ensembles}

Ensembles are open, by audition, to all Coe students, as well as being
required of all students pursuing a degree in music. Credit for
participation in these regularly-offered ensembles is available under
the following schedule. Students are encouraged to participate in these
ensembles as often as possible. Additional ensembles, offered on a
periodic basis, are listed in the \emph{Coe College Music Department
Faculty/Student Handbook}. MUA 100 Concert Band MUA 101 Jazz Ensemble
MUA 102 Jazz \& Improvisation Lab MUA 103 Symphony Orchestra MUA 104
Concert Choir MUA 105 Crimson Singers MUA 110 Vocal Jazz Ensemble MUA
111 Coe Handbell Ensemble +++MISSING INFO: c.mua130V.long +++ +++MISSING
INFO: c.mua131V.long +++ MUA 284 Applied Music: Special Topics

\subsection{GENERAL SURVEY COURSES}\label{general-survey-courses}

\begin{verbatim}
MU 100 Experiencing Music
MU 101 Music Fundamentals
MU 140 Film Music
MU 151 Record Label
MU 251 Record Label Seminar
MU 157 Introduction to Jazz History
MU 161 Hip Hop Workshop
MU 166 Topics in Music:Non-Western Perspct
MU 270 Musical Theatre Acting
MU 284 Topics in Music
+++MISSING INFO: c.mua133V.long +++
+++MISSING INFO: c.mua134V.long +++
\end{verbatim}

\subsection{THEORY COURSES}\label{theory-courses}

\begin{verbatim}
MU 109 Theory of Music I
+++MISSING INFO: c.mu109L.long +++
MU 209 Theory of Music II
+++MISSING INFO: c.mu209L.long +++
MUA 211 Group Sight-Singing Lessons
MU 309 Theory of Music III
+++MISSING INFO: c.mu309L.long +++
MU 409 Theory of Music IV
+++MISSING INFO: c.mu409L.long +++
\end{verbatim}

\subsection{COMPOSITION COURSES}\label{composition-courses}

\begin{verbatim}
MU 195 Music Production
MUA 413 Orchestration
MUA 423 Saxophone (Adv Std)
\end{verbatim}

\subsection{CONDUCTING COURSES}\label{conducting-courses}

\begin{verbatim}
MU 285 Conducting I
MU 385 Conducting II
\end{verbatim}

\subsection{HISTORY COURSES}\label{history-courses}

\begin{verbatim}
MU 255 Music History & Literature I
MU 355 Music History & Literature II
MU 458 Music History & Literature III
\end{verbatim}

\subsection{MUSIC EDUCATION COURSES}\label{music-education-courses}

\begin{verbatim}
MU 205 Practicum in Music Education
MU 360 Elementary & General Music Methods
MU 361 Choral Music Methods
MU 362 InstrMusMethChrlTeachr
MU 363 Instrumental Music Methods
MU 364 ChrlMusMethInstTeach
MU 421 Student Teaching Elementary Music
MU 422 Student Teaching Secondary Music
+++MISSING INFO: c.mua104K.long +++
\end{verbatim}

\subsection{ADVANCED STUDY IN MUSIC}\label{advanced-study-in-music}

\begin{verbatim}
MUA 490 Senior Recital
MU 444 Ind Study-Music
\end{verbatim}

\subsection{Courses in Music}\label{courses-in-music}

\textbf{MU Prefix} - \textbf{MU 100 Experiencing Music} Provides a
framework for informed music listening and for developing an
appreciation for a diverse variety of musical styles. Emphasis is on
traditional classical music, with some attention to jazz, electronic,
and avant-garde styles. Attendance at live concerts is an integral part
of the course. This course does not satisfy any of the requirements for
a major in music. - \textbf{MU 101 Music Fundamentals} An introduction
to the materials of music and an understanding of the musical system.
The course includes basic keyboard knowledge as well as beginning song
writing. - \textbf{MU 109 Theory of Music I} For students majoring in
music and other students with background in music. Develops keyboard
harmony, melodic and harmonic dictation, improvisation, four-part
writing up to and including dominant seventh chords, and introduction to
music notation software. Students seeking a BA in Music or one of the
Bachelor Music degrees must also register for an appropriate Aural
Skills Lab course. Prerequisite: Music Theory Placement exam or consent
of the instructor. - \textbf{+++MISSING INFO: c.mu109L.long +++}
+++MISSING INFO: c.mu109L.desc +++ - \textbf{MU 140 Film Music} An
introduction to the history, principles, and techniques of music in
film. In addition to lecture, film viewing and analysis, critical
reading and response, student presentation, and short film-music
projects elucidate the genre. - \textbf{MU 151 Record Label}
Team-oriented class environment offering students practical experience
operating 399 Records, a student-run record label. Students make all
business and creative decisions, including discovering and developing
performing artists on campus, recording, promoting and distributing a
finished music product each term. May be taken more than once. (0.5
course credit) - \textbf{MU 157 Introduction to Jazz History} Surveys
many styles of jazz by studying them in a historical perspective.
Listening, discussion, and lecture components are emphasized. (Offered
May Term only) - \textbf{MU 161 Hip Hop Workshop} Blend of seminar and
studio sessions focusing on hip-hop repertoire, beat arrangement, and
lyrical structure. Additional emphasis on vocal recording and editing
proficiency, as well as active listening exercises to better apply music
production techniques to original student works. Participation in the
creative process is required, and collaboration among students is
encouraged. - \textbf{MU 166 Topics in Music:Non-Western Perspct} A
topics course that explores the music of one of more non-Western
cultures, largely from a musicological viewpoint. Topics could include
survey courses, such as World Music, or courses focused on the music of
one particular continent, country, or sub-group. Prerequisite: consent
of instructor. (Offered on an occasional basis) - \textbf{MU 195 Music
Production} Exploration of audio engineering and recording practices,
with special attention to electronic and popular music. Overview of
intellectual property in the arts, the evolution of audio technology,
and cultural movements that influenced new forms of experimental and
commercial music. Students are introduced to the fundamentals of
acoustics and sound design through audio production software, with
hands-on projects detailing methods for producing music in a variety of
genres. - \textbf{MU 202 Arts Administration} Introduces principles and
the development of skills associated with the management of arts
organizations. Financial and program planning, audience engagement,
marketing and publicity. Supplemental guest lectures by professionals
highlight arts advocacy and specific challenges facing nonprofits. -
\textbf{MU 205 Practicum in Music Education} For students considering
the teaching profession. Class discussions and reading assignments
explore the purpose of music education in the public schools, requisites
of good teachers and good teaching, and basic philosophies and methods
of music education. Students spend a minimum of 60 hours observing and
assisting public school music teachers in a range of activities.
Prerequisites: Educational Foundations (EDU-205) and admission to the
Teacher Education Program, or consent of Department Chair. (Offered
Spring Term) - \textbf{MU 209 Theory of Music II} Continuation of Theory
of Music I. Prerequisite: Theory of Music I (MU-109) or consent of
instructor. - \textbf{+++MISSING INFO: c.mu209L.long +++} +++MISSING
INFO: c.mu209L.desc +++ - \textbf{MU 251 Record Label Seminar} Same as
MU-151, but at an advanced level. In addition to a weekly seminar
providing intensive discussion of music industry topics, students
oversee recording sessions and lead development meetings for 399
Records, as well as conduct independent research suitable to career
aspirations. Students may not register for both MU-151 and MU-251 in the
same term. May be taken more than once for credit for a maximum of 2.0
credits. Prerequisites: junior standing and MU-151 Record Label. (0.5
course credit) - \textbf{MU 255 Music History \& Literature I} Study of
the evolution of western music and musical systems, forms, styles, and
media from ancient Near East, Greece, and Rome through the first half of
the 18th century. Prerequisite: Theory of Music II (MU- 125) or consent
of instructor. - \textbf{MU 270 Musical Theatre Acting} A
performance-based studio course focusing on the development of basic
skills necessary for musical theatre performance. Students become
familiar with the specialized requirements necessary for the merging of
singing with dramatic action. Periodic performance projects (solos,
duets, and ensemble numbers---some including dialogue) are supplemented
by student research projects. The course is also designed to introduce
students to a wide-ranging repertoire of available audition material.
Additional rehearsal time outside of class is required. - \textbf{MU 284
Topics in Music} A course of selected focus that centers on a particular
musical issue, problem, theory, or methodology. Topics vary, and they
include, but are not limited to, Advanced Form and Analysis, Rock and
Roll History, Music of the 1960's, and Modern Musical Theatre. May be
taken more than once for credit, provided the topics are substantially
different. Prerequisite: consent of instructor. (Offered on an
occasional basis) - \textbf{MU 285 Conducting I} Basic conducting
techniques: reading, analysis, and interpretation of choral, band, and
orchestral literature. Laboratory experience with college ensembles.
Prerequisite: Theory of Music III (MU-215) or consent of instructor. -
\textbf{MU 309 Theory of Music III} Continuation of Theory of Music II,
including musicianship training, form and analysis, chromatic harmony,
and the basic compositional tools of the 18th and 19th centuries.
Prerequisite: Theory of Music II (MU-209) or consent of instructor. -
\textbf{+++MISSING INFO: c.mu309L.long +++} +++MISSING INFO:
c.mu309L.desc +++ - \textbf{MU 355 Music History \& Literature II} Study
of western music of the 18th and 19th centuries in Europe and America.
Prerequisite: Theory of Music II (MU-125) or consent of instructor. -
\textbf{MU 360 Elementary \& General Music Methods} Objectives,
problems, and methods of teaching elementary music and general music in
the schools. Course surveys elementary and general music curricula and
develops a functional knowledge of: organization and management; fretted
and classroom instruments; methods of teaching singing, rhythmic, and
listening activities expected. Directed observation in elementary
schools required. Prerequisite: Practicum in Music Education (MU-205).
(0.5 course credit) (Offered Spring Term) - \textbf{MU 361 Choral Music
Methods} Objectives, problems, and methods of teaching choral music in
the schools. General survey of elementary and secondary choral music
curricula and develops a functional knowledge of: organization and
management; the changing voice; beginning, intermediate, and advanced
choral techniques. Directed observation in elementary and secondary
schools required. Credit is given for Choral Music Methods (MU-361) or
Choral Music Methods for the Instrumental Teacher (MU-364), not both.
Prerequisite: Practicum in Music Education (MU-205) (Offered Spring
Term) - \textbf{MU 362 InstrMusMethChrlTeachr} Objectives, problems, and
methods of teaching instrumental music in the schools. General survey of
elementary and secondary instrumental music curricula. Students are
introduced to: concert band and orchestral techniques; jazz band
techniques; marching band techniques; beginning, intermediate, and
advanced band and orchestral techniques. Credit is given for
Instrumental Music Methods for the Choral Teacher (MU-362) or
Instrumental Music Methods (MU-363), not both. Prerequisite: Practicum
in Music Education (MU-205) (0.5 course credit) (Offered Spring Term) -
\textbf{MU 363 Instrumental Music Methods} Objectives, problems, and
methods of teaching instrumental music in the schools. General survey of
elementary and secondary instrumental music curricula and develops a
functional knowledge of: organization and management; concert band and
orchestral techniques; jazz band techniques; marching band techniques;
beginning, intermediate, and advanced band and orchestral techniques.
Directed observation in elementary and secondary schools required.
Credit is given for Instrumental Music Methods for the Choral Teacher
(MU-362) or Instrumental Music Methods (MU-363), not both. Prerequisite:
Practicum in Music Education (MU-205). (Offered Spring Term) -
\textbf{MU 364 ChrlMusMethInstTeach} Objectives, problems, and methods
of teaching choral music in the schools. General survey of elementary
and secondary choral music curricula. Students are introduced to the
changing voice and to beginning, intermediate, and advanced choral
techniques. Credit is given for Choral Music Methods (MU-361) or Choral
Music Methods for the Instrumental Teacher (MU-364), not both.
Prerequisite: Practicum in Music Education (MU-205). (0.5 course credit)
(Offered Spring Term) - \textbf{MU 385 Conducting II} Continuation of
Conducting I. Prerequisite: Conducting I (MU-285). - \textbf{MU 409
Theory of Music IV} Further investigation of tonal music, including jazz
harmony, harmonic and formal analysis, the rudiments of 18th-century
counterpoint, and an introduction to contemporary music. Prerequisite:
Theory of Music III (MU-309) or consent of instructor. -
\textbf{+++MISSING INFO: c.mu409L.long +++} +++MISSING INFO:
c.mu409L.desc +++ - \textbf{MU 421 Student Teaching Elementary Music}
Directed observation and student teaching in the first six grades.
Scheduled daily for seven weeks, approximately six hours per day.
Prerequisite: satisfactory completion of all other requirements for the
Bachelor of Music in Music Education. (2.0 course credits) (Offered Fall
Term) - \textbf{MU 422 Student Teaching Secondary Music} Directed
observation and student teaching in middle school and high school.
Scheduled for seven weeks, approximately six hours per day.
Prerequisite: satisfactory completion of all other requirements for the
Bachelor of Music in Music Education. (2.0 course credits) (Offered Fall
Term) - \textbf{MU 444 Ind Study-Music} Independent work on a selected
project in music under the direction of a faculty member of the
department. May be taken for an X status grade with consent of
instructor prior to registration. Prerequisites: background courses and
consent of department chair. - \textbf{MU 458 Music History \&
Literature III} Study of the composers and musical developments in
Europe and America in the 20th and 21st centuries, together with the
influences of music from other world cultures. Prerequisite: Theory of
Music II (MU-209) or consent of instructor.

\textbf{MUA Prefix} - \textbf{MUA 100 Concert Band} Open to all
woodwind, brass, and percussion players. The ensemble offers students
the opportunity to perform some of the great band repertoire of the 20th
and 21st centuries, as well as carefully selected transcriptions of
orchestral repertoire. A select Wind Ensemble within the Concert Band
occasionally performs additional selections from the contemporary
repertoire. The Concert Band tours on a regular basis. (0.3 course
credit) - \textbf{MUA 101 Jazz Ensemble} Membership determined by an
audition of interested saxophone, trombone, trumpet, piano, bass,
guitar, drum set, and auxiliary percussion players. The ensemble offers
students the opportunity to perform varied selections from the jazz
repertoire of the 20th and 21st centuries, as well as the opportunity to
improvise in the jazz idiom. The Jazz Ensemble tours on a regular basis
and performs with three to five guest artists each year. (0.2 course
credit) - \textbf{MUA 102 Jazz \& Improvisation Lab} Open to all
interested musicians. The ensemble offers students the opportunity to
perform varied selections from the jazz repertoire of the 20th and 21st
centuries, as well as the opportunity to improvise in the jazz idiom.
The Lab gives students an outlet to practice a secondary instrument and
also allows music education majors the opportunity to rehearse and
conduct a jazz ensemble. (0.1 course credit) - \textbf{MUA 103 Symphony
Orchestra} Membership determined by an audition of interested string,
woodwind, brass, and percussion players. The ensemble offers students
the opportunity to perform varied selections from the symphonic
repertoire of the 18th through the 21st centuries, in both full
orchestra and string orchestra formats. The Symphony Orchestra also
occasionally performs works that feature soloists or combines with the
Concert Choir to present masterworks. (0.3 course credit) - \textbf{MUA
104 Concert Choir} Membership determined by audition. The ensemble
offers students the opportunity to perform a wide variety of repertoire
covering many styles from the Renaissance through the 21st centuries.
The Concert Choir also occasionally combines with the Symphony Orchestra
to present large masterworks, and tours on a regular basis. (0.3 course
credit) - \textbf{MUA 105 Crimson Singers} Presents concerts each term,
on and off campus. Open to all interested musicians who sing. The
Crimson Singers is a small auditioned, mixed singing ensemble focusing
on Broadway, jazz, and popular music styles. Some selections will be
choreographed. Participants will gain experience in show choir, jazz,
and a cappella singing and performing styles. Previous show choir is
preferable; it is not required. Students in the Crimson Singers are also
encouraged to enroll in the Concert Choir. Registration for this course
is by audition and the consent of the instructor. (0.3 course credit) -
\textbf{MUA 110 Vocal Jazz Ensemble} Membership determined by audition.
The Vocal Jazz Ensemble offers students the opportunity to perform in a
wide variety of vocal and choral styles, with an emphasis on repertoire
for smaller vocal ensemble including vocal jazz, madrigals, pop, and
music of other cultures. The Vocal Jazz Ensemble may perform either on
or off campus and tours on an occasional basis. (0.2 course credit) -
\textbf{MUA 111 Coe Handbell Ensemble} Provides students the unique
opportunity to play English Handbells in a team-ensemble experience. The
course focuses on basic through advanced ringing techniques,
music-reading skills, rhythmic skills, and coordination skills.
Prerequisites: Music Fundamentals (MU-055) or Theory of Music I
(MU-115), prior experience in a handbell or other music ensemble, or
consent of instructor. (0.20 course credit) - \textbf{+++MISSING INFO:
c.mua130V.long +++} +++MISSING INFO: c.mua130V.desc +++ -
\textbf{+++MISSING INFO: c.mua131V.long +++} +++MISSING INFO:
c.mua131V.desc +++ - \textbf{+++MISSING INFO: c.mua133V.long +++}
+++MISSING INFO: c.mua133V.desc +++ - \textbf{+++MISSING INFO:
c.mua134V.long +++} +++MISSING INFO: c.mua134V.desc +++ - \textbf{MUA
211 Group Sight-Singing Lessons} Study of aural skills in a group
setting. Develops ear training and sight-singing proficiencies.
Prerequisite: Theory of Music II (MU-209). (0.2 course credit) -
\textbf{MUA 284 Applied Music: Special Topics} Private study in Applied
Music topics not offered in MUA course listings for music majors or
other interested and qualified students. The course provides extended,
yet tangible, instruction and/or research pertaining to the student's
specific applied music specialty or related music interest area.
Prerequisites: consent of instructor and department chair. (0.3 and 0.6
course credit) - \textbf{MUA 413 Orchestration} None - \textbf{MUA 423
Saxophone (Adv Std)} None - \textbf{MUA 490 Senior Recital} A
full-length senior recital in composition or performance area. S/U basis
only. Prerequisite: consent of instructor.

\section{Neuroscience (Collateral Major)}\label{sec-neuroscience}

Baker (Administrative Coordinator) Concurrent completion of a primary
major in biology, chemistry, or psychology is required. A minimum of six
course credits must be taken that do not count toward the student's
primary major.

\subsection{Collateral Major in
Neuroscience}\label{collateral-major-in-neuroscience}

\begin{enumerate}
\def\labelenumi{\arabic{enumi}.}
\tightlist
\item
  +++MISSING INFO: c.bio145/145L.long +++
\item
  +++MISSING INFO: c.bio155/155L.long +++
\item
  +++MISSING INFO: c.bio375/375L.long +++
\item
  +++MISSING INFO: c.chm121/121L.long +++
\item
  +++MISSING INFO: c.chm122/122L.long +++
\item
  PSY 100 Introductory Psychology
\item
  PSY 250 Biopsychology
\item
  PSY 350 Drugs \& Behavior
\item
  PSY 450 Behavioral Neuroscience
\item
  \textbf{Four} courses chosen from the lists below. Unless explicitly
  listed, associated laboratories are recommended, but not required.

  \begin{enumerate}
  \def\labelenumii{\alph{enumii}.}
  \tightlist
  \item
    Biology BIO 202 Topics in Evolution BIO 215 Human Anatomy BIO 235
    Genetics +++MISSING INFO: c.bio285/285L.long +++ +++MISSING INFO:
    c.bio345/345L.long +++ BIO 415 Developmental Biology BIO 435 Cell
    Physiology +++MISSING INFO: c.bio455/455L.long +++
  \item
    Chemistry +++MISSING INFO: c.chm211/211L.long +++ CHM 221 Organic
    Chemistry I CHM 321 Organic Chemistry II CHM 421 Advanced Organic
    Chemistry +++MISSING INFO: c.chm431/431L.long +++ CHM 432 Protein
    Biochemistry
  \item
    Philosophy PHL 245 Minds, Brains, and Robots
  \item
    Psychology PSY 205 Developmental Psychology PSY 235 Abnormal
    Psychology PSY 315 Learning \& Behavior +++MISSING INFO:
    c.psy325/325L.long +++ PSY 455 Advanced Experimental Psychology
    (when research topic is appropriate, as determined by the
    neuroscience administrative coordinator) PSY 464 Seminar in
    Psychology (when seminar topic is appropriate, as determined by the
    neuroscience administrative coordinator)
  \end{enumerate}
\end{enumerate}

\section{Nursing}\label{nursing}

Bursch (Chair), Dehner, Guthrie, Kittrell, Siems, Umbarger-Mackey

A description of policies unique to the nursing department is in the
\emph{Nursing Department Student Policies Manual}. Copies of the manual
are available in the Nursing Department Office, Stuart Hall 415, and
online.

\subsection{The Bachelor of Science in Nursing
Program}\label{the-bachelor-of-science-in-nursing-program}

The baccalaureate nursing program is designed to prepare students for
practice as professional nurses in a variety of settings. The upper
division nursing courses draw upon broad and diverse knowledge gained
from the liberal arts foundation to support the educational outcomes.

Clinical experiences in the program include working with clients across
the age span in a wide spectrum of practice sites. Nurse preceptors are
utilized at clinical sites to provide the maximum amount of individual
supervision and educational opportunity to students. Coe College nursing
faculty provide clinical expertise and education by overseeing student
clinical experiences, evaluating and promoting preceptor performance,
and engaging in ongoing dialogue with students regarding the application
of theory into practice.

Graduates of the Bachelor of Science in Nursing degree program are
eligible to take the state board licensing examination for Registered
Nurses. They are also eligible for admission to graduate programs in
nursing and to advanced nurse practitioner programs. The Coe College
nursing program is approved by the Iowa Board of Nursing and accredited
by the Commission on Collegiate Nursing Education (CCNE).

Coe College offers two paths to enter the nursing program. Standard
entry is for sophomore level (or higher) college students and direct
entry is for selected high school seniors. Details for both plans are
found in the \emph{Nursing Department Student Policies Manual}. Copies
of the manual are available in the Nursing Department Office. At the end
of the sophomore year, a student must possess valid licensure,
e.g.~Licensed Practical Nurse (LPN) or Certified Nursing Assistant
(Direct Care Worker). A student with a revoked license from any state
will NOT be admitted into the nursing program. In addition, a clinical
component may not be taken by a person: a) who had been denied licensure
by the State Board of Nursing, b) whose licensure is currently
suspended, surrendered or revoked in any United States jurisdiction, c)
whose licensure/ registration is currently suspended, surrendered or
revoked in another country due to disciplinary action.

The nursing department Admission, Promotion, and Retention committee
reviews applications and selects candidates who are best qualified to
meet the standards of the nursing profession. Admission to the BSN
degree program is competitive. Those applicants who appear to be the
most qualified will be admitted. Standard and direct entry students are
expected to maintain the eligibility requirements as outlined in the
\emph{Nursing Department Student Policies Manual}.

\subsection{Departmental Notes:}\label{departmental-notes-1}

In order to successfully complete a course and be promoted to successive
courses, students must:

\begin{verbatim}
- Achieve a minimum grade of C (2.0) in all nursing courses; a grade of C- (1.7) or lower requires the student to repeat the course.
- Achieve a cumulative average of 72% on all exams in a given course.
\end{verbatim}

Candidates for the \textbf{Bachelor of Science in Nursing} degree must
satisfactorily complete:

\begin{enumerate}
\def\labelenumi{\arabic{enumi}.}
\item
  The general education requirements (see
  Chapter~\ref{sec-general-education-courses} ).
\item
  \textbf{Eight} required supporting courses:
\end{enumerate}

Biology

Chemistry

Psychology

Sociology

Statistics

BIO 145 Cellular \& Molecular Biology BIO 195 Introduction to
Microbiology +++MISSING INFO: c.bio215\_215l.long +++ BIO 225 Human
Physiology

+++MISSING INFO: c.bio111\_111l.long +++

PSY 100 Introductory Psychology

SOC 107 Introductory Sociology

+++MISSING INFO: c.sta300.long +++ \textbf{or} STA 100 Statistical
Reasoning I-Foundations (7 weeks) \textbf{and} STA 110 Stats IIA:
Inferential Reasoning

Biology BIO 145 Cellular \& Molecular Biology BIO 195 Introduction to
Microbiology +++MISSING INFO: c.bio215\_215l.long +++ BIO 225 Human
Physiology

Chemistry +++MISSING INFO: c.bio111\_111l.long +++

Psychology PSY 100 Introductory Psychology

Sociology SOC 107 Introductory Sociology

Statistics +++MISSING INFO: c.sta300.long +++ \textbf{or} STA 100
Statistical Reasoning I-Foundations (7 weeks) \textbf{and} STA 110 Stats
IIA: Inferential Reasoning

\begin{enumerate}
\def\labelenumi{\arabic{enumi}.}
\setcounter{enumi}{2}
\item
  NUR 100 Nursing Issues (0.2 course credit) (Pre-nursing students must
  enroll each term of the first year. A maximum of two seminars may be
  counted for credit toward graduation.)
\item
  NUR 200 Nursing Issues II (0.2 course credit) (Pre-nursing students
  must enroll each term of the sophomore year. A maximum of two seminars
  may be counted for credit toward graduation.)
\item
  \textbf{Nine} non-clinical theory course credits: NUR 300 Art \&
  Science of Nursing NUR 305 Information Literacy \& Management
  +++MISSING INFO: c.nur345\_345l.long +++ NUR 360 Pharmacological
  Principles NUR 375 Legal \& Ethical Issues in Nursing NUR 425 Nursing
  Research NUR 430 Community \& Population Oriented Nur NUR 431 Wellness
  in Aging \& Chronicity NUR 495 Maternal Newborn Nursing
\item
  \textbf{Four} two-course-credit clinical courses: +++MISSING INFO:
  c.nur315\_315l.long +++ +++MISSING INFO: c.nur355\_355l.long +++
  +++MISSING INFO: c.nur415\_415l.long +++ +++MISSING INFO:
  c.nur455\_455l.long +++
\item
  A formal NCLEX review course (at the student's expense)
\item
  All required supporting courses and required nursing courses listed
  above, with no one specific course repeated more than once and with no
  more than two different courses repeated.
\end{enumerate}

\subsection{Courses in Nursing}\label{courses-in-nursing}

\begin{itemize}
\tightlist
\item
  \textbf{NUR 100 Nursing Issues} +++MISSING INFO: c.dsnur100.desc +++\\
\item
  \textbf{NUR 137 Human Sexuality} See also PSY 137 Human Sexuality ,
  Section~\ref{sec-psychology} See also Psychology (PSY-137), p.~170
  Examines human sexuality from the psychosocial, biophysiological, and
  cultural perspective. Topics include, but are not limited to, cultural
  and historical influences on our current understanding and attitudes
  toward the human sexual experience; the development of gender roles as
  they impact upon political, work, and social relationships; cultural
  aspects of sexuality including intimacy, courtship, marriage, and
  procreation; and sexuality during developmental changes and
  alterations in health such as infertility, pregnancy, abortion,
  cancer, AIDS, and others. Prerequisite: sophomore standing.
\item
  \textbf{NUR 200 Nursing Issues II} An introduction to the nursing
  profession. The seminar provides an opportunity for pre-nursing
  students to explore the profession of nursing through discussions
  regarding the domains of nursing knowledge, nursing roles, and nursing
  skills. May be taken for credit twice. Prerequisite: Nursing Issues
  (NUR-100) or sophomore standing. (0.2 course credits)
\item
  \textbf{NUR 215 Devel Relationship} Investigates theories for
  successful relationships. The emphasis is on self learning and
  application of principles involved in healthy and dysfunctional
  relationships. The students critically review popular literature
  versus scientific research related to relationship theory.
  Prerequisite: sophomore standing.\\
\item
  \textbf{NUR 268 Cult Diver \& Health} A study of the health practices
  of diverse cultures within the United States. Culture guides problem
  solving with regard to life choices, including health. This course
  examines how culture affects decisions about health and health care.
  Prerequisite: sophomore standing.
\item
  \textbf{NUR 297 Parent Child Relationships} A study of the historical,
  cultural, ethnic, and religious perspectives on parenting in America,
  the effects of stress and change on parenting ability, and the
  challenges and rewards of parenting as children and parents move
  across the lifespan and experience changes in family composition
  (blended, single-parent, gay and lesbian), health (sandwich generation
  and aging), and lifestyle. Prerequisite: sophomore standing.
\item
  \textbf{NUR 300 Art \& Science of Nursing} Introduces the nursing
  student to the nursing profession. The concepts of . professional
  responsibility, accountability, human development, spirituality,
  values and beliefs, diversity, and death and dying are discussed.
  Students learn to apply the nursing process, principles of
  teaching/learning, and therapeutic communication in the practice of
  individualized nursing care. Preventative nursing strategies in caring
  for individuals with reduced mobility are introduced. Prerequisite:
  admission to the Bachelor of Science in Nursing degree.
\item
  \textbf{NUR 305 Information Literacy \& Management} Introduces the
  student to the intellectual language, vocabulary, and expectations
  used in making nursing decisions in practice. Discusses concepts such
  as data gathering using search methods, organizing, synthesizing and
  critical evaluation Data information, knowledge, and standardized
  nursing language are discussed. Prerequisite: acceptance into the
  Bachelor of Science Nursing degree or consent of instructor. (0.5
  course credits)
\item
  \textbf{+++MISSING INFO: c.nur315\_315l\_315c.long +++} Provides the
  foundation for the integration of assessment data and
  pathophysiological concepts in the application of the nursing process.
  Focuses on comprehensive health assessment of diverse individuals
  across the lifespan. Assessment skills, health histories, and physical
  exams are practiced. Clinical component provides an opportunity to
  reinforce health assessment skills with diverse individuals across the
  lifespan as well as implementing safe, basic patient-centered care.
  Prerequisites: admission to the Bachelor of Science in Nursing degree;
  previous or concurrent registration in Art and Science of Nursing
  (NUR-300). (2.0 course credits)
\item
  \textbf{NUR 345 Mental Health Nursing} Introduces the nursing student
  to the care of persons with at-risk behavioral responses to life
  processes. The student analyzes theoretical and empirical knowledge
  from liberal arts and sciences as it applies to diagnosing and
  treating at-risk behavioral responses to life processes. The students
  study such topics as alterations in cognition and thought processes,
  coping responses, self-perception and violence toward self and others.
  A major component is the use of therapeutic communication skills to
  provide support that reduce risk, promote positive coping, and
  reinforce accurate perceptions in patients with alterations in mental
  health. Prerequisite: Art and Science of Nursing (NUR-300).\\
\item
  \textbf{+++MISSING INFO: c.nur355\_335l\_335c.long +++} Focuses on
  holistic care of diverse individuals and families across the life
  span. Applies the nursing process, emphasizing primary, secondary, and
  tertiary nursing interventions. in the care of individuals
  experiencing actual or risk for alterations in renal,
  gastrointestinal, and musculoskeletal systems. Surgical, dietary, and
  pharmacological management is integrated. Clinical components
  reinforce application of the nursing process in the delivery of safe,
  evidence-based, holistic care.. Prerequisites: Art and Science of
  Nursing (NUR-300); Pathophysiology and Assessment/Clinical Application
  (NUR-315); Information Management and Patient Care Technology
  (NUR-305). (2.0 course credits)
\item
  \textbf{NUR 360 Pharmacological Principles} Basic concepts and
  principles of administration, pharmacokinetics, pharmacodynamics, and
  application to basic biophysical concepts for specific pharmacological
  interventions. A brief overview of the mechanisms of action of select
  classifications of drugs is included. Prerequisite: acceptance into
  the Bachelor of Science in Nursing Degree or consent of instructor.
  (0.5 course credits)\\
\item
  \textbf{NUR 375 Legal \& Ethical Issues in Nursing} Focuses on the
  complexity of moral, legal, and ethical issues in health care. Topics
  include the legislative and regulatory processes governing healthcare,
  appraisal of legal risks, ethical principles, current ethical debates
  in healthcare, and developing trends and ethical conflicts. Students
  examine and explore one topic in depth. Prerequisite:acceptance into
  the Bachelor of Science Nursing Degree or consent of instructor .
\item
  \textbf{NUR 387 Alternative Therapies for Hlth/Heal} Examines
  available alternative and complimentary therapies. Risks and benefits
  of these modalities are assessed to determine if there are solid,
  scientific rationales for them. Therapies include dietary supplements,
  mind-body interventions (e.g., meditation), body based methods (e.g.,
  massage), and energy therapies (e.g., Reiki). Prerequisite: sophomore
  standing.
\item
  \textbf{+++MISSING INFO: c.nur415\_415l\_415c.long +++} Advances and
  continues the focus of holistic care for diverse individuals,
  families, groups, and populations across the life span. Applies the
  nursing process, emphasizing primary, secondary and tertiary nursing
  interventions, in care of individuals experiencing actual or risk for
  alterations in integumentary, pulmonary, cardiovascular, endocrine,
  sensory/neurological, and immune systems. Surgical, dietary and
  pharmacological management is integrated. Clinical components
  reinforce application of the nursing process in the delivery of safe,
  evidence-based, holistic, patient-centered care. Prerequisites:
  Introductory Concepts in Nursing/Clinical Application (NUR-355). (2.0
  course credits)
\item
  \textbf{NUR 425 Nursing Research} Study of the research process, the
  language of research, hypothesis formulation and testing, data
  collection, and analysis as they relate to the profession of nursing.
  Discussion of the nurse as a consumer of research with critical
  evaluation of selected research endeavors. Nursing research project
  required. The intent of this learning experience is to expose the
  student to the basic steps of the research process and their
  relationship to nursing. Prerequisites: R.N., junior, or senior
  standing in the nursing program, and Statistical Methods in the
  Behavioral Sciences (PSY-215) or Statistical Reasoning-Statistical
  Foundations (STA-100) (7 weeks) and Statistical Reasoning IIA
  (STA-110) (7 weeks).
\item
  \textbf{NUR 430 Community \& Population Oriented Nur} Focuses on
  community and population health promotion, and disease/injury
  prevention. Community oriented nursing roles are discussed. Topics
  include levels of prevention, risk analyses, harm reduction,
  causality, epidemiology, biostatistics, study designs, and sources of
  data applied to population health. Current issues related to disease
  control and surveillance, screening programs, clinical
  decision-making, health planning, and evaluation are addressed.
  Clinical application includes community observation experiences.
  Prerequisite: previous or concurrent registration in Nursing Research
  (NUR-425).
\item
  \textbf{NUR 431 Wellness in Aging \& Chronicity} Examines normal
  versus abnormal aspects of the aging process. Common health problems
  of the elderly are discussed. Addresses evidence-based strategies to
  promote wellness and to assist those living with chronic illnesses.
  Explores community resources to meet the holistic health needs of
  diverse individuals, groups, and families. Topics include the impact
  of culture, gender, stigma, and socioeconomic status on communication
  and care. Clinical application includes well elderly visits in the
  community. Prerequisite: Advanced Concepts in Nursing: Clinical
  Application (NUR-415) or Community and Population Oriented Nursing
  (NUR-430).
\item
  \textbf{NUR 444 Ind Study-Nursing} Guided study of individually chosen
  topic in nursing with a nursing department faculty member. May be
  taken for an X status grade with consent of instructor prior to
  registration. Prerequisites: junior standing and consent of
  instructor.
\item
  \textbf{NUR 455 Leadership \& Cont Issues in Nursing} +++MISSING INFO:
  c.nur455\_445c.desc +++
\item
  \textbf{NUR 494 Internship in Nursing} A clinical practicum on an
  inpatient health care unit supervised by a faculty member of the
  department and a professionally prepared R.N. preceptor. A minimum of
  140 hours on-site experience is required. S/U basis only.
  Prerequisite: Introductory Concepts in Nursing/Clinical Applications
  (NUR-355/-355L) and consent of department chair.
\item
  \textbf{NUR 495 Maternal Newborn Nursing} Analyzes previously learned
  nursing knowledge and skills to provide safe, holistic
  patient-centered care for diverse individuals and families during
  normal and high risk pregnancy, labor, delivery, and the
  postpartum-neonatal period of life. Includes concepts and issues in
  reproductive health of men and women using a developmental framework.
  Surgical, dietary, and pharmacological management are integrated.
  Clinical component includes patient simulation experiences.
  Prerequisite: Advanced Concepts in Nursing/Clinical Application
  (NUR-415).
\end{itemize}

\section{Organizational Science (Collateral
Major)}\label{sec-organizational-science}

Farrell (Administrative Coordinator)

The Organizational Science major is a collateral and interdisciplinary
program of study designed to acquaint students with human behavior in
organizations from both theoretical and applied perspectives. The
program aims to prepare students to carry out various human resources,
organizational development, customer service, and institutional research
functions in a variety of organizational settings. It also prepares
students for graduate study in the fields of Industrial/Organizational
(I/O) Psychology, Human Resources, Quantitative Methods, and other
similar fields. Required courses in quantitative methods form the core
of the major, complemented by required and elective coursework on
relevant topics within the disciplines of Psychology and Business
Administration that are central to the field. An internship or research
experience is also required.

\subsection{Collateral Major in Organizational
Science}\label{collateral-major-in-organizational-science}

A major in organizational science requires a minimum cumulative 2.0 GPA
in all courses counted toward the major.

Concurrent completion of a primary major in psychology is required.

\begin{enumerate}
\def\labelenumi{\arabic{enumi}.}
\item
  \textbf{All} of the following quantitative/methodological core
  courses:

  \begin{itemize}
  \tightlist
  \item
    BUS 340 Applied Regression Analysis
  \item
    PSY 475 Testing \& Measurement
  \end{itemize}
\item
  \textbf{All} of the following organizational core courses:

  \begin{itemize}
  \tightlist
  \item
    BUS 300 Human Resource Management
  \item
    BUS 315 Business Law I
  \item
    PSY 245 Organizational Psychology
  \item
    PSY 465 Industrial Psychology
  \end{itemize}
\item
  \textbf{Four} elective courses to be chosen as indicated from the two
  lists below.

  \begin{enumerate}
  \def\labelenumii{\alph{enumii}.}
  \tightlist
  \item
    \textbf{At least two} of the following courses:

    \begin{itemize}
    \tightlist
    \item
      BUS 250 Principles of Management
    \item
      BUS 375 Business Ethics
    \item
      BUS 387 Adv Top:Human Res Mgt
    \item
      BUS 395 Organizational Behavior
    \item
      BUS 410 Business Law II
    \item
      BUS 437 Strategic Compensation
    \item
      BUS 457 Employment and Discrimination Law
    \item
      BUS 464 Seminar in Management
    \end{itemize}
  \item
    \textbf{No more than two} of the following courses:

    \begin{itemize}
    \tightlist
    \item
      PSY 255 Social Psychology
    \item
      PSY 315 Learning \& Behavior
    \item
      +++MISSING INFO: c.psy355\_355l.long +++
    \item
      PSY 464 Seminar in Psychology (if topic is appropriate)
    \end{itemize}
  \end{enumerate}
\item
  \textbf{One} of the following, subject to prior approval by the
  Organizational Science administrative coordinator:

  \begin{itemize}
  \tightlist
  \item
    BUS 454 Research in Business (with business administration
    departmental approval)
  \item
    BUS 494 Internship in Business (with business administration
    departmental approval)
  \item
    PSY 455 Advanced Experimental Psychology
  \item
    PSY 494 Internship in Psychology
  \end{itemize}
\end{enumerate}

\section{Philosophy}\label{philosophy}

Hoover, Lemos.

The philosophy and religion department offer courses designed to lead
students to reflect on their views concerning fundamental issues in life
and thought. Since both the philosophical and religious traditions have
had a central place in and an enormous influence upon the development of
human culture, any student seeking a liberal education, whatever the
major discipline, will profit from the departmental offerings.

\subsection{Philosophy Major}\label{philosophy-major}

A grade of ``C'' (2.0) or higher must be earned in all courses counted
toward a major in philosophy.

\begin{enumerate}
\def\labelenumi{\arabic{enumi}.}
\item
  \textbf{Two} of the following, one of which must be either PHL 240
  Early Modern Philosophy or PHL 335 Late Modern Philosophy:

  \begin{itemize}
  \tightlist
  \item
    PHL 220 Ancient Greek Philosophy
  \item
    PHL 230 Medieval Philosophy
  \item
    PHL 240 Early Modern Philosophy
  \item
    PHL 335 Late Modern Philosophy
  \end{itemize}
\item
  \textbf{Three} additional 300-level philosophy courses
\item
  \textbf{Five} additional philosophy courses, at least two of which
  must be numbered 200 or above
\item
  PHL 490 Philosophy Colloquium (non-credit bearing)
\end{enumerate}

\textbf{NOTE:} \emph{The recommended beginning course in philosophy for
those contemplating a major in philosophy is PHL 105 Introduction to
Philosophy:. However, any of the other 100-level courses are also
suitable first courses. Some courses numbered between 200 and 299 may be
suitable first courses for students with sophomore standing.}

\subsection{Philosophy Minor}\label{philosophy-minor}

\begin{enumerate}
\def\labelenumi{\arabic{enumi}.}
\item
  \textbf{Two} of the following, one of which must be either PHL 240
  Early Modern Philosophy or PHL 335 Late Modern Philosophy :

  \begin{itemize}
  \tightlist
  \item
    PHL 220 Ancient Greek Philosophy
  \item
    PHL 230 Medieval Philosophy
  \item
    PHL 240 Early Modern Philosophy
  \item
    PHL 335 Late Modern Philosophy
  \end{itemize}
\item
  \textbf{One} additional 300-level philosophy course
\item
  \textbf{Three} additional philosophy courses
\end{enumerate}

\subsection{Courses in Philosophy}\label{courses-in-philosophy}

\subsubsection*{Introductory courses}\label{introductory-courses}
\addcontentsline{toc}{subsubsection}{Introductory courses}

\textbf{Category A: General Introductions}

The following courses, all numbered PHL-105, are different versions of
the same course. They share overlapping content and focus on the
development of the same skills. Students can receive credit for only one
PHL 105 Introduction to Philosophy: course. - \textbf{PHL 105
Introduction to Philosophy:} Minds: Examines what philosophers think
about some of the most basic questions in life. What is real? What does
it mean to have a mind? Could minds have an existence independent from
our bodies? Is there anything we can know with absolute certainty? Are
there objective moral values, and if so, could we know what they are?
May not be taken more than once for credit.

God: Focuses on questions about the nature and existence of God, human
nature, personhood, and free will. Is there evidence for the existence
of God and is the presence of suffering inthe world consistent with
God's existence? Are human beings merely material beings or might we
have non-physical minds or souls? What are persons and what constitutes
personal identity over time? What does it mean to have free will? Do we
possess free will? May not be taken more than once for credit.

Masterworks: Central philosophical debates encountered though the
examination of important classical and modern texts from the history of
philosophy. Readings in the course focus on several major works from
philosophers such as Plato, Descartes, Hume, Marx, and Sartre who
discuss the meaning of life, belief in God, the mind-body problem,
relativism of truth, and other important philosophical concerns. May not
be taken more than once for credit.

Science Fiction: Uses science fiction as a springboard for thinking
about classic issues and problems in philosophy. Sci-fi classics such as
The Matrix, Bladerunner, and Surrogates, raise fundamental philosophical
questions such as: What the difference between appearance and reality
and how can we distinguish between them? What is it to be a person?
Should we think that all and only human beings can be person? What is a
good human life? Is a pleasant life sufficient for living a good life?
In this course such questions are examined through the lens of both
philosophy and science fiction. May not be taken more than once for
credit.\\
- \textbf{PHL 105 Introduction to Philosophy:} Minds: Examines what
philosophers think about some of the most basic questions in life. What
is real? What does it mean to have a mind? Could minds have an existence
independent from our bodies? Is there anything we can know with absolute
certainty? Are there objective moral values, and if so, could we know
what they are? May not be taken more than once for credit.

God: Focuses on questions about the nature and existence of God, human
nature, personhood, and free will. Is there evidence for the existence
of God and is the presence of suffering inthe world consistent with
God's existence? Are human beings merely material beings or might we
have non-physical minds or souls? What are persons and what constitutes
personal identity over time? What does it mean to have free will? Do we
possess free will? May not be taken more than once for credit.

Masterworks: Central philosophical debates encountered though the
examination of important classical and modern texts from the history of
philosophy. Readings in the course focus on several major works from
philosophers such as Plato, Descartes, Hume, Marx, and Sartre who
discuss the meaning of life, belief in God, the mind-body problem,
relativism of truth, and other important philosophical concerns. May not
be taken more than once for credit.

Science Fiction: Uses science fiction as a springboard for thinking
about classic issues and problems in philosophy. Sci-fi classics such as
The Matrix, Bladerunner, and Surrogates, raise fundamental philosophical
questions such as: What the difference between appearance and reality
and how can we distinguish between them? What is it to be a person?
Should we think that all and only human beings can be person? What is a
good human life? Is a pleasant life sufficient for living a good life?
In this course such questions are examined through the lens of both
philosophy and science fiction. May not be taken more than once for
credit. - \textbf{PHL 105 Introduction to Philosophy:} Minds: Examines
what philosophers think about some of the most basic questions in life.
What is real? What does it mean to have a mind? Could minds have an
existence independent from our bodies? Is there anything we can know
with absolute certainty? Are there objective moral values, and if so,
could we know what they are? May not be taken more than once for credit.

God: Focuses on questions about the nature and existence of God, human
nature, personhood, and free will. Is there evidence for the existence
of God and is the presence of suffering inthe world consistent with
God's existence? Are human beings merely material beings or might we
have non-physical minds or souls? What are persons and what constitutes
personal identity over time? What does it mean to have free will? Do we
possess free will? May not be taken more than once for credit.

Masterworks: Central philosophical debates encountered though the
examination of important classical and modern texts from the history of
philosophy. Readings in the course focus on several major works from
philosophers such as Plato, Descartes, Hume, Marx, and Sartre who
discuss the meaning of life, belief in God, the mind-body problem,
relativism of truth, and other important philosophical concerns. May not
be taken more than once for credit.

Science Fiction: Uses science fiction as a springboard for thinking
about classic issues and problems in philosophy. Sci-fi classics such as
The Matrix, Bladerunner, and Surrogates, raise fundamental philosophical
questions such as: What the difference between appearance and reality
and how can we distinguish between them? What is it to be a person?
Should we think that all and only human beings can be person? What is a
good human life? Is a pleasant life sufficient for living a good life?
In this course such questions are examined through the lens of both
philosophy and science fiction. May not be taken more than once for
credit. - \textbf{PHL 105 Introduction to Philosophy:} Minds: Examines
what philosophers think about some of the most basic questions in life.
What is real? What does it mean to have a mind? Could minds have an
existence independent from our bodies? Is there anything we can know
with absolute certainty? Are there objective moral values, and if so,
could we know what they are? May not be taken more than once for credit.

God: Focuses on questions about the nature and existence of God, human
nature, personhood, and free will. Is there evidence for the existence
of God and is the presence of suffering inthe world consistent with
God's existence? Are human beings merely material beings or might we
have non-physical minds or souls? What are persons and what constitutes
personal identity over time? What does it mean to have free will? Do we
possess free will? May not be taken more than once for credit.

Masterworks: Central philosophical debates encountered though the
examination of important classical and modern texts from the history of
philosophy. Readings in the course focus on several major works from
philosophers such as Plato, Descartes, Hume, Marx, and Sartre who
discuss the meaning of life, belief in God, the mind-body problem,
relativism of truth, and other important philosophical concerns. May not
be taken more than once for credit.

Science Fiction: Uses science fiction as a springboard for thinking
about classic issues and problems in philosophy. Sci-fi classics such as
The Matrix, Bladerunner, and Surrogates, raise fundamental philosophical
questions such as: What the difference between appearance and reality
and how can we distinguish between them? What is it to be a person?
Should we think that all and only human beings can be person? What is a
good human life? Is a pleasant life sufficient for living a good life?
In this course such questions are examined through the lens of both
philosophy and science fiction. May not be taken more than once for
credit. - \textbf{PHL 105 Introduction to Philosophy:} Minds: Examines
what philosophers think about some of the most basic questions in life.
What is real? What does it mean to have a mind? Could minds have an
existence independent from our bodies? Is there anything we can know
with absolute certainty? Are there objective moral values, and if so,
could we know what they are? May not be taken more than once for credit.

God: Focuses on questions about the nature and existence of God, human
nature, personhood, and free will. Is there evidence for the existence
of God and is the presence of suffering inthe world consistent with
God's existence? Are human beings merely material beings or might we
have non-physical minds or souls? What are persons and what constitutes
personal identity over time? What does it mean to have free will? Do we
possess free will? May not be taken more than once for credit.

Masterworks: Central philosophical debates encountered though the
examination of important classical and modern texts from the history of
philosophy. Readings in the course focus on several major works from
philosophers such as Plato, Descartes, Hume, Marx, and Sartre who
discuss the meaning of life, belief in God, the mind-body problem,
relativism of truth, and other important philosophical concerns. May not
be taken more than once for credit.

Science Fiction: Uses science fiction as a springboard for thinking
about classic issues and problems in philosophy. Sci-fi classics such as
The Matrix, Bladerunner, and Surrogates, raise fundamental philosophical
questions such as: What the difference between appearance and reality
and how can we distinguish between them? What is it to be a person?
Should we think that all and only human beings can be person? What is a
good human life? Is a pleasant life sufficient for living a good life?
In this course such questions are examined through the lens of both
philosophy and science fiction. May not be taken more than once for
credit.

\textbf{Category B: Ethical and Political Issues} - \textbf{PHL 128
Morality \& Moral Controversies} A critical examination of important
moral issues facing contemporary society. The course uses a variety of
common ethical theories. Possible topics include environmental ethics,
euthanasia, animal rights, humanitarian aid, abortion, and capital
punishment.\\
- \textbf{PHL 138 Freedom, State, and Society} Addresses questions of
our relation to society and to the state in particular. How much freedom
should individuals be given over their own lives? What would an ideal
society look like? What demands could social and political institutions
legitimately make on us in the name of social order? Works from such
philosophers as Plato, Rousseau, and Marx may be considered alongside
literary texts such as Utopia or Brave New World. - \textbf{PHL 165
Bio-medical Ethics} Addresses a variety of issues in medical ethics and
introduces various moral frameworks for thinking about these issues.
Students are introduced to ethical theories, such as utilitarianism and
Kantianism, and how they can be applied in the context of medical
ethics. Topics addressed in the course are likely to include: abortion,
euthanasia, peternalism, and patient autonomy, organ transplats and
scarce medical resources, and genetic screening, among others. -
\textbf{PHL 205 Environmental Ethics} Serves as a general introduction
to environmental ethics. Students receive instruction in ethical theory
and how it can be applied to issues in environmental ethics. Some topics
likely to be addressed in the course are: defining our obligations to
future generations, the definition of wilderness, sustainable
agriculture, animal rights, anthropocentrism, the nature of the value of
wilderness, environmental holism, and ecofeminism.

\textbf{Category C: Logic} - \textbf{PHL 115 Logic} An introduction to
the discipline of logic on an elementary level. This course introduces
skills that are essential to good critical reasoning---how to detect
forms of arguments, how to test for validity, and how to construct valid
arguments. This course focuses on both formal and informal logic.

\paragraph*{Intermediate courses}\label{intermediate-courses}
\addcontentsline{toc}{paragraph}{Intermediate courses}

\begin{itemize}
\tightlist
\item
  \textbf{PHL 206 Buddhist Thought} See also REL 206 Buddhist Thought ,
  Section~\ref{sec-religion} A survey of major issues in Buddhist
  philosophy, including ethics, emptiness, idealism, the nature of mind,
  and the nature of reality. The course focuses on Indian Buddhist
  philosophical schools and also explores distinctive philosophical
  ideas from Buddhist traditions in China, Japan, and Tibet.
  Prerequisite: Eastern Religions (REL-036), or Buddhism (REL-116), or
  consent of instructor.\\
\item
  \textbf{PHL 210 Topics in Philosophy} An examination of a selected
  topic in philosophy. Topics vary depending on the instructor. May be
  taken more than once for credit, provided the topics are substantially
  different.
\item
  \textbf{PHL 220 Ancient Greek Philosophy} A survey of the central
  ideas and figures in the philosophy of the ancient Greek world.
  Figures studied include the pre-Socratic philosophers, Socrates,
  Plato, and Aristotle.
\item
  \textbf{PHL 230 Medieval Philosophy} A survey of the major
  philosophical and theological ideas of the Middle Ages. Special
  emphasis is placed on the writings of such thinkers as Augustine,
  Anselm, and Thomas Aquinas. (This course is also applicable to the
  major in Religion.)
\item
  \textbf{PHL 235 Philosophy of Science} An examination of the nature of
  scientific activity and theory. Views of scientific method are
  considered, in addition to the ways in which scientific theories
  develop. Both classical theories of science focusing on the structure
  of scientific explanation and more recent views focusing on the
  dynamic nature of science are considered. Attention is also given to
  the question of whether different branches of science have different
  types of explanation. Some previous experience with science helpful.
\item
  \textbf{PHL 240 Early Modern Philosophy} An examination of the
  metaphysical and epistemological theories of major European
  philosophers of the 17th and 18th centuries. Readings are drawn from
  the works of Descartes, Spinoza, Leibniz, Locke, Berkeley, Hume, and
  Kant.
\item
  \textbf{PHL 245 Minds, Brains, and Robots} An examination of the
  central issues in the philosophy of mind. The primary focus of this
  course is on the nature of consciousness and its relation to the
  physical processes of the body. Questions to be addressed include the
  following: are the mind and brain distinct entities? Can awareness be
  produced by non-brain-like things, particularly machines? Does it make
  sense to think of the self as a unitary entity that underlies one's
  many experiences? In what sense, if any, do persons possess free will?
\item
  \textbf{PHL 255 Existentialism} An examination of the writings of
  major figures representing modern existentialist views. This course
  includes both philosophical and literary texts, primarily from authors
  of the 20th century, such as Heidegger, Sartre, and Camus, and also
  traces the origins of this movement from 19th century figures, such as
  Kierkegaard and Nietzsche. Readings will explore themes such as
  finitude, authenticity, the absurd, bad faith, freedom and
  responsibility.
\item
  \textbf{PHL 265 Political Philosophy} An investigation of the central
  issues in social and political philosophy concerning the individual's
  relation to society and to the state in particular. Questions may
  include: on what basis can states legitimately exercise authority over
  individuals? What are the proper ends and limits of state authority?
  What principles should our society pursue in allocating goods such as
  property, education, health, and welfare?
\item
  \textbf{PHL 270 Ethical Theory} An examination of central normative
  and meta-ethical theories. Some questions that may be addressed in the
  course are: should we seek to maximize the happiness of the universe
  in whatever we do? Are some actions good in themselves regardless of
  their consequences? Are there absolute moral truths? Or, are all moral
  truths relative? What justifies our moral claims, if anything does?
  Prerequisite: at least one previous course in philosophy or consent of
  instructor.
\item
  \textbf{PHL 277 Philosophy of Gender \& Race} An examination of
  various issues involving the manner in which gender and race have been
  conceptualized in Western thought. This course considers ways in which
  gender and race pose problems for traditional conceptions of justice
  and equality will explore a variety of responses to these challenges
  by social and political theorists.
\item
  \textbf{PHL 285 Law, Morality \& Punishment} An introduction to the
  central issues in the philosophy of law. This course includes a survey
  of central theories on the nature of law, such as natural law,
  positive law, and legal realism. It also covers such topics as the
  relationship between law and morality and various philosophical views
  on the nature and justification of punishment.
\end{itemize}

\paragraph*{Advanced seminars}\label{advanced-seminars}
\addcontentsline{toc}{paragraph}{Advanced seminars}

\begin{itemize}
\tightlist
\item
  \textbf{PHL 305 Contemporary Continental Philosophy} An examination of
  central philosophical topics and themes of philosophy on the European
  continent since 1900 e.g., structuralism, critical theory, and
  post-structuralism. Readings are drawn from such philosophers as
  Saussure, Habermas, Derrida, Foucault, and Lyotard. Prerequisite: two
  courses in philosophy or consent of instructor.
\item
  \textbf{PHL 315 Advanced Topics in Philosophy} An examination of a
  selected topic in philosophy. Topics vary depending on the instructor.
  May be repeated for credit, provided the topics are substantially
  different. Prerequisite: two courses in philosophy or consent of
  instructor.
\item
  \textbf{PHL 320 Seminar in Ethics} An advanced research seminar
  dealing with central topics in normative ethics and/or metaethics.
  Topics addressed may include: utilitarianism, Kantianism, Aristotelian
  virtue theory, moral realism, ethical relativism, etc. In this course
  emphasis is placed on the development of student research projects on
  central topics in ethical theory. Prerequisite: two courses in
  philosophy or consent of instructor.
\item
  \textbf{PHL 335 Late Modern Philosophy} An examination of the works of
  influential European thinkers of the late 18th and 19th century. The
  course begins with the study of German idealism, a movement that
  includes philosophers such as Kant, Fichte, and Hegel, followed by an
  examination of later 19th-century figures such as Marx and Nietzsche.
  Prerequisite: at least one previous course in philosophy or consent of
  instructor.
\item
  \textbf{PHL 345 Philosophy of Language} Questions how linguistic signs
  allow us to communicate meaning. Does meaning become established by
  reference to objective content that is independent of individual
  speakers? Or, is meaning a function of private mental states in the
  minds of language users? These questions belong to the field of
  semantics or philosophy of language, which studies the nature of
  meaning and reference in linguistic systems. Course materials
  primarily consists of works from contemporary philosophy. Prerequiste:
  two courses in philosophy or consent of instructor.
\item
  \textbf{PHL 355 Seminar in Metaphysics} An advanced research seminar
  dealing with cetral topics in metaphysics. Topics addressed may
  include: the nature and existence of God, the nature of the self and
  personal identity, free will, etc. In this course emphasis is placed
  on the development of student research projects on central topics in
  metaphysics. Prerequisite: two courses in philosophy or consent of
  instructor.
\item
  \textbf{PHL 365 Philosophy of Art and Aesthetic Exp} None
\item
  \textbf{PHL 394 Directed Learning in Philosophy} A course of readings
  selected by the student and instructor to fit the individual student's
  particular interests and educational needs. Readings may focus on
  either a philosophical problem or one or more philosophers.
  Prerequisite: consent of instructor.
\item
  \textbf{PHL 444 Ind Study-Philos} Independent study in some
  philosophical problem or the thought of some major philosopher, under
  the direction of a faculty member of the department. May be taken for
  an X status grade with consent of instructor prior to registration.
  Prerequisite: consent of instructor.
\item
  \textbf{PHL 464 Seminar in Philosophy of Religion} An advanced
  research seminar dealing with central topics in the philosophy of
  religion. Topics addressed may include: the evidence for God,
  non-evidential defenses of the rationality of religious belief,
  miracles, the problem of evil, etc. In this course emphasis is placed
  on the development of student research projects on central topics in
  the philosophy or religion. Prerequisites: two courses in philosophy
  or consent of instructor.
\item
  \textbf{PHL 490 Philosophy Colloquium} Majors are required to submit
  10-15 pages of finished philosophical prose, and present their work
  orally to students and faculty. Although the Colloquium is usually
  taken during Spring Term of the senior year, it is open to all juniors
  and seniors with appropriate background in philosophy. Satisfactory
  completion of the Colloquium is required for graduation with a major
  in philosophy. S/U basis only.
\item
  \textbf{PHL 494 Internship in Philosophy} Exploration of a career area
  related to the student's interest in philosophy supervised by a
  faculty member of the department in cooperation with the Internship
  Specialist. A minimum of 140 hours on-site experience is required. S/U
  basis only. This course does not satisfy any of the requirements for a
  major in philosophy. Prerequisites: declared major in philosophy,
  junior standing, and consent of department chair.
\end{itemize}

\section{Physics}\label{physics}

Affatigato (Chair), Akgun, Baehr, Bragatto, Duru, Feller, Wetzel

The physics department serves a variety of students with a balanced
program, giving equal emphasis to the needs of the technically and the
non-technically oriented.

\subsection{Physics Major}\label{physics-major}

A major in physics requires a minimum cumulative 2.0 GPA in all courses
counted toward the major.

\begin{enumerate}
\def\labelenumi{\arabic{enumi}.}
\item
  +++MISSING INFO: c.phy185\_185l.long +++
\item
  +++MISSING INFO: c.phy195\_195l.long +++
\item
  PHY 231 Mathematical Methods for Physicists
\item
  +++MISSING INFO: c.phy235\_235l.long +++
\item
  PHY 265 Electromagnetism
\item
  \textbf{Two} of the following courses:

  \begin{itemize}
  \tightlist
  \item
    PHY 275 Mechanics Formulations
  \item
    PHY 315 Thermodynamics \& Stat Mech
  \item
    PHY 335 Quantum Mechanics
  \item
    PHY 425 Solid State Physics
  \end{itemize}
\item
  \textbf{Four} additional physics courses, all of which must be
  numbered 150 or above.
\item
  Comprehensive evaluation Satisfactory completion of written and oral
  examinations
\end{enumerate}

\textbf{\emph{Recommended:}} - CS 125 Computer Science I - +++MISSING
INFO: c.chm121\_121l.long +++

\subsection{Physics Major}\label{physics-major-1}

A minor in Physics requires a cumulative 2.0 GPA in all courses counted
toward the major.

\begin{enumerate}
\def\labelenumi{\arabic{enumi}.}
\item
  +++MISSING INFO: c.phy185\_185l.long +++ or +++MISSING INFO:
  c.phy165\_165l.long +++
\item
  +++MISSING INFO: c.phy195\_195l.long +++ or +++MISSING INFO:
  c.phy175\_175l.long +++
\item
  +++MISSING INFO: c.phy235\_235l.long +++
\item
  PHY 265 Electromagnetism
\item
  \textbf{Two} additional physics course approved by the department,
  both of which must be numbered 150 and above.
\end{enumerate}

\subsection{Courses in Physics}\label{courses-in-physics}

\begin{itemize}
\tightlist
\item
  \textbf{PHY 105 Physics: An Historical Approach} A course emphasizing
  important developments in physics from the time of Aristotle to the
  20th century. Special attention is given to significant conceptual
  developments and major technological advances. Readings are selected
  from writings of some of the major figures in the history of physics,
  as well as modern commentators. The class experience includes
  reenactments of some historically significant experiments.
\item
  \textbf{PHY 114 Modern Astronomy} An introduction to the objects and
  phenomena found in the universe, including the solar system, planets,
  moons, comets, meteors, the sun, stars, birth and death of stars,
  neutron stars, pulsars, black holes, galaxies, quasars, and
  cosmological evolution. Laboratory activities required. This course
  satisfies the general education laboratory science requirement.
  Prerequisite: competence in algebra.\\
\item
  \textbf{PHY 121 Everyday Physics} Covers the principle of physics we
  use in our daily life. Examples from everyday experience are used to
  explain the fundamental principles of linear and rotational motion,
  momentum, forces, energy, as well as electricity, magnetism, and
  medical applications.
\item
  \textbf{+++MISSING INFO: c.phy155\_155l.long +++} +++MISSING INFO:
  c.phy155\_155l.desc +++
\item
  \textbf{PHY 161 Energy: Science and Technology} The physics and
  technology of energy generation, consumption, and conservation. Covers
  a wide range of energy sources, including fossil fuels, hydropower,
  solar energy, wind energy, bioenergy, and nuclear energy. Surveys the
  efficiencies and environmental impacts of energy use in
  transportation, manufacturing, and buildings.\\
\item
  \textbf{+++MISSING INFO: c.phy165\_165l.long +++} +++MISSING INFO:
  c.phy165\_165l.desc +++
\item
  \textbf{+++MISSING INFO: c.phy175\_175l.long +++} +++MISSING INFO:
  c.phy175\_175l.desc +++
\item
  \textbf{PHY 181 Materials Science I} Introduces the relationship of
  atomic arrangement with microscopic and macroscopic material
  properties using fundamental of physics and chemistry. Covers the
  crystalline and amorphous structures, the defects and grain
  boundaries. The emphasis is placed on relating the chemical
  composition, structure, and the material properties.
\item
  \textbf{+++MISSING INFO: c.phy185\_185l.long +++} +++MISSING INFO:
  c.phy185\_185l.desc +++\\
\item
  \textbf{+++MISSING INFO: c.phy195\_195l.long +++} +++MISSING INFO:
  c.phy195\_\_195l.desc +++
\item
  \textbf{PHY 201 Materials Science 2} Incorporates the time dependent
  phenomena in solids: phase transformation, crystal nucleation and
  growth. Strong focus is placed on the phase diagrams and the technical
  methods on instrumental analysis of solid materials. Common material
  properties presented with focus on practical measurements: mechanical,
  thermal, optical, magnetic, and electrical properties. Prerequisite:
  Materials Science 1 (PHY-181) or consent of instructor.
\item
  \textbf{PHY 211 Glass Science} Introduces the physics and chemistry
  behind the formation and study of glassy materials. The course covers
  glass making, glass structure and surfaces, property characterization,
  a wide array of industrial and scientific applications, as well as
  modern experimental techniques. It is especially suitable in
  preparation for glass research, and for students interested in a
  possible career in materials research and/or condensed matter physics.
  Prerequisite: General Physics II \& Laboratory (PHY-195/-195L), or
  consent of instructor.
\item
  \textbf{+++MISSING INFO: c.phy221.long +++} +++MISSING INFO:
  c.phy221.desc +++\\
\item
  \textbf{PHY 231 Mathematical Methods for Physicists} Studies areas of
  mathematics which are of fundamental importance in the physical
  sciences. Topics include complex variables, Fourier analysis,
  eigenvalue problems, and vector calculus. Includes one computer
  laboratory session per week. Previous experience in calculus
  recommended. Prerequisite: Calculus II (MTH-145), or consent of
  instructor.
\item
  \textbf{PHY 235 Modern Physics} Introductory study of the phenomena,
  techniques, and models of modern physics including quantum phenomena,
  special relativity physics, and their interpretive models. Laboratory
  activities required. Prerequisites: Calculus II (MTH-145) and General
  Physics II (PHY-225) or consent of instructor.
\item
  \textbf{PHY 235L Modern Physics Lab} Introductory study of the
  phenomena, techniques, and models of modern physics including quantum
  phenomena, special relativity physics, and their interpretive models.
  Laboratory activities required. Prerequisites: Calculus II (MTH-145)
  and General Physics II \& Laboratory (PHY-195/-195L) or consent of
  instructor.
\item
  \textbf{PHY 241 Introduction to Astrophysics} Covers the fundamental
  concepts in astrophysics: The tools of astronomy, celestial mechanics,
  interaction of light and matter, telescopes, nature of the stars and
  their classifications. Other topics include the general overview of
  the solar system, and the binary systems. Prerequisite: General
  Physics II \& Laboratory (PHY-195/-195L) or consent of instructor.
\item
  \textbf{PHY 251 Stars \& Galaxies} Focuses on the star formation and
  evolution, as well as the galactic evolution. The formation of the
  solar system and its implications, the degenerate remnants, and the
  Milky Way galaxy are covered. Prerequisite: General Physics II \&
  Laboratory (PHY-195/-195L) or consent of instructor.
\item
  \textbf{+++MISSING INFO: c.phy255\_355.long +++} +++MISSING INFO:
  c.phy255\_355.desc +++
\item
  \textbf{PHY 265 Electromagnetism} Electromagnetic phenomena at the
  intermediate level, including circuits, static and quasi-static
  fields, Maxwell's equations, radiation, and selected topics in
  properties of materials. Special topics in vector algebra, scalar and
  vector point functions, and differential vector calculus are developed
  and used. Prerequisites: General Physics II \& Laboratory
  (PHY-195/-195L) and Calculus II (MTH-145) or consent of instructor.
\item
  \textbf{PHY 275 Mechanics Formulations} The Newtonian, Lagrangian, and
  Hamiltonian formulations of the laws of motion. Applications to
  systems of particles, extended objects, and oscillatory systems.
  Prerequisites: General Physics II \& Laboratory (PHY-195/-195L) and
  Calculus II (MTH-145) or consent of instructor.
\item
  \textbf{PHY 301 Optics \& Waves} A treatment of the theory of modern
  optics, wave theory, as well as mechanical and electromagnetic
  oscillations. Introducing a variety of topics, including geometrical
  and physical optics, mathematics of wave motion, propagation,
  reflection, refraction, phenomenon of resonance in oscillations, and
  Fourier formalism. Prerequisites: Electromagnetism (PHY-265) or
  consent of instructor.
\item
  \textbf{PHY 311 Renewable Energy (WE)} Reviews the scientific
  fundamentals of renewable energy production. Basic thermodynamic
  principles of the heat engines, the hydrogen production and storage
  methods are covered. Renewable energy sources including solar,
  biomass, wind, and ocean engine designs are discussed in detail. Also
  includes some hands-on experiments on solar, wind, and fuel cell
  systems. Prerequisite: General Physics II \& Laboratory
  (PHY-195/-195L), or consent of instructor.
\item
  \textbf{PHY 315 Thermodynamics \& Stat Mech} An introduction to
  fundamental concepts such as temperature, phase transitions, the
  First, Second, and Third Laws of thermodynamics, and the work/ entropy
  relationship. The Statistical Mechanics half covers a mathematical
  treatment of partition functions, thermal properties of solids, and
  critical-point transitions. Prerequisites: General Physics II
  (PHY-225) and Calculus III (MTH-255) or consent of instructor.
\item
  \textbf{PHY 321 Health Physics} Studies the use of physics in
  medicine. The basic principles of the medical physics applications,
  such as radiation therapy, dosimetry, computed tomography (CT),
  positron emission tomography (PET), single photon emission
  spectroscopy (SPECT), Magnetic Resonance Imaging (MRI), Nuclear
  Magnetic Resonance (NMR), and crystallography are covered.
  Prerequisite: General Physics II \& Laboratory (PHY-195/-195L), or
  consent of instructor.
\item
  \textbf{PHY 335 Quantum Mechanics} An introduction to the formal
  treatment of quantum mechanics. This course covers the Schrodinger
  wave equation, the Dirac Braket notation, operator formalism, spin and
  angular momentum, the wave equation in one and three dimensions, and
  perturbation theory. Prerequisites: Modern Physics (PHY-235) and
  Calculus III (MTH-255) or consent of instructor.
\item
  \textbf{PHY 341 Space Plasma Physics} Focuses on the behavior of
  plasma in space environment. Covers the plasma parameters, waves, the
  planetary atmospheres, ionospheres, and magnetospheres. Prerequisite:
  Introduction to Astrophysics (PHY-241) or consent of instructor.
\item
  \textbf{PHY 411 Robotics \& Advanced Electronics} Covers both software
  and hardware techniques in physics. The software portion includes
  sophisticated operations datasets, such as Fourier transforms,
  nonlinear fits, residual analysis, statistical and characterization.
  Most of the course is dedicated to advanced electronics,
  microcontrollers, and computer-control of data acquisition. This
  hardware portion starts with discussions of operational amplifiers,
  bandpass filters, transducers, and other advanced analog electronic
  concepts. The course then moves on to the use of microcontrollers to
  acquire data and to perform actions in response to the inputs.
  Stepping and servo motors, sensors, and other input/output devices are
  some of the topics that are covered during this stage. Project-based
  work is an important part of the course. Prerequisite: General Physics
  II \& Laboratory (PHY-195/-195L), Electronics (PHY-155) and some
  background in programming are also recommended.
\item
  \textbf{PHY 425 Solid State Physics} Study of the structure and
  properties of crystalline and amorphous solids. The main topics
  include crystal structure and quantized vibrations (phonons);
  electronic band structure and its relation to electrical, thermal, and
  optical behavior; semiconductors and superconductors. Prerequisites:
  Modern Physics \& Laboratory (PHY-235/-235L) and Calculus III
  (MTH-255) or consent of instructor.
\item
  \textbf{PHY 441 Relativity and Cosmology} An introduction to
  Einstein's general theory of relativity, as well as the intervals,
  geodesics, black holes, and close binary star systems. The course
  starts with an introduction to tensor calculus, then covers the
  Newtonian, relativistic and observational cosmology, and the physics
  of the early Universe. Prerequisite: Modern Physics and laboratory
  (PHY-235/-235L) and Mathematical Methods for Physicists Sciences
  (PHY-231), or consent of instructor.
\item
  \textbf{PHY 444 Ind Study-Physics} Independent study of topics under
  the guidance of the department: experimental or pedagogical research
  on a problem predefined by the student in consultation with the
  department. May be taken for an X status grade with consent of
  instructor prior to registration. Prerequisites: demonstrated
  initiative and self-discipline, four courses in physics, and consent
  of department chair.
\item
  \textbf{PHY 451 Particle Physics} Reviews the Standard Model, particle
  detection techniques, and the particle physics experiments. The
  nuclear weak, strong, and the electromagnetic interactions, Feynman
  diagrams, quark model, relativistic kinematics are also covered.
  Prerequisite: Modern Physics PHY-235, or consent of instructor.
\item
  \textbf{PHY 454 Summer Research} May be taken more than once for
  credit for a maximum of 2.0 credits. Prerequisite: consent of
  instructor.
\item
  \textbf{PHY 464 Junior-Senior Seminar I} Presentations and discussions
  of advanced topics unavailable through the regular catalog offerings,
  and appropriate to students enrolled. Prerequisites: Electromagnetism
  (PHY-265) and consent of instructor.
\item
  \textbf{PHY 484 Advanced Topics in Physics} Covers several
  instructional modules. The content corresponds to material usually
  covered in the second-semester of an advanced, year-long course. Core
  modules include quantum mechanics (e.g., Bell's theorem) and
  electromagnetism (e.g., stress tensor). The four remaining modules are
  chosen from topics that include optics, fluid dynamics, experimental
  techniques, nuclear physics, geophysics, advanced classical mechanics,
  waves, statistical mechanics, and atomic physics. Prerequisite:
  Quantum Mechanics (PHY-335) and Electromagnetism (PHY-265), or consent
  of instructor.
\item
  \textbf{OCC 365 Oak Ridge Science Semester} St.~Clair (Program
  Director). The Oak Ridge Science Semester is designed to enable
  qualified undergraduates to study and conduct research in a
  prestigious and challenging scientific environment. As members of a
  research team working at the frontiers of knowledge, participants
  engage in long-range investigations using the facilities of the Oak
  Ridge National Laboratory (ORNL) near Knoxville, Tennessee. The
  majority of a student's time is spent in research with an advisor
  specializing in biology, engineering, mathematics, or the physical or
  social sciences. Students also participate in an interdisciplinary
  seminar designed to broaden their exposure to developments in their
  major field and related disciplines. In addition, each student chooses
  an elective from a variety of advanced courses. The academic program
  is enriched in informal ways by guest speakers, departmental
  colloquia, and the special interests and expertise of the ORNL staff.
  Administered by Denison University, Oak Ridge Science Semester is
  recognized by both ACM and GLCA. Learn more at
  http://www.acm.edu/programs/15/oakridge/index.html.
\end{itemize}

\section{Political Science}\label{political-science}

Barrow, Lanegran (Chair), B. Nesmith.

The department of political science emphasizes the breadth of political
science and presents politics as a worldwide phenomenon. The department
nurtures active and responsible habits of citizenship, encouraging
service learning and the development of political values, while offering
students a variety of opportunities to study politics outside the
classroom.

\subsection{Political Science Major}\label{political-science-major}

A major in political science requires a cumulative 2.0 GPA in all
courses counted toward the major.

A major in political science requires ten courses, including at least
three 300- or 400-level courses.

\begin{enumerate}
\def\labelenumi{\arabic{enumi}.}
\item
  POL 108 Introduction to Politics
\item
  POL 115 American National Gov \& Pol
\item
  \textbf{One} political theory course:

  \begin{itemize}
  \tightlist
  \item
    POL 405 Contemporary Political Theory
  \item
    POL 435 Ancient \& Medieval Pol Thry
  \item
    POL 445 Modern Political Theory
  \end{itemize}
\item
  \textbf{One} additional American government course:

  \begin{itemize}
  \tightlist
  \item
    POL 207 Religion \& American Politics
  \item
    POL 245 Political Parties \& Elections
  \item
    POL 277 Women \& Poltics in US
  \item
    POL 325 The American Congress
  \item
    POL 345 American Presidency
  \item
    POL 350 US Social Policy Process
  \end{itemize}
\item
  \textbf{Two} comparative or international politics courses:

  \begin{itemize}
  \tightlist
  \item
    POL 248 Political Violence and the Violent
  \item
    POL 258 World Politics
  \item
    POL 266 Latin American Politics
  \item
    POL 276 African Politics
  \item
    POL 286 Asian Politics
  \item
    POL 298 European Politics
  \item
    POL 305 Terrorism
  \item
    POL 310 International Organizations\\
  \item
    POL 365 American Foreign Policy
  \item
    POL 386 International Development
  \item
    POL 398 Religion \& World Politics
  \end{itemize}
\item
  \textbf{Four} additional political science courses
\end{enumerate}

Satisfactory work in Topics in Political Science (POL-284/-296) may be
used, with consent of department chair, to satisfy any departmental
requirement.

\subsection{Political Science Minor}\label{political-science-minor}

A minor in political science requires six courses, including at least
two 300- or 400-level courses.

\begin{enumerate}
\def\labelenumi{\arabic{enumi}.}
\item
  POL 108 Introduction to Politics
\item
  POL 115 American National Gov \& Pol
\item
  \textbf{One} political theory course:

  \begin{itemize}
  \tightlist
  \item
    POL 405 Contemporary Political Theory
  \item
    POL 435 Ancient \& Medieval Pol Thry
  \item
    POL 445 Modern Political Theory
  \end{itemize}
\item
  \textbf{One} additional American government course:

  \begin{itemize}
  \tightlist
  \item
    POL 207 Religion \& American Politics
  \item
    POL 245 Political Parties \& Elections
  \item
    POL 277 Women \& Poltics in US
  \item
    POL 325 The American Congress
  \item
    POL 345 American Presidency
  \item
    POL 350 US Social Policy Process
  \end{itemize}
\item
  \textbf{One} comparative or international politics course:

  \begin{itemize}
  \tightlist
  \item
    POL 248 Political Violence and the Violent
  \item
    POL 258 World Politics
  \item
    POL 266 Latin American Politics
  \item
    POL 276 African Politics
  \item
    POL 286 Asian Politics
  \item
    POL 298 European Politics
  \item
    POL 305 Terrorism
  \item
    POL 310 International Organizations
  \item
    POL 365 American Foreign Policy
  \item
    POL 386 International Development
  \item
    POL 398 Religion \& World Politics
  \end{itemize}
\item
  \textbf{One} additional political science course
\end{enumerate}

Satisfactory work in Topics in Political Science (POL-284/-296) may be
used, with consent of department chair, to satisfy any departmental
requirement.

\subsection{Courses in Political
Science}\label{courses-in-political-science}

\begin{itemize}
\tightlist
\item
  \textbf{POL 108 Introduction to Politics} Compares societies and
  states across regions, cultures, and time spans, in an attempt to
  understand what governments have in common, how they differ, and why.
  Includes such specific topics as democracy and fascism, nationalism,
  human rights, post-communist states, and post-cold war international
  politics. Challenges students to look beyond the day's headlines,
  learn from other peoples' politics, and develop political
  self-awareness.\\
\item
  \textbf{POL 115 American National Gov \& Pol} Constitutional,
  institutional, and political dimensions, and principal contemporary
  problems of the government of the United States.
\item
  \textbf{POL 207 Religion \& American Politics} Examines several points
  of tension at the intersection of the religious and political spheres.
  Explores the connections between religious movements and political
  beliefs in American history, evolving understandings of the
  Constitution's religious freedom clauses, and the complicating effects
  on politics of America's increasing religious and cultural
  diversity.\\
\item
  \textbf{POL 210 Environmental Politics} Brings multiple perspectives
  to bear and provides a solid foundation for understanding the politics
  and complexities of environmental issues. Examines actors and issues
  in environmental policy-making at various levels of government, from
  the local to the national to the global. Analyzes the reasons for and
  hindrances to collective action. Students will acquire some tools of
  ``practical politics,'' including political communications. No
  prerequisites, but Introduction to Politics (POL-108) is recommended.
\item
  \textbf{POL 245 Political Parties \& Elections} The development and
  nature of political parties; state, local, and national party
  organizations; parties in government; voting behavior; campaigns and
  nominations. The course includes an introduction to election data sets
  and original research using quantitative research methods.
  Prerequisite: American National Government and Politics (POL-115) or
  consent of instructor. (Offered alternate years)\\
\item
  \textbf{POL 248 Political Violence and the Violent} Focuses on
  politically motivated violence by and against states, groups, and
  individuals, with attention to theories that explain the persistence
  of such violence. Examines such phenomena as traditional warfare,
  guerrilla warfare, coups d'état, rebellions, torture, and terrorism
  and the people, politics, ideals, and ideologies behind them.
  Prerequisite: Introduction to Politics (POL-108) or consent of the
  instructor. (Offered alternate years)\\
\item
  \textbf{POL 258 World Politics} Survey of the basic factors of
  international politics, including the character of the state system
  and international economic relations, the role of force, the role of
  diplomacy and negotiation, and an examination of the formulation of
  foreign policy within domestic political systems.\\
\item
  \textbf{POL 266 Latin American Politics} Focuses on two of the most
  exciting and dynamic features of contemporary Latin American politics:
  the ``wave'' of democratization that is washing over the region and
  the changing relationship between religion and politics. Topics
  include political culture, political economy, political violence, the
  impact of U.S. policies, the Catholic church's role in politics, and
  how the rise of Protestantism affects church-state relations.
  Prerequisite: Introduction to Politics (POL-108) or consent of
  instructor. (Offered alternate years)
\item
  \textbf{POL 276 African Politics} An introductory survey of
  post-independence political patterns and processes in Africa.
  Similarities and differences across the continent are highlighted
  while a small number of select countries are studied indepth.
  Attention is given to the legacy of the colonial period,
  democratization, the challenges of violence and illegitimate
  governance, and the impact of the modern global economy on life in
  Africa. Prerequisite: Introduction to Politics (POL-108) or consent of
  instructor. (Offered alternate years)\\
\item
  \textbf{POL 277 Women \& Poltics in US} Examines three aspects of the
  dynamic between women and the US political process: women as political
  leaders, women as voters and activists, and the impact of policies on
  women and their everyday lives. The course addresses general theories
  of elections, feminist politics, and political behavior and discusses
  a number of specific policy issues such as reproductive health and
  international affairs. Prerequisite: American National Government and
  Politics (POL-115) or consent of instructor. (Offered alternate years)
\item
  \textbf{POL 284 Topics in Political Science} Intensive reading, study,
  writing, and discussion dealing with various political science
  subjects. Examples of recent topics include political violence,
  environmental politics, and Mexican politics. This course may count
  toward a political science major, depending on course content, as
  either an American government course or as a comparative or
  international politics course. May be taken more than once for credit,
  provided the topics are substantially different.
\item
  \textbf{POL 286 Asian Politics} Examines the broad variety of Asian
  political systems through case studies of selected countries that are
  authoritarian, communist, transitioning, or established democracies.
  Issues confronted include: the East Asian economic miracle, the
  character of Asian democracies, and the role of ethnicity and religion
  in politics. The tension in the region between economic growth and
  political control receives particular attention. Prerequisite:
  Introduction to Politics (POL-108) or consent of instructor. (Offered
  alternate years)\\
\item
  \textbf{POL 296 Topics Pol Sci: Non-West Persp} Same as Topics in
  Political Science (POL-284) except the course focuses on topics
  related to non-Western cultures.\\
\item
  \textbf{POL 298 European Politics} Addresses political and economic
  continuities, changes, and challenges in modern Europe. The course
  examines the political structures and policy challenges of select
  states in the region including Great Britain, Germany and France.
  Other topics are the political dynamics of the expanding European
  Union as well as transitions to democracy and capitalism in Russia and
  Central and Eastern European states. Prerequisite: Introduction to
  Politics (POL-108) or consent of instructor. (Offered alternate
  years)\\
\item
  \textbf{POL 305 Terrorism} An advanced-level examination of terrorism
  and global responses to it. Topics include the history of terrorism, a
  variety of domestic and international terrorist groups, and how
  terrorism is changing in the post-Cold War era. Prerequisite:
  Introduction to Politics (POL-108) or consent of instructor. (Offered
  alternate years)
\item
  \textbf{POL 310 International Organizations} Examines the role of
  international organizations in international relations. The central
  question is whether organizations like the United Nations and the
  European Union are tools of their member states or actors that rival
  the power of nation-states in international relations. Students
  participate in simulations of international organizations.
  Prerequisite: Introduction to Politics (POL-108) or consent of
  instructor.\\
\item
  \textbf{POL 325 The American Congress} Examines the American
  legislative process at the national level, with special attention to
  the constitutional origins of Congress, consideration of legislation
  by Congress, and the relationship of Congress to other political
  actors, as well as current policy issues. Prerequisite: American
  National Government and Politics (POL-115) or consent of instructor.
\item
  \textbf{POL 335 Constitution \& Roles of Govt} Examination of original
  court opinions and political writings focusing on the nature and
  sources of Supreme Court authority; the structure of government;
  judicial review; commerce, taxing, spending, and war powers; with
  special emphasis upon separation of powers: the President, Congress,
  and the Court. Prerequisite: junior standing or consent of instructor.
  (Offered alternate years)
\item
  \textbf{POL 345 American Presidency} The President as chief executive,
  commander-inchief, chief diplomat, chief legislator, party leader,
  head of state; the institutionalized presidency. The course includes
  doing original research using historical case studies. Prerequisite:
  American National Government and Politics (POL-115) or consent of
  instructor.
\item
  \textbf{POL 350 US Social Policy Process} Familiarizes students with
  the bureaucratic process through which national-level public policy is
  formulated in the United States, and gives students expertise in the
  challenges, history, successes, and shortcomings of select social
  policies. Policies examined may include education, public health,
  social security and welfare. Students participate in primary research
  examining the implementation of national policies in Iowa.
  Prerequisite: American National Government and Politics (POL-115) or
  consent of instructor. (Offered alternate years)\\
\item
  \textbf{POL 365 American Foreign Policy} Examination of the
  institutional setting in which foreign policy is formulated, the
  political dynamics of policy formulation, and case studies of American
  foreign policy since World War II. Prerequisite: Introduction to
  Politics (POL-108) or consent of instructor. (Offered alternate
  years)\\
\item
  \textbf{POL 375 Constitution \& Individual Liberties} Examination of
  original court opinions and political writings focusing upon the
  procedural contents of due process, equal protection under the law,
  post-Civil War amendments, and civil rights legislation, with special
  emphasis upon freedoms of religion and expression. Prerequisite:
  junior standing or consent of instructor. (Offered alternate years)\\
\item
  \textbf{POL 386 International Development} Addresses controversies in
  international development, such as what is to be developed, for whom,
  and whether development means Westernization. Topics include how we
  measure development; foreign aid and debt; the roles of the World
  Bank, International Monetary Fund and USAID, as well as
  non-governmental organizations; and conflicting theories of
  development. Readings and discussions also touch on pressing ethical
  issues, most basically whether citizens have any moral responsibility
  to people who live beyond the nation's borders. Prerequisite:
  Introduction to Politics (POL-108) or consent of instructor. (Offered
  alternate years)
\item
  \textbf{POL 398 Religion \& World Politics} Seminar addressing such
  issues as the ways in which religion enters world politics (and vice
  versa), when and where religion has been a force for peacemaking or
  for conflict, and why religion is so often ignored or misunderstood by
  Western foreign policy theorists and practitioners. Includes student
  research projects. Prerequisite: Introduction to Politics (POL-108) or
  consent of instructor. (Offered alternate years)\\
\item
  \textbf{POL 405 Contemporary Political Theory} Survey and
  argumentative analysis of the ideas of major political thinkers since
  1900. Selections are made from such authors as Hannah Arendt, John
  Dewey, Sigmund Freud, Martin Luther King Jr., Robert Nozick, John
  Rawls, and Jean-Paul Sartre. Prerequisites: Introduction to Politics
  (POL-108) and junior standing, or consent of instructor.
\item
  \textbf{POL 435 Ancient \& Medieval Pol Thry} Historical survey and
  argumentative analysis of the ideas of great political thinkers from
  the 5th century B.C.E. to the 15th century C.E. Selections are made
  from such authors as Plato, Aristotle, Confucius, the Biblical
  writers, Augustine, and Thomas Aquinas. Prerequisites: Introduction to
  Politics (POL-108) and junior standing, or consent of instructor.\\
\item
  \textbf{POL 444 Ind Study-Pol Sci} Independent study and research,
  under the direction of a faculty member of the department, in some
  area of political science. May be taken for an X status grade with
  consent of instructor prior to registration. Prerequisites: two
  completed courses in the relevant area at the 300- or 400-level and
  consent of department chair.
\item
  \textbf{POL 445 Modern Political Theory} Historical survey and
  argumentative analysis of the ideas of the most important political
  thinkers from the 16th to the 19th centuries. Selections are made from
  such authors as Machiavelli, Hobbes, Locke, Rousseau, Marx, and Mill.
  Prerequisites: Introduction to Politics (POL-108) and junior standing,
  or consent of instructor.
\item
  \textbf{POL 494 Internship in Political Science} Substantial work or
  participation in an office, organization, or activity concerned with
  government and politics, such as a congressional, federal, state, or
  local government office, a political campaign, or an active interest
  group. A minimum of 140 hours on-site experience is required. S/U
  basis only. One course credit toward a political science major for
  successful completion, unless The Washington Experience (WSH-494) is
  completed for credit toward a major. Prerequisite: consent of
  department chair.
\item
  \textbf{WSH 284 Topics in Washington DC} See description,
  @washington-term
\item
  \textbf{WSH 286 Topics in Washington, D.C:NWP} See description,
  @washington-term
\item
  \textbf{WSH 464 Internship Seminar} See description, @washington-term
\item
  \textbf{WSH 494 Washington Experience} See description,
  @washington-term
\end{itemize}

\section{Public Relations (Collateral
Major)}\label{public-relations-collateral-major}

Carstens (Administrative Coordinator).

The Public Relations major prepares students for a career in public
relations and related communication fields. The major integrates
coursework in business administration, professional writing, graphic
arts, and other disciplines relevant to public relations. The curriculum
encourages the development of skills and perspectives desirable for
learning to manage the successful communication between an organization
and its publics.

Students wishing to complete this major must consult with the Public
Relations administrative coordinator no later than the first term of
their junior year. Students wishing to register for any of the art
courses that count toward this major should consult with the art and art
history department prior to course registration.

\subsection{Collateral Major in Public
Relations}\label{collateral-major-in-public-relations}

A major in Public Relations requires a cumulative 2.0 GPA in all courses
counted toward the major.

Concurrent completion of any of the majors listed on p.~of the Catalog
is required.

\begin{enumerate}
\def\labelenumi{\arabic{enumi}.}
\item
  \textbf{One} of the following:

  \begin{itemize}
  \tightlist
  \item
    ART101 Art Appreciation
  \item
    ART 145 Digital Studio
  \item
    ART 155 Photography: Light Writing
  \item
    ART 363 Graphic Design Studio
  \end{itemize}
\item
  PR 205 Public Relations
\item
  BUS 330 Principles of Marketing
\item
  BUS 460 Advertising
\item
  \textbf{One} of the following:

  \begin{itemize}
  \tightlist
  \item
    COM 241 Intro to Multimedia Journalism
  \item
    RHE 225 Journalism/Media Wtg Wksp
  \end{itemize}
\item
  \textbf{One} of the following (producing a portfolio of writings
  related to the field of public relations):

  \begin{itemize}
  \tightlist
  \item
    BUS 461 Marketing Decision Making/Analysis
  \item
    RHE 415 How Writers Write
  \end{itemize}
\item
  \textbf{One} of the following:

  \begin{itemize}
  \tightlist
  \item
    BUS 494 Internship in Business (with public relations or advertising
    as a major component
  \item
    INT 499 Summer Internship (0.0 credit) (with public relations as a
    major component
  \item
    PR 494 Internship in Public Relations
  \end{itemize}
\item
  \textbf{Three} of the following. No more than two courses may be
  selected from within any one department. (Students majoring in
  Business Administration may select no more than one course with either
  a BUS or an ACC prefix. Students majoring in Communication Studies or
  in Writing may select no more than one course with either a COM or an
  RHE prefix.)

  \begin{itemize}
  \tightlist
  \item
    ACC 171 Principles of Accounting I
  \item
    ART101 Art Appreciation (if not used to satisfy \#1)
  \item
    ART 145 Digital Studio (if not used to satisfy \#1)
  \item
    ART 155 Photography: Light Writing (if not used to satisfy \#1)
  \item
    ART 363 Graphic Design Studio (if not used to satisfy \#1)
  \item
    BUS 250 Principles of Management
  \item
    BUS 250 Principles of Management
  \item
    BUS 375 Business Ethics
  \item
    BUS 461 Marketing Decision Making/Analysis (if not used to satisfy
    \#6)
  \item
    BUS 464 Seminar in Management , subject to topic approval by PR
    administrative coordinator
  \item
    BUS 465 Advanced Topics in Marketing , subject to topic approval by
    PR administrative coordinator
  \item
    COM 125 Fundamentals of Public Speaking
  \item
    COM 157 Introduction to Media Analysis
  \item
    COM 337 Persuasion
  \item
    COM 341 Digital Storytelling
  \item
    RHE 225 Journalism/Media Wtg Wksp (if not used to satisfy \#5)
  \item
    RHE 265 Professional Writing
  \item
    RHE 415 How Writers Write (if not used to satisfy \#6)
  \end{itemize}
\end{enumerate}

\subsection{Courses in Public
Relations}\label{courses-in-public-relations}

\begin{itemize}
\tightlist
\item
  \textbf{PR 205 Public Relations} A study of the key concepts and
  processes of public relations used in corporate, not-for-profit, and
  government organizations. Topics include planning, research,
  communication/ media channels, campaigns, crisis communication, and
  public relations ethics. The historical development of public
  relations, current trends in public relations, and international
  issues in public relations are also coverd. This course does not
  satisfy any of the requirements for a major in business
  administration. Prerequisite: sophomore standing.\\
\item
  \textbf{PR 494 Internship in Public Relations} An internship with a
  focus on public relations supervised by the Public Relations
  administrative coordinator or by faculty teaching in the public
  relations major. A minimum of 140 hours on-site experience is
  required. S/U basis only. One credit may be counted toward a major in
  public relations with consent of the Public Relations administrative
  coordinator. Prerequisite: junior standing and consent of the public
  relations administrative coordinator.
\end{itemize}

\section{Psychology}\label{sec-psychology}

Baker, Brown, Castillo, Chihak, Farrell (Chair, Fall), Kelly (Chair,
Spring), Lee, Recker, Stephenson

Psychology is the scientific study of behavior and mental
processes---the basis for both a field of scientific knowledge and of
professional application. Both required and elective courses in
psychology are grounded in the scientific approach. As an important tool
for the understanding of both theory and data, the study of basic
statistical and methodological concepts is included among courses
required of all students majoring in psychology.

In addition to a major in \textbf{Psychology}, the College also offers
collateral majors in \textbf{Neuroscience} (see
Section~\ref{sec-neuroscience} ) and \textbf{Organizational Science}
(see Section~\ref{sec-organizational-science} ).

\subsection{Secondary Education Certification in
Psychology}\label{secondary-education-certification-in-psychology}

Students seeking certification to teach psychology at the secondary
level are strongly encouraged to speak with an advisor in Education as
early as possible in their program of studies.

\subsection{Psychology Major}\label{psychology-major}

A major in Psychology requires a cumulative 2.0 GPA in all courses
counted toward the major.

\begin{enumerate}
\def\labelenumi{\arabic{enumi}.}
\item
  PSY 100 Introductory Psychology
\item
  PSY 200 Research Methods
\item
  PSY 215 Topics in Diversity \& Inclusion (7 weeks) (0.5 credits)
\item
  PSY 295 Applied Contemporary Psychology
\item
  PSY 300 Stat Methods and Data Analysis
\item
  PSY 464 Seminar in Psychology
\item
  \textbf{One}of the following:

  \begin{itemize}
  \tightlist
  \item
    PSY 455 Advanced Experimental Psychology
  \item
    PSY 494 Internship in Psychology
  \end{itemize}
\item
  \textbf{One} of the following:

  \begin{itemize}
  \tightlist
  \item
    PSY 205 Developmental Psychology
  \item
    PSY 235 Abnormal Psychology
  \end{itemize}
\item
  \textbf{One} of the following:

  \begin{itemize}
  \tightlist
  \item
    PSY 245 Organizational Psychology
  \item
    PSY 255 Social Psychology
  \end{itemize}
\item
  \textbf{One} of the following:

  \begin{itemize}
  \tightlist
  \item
    PSY 205 Developmental Psychology
  \item
    PSY 260 Cognitive Psychology
  \end{itemize}
\item
  \textbf{One} of the following:

  \begin{itemize}
  \tightlist
  \item
    +++MISSING INFO: c.psy325\_325l.long +++
  \item
    +++MISSING INFO: c.psy335\_335l.long +++
  \item
    +++MISSING INFO: c.psy355\_355l.long +++
  \end{itemize}
\item
  \textbf{One} of the following:

  \begin{itemize}
  \tightlist
  \item
    PSY 315 Learning \& Behavior
  \item
    PSY 350 Drugs \& Behavior
  \item
    PSY 415 Counseling Psychology
  \item
    PSY 450 Behavioral Neuroscience
  \item
    PSY 465 Industrial Psychology
  \item
    PSY 475 Testing \& Measurement
  \end{itemize}
\end{enumerate}

\subsection{Courses in Psychology}\label{courses-in-psychology}

\begin{itemize}
\tightlist
\item
  \textbf{PSY 100 Introductory Psychology} Basic concepts, theories, and
  methods in the study of behavior and mental processes. Provides a
  basic understanding of psychology for interested students, who may
  take this as their only course in psychology, as well as for future
  majors.\\
\item
  \textbf{PSY 137 Human Sexuality} See also Nursing (NUR-137), p.~154
  This course does not satisfy any of the requirements for a major in
  psychology.
\item
  \textbf{PSY 200 Research Methods} Discussion of and experience in
  designing research studies, collecting and analyzing data, and
  preparing research reports in psychology. Coverage includes
  descriptive, correlational, quasi-experimental, and experimental
  methods, and basic statistical analysis using SPSS. Prerequisites:
  Introductory Psychology (PSY-115).\\
\item
  \textbf{PSY 205 Developmental Psychology} Consideration of the major
  principles of maturation from conception to death. Critical evaluation
  of contemporary theories in physical, sensory, cognitive, emotional,
  and social development. Special attention to empirical, experimental,
  and theoretical literature related to the developmental process.
  Prerequisite: Introductory Psychology (PSY-100).
\item
  \textbf{PSY 208 Gender Psychology} Psychological perspectives on the
  differences and similarities between females and males. Examination of
  theory and research includes topics such as: hormones and brain
  structure, intelligence, education, social roles, stereotypes,
  emotion, health, employment, and relationships. Prerequisite:
  Introductory Psychology (PSY-100). (Offered on an occasional basis)\\
\item
  \textbf{PSY 215 Topics in Diversity \& Inclusion} Introduces the
  critical role of socio-cultural context in the understanding of human
  behavior, emotion, and thought processes. Addresses issues related to
  diversity and inclusion within a particular subfield of psychology.
  Topics may include sexuality, health, power \& prejudice,
  discrimination, and cultural development. May only be taken once for
  credit. Prerequisite: Introductory Psychology (PSY-100). (0.5 course
  credit)
\item
  \textbf{PSY 235 Abnormal Psychology} Study of the diagnosis, etiology,
  explanation, and treatment of major mental disorders. Focus is on
  understanding the interplay of biological and psychological forces in
  the development and treatment of disorders, with emphasis on research
  findings. Appropriate for first-year students and sophomores.
  Prerequisite: Introductory Psychology (PSY-115).\\
\item
  \textbf{PSY 245 Organizational Psychology} Scientific study of how
  human attitudes and behavior are affected by organizational
  characteristics. Involves the application of psychological research
  and theories in organizational settings. Major topics include
  motivation, leadership, team performance, job attitudes,
  organizational justice, and organizational culture. Prerequisite:
  Introductory Psychology (PSY-115).
\item
  \textbf{PSY 250 Biopsychology} Introduces students to the biological
  bases of behavior and mental processes. This course emphasizes the
  cell biology of neurons, neural communication, and the organization of
  the nervous system. The neurological basis of psychological processes
  such as sensation, learning, memory, and cognition are discussed.
  Prerequisite: Introductory Psychology (PSY-100).\\
\item
  \textbf{PSY 255 Social Psychology} Examination of individual human
  behavior as it is influenced by social variables. Topics covered
  include person perception, conformity, attitudes, prejudice,
  persuasion, helping, aggression, and group processes. Experimental
  research methods and findings are given emphasis. Prerequisite:
  Research Methods (PSY-200) or consent of instructor.
\item
  \textbf{+++MISSING INFO: c.psy60.long +++} Explores current theories,
  research findings, and applications in the areas of attention,
  perception, consciousness, knowledge representation, memory processes,
  language comprehension and production, inductive and deductive
  reasoning, evaluation and decision making, human and artificial
  intelligence, problem solving and creativity, and cross-cultural
  cognition. Prerequisite: Introductory Psychology (PSY-100).
\item
  \textbf{PSY 295 Applied Contemporary Psychology} Addresses complex
  interpersonal, cultural, ethical, and legal issues that may arise in
  psychology-related professional settings, using established
  theoretical and practical frameworks. Taken prior to Internship in
  Psychology (PSY-494), this course allows students to explore various
  psychology-related careers and develop some of the applied knowledge
  necessary for entry into such careers. Prerequisites: Introductory
  Psychology (PSY-100), a declared major in psychology, or consent of
  instructor.\\
\item
  \textbf{PSY 300 Stat Methods and Data Analysis} Core topics include
  the theoretical foundations of estimation, variability, and
  inferential statistics critical for statistical literacy. Focus is on
  the development of proficiency in data analysis using SPSS,
  interpretation of analyses, graphical representation of data, and
  written communication of results. Prerequisite: Research Methods
  (PSY-200).
\item
  \textbf{PSY 315 Learning \& Behavior} Discussion of how behavior
  changes as a result of our experiences. The course focuses on roles of
  respondent and operant learning in the development and expression of
  adaptive and maladaptive behaviors and emotional responses. Students
  are provided opportunities to discover how learning principles are
  applied in contemporary behavior modification and behavior therapy.
  The role of learning is discussed in contexts such as health-related
  behaviors, sex and love, self-control, drug addiction, and
  psychological disorders. Prerequisite: Research Methods (PSY-200).\\
\item
  \textbf{PSY 325 Health Psychology} An introduction to scientific
  research and theory on the relationship between physical health and
  mental processes, emotion, and behavior. Topics include stress,
  coping, compliance with medical advice, health promotion, disease
  prevention, pain, chronic illness (e.g., cardiovascular disease,
  cancer, AIDS), and health behaviors (e.g., smoking, diet, exercise).
  Prerequisites: Research Methods (PSY-200) and Introduction to
  Biopsychology (PSY-250) or consent of instructor.
\item
  \textbf{PSY 335 Sensation \& Perception} Explores current theories,
  research findings, and laboratory applications related to how
  individuals detect and perceive sensory information in the
  environment. An overarching theme is how organisms appear to generate
  accurate percepts despite the limited and ambiguous nature of incoming
  sensory information. While the primary focus is on human vision, the
  most extensively studied of the senses, other systems are discussed.
  Course material covers basic biological structures
  (e.g.~photoreceptors) that detect and transduce environmental energy
  into electrical impulses transmitted throughout the nervous system.
  Three hours of lecture and three hours of laboratory per week.
  Prerequisites: Statistical Methods and Data Analysis (PSY-300).\\
\item
  \textbf{PSY 350 Drugs \& Behavior} Explores how psychoactive drugs
  affect the nervous system. Concepts particularly relevant to a wide
  variety of psychological, sociological, and health-related careers in
  which clients are commonly taking drugs, therapeutically or
  recreationally. Course focuses on factors that influence the
  variability of drug effects, including neural, pharmacological, and
  psychological mechanisms. Major topics include the problem and
  implications of categorizing drugs, basic neural function, principles
  of pharmacology, and physiological and psychological aspects of
  addiction. Selected psychotherapeutic drugs and legal and illegal
  drugs of abuse are surveyed. Prerequisite: Introduction to
  Biopsychology (PSY-250) or Integrated Human Physiology (BIO-375).
\item
  \textbf{PSY 354 Research Participation} Research and investigation of
  an area of interest supervised by a faculty member of the department.
  The student must obtain approval of a specific plan and complete the
  necessary arrangements prior to the term of registration for the
  course. S/U basis only. Prerequisites: Research Methods (PSY-200) and
  consent of instructor.
\item
  \textbf{PSY 355 Personality} Explores current theories, research
  findings, and laboratory applications related to the structure,
  development, and dynamic processes underlying the differences in how
  people act, think, and feel. Includes methods of constructing and
  evaluating personality assessment techniques. Three hours of lecture
  and three hours of laboratory per week. Prerequisite: Statistical
  Methods and Data Analysis (PSY-300).\\
\item
  \textbf{PSY 415 Counseling Psychology} A general introduction to the
  field of counseling. Topics include ethical principles of the
  counseling profession, legal issues and licensing, counseling in a
  diverse and multicultural society, and effectiveness of various forms
  of therapy. Major theoretical approaches including psychoanalytic,
  humanistic, existential, cognitive-behavioral, couples and family
  systems are covered. Prerequisites: Research Methods (PSY-200) and
  Abnormal Psychology (PSY-235).
\item
  \textbf{PSY 444 Ind Study-Psychology} Independent reading and the
  preparation of a proposal, with consent of psychology department
  faculty required prior to the term of registration. May be taken for
  an X status grade with consent of instructor prior to registration.
  Prerequisites: Research Methods (PSY-200), a declared major in
  psychology and consent of instructor.\\
\item
  \textbf{PSY 450 Behavioral Neuroscience} Further explores relationship
  between the nervous system and behavior begun in Introduction to
  Biopsychology (PSY-225). The course provides a more in-depth study of
  neural function and explores many new areas. Focuses on development of
  the nervous system, neural communication, neuroanatomy, hierarchical
  and parallel organization, neural plasticity, sensorimotor function,
  and neurohormonal influences on sexual development and behavior.
  Prerequisites: junior standing and either Introduction to
  Biopsychology (PSY-250) or Integrated Human Physiology (BIO-375).
\item
  \textbf{PSY 455 Advanced Experimental Psychology} A capstone course
  for students interested in conducting psychological research. Topics
  include legal and ethical responsibilities in psychological research,
  conducting literature reviews, research design, use of statistical
  software (e.g., SPSS and SAS), interpretation of statistical results,
  and clear communication and presentation of scientific information.
  Students also present their research findings in a public forum. S/U
  basis only. May be taken more than once for credit. A maximum of one
  course credit may be counted toward a major in psychology.
  Prerequisites: Research Methods (PSY-200) and consent of instructor.
\item
  \textbf{PSY 464 Seminar in Psychology} Intensive study of a topic
  selected by the instructor. May be taken more than once for credit,
  provided the topics are substantially different. Prerequisites:
  Statistical Methods and Data Analysis (PSY-300) or consent of
  instructor.\\
\item
  \textbf{PSY 465 Industrial Psychology} The scientific study of making
  decisions about and developing people within organizations. Examines
  psychological perspective, procedures aimed at improving productivity
  and fairness in work settings. Students in the course complete
  multiple applied projects. Major topics include job analysis, employee
  selection, performance evaluation, and employee training and
  development. Prerequisite: Statistical Methods and Data Analysis
  (PSY-300) or consent of instructor.
\item
  \textbf{PSY 475 Testing \& Measurement} Investigation of classical
  measurement theory, focusing on issues of reliability, validity, and
  item characteristics, and of some of the most commonly used tests in
  educational, industrial, and clinical settings. Covers appropriate
  methods of constructing and evaluating classroom measurement
  instruments and explores ethical, legal, and financial issues in
  testing. Prerequisite: Research Methods (PSY-200) or one term of
  college-level statistics.
\item
  \textbf{PSY 494 Internship in Psychology} On-site work experience in
  psychology under the direction of the on-site supervisor and a faculty
  member of the department. A minimum of 140 hours on-site experience is
  required. S/U basis only. One course credit of Internship in
  Psychology may be counted toward a major in psychology. Prerequisites:
  Applied Contemporary Psychology (PSY-295), a declared major in
  psychology, and consent of instructor.
\end{itemize}

\section{Religion}\label{sec-religion}

Chaplin, Hatchell (Chair), Kensky.

The philosophy and religion department offer courses designed to lead
students to reflect on their views concerning fundamental issues in life
and thought. Since both the philosophical and religious traditions have
had a central place in and an enormous influence upon the development of
human culture, any student seeking a liberal education, whatever the
major discipline, will profit from the departmental offerings.

\subsection{Religion Major}\label{religion-major}

A grade of ``C'' (2.0) or higher must be earned in all courses counted
toward a major in religion.

\begin{enumerate}
\def\labelenumi{\arabic{enumi}.}
\item
  \textbf{One} of the following:

  \begin{itemize}
  \tightlist
  \item
    REL 101 Introduction to Religion
  \item
    REL 103 Belief \& Unbelief
  \end{itemize}
\item
  REL 106 Eastern Religions
\item
  REL 108 Western Religions
\item
  \textbf{One} of the following:

  \begin{itemize}
  \tightlist
  \item
    REL 105 Introduction to Hebrew Bible
  \item
    REL 115 Introduction to New Testament
  \end{itemize}
\item
  \textbf{One} of the following:

  \begin{itemize}
  \tightlist
  \item
    REL 116 Buddhism
  \item
    REL 136 Religions of China
  \item
    REL 196 Hinduism
  \end{itemize}
\item
  \textbf{Five} additional religion courses, at least three of which are
  numbered 300 or above
\end{enumerate}

\subsection{Religion Minor}\label{religion-minor}

\begin{enumerate}
\def\labelenumi{\arabic{enumi}.}
\tightlist
\item
  \textbf{One} of the following:

  \begin{itemize}
  \tightlist
  \item
    REL 106 Eastern Religions
  \item
    REL 108 Western Religions
  \end{itemize}
\item
  \textbf{Four} additional religion courses
\end{enumerate}

\subsection{Courses in Religion by Content
Area}\label{courses-in-religion-by-content-area}

Special attention should be given to the numerical ordering of the
courses listed below:

\begin{enumerate}
\def\labelenumi{\arabic{enumi}.}
\item
  Courses numbered between 100 and 199 include general introductions to
  religion (REL 101 Introduction to Religion , REL 103 Belief \&
  Unbelief , REL 106 Eastern Religions , and REL 108 Western Religions
  as well as introductory courses focused on specific traditions.
\item
  Courses numbered 200-299 are topical courses especially suitable for
  sophomore level and above.
\item
  Courses numbered 300-399 are advanced courses with prerequisites.
\end{enumerate}

Recommended beginning courses for those contemplating a major in
religion are: REL 101 Introduction to Religion , REL 103 Belief \&
Unbelief , REL 106 Eastern Religions , or REL 108 Western Religionsm.
However, other courses numbered below 200 are also suitable introductory
courses.

\begin{itemize}
\tightlist
\item
  \textbf{Survey courses}

  \begin{itemize}
  \tightlist
  \item
    REL 101 Introduction to Religion
  \item
    REL 103 Belief \& Unbelief
  \item
    REL 106 Eastern Religions
  \item
    REL 108 Western Religions
  \end{itemize}
\item
  \textbf{Individual traditions}

  \begin{itemize}
  \tightlist
  \item
    REL 116 Buddhism
  \item
    REL 128 Judaism
  \item
    REL 136 Religions of China
  \item
    REL 138 Modern Judaism
  \item
    REL 148 Islam
  \item
    REL 178 Christianity
  \item
    REL 196 Hinduism
  \item
    REL 206 Buddhist Thought
  \item
    REL 215 The Rise of Christianity
  \item
    REL 226 Religions of China:Daoism
  \item
    REL 236 Zen Buddhism
  \item
    REL 336 Tibetan Buddhist Cultrue
  \end{itemize}
\end{itemize}

\begin{itemize}
\tightlist
\item
  \textbf{Judeo-Christian scriptures}

  \begin{itemize}
  \tightlist
  \item
    REL 105 Introduction to Hebrew Bible
  \item
    REL 115 Introduction to New Testament
  \item
    REL 310 Early Christian Gospels
  \item
    REL 330 Topics in Hebrew Bible
  \item
    REL 365 The Letter of Paul
  \end{itemize}
\item
  \textbf{Topical courses}

  \begin{itemize}
  \tightlist
  \item
    REL 217 Religion in America
  \item
    REL 295 Topics in Religion
  \item
    REL 296 Topics in Religion NWP
  \item
    REL 338 Modern Religious Thought
  \item
    REL 385 Advanced Topics in Religion
  \item
    +++MISSING INFO: c.rel386.long +++
  \item
    REL 394 Directed Learning in Religion
  \item
    REL 444 Ind Study-Relig
  \item
    REL 494 Internship in Religion
  \end{itemize}
\end{itemize}

\begin{itemize}
\tightlist
\item
  \textbf{Survey courses}

  \begin{itemize}
  \tightlist
  \item
    REL 101 Introduction to Religion
  \item
    REL 103 Belief \& Unbelief
  \item
    REL 106 Eastern Religions
  \item
    REL 108 Western Religions
  \end{itemize}
\item
  \textbf{Individual traditions}

  \begin{itemize}
  \tightlist
  \item
    REL 116 Buddhism
  \item
    REL 128 Judaism
  \item
    REL 136 Religions of China
  \item
    REL 138 Modern Judaism
  \item
    REL 148 Islam
  \item
    REL 178 Christianity
  \item
    REL 196 Hinduism
  \item
    REL 206 Buddhist Thought
  \item
    REL 215 The Rise of Christianity
  \item
    REL 226 Religions of China:Daoism
  \item
    REL 236 Zen Buddhism
  \item
    REL 336 Tibetan Buddhist Cultrue
  \end{itemize}
\item
  \textbf{Judeo-Christian scriptures}

  \begin{itemize}
  \tightlist
  \item
    REL 105 Introduction to Hebrew Bible
  \item
    REL 115 Introduction to New Testament
  \item
    REL 310 Early Christian Gospels
  \item
    REL 330 Topics in Hebrew Bible
  \item
    REL 365 The Letter of Paul
  \end{itemize}
\item
  \textbf{Topical courses}

  \begin{itemize}
  \tightlist
  \item
    REL 217 Religion in America
  \item
    REL 295 Topics in Religion
  \item
    REL 296 Topics in Religion NWP
  \item
    REL 338 Modern Religious Thought
  \item
    REL 385 Advanced Topics in Religion
  \item
    +++MISSING INFO: c.rel386.long +++
  \item
    REL 394 Directed Learning in Religion
  \item
    REL 444 Ind Study-Relig
  \item
    REL 494 Internship in Religion
  \end{itemize}
\end{itemize}

\subsection{Courses in Religion}\label{courses-in-religion}

\begin{itemize}
\tightlist
\item
  \textbf{REL 101 Introduction to Religion} Introduces students to
  thinking about religion as a category of human experience, both in
  terms of foundational beliefs and how those beliefs are situated in
  practice. The course examines methods of studying religion as well as
  essential questions regarding the nature of religion.\\
\item
  \textbf{REL 103 Belief \& Unbelief} Discussion oriented course
  focusing on the dynamics of faith and of atheism. Special attention to
  traditional proofs for God's existence, the problems of evil and the
  afterlife, and the nature of religious experience.
\item
  \textbf{REL 105 Introduction to Hebrew Bible} A literary and
  theological overview of the first five books of the Hebrew Bible (the
  Pentateuch or Torah) in the context of their historical development
  and their formative impact on the rest of Israelite scripture (the
  Prophets and Writings). It is recommended that this course be taken
  prior to other Biblical studies courses.\\
\item
  \textbf{REL 106 Eastern Religions} An introductory survey of some of
  the major religions of the Indian subcontinent and the Far East.
  Religions to be discussed include Hinduism, Buddhism, Confucianism and
  Taoism.
\item
  \textbf{REL 108 Western Religions} An introductory survey of the three
  major Abrahamic religions (Judaism, Christianity and Islam), both in
  their historical development and their contemporary expressions.\\
\item
  \textbf{REL 115 Introduction to New Testament} A literary and
  theological overview of the Christian scriptures (the Gospels and
  Acts, the Pauline, Johannine and catholic epistles, and the Apocalypse
  of John) in the context of the origins and early historical
  development of Christianity.
\item
  \textbf{REL 116 Buddhism} An introduction to the Buddhist religion,
  including its history, philosophy, ritual, meditation, and popular
  practice. Course materials include Buddhist histories and religious
  texts, as well as contemporary anthropological materials and film.
\item
  \textbf{REL 128 Judaism} The basic beliefs and practices of Judaism,
  from the prophetic period to the present. This course and the Modern
  Judaism course form a program in Jewish Studies which is supported by
  The Sinaiko Endowment.
\item
  \textbf{REL 136 Religions of China} An introduction to religion in
  China, with particular focus on the three major traditions of
  Confucianism, Taoism, and Buddhism. Course materials include readings
  from major texts of each tradition, as well as histories,
  anthropological studies, literature, and film.\\
\item
  \textbf{REL 138 Modern Judaism} A study of selected issues in
  Enlightenment or post Enlightenment Judaism as reflected, for example,
  in the history of the Jewish people, rabbinic teachings and Jewish
  theological scholarship, or Jewish literature. This course and the
  Judaism course form a program in Jewish Studies which is supported by
  The Sinaiko Endowment.
\item
  \textbf{REL 148 Islam} An introductory overview of Islam as an
  Abrahamic faith, a global civilization, and an integral facet of the
  American religious experience.\\
\item
  \textbf{REL 178 Christianity} A study of the beliefs and practices of
  Christianity from its earliest formulations to the modern world.
  Special attention is paid to essential tenets of Christian faith,
  elements of Christian practice, and divergences between Catholic,
  Protestant, and Orthodox Christianities.
\item
  \textbf{REL 196 Hinduism} An introduction to the Hindu religion,
  including its history, philosophy, ritual, meditation, and popular
  practice. Course materials include Hindu histories and religious
  texts, as well as contemporary anthropological materials, literature,
  and film.
\item
  \textbf{REL 206 Buddhist Thought} A survey of major issues in Buddhist
  philosophy, including ethics, emptiness, idealism, the nature of mind,
  and the nature of reality. The course focuses on Indian Buddhist
  philosophical schools and also explores distinctive philosophical
  ideas from Buddhist traditions in China, Japan, and Tibet.
  Prerequisite: Eastern Religions (REL-036), or Buddhism (REL-116), or
  consent of instructor.
\item
  \textbf{REL 215 The Rise of Christianity} An examination of how
  Christianity grew from a small band of Jewish followers of Jesus to
  the dominant religion in the Roman Empire. Attention is paid to
  crucial figures such as Paul of Tarsus, Irenaeus of Lyons, and
  Augustine of Hippo, among others.\\
\item
  \textbf{REL 217 Religion in America} Examines the varieties of
  American religious experience, from the religion of the Puritans to
  the 21st century. Attention is paid both to normative and minority
  traditions, with a look at the growing Evangelical and Muslim
  communities in America today.
\item
  \textbf{REL 226 Religions of China:Daoism} An introduction to China's
  Daoist tradition of Buddhism, beginning with its early literature like
  the Dao-de-jing and the Zhuang-zi and examining several later Daoist
  movements in China. The course also examines other Chinese religions
  and intellectual traditions that have influenced Daoism, including
  Confucianism and Buddhism. Course materials include histories,
  translations of Daoist literature, accounts of contemporary Daoists,
  and film.\\
\item
  \textbf{REL 236 Zen Buddhism} An introduction to the Zen tradition of
  Buddhism beginning with its origins in China and also examining its
  traditions in Japan. The course examines other Chineses religious and
  intellectual traditions that helped shape the Zen tradition, with
  particular influence on Daoism. Course materials include histories,
  translations of Zen literature, autobiography, and film.
\item
  \textbf{REL 295 Topics in Religion} An examination of a selected topic
  in religious studies. Content varies from year to year. May be taken
  for credit more than once.
\item
  \textbf{REL 296 Topics in Religion NWP} Same as Topics in Religion
  REL295, except the course focuses on topics related to non-western
  cultures. Content varies from year to year. May be taken for credit
  more than once.
\item
  \textbf{REL 310 Early Christian Gospels} An examination of the
  literary genre in early Christianity, focusing on both canonical
  (Mark, Matthew, Luke, John) and non-canonical Gospels, including the
  Gospel of Truth and the Gospel of Thomas. Why did early Christians
  utilize this genre to communicate traditions about Jesus of Nazareth?
  Prerequisite: Introduction to Hebrew Bible (REL-105) or Introduction
  to New Testament (REL-115) or consent of instructor.
\item
  \textbf{REL 330 Topics in Hebrew Bible} An advanced course in an
  aspect of critical study of the Hebrew Bible. Potential topics include
  Prophecy, Wisdom Literature, and Women in the Bible. Prerequisite:
  Introduction to Hebrew Bible (REL-105) or consent of instructor.
\item
  \textbf{REL 336 Tibetan Buddhist Cultrue} Introduces students to the
  lived experience of Buddhists on the Tibetan plateau and in Nepal. THe
  course discusses the history of religion in Tibet, as well as the
  major doctrines of Tibetan Buddhism. Particular attention is also paid
  toTibetan religious culture and popular religious practices. Course
  materials include Tibetan literature, histories, biographies, and
  film, as well as anthropological studies of Tibetan societies.
\item
  \textbf{REL 338 Modern Religious Thought} A survey of the religious
  thinkers and themes of the 20th century. Various religious outlooks,
  ranging from conservative to radical, are explored, as are alternative
  conceptions of God, religion, and salvation. Prerequisite: one course
  in religion or consent of instructor.
\item
  \textbf{REL 365 The Letter of Paul} An examination of the 13 letters
  attributed to Paul of Tarsus in the New Testament as well as biblical
  and extra-biblical sources for the life of this crucial figure who
  spread Christianity around the Roman Empire. Prerequisite:
  Introduction to Hebrew Bible (REL-105) or Introduction to New
  Testament (REL-115) or consent of instructor.
\item
  \textbf{REL 385 Advanced Topics in Religion} Seminar examining a
  selected topic in religious studies. Content varies from year to year.
  May be taken for credit more than once. Prerequisite: one course in
  religion or consent of instructor.\\
\item
  \textbf{+++MISSING INFO: c.rel386.long +++} +++MISSING INFO:
  c.rel386.desc +++
\item
  \textbf{REL 394 Directed Learning in Religion} A course of directed
  learning designed by the student and instructor to fit the individual
  student's particular interests and educational needs. Prerequisite:
  consent of instructor.
\item
  \textbf{REL 444 Ind Study-Relig} Independent study under the direction
  of a faculty member of the department in an area selected by the
  student. May be taken for an X status grade with consent of instructor
  prior to registration. Prerequisite: consent of instructor.
\item
  \textbf{+++MISSING INFO: c.re494.long +++} Exploration of a career
  area related to the student's interest in religion supervised by a
  Religion faculty member in cooperation with the Internship Specialist.
  A minimum of 140 hours on-site experience is required. S/U basis only.
  This course does not satisfy any of the requirements for a major in
  Religion. Prerequisites: declared major in Religion, junior standing,
  and consent of department chair.
\end{itemize}

\section{Rhetoric}\label{rhetoric}

Carr (Chair), Donofrio, Harmsen, J. Nesmith, Opayemi, Spikes

The Rhetoric Department offers a major and a minor in both
\textbf{Communication Studies} (see \textbf{?@sec-communication-studies}
) and \textbf{Writing} (see \textbf{?@sec-writing} ) and is responsible
for offering courses in the study and practice of academic prose,
non-fiction writing, journalism, and communication studies.

\section{Social \& Criminal Justice}\label{social-criminal-justice}

L. Barnett, J. Christensen, Lemos, McNabb (Administrative Coordinator).

The Social \& Criminal Justice Program offers students an opportunity to
immerse themselves in an interdisciplinary major that draws from Coe's
rich tradition in the liberal arts, as well as the pre-professional
opportunities at the college. Core courses in the major address
multifaceted questions surrounding restorative, retributive, procedural,
and distributive justice. While many programs addressing such issues are
housed in the field of criminal justice studies, the SCJ faculty teach
topics within the major from varying perspectives and disciplines,
introducing students to the ways in which matters of justice are at work
in --- and essential to --- many areas of study.

The major is both local and global in nature, exhibiting the
possibilities and limitations of social and criminal justice initiatives
in the Cedar Rapids community, while also navigating matters of human
rights and comparative justice systems that reveal our connections to
the broader world.

Through interdisciplinary engagement and a required practicum, the SCJ
program demonstrates for students how a multiplicity of voices and
viewpoints can help to shape new ideas about the impact of social
justice initiatives upon the criminal justice system, and the
foundations of individual and collective community engagement.

\subsection{Social \& Criminal Justice
Major}\label{social-criminal-justice-major}

A major in Social \& Criminal Justice requires a cumulative 2.0 GPA in
all courses counted toward the major.

\begin{enumerate}
\def\labelenumi{\arabic{enumi}.}
\item
  SCJ 101 Intro to Social \& Criminal Justice
\item
  SCJ 201 Law Enforcement \& Corrections
\item
  +++MISSING INFO: c.scj235.long +++
\item
  PHL 285 Law, Morality \& Punishment
\item
  SCJ 350 Human Rights \& Comparative Justice
\item
  SOC 351 Criminology
\item
  \textbf{One} of the following:

  \begin{itemize}
  \tightlist
  \item
    STA 100 Statistical Reasoning I-Foundations (7 weeks) \textbf{and}
    STA 110 Stats IIA: Inferential Reasoning (7 weeks)
  \item
    STA 100 Statistical Reasoning I-Foundations (7 weeks) \textbf{and}
    STA 130 Stats IIB: Experimental Design (7 weeks)
  \item
    +++MISSING INFO: c.sta190.long +++
  \item
    PSY 300 Stat Methods and Data Analysis
  \end{itemize}
\item
  \textbf{Three} courses, chosen from the following, in consultation
  with the student's advisor and/or the Social \& Criminal Justice
  administrative coordinator. No more than two courses may be selected
  from the same prefix.

  \begin{itemize}
  \tightlist
  \item
    ACC 313 Fraud Examination
  \item
    ARH 107 Gender and Art
  \item
    COM 237 Interpersonal Communication
  \item
    COM 357 Sex, Race, \& Gender in Media
  \item
    COM 361 Communication \& Social Change
  \item
    COM 362 U.S. Public Address
  \item
    ECO 115 Intro Political Econ
  \item
    EDU 187 Human Relations
  \item
    ENG 127 Social Justice and Literature
  \item
    ENG 146 Intro Postcolonial Literature
  \item
    GS 107 Intro Gender \& Sexuality Studies
  \item
    HIS 145 History of United States to 1865
  \item
    HIS 155 History of United States since 1865
  \item
    HIS 347 African American History
  \item
    HIS 257 Native American History
  \item
    HIS 297 Women in America
  \item
    HIS 325 Recent American History I
  \item
    IS 126 HumanRightsBurmeseMigrant
  \item
    PHL 128 Morality \& Moral Controversies
  \item
    PHL 205 Environmental Ethics
  \item
    PHL 265 Political Philosophy
  \item
    PHL 270 Ethical Theory
  \item
    PHL 277 Philosophy of Gender \& Race
  \item
    POL 305 Terrorism
  \item
    POL 350 US Social Policy Process
  \item
    POL 375 Constitution \& Individual Liberties
  \item
    PSY 235 Abnormal Psychology
  \item
    PSY 245 Organizational Psychology
  \item
    PSY 255 Social Psychology
  \item
    PSY 315 Learning \& Behavior
  \item
    PSY 350 Drugs \& Behavior
  \item
    PSY 415 Counseling Psychology
  \item
    REL 217 Religion in America
  \item
    SCJ 190 Topics in Social \& Criminal Justice
  \item
    SCJ 220 Juvenile Delinq \& the Justice Sys
  \item
    SCJ 301 Criminal Law \& the 4th, 5th \& 6th
  \item
    SCJ 390 Adv Topics in Soc \& Crim Just
  \item
    SOC 207 Sociology of the Family
  \item
    SOC 247 Sociology of Race
  \item
    SOC 207 Sociology of the Family
  \item
    SOC 247 Sociology of Race
  \item
    SOC 328 Urban Sociology
  \item
    SOC 338 Political Sociology
  \item
    SOC 355 Deviant Behavior
  \item
    SOC 425 Social Change
  \item
    A course approved by the Social \& Criminal Justice administrative
    coordinator
  \end{itemize}
\item
  \textbf{One} of the following capstone projects:

  \begin{itemize}
  \tightlist
  \item
    SCJ 444 Independent Study in Social and Cri
  \item
    SCJ 494 Internship in Social and Criminal J
  \item
    OCC 323 OCC Chicago: Urban Studies (if internship/independent study
    has SCJ focus)
  \item
    WSH 464 Internship Seminar (if internship has SCJ focus)
  \item
    As a capstone experience, it is expected that students will carry
    out the project during the junior or senior year, after completing
    the majority of the required coursework. Students are expected to
    consult with the SCJ administrative coordinator prior to beginning
    their projects.
  \end{itemize}
\item
  SCJ 490 Social \& Criminal Just Colloquium -- Non-Credit Bearing
\end{enumerate}

\subsection{Courses Social \& Criminal
Justice}\label{courses-social-criminal-justice}

\begin{itemize}
\tightlist
\item
  \textbf{SCJ 101 Intro to Social \& Criminal Justice} Explores
  components of justice. Focus is on understanding the nature and goals
  of social and criminal justice issues and policies including an
  analysis of globalization, consumer culture, and social privilege.
  Considers the history, structure, functions, and philosophy of
  justice. This course includes a required civic engagement component.\\
\item
  \textbf{SCJ 190 Topics in Social \& Criminal Justice} A focused
  examination of an issue, problem, theory, or methodology related to
  social and criminal justice. Content varies and is determined by the
  instructor. May be taken more than once for credit, provided the
  topics are substantially different (0.5 or 1.0 course credit).
\item
  \textbf{SCJ 201 Law Enforcement \& Corrections} Examines the structure
  and function of law enforcement policies and techniques along with
  practices, policies, and agencies involved in corrections systems.
  Review the histrical development of law enforcement and corrections,
  including challenges facing correctional populations. Explores the
  principles and practices of treatment accorded to suspects and
  offenders in various settings. Prerequisite: Introduction to Social
  and Criminal Justice (SCJ-101).\\
\item
  \textbf{SCJ 220 Juvenile Delinq \& the Justice Sys} Overview of
  delinquent behavior and juvenile justice system responses to
  delinquency. Addresses historical precedents and philosophical reasons
  for treating juveniles differently from adults. Also considers
  methodological problems and theoretical controversies in delinquency
  research, and the goals and effectiveness of juvenile justice systems.
  Prerequisite: Introduction to Social and Criminal Justice (SCJ-101).
\item
  \textbf{SCJ 301 Criminal Law \& the 4th, 5th \& 6th} Study of criminal
  law including the general elements of crime, the specific criminal
  offenses, legal justification defenses, and conspiracy. Additionally,
  the course offers a general understanding of accused rights under the
  Fourth, Fifth, Sixth Amendments of the U.S. Constitution.
  Prerequisite: junior standing or consent of instructor.\\
\item
  \textbf{SCJ 350 Human Rights \& Comparative Justice} Introduces
  philosophical and legal questions surrounding ``human rights'',
  analyzing the ways that the language of human rights permeates
  questions of civil rights and social justice in both international and
  domestic settings.
\item
  \textbf{SCJ 390 Adv Topics in Soc \& Crim Just} Advanced study of an
  issue, problem, theory, or methodology related to social and criminal
  justice. Content varies and is determined by the instructor. May be
  taken more than once for credit, provided the topics are substantally
  different. Prerequisite: Introduction to Social and Criminal Justice
  (SCJ-101) and Methods of Sociological Research (SOC-235). 0.5 or 1.0
  course credit.\\
\item
  \textbf{SCJ 444 Independent Study in Social and Cri} Independent study
  under faculty guidance of a research problem chosen by the student.
  May be taken for an X status grade with consent of instructor prior to
  registration. Prerequisites: Methods of Sociological Research
  (SOC-235), junior standing, and declared major in Social \& Criminal
  Justice.
\item
  \textbf{SCJ 490 Social \& Criminal Just Colloquium} Majors discuss and
  present their work that is a requirement of their capstone projects.
  It is expected that Colloquium is taken during or following the
  completing of the capstone internship/independent study. Satisfactory
  completion of the Colloquium is required for graduation with a major
  in Social \& Criminal Justice. S/U basis only. (0.0 course credit)
\item
  \textbf{SCJ 494 Internship in Social and Criminal J} Investigation of
  a career area related to the student's interest in social and criminal
  justice supervised by a faculty member in cooperation with the Center
  for Creativity and Careers. A minimum 140 hours of on-site experience
  is required. S/U basis only. Prerequisites: junior standing, declared
  major in Social \& Criminal Justice, and consent of Social \& Criminal
  Justice administrative coordinator.
\end{itemize}

\section{Secondary Education (Minor
Only)}\label{secondary-education-minor-only}

See Education , Section~\ref{sec-education}

\section{Sports Management (Courses
Only)}\label{sports-management-courses-only}

See Sports Management Concentration in Business,
\textbf{?@sec-sports-management-concentration}

\subsection{Courses in Sports
Management}\label{courses-in-sports-management}

\begin{itemize}
\tightlist
\item
  \textbf{SMT 100 Introduction to Sports Management} An introductory
  course in sports management. This course provides an overview of the
  historical, legal, economic, political, and social-cultural issues
  that shape the field of sports management. The skills needed for
  managing sports organizations, from recreational to collegiate and
  professional, are also introduced. The course will look at the key
  functional areas of sport management such as marketing, communication,
  finance and economics and facility and event management.\\
\item
  \textbf{SMT 300 Sport/Recreation Event Management} An introduction to
  the principles and procedures for preparing, planning, operating,
  managing, and evaluating events and venues in sports settings.
  Students will gain a greater understanding of event and venue
  management and the total operation of sports organizations.
  Prerequisite: SMT-100: Introduction to Sports Management
\item
  \textbf{SMT 454 Seminar in Sports Management} A
  lecture/discussion-based course in which relevant and current research
  and industry trends are thoroughtly reviewed and analyzed. Students
  are encouraged to take this course near the end of their careers as
  undergraduate students in the sports management concentration.
\item
  \textbf{SMT 494 Internship in Sports Management} An internship with a
  focus on sports management supervised by a faculty member teaching
  within the sports management concentration. A minimum of 140 hours
  on-site experience is required. Students must document and analyze the
  experience through a journal and final report or through other written
  work as assigned by the faculty supervisor. S/U basis only. A maximum
  of one credit may count toward the concentration with the approval of
  the administrative coordinator(s) of the concentration. Prerequisites:
  junior standing and consent of administrative coordinator(s).
\end{itemize}

\section{Sociology}\label{sociology}

L. Barnett, Boguslaw, Fairbanks (Chair, Fall), K. Rodgers (Chair,
Spring).

The sociology department, which also offers a minor in anthropology,
offers a rigorous curriculum in support of Coe's mission of providing
students with a high-quality liberal arts education and preparing
students intellectually, professionally, and socially to lead productive
and satisfying lives in the global society of the 21\textsuperscript{st}
century. The department serves this mission through an integrated series
of courses designed to promote students' awareness and understanding
21\textsuperscript{st} century social problems, including the enduring
presence of social inequality.

\subsection{Sociology Major}\label{sociology-major}

A major in Sociology requires a cumulative 2.0 GPA in all courses
counted toward the major.

\begin{enumerate}
\def\labelenumi{\arabic{enumi}.}
\item
  SOC 107 Introductory Sociology
\item
  SOC 235 Methods of Sociological Research
\item
  SOC 450 Sociological Theory
\item
  SOC 464 Capstone Seminar in Sociology
\item
  \textbf{Five} additional sociology courses, including at least one
  course numbered 400--449 (not including SOC-444 Independent Study,
  which does not count towards the major)
\item
  \textbf{One} of the following:

  \begin{itemize}
  \tightlist
  \item
    STA 100 Statistical Reasoning I-Foundations (7weeks) and STA 110
    Stats IIA: Inferential Reasoning (7 weeks)
  \item
    +++MISSING INFO: c.staa100.long +++ (7weeks) and STA 130 Stats IIB:
    Experimental Design (7 weeks)
  \end{itemize}
\end{enumerate}

\subsection{Sociology Minor}\label{sociology-minor}

\begin{enumerate}
\def\labelenumi{\arabic{enumi}.}
\item
  SOC 107 Introductory Sociology
\item
  SOC 235 Methods of Sociological Research
\item
  SOC 450 Sociological Theory
\item
  \textbf{Three} additional sociology courses, including at least one
  course numbered 400--449 (not including SOC 444 Ind Study-Soc , which
  does not count towards the major) The following courses do not satisfy
  any of the requirements for a major or minor in sociology:

  \begin{itemize}
  \tightlist
  \item
    SOC 275 Directed Readings in Sociology
  \item
    SOC 365 Research Participation I or SOC 375 Research Participation
    II
  \item
    SOC 444 Ind Study-Soc
  \item
    SOC 494 Internship in Sociology
  \item
    SOC 499 Career Related Independent Invest
  \end{itemize}
\end{enumerate}

\subsection{Courses in Sociology}\label{courses-in-sociology}

\begin{itemize}
\tightlist
\item
  \textbf{SOC 107 Introductory Sociology} An introduction to sociology,
  the scientific study of human social behavior. The course explores the
  place of the discipline in the social sciences and the interplay of
  theory and empirical evidence in building an understanding of society,
  and it provides exposure to a wide range of research topics and
  results from different areas of study in the field.\\
\item
  \textbf{SOC 207 Sociology of the Family} Sociological description and
  analysis of the family as a major social institution. Topics covered
  may include, but are not limited to: the various ways in which
  ``families'' are defined, and the different forms that families take;
  the relationship of family life to social processes such as
  socialization, stratification, and modernization, and the role of
  families in the interdependent network of social institutions.
  Prerequisite: Introductory Sociology (SOC-107) or consent of
  instructor.
\item
  \textbf{SOC 217 Sociology of Religion} Examines the ways in which
  religion affects---and is affected by---the social context in which it
  occurs. Although the primary emphasis is on religion in the
  contemporary United States, considerable cross-cultural and historical
  material is included as well. The concept of secularization (that both
  the public and private impact of religion decline as societies
  modernize) is examined in light of empirical evidence. Prerequisite:
  Introductory Sociology (SOC-107) or consent of instructor. (Offered on
  an occasional basis)\\
\item
  \textbf{SOC 226 Gender and Globalizaton} Examines processes of
  globalization through a gendered lens. Focus on the ways global issues
  such as labor, intimacy, poverty, pop culture, and environmental
  degradation affect people of different genders in various ways. Also
  notes how people respond to the forces of globalization and addresses
  the effects of globalization on the lives of people in the non-Western
  world. Prerequisite: Introductory Sociology (SOC-107), or consent of
  instructor.
\item
  \textbf{SOC 235 Methods of Sociological Research} An exploration of
  the various strategies researchers employ to gather information and
  test hypotheses about the social world. Topics include data
  collection, sampling, conceptualization and measurement, and both
  qualitative and quantitative methods of analysis. Prerequisite:
  Introductory Sociology (SOC-107), sophomore standing or consent of
  instructor.\\
\item
  \textbf{SOC 236 Topics in Sociology NWP} Same as Topics in Sociology
  (SOC-237) except topic(s) relate to non-Western cultures.
  Prerequisite: Introductory Sociology (SOC-107) or consent of
  instructor.
\item
  \textbf{SOC 237 Topics in Sociology:U S Pluralism} A course of
  selected focus that centers on a particular sociological issue,
  problem, theory, or methodology related to United States Pluralism.
  Content varies and is determined by the instructor. May be repeated
  for credit, provided the topics are substantially different.
  Prerequisite: Introductory Sociology (SOC-107) or consent of
  instructor. (Offered on an occasional basis)\\
\item
  \textbf{SOC 238 Topics in Soc Div West Perspectives} Same as Topics in
  Sociology (SOC-237) except topic(s) relate to diverse Western
  cultures. Prerequisite: Introductory Sociology (SOC 107)
\item
  \textbf{SOC 247 Sociology of Race} The purpose of this course is
  twofold: first, students will develop an understanding of the social,
  political, and economic pressures shaping definitions of `race' over
  time; second, we will explore the intersections of `race' with
  ethnicity, gender, class, and sexuality. Students will examine these
  via in-depth evaluations of the following topics: the historical
  emergence of `race,' contemporary and historical whiteness, harmonious
  and discordant inter-racial interactions, stereotyping,
  discrimination, racism, race and crime, economic inequality, and
  changing racial demographics. Efforts will be made throughout the
  course to make the material relevant to students' lives.
\item
  \textbf{SOC 275 Directed Readings in Sociology} Study of major
  literature on a selected topic in sociology directed by a sociology
  department faculty member. This course does not satisfy any of the
  requirements for a major or minor in sociology. Prerequisites:
  declared sociology major, minor, or elementary education emphasis,
  sophomore standing, or consent of department.\\
\item
  \textbf{SOC 328 Urban Sociology} Analysis of cities as they affect
  social behavior, and the study of the urban form as it is produced and
  modified by wider changes in the world economy. Emphasis is
  theoretical, historical, and comparative. Problems addressed include
  restructuring, poverty, and underdevelopment. Prerequisites:
  Introductory Sociology (SOC-107), sophomore standing or consent of
  instructor.
\item
  \textbf{SOC 338 Political Sociology} Examines the interconnections
  between politics and society. Emphasis is on the relationship between
  the state and social structures of capatalist societies. Theoretical,
  historical, and comparative materials are considered. Topics may
  include polititcal power, the polititcs of the welfare state,
  policy-making, and political participation. Prerequisite: Introductory
  Sociology (SOC 107) or consent of the instructor.\\
\item
  \textbf{SOC 351 Criminology} Examines the social meaning of criminal
  behavior. Looks at the relationship between crime and society - in
  particular, how the production and destribution of economic,
  political, and cultural resources shape the construction of ``law''
  and ``crime''. Includes a comparison of different types of crime,
  criminals, and victims, as well as at efforts to understand and
  control them. Prerequisite: Introductory Sociology (SOC-107).
\item
  \textbf{SOC 355 Deviant Behavior} Emphasis on theories and research in
  understanding deviant behaviors and deviant careers. Approaches
  include learning, strain, conflict, and labeling theories.
  Prerequisite: Introductory Sociology (SOC-107), consent of instructor.
\item
  \textbf{SOC 365 Research Participation I} Individual or group
  investigation with a sociology department faculty member on a research
  topic or topics of mutual interest. The student must obtain approval
  for a specific project and make necessary arrangements prior to the
  term of registration for the course. This course does not satisfy any
  of the requirements for a major or minor in sociology. Prerequisites:
  Introductory Sociology (SOC-107) and consent of instructor. (Offered
  by arrangement)\\
\item
  \textbf{SOC 375 Research Participation II} A continuation of Research
  Participation I. The student must obtain approval for a specific
  project and make necessary arrangements prior to the term of
  registration for the course. This course does not satisfy any of the
  requirements for a major or minor in sociology. Prerequisites:
  Research Participation I (SOC-365) and consent of instructor. (Offered
  by arrangement)
\item
  \textbf{SOC 417 Sociology of Sex \& Sexuality} Examines how social
  contexts in the United States shape sexuality. Gender is a significant
  theme throughout the course, with a focus on topics such as past and
  current research about sexual behavior and identities; the social
  construction of sexual orientations; connections among thnicity, race,
  and sexulaity; domestic and international sex work; sexual violence.
  Prerequisite: Methods of Sociological Research (SOC 235) or consent of
  instructor.\\
\item
  \textbf{SOC 425 Social Change} Theories of change applied to
  substantive areas (for example, modernization, economic development or
  restructuring, social values, and social definitions), selected by the
  instructor. The course addresses the differential experiences of the
  consequences of change among various social groups. Prerequisite:
  Methods of Social Research (SOC-235) or consent of instructor.
  (Offered alternate years)
\item
  \textbf{SOC 435 Advanced Topics in Sociology} Examines a particular
  sociological issue, problem, theory, or methodology. Content varies
  and is determined by the instructor. May be taken more than once for
  credit, provided the topics are substantially different. Prerequisite:
  Methods of Sociological Research (SOC-235) or consent of instructor.
\item
  \textbf{SOC 444 Ind Study-Soc} None\\
\item
  \textbf{SOC 445 Small Groups} Theory and research on the processes of
  social interaction and social psychology with special application to
  studies of groups. Prerequisite: Methods of Social Research (SOC-235)
  or consent of instructor.
\item
  \textbf{SOC 450 Sociological Theory} Examines the major paradigms and
  theories in contemporary sociology. The course places particular
  emphasis on the ways in which those paradigms and theories
  affect---and are affected by---the process of empirical research.
  Prerequisites: Methods of Sociological Research (SOC-235) and three
  other sociology courses above the introductory level or consent of
  instructor.\\
\item
  \textbf{SOC 464 Capstone Seminar in Sociology} Concepts, theories, and
  methodologies learned in previous sociology courses are applied to the
  intensive study of a topic selected by the instructor. Prerequisite:
  Sociological Theory (SOC-450), and senior standing, or consent of
  department chair.
\item
  \textbf{SOC 494 Internship in Sociology} A field placement with a
  career-related organization. A minimum of 140 hours on-site experience
  is required. P/NP basis only. This course does not satisfy any of the
  requirements for a major or minor in sociology. Prerequisite: declared
  major in sociology, junior standing, or consent of department chair.
\end{itemize}

\section{Spanish/Spanish Studies}\label{spanishspanish-studies}

Peach, Muñoz Pérez, Rodríguez Moreno (Program Coordinator).

\subsection{Spanish Major}\label{spanish-major}

A grade of ``C'' (2.0) or higher must be earned in all courses counted
toward a major in Spanish.

Students who major in Spanish complete a minimum of eight courses in
Spanish beyond SPA 225 Intermediate Spanish II , though SPA 148 Spanish
Literature in Translation may be taken as one of the eight. SPA 115
Elementary Spanish I , SPA 125 Elementary Spanish II , SPA 135 Spanish
Review \& Preparation , SPA 215 Intermediate Spanish I , and SPA 225
Intermediate Spanish II are regarded as skill-building courses, and do
not count toward the major in Spanish. Spanish students are urged to
substitute study abroad in a program approved by the College and the
department for up to three courses for the major. One Hispanic
literature course (SPA-336 or above) must be taken in the senior year.

+++MISSING INFO: c.spa464.long +++ is required during the final spring
term before graduation. Students selecting Spanish as a second teaching
field should complete a minimum of four Spanish courses numbered 315 or
above. Any advanced courses taken for the second teaching field should
include SPA 315 Spanish Composition \& Conversation .

\subsection{Spanish Minor}\label{spanish-minor}

A grade of ``C'' (2.0) or higher must be earned in all courses counted
toward a minor in Spanish.

The minor in Spanish requires the completion of four courses beyond SPA
225 Intermediate Spanish II , though with the consent of department
chair, SPA 148 Spanish Literature in Translation may be taken as one of
the four. SPA 115 Elementary Spanish I , SPA 125 Elementary Spanish II ,
SPA 135 Spanish Review \& Preparation , SPA 215 Intermediate Spanish I ,
and SPA 225 Intermediate Spanish II are regarded as skill-building
courses, and do not count toward the minor in Spanish.

\subsection{Spanish Studies Major}\label{spanish-studies-major}

A grade of ``C'' (2.0) or higher must be earned in all courses counted
toward a major in Spanish Studies.

The Spanish Studies major requires: 1) successful completion of four
courses beyond SPA 225 Intermediate Spanish II , with SPA 336 Hispanic
Life/Cult-Latin America or SPA 338 Hispanic Life/Culture:Europe
recommended as one of the four; and 2) successful completion of a
departmentally approved list of five courses proposed by the student
which exhibits both internal coherence and relevance to the coursework
in Spanish. Typically, application to the department for the approval of
the interdisciplinary elected courses is to be made no later than the
Spring Term of the junior year. SPA 115 Elementary Spanish I , SPA 125
Elementary Spanish II , SPA 135 Spanish Review \& Preparation, SPA 215
Intermediate Spanish I , and SPA 225 Intermediate Spanish II are
regarded as skill-building courses, and do not count toward the major in
Spanish Studies.

\subsection{Courses in Spanish}\label{courses-in-spanish}

\begin{itemize}
\tightlist
\item
  \textbf{+++MISSING INFO: c.spa115\_125.long +++} +++MISSING INFO:
  c.spa115\_125.desc +++\\
\item
  \textbf{+++MISSING INFO: c.spa129.long +++} +++MISSING INFO:
  c.spa129.desc +++
\item
  \textbf{SPA 135 Spanish Review \& Preparation} Designed to prepare
  students who have had some limited experience in Spanish to enter
  SPA-215 the following term. Strong students with one year of secondary
  school Spanish should enter this course. This class is also
  appropriate for students with two years of secondary school Spanish or
  for those for whom some time has elapsed since their earlier study of
  Spanish. This course provides review and elementary college-level
  preparation in Spanish grammar, vocabulary, speaking, and writing.\\
\item
  \textbf{SPA 148 Spanish Literature in Translation} Reading, in
  translation, of a selection of works by major Hispanic authors.
  Content varies from term to term. In most cases selections center on a
  theme, a time period, or a genre. Contact the instructor for specific
  information about course content. Students preparing a Spanish major
  or minor must write a paper in Spanish. Taught in English.
\item
  \textbf{SPA 215 Intermediate Spanish I} Continuation of Elementary
  Spanish, with emphasis on oral practice and a review of grammatical
  structures. Prerequisite: Elementary Spanish II (SPA-125) or Spanish
  Review and Preparation (SPA-135) or consent of instructor.\\
\item
  \textbf{SPA 258 Spanish Lang Learn in Spain} Features immersive
  leaning of Spanish in the context of a study abroad course supervised
  by a Coe faculty member. It includes three hours of daily instruction
  in Spanish with an additional hour studying the culture and traditions
  of the country. Prerequisite: consent of instructor. (Offered May Term
  only)
\item
  \textbf{SPA 315 Spanish Composition \& Conversation} A course
  stressing the skills of speaking and writing. Some third-year grammar
  is studied, with an emphasis on the key problematic structures of the
  Spanish language. Students write frequent compositions, and class time
  concentrates on conversational activities. Prerequisite: Intermediate
  Spanish II (SPA-225).\\
\item
  \textbf{SPA 325 Spanish Language \& Literature} Continuation of
  Spanish Composition and Conversation, with an emphasis on speaking and
  writing. A wide range of short literary texts is also studied.
  Prerequisite: Intermediate Spanish II (SPA-315).
\item
  \textbf{SPA 330 Spanish for Heritage Speakers} Focuses on the specific
  linguistic and communicative needs of heritage speakers of Spanish in
  the context of exploring issues of identity, immigration and community
  in contemporary Latino communities in the U.S. Assessments based on an
  exit interview and a portfolio of formal writing. Prerequisite:
  Spanish placement test or approval of the instructor.\\
\item
  \textbf{SPA 336 Hispanic Life/Cult-Latin America} An interdisciplinary
  overview of the culture and civilization of Latin America as portrayed
  in literature, art, architecture, and film. Particular attention is
  given to the way the past has created and shaped contemporary Hispanic
  culture in Latin America. Prerequisite: Spanish Composition and
  Conversation (SPA-325).
\item
  \textbf{SPA 338 Hispanic Life/Culture:Europe} Same as Hispanic Life
  and Culture (SPA-336) except focus of the course is European culture
  and civilization as portrayed in literature, art, architecture, and
  film. Particular attention is given to the way the past has created
  and shaped Hispanic culture in Spain.\\
\item
  \textbf{SPA 339 Spanish for Health Care} Focuses on providing Spanish
  language proficiency for students planning careers in the area of
  healthcare. Emphasis is placed on the healthcare vocabulary and the
  culture of the Latino population living in the United States. Local
  outreach is a required component of this course. Students visit local
  health clinics, where they are able to volunteer, using their
  knowledge of Spanish and working as translators. Prerequisite:
  Intermediate Spanish II (SPA-225) or consent of instructor.
\item
  \textbf{SPA 345 Introduction to Hispanic Literature} An introduction
  to the study of Peninsular and Spanish-American literary works, with
  an emphasis on basic critical concepts, terminology, and methods. A
  wide range of texts is studied, beginning with poems and short
  narratives, and ending with longer works. Prerequisite: Spanish
  Composition and Conversation (SPA-315) or Spanish Language and
  Literature (SPA-325). (Offered alternate years)\\
\item
  \textbf{SPA 349 Business Spanish} Focuses on providing Spanish
  language proficiency for students planning careers in business.
  Emphasis is placed on the different cultures of Spanish-speaking
  countries in the business world. Local outreach is a required
  component of this course. Students visit local businesses, where they
  learn about the culture of the Spanish-speaking business world.
  Prerequisite: Intermediate Spanish II (SPA-225) or consent of
  instructor.
\item
  \textbf{SPA 394 Directed Learning: Spanish} For students wishing to
  investigate a particular aspect of Spanish literature unavailable
  through the regular sequence of courses offered. Periodic conferences
  and papers are required. May be taken more than once for credit.
  Prerequisite: Spanish Composition and Conversation (SPA-315) or
  Spanish Language and Literature (SPA-325).\\
\item
  \textbf{SPA 412 Ecologies:Latin America Environ Lit} Ecologies: Latin
  American Environmental Literature and Cultural Works (WE) Examines
  Latin American writers and artists who have long focused on nature as
  a means to address pressing political, social, and ethical issues
  through literature, ethnographic texts, film and the visual arts.
  Readings in this class will trace connections between environmental
  thought and the region's long and multi-layered history of
  colonialism. Prerequisite: Introduction to Hispanic Literature
  (SPA-345) or Hispanic Life and Culture (SPA-336 / SPA-338).
\item
  \textbf{SPA 418 Gender \& Sexuality in Hispanic Wrld} Examines
  constructions and representations of gender, sexuality, and power in
  Latin American and/or Spanish literature and cultures with particular
  emphasis on intersections with race/ethnic positioning, sexual
  identity, and social class. The course adopts an interdisciplinary
  approach to explore the role and contributions of artists, as well as
  the reception of their work, with regard to issues surrounding
  cultural representations of race and ethnicity, masculinity and
  femininity, gender and sexual identities and LGTBQ+ communities,
  nationalism and citizenship, and social movements. Prerequisite:
  Introduction to Hispanic Literature (WE) (SPA-345) or Hispanic Life
  and Culture (SPA-336 / SPA-338).
\item
  \textbf{SPA 421 Nation, History, and Literature} Focuses on the
  literature and culture of one nation in the Hispanic world,
  highlighting their relationship to the specific national history of
  the country. This class will study a variety of genres: short stories,
  poetry, and theater. Prerequisite: Introduction to Hispanic Literature
  (WE) (SPA-345) or Hispanic Life and Culture (SPA-336 / SPA-338).
\item
  \textbf{SPA 428 Indigeneity, Blackness, \& EthnicLit} Examines voices
  of ethnic cultures that have creatively responded to their social and
  political contexts through literary analysis as well as visual,
  musical, and other symbolic and representational literacies.
  Prerequisite: Introduction to Hispanic Literature (WE) (SPA-345) or
  Hispanic Life and Culture (SPA-336 / SPA-338).
\item
  \textbf{SPA 431 Hispanic Graphic Novel:Theory\& Prac} Focuses on
  providing students with the knowledge to understand cartoons and
  graphic novels produced in Spanish as a particular medium in relation
  to a historical context. The class will study theory of comics, the
  tradition of Hispanic comics, and will hold a workshop where the
  students learn techniques to develop their own story in a comic
  format. Prerequisite: Introduction to Hispanic Literature (WE)
  (SPA-345) or Hispanic Life and Culture (SPA-336 / SPA-338)
\item
  \textbf{SPA 432 Representations of Violence} Explores the
  relationships between a variety of cultural productions (film,
  literature, poetry, painting, theater, performance, TV productions,
  etc.) and specific accounts of personal and collective violence in
  socio-political conflicts in the Hispanic world. Concepts such as
  memory, trauma, mourning, and pain will be a central part of this
  course. Prerequisite: Introduction to Hispanic Literature (SPA-345) or
  Hispanic Life and Culture (SPA-336 / SPA338).
\item
  \textbf{SPA 442 Popular Culture, Media, \& Cultural} Focuses on
  popular culture with an emphasis on the role of visual and material
  culture in the Hispanic world as contested territory between power
  structures and resistance movements. A variety of cultural productions
  may be included: film, literature, music, internet blogs, YouTube
  videos, publicity, folklore, dance, TV productions, comic strips, etc.
  This class will focus on a selected time period and/or geographical
  region of the Spanish-speaking world. May be taken more than once for
  credit, if focused on a different region or period. Prerequisite:
  Introduction to Hispanic Literature (WE) (SPA-345) or Hispanic Life
  and Culture (SPA-336 / SPA-338).
\item
  \textbf{SPA 443 Cinema \& Politics of Representation} Provides a
  historical and thematic overview of cinema in a selected time period
  and/or geographical region of the Spanish-speaking world to deepen
  understanding of and engagement with important and complex cultural
  issues in relation to both aesthetics and politics, and offers both
  offers tools and guidance for discussing and writing about film. May
  be taken more than once for credit, if focused on a different region
  or period. Prerequisite: Introduction to Hispanic Literature (WE)
  (SPA-345) or Hispanic Life and Culture (SPA-336 / SPA-338).\\
\item
  \textbf{SPA 444 Ind Study-Spanish} Independent investigation of a
  selected project in Spanish under the direction of a faculty member of
  the department. May be taken for an X status grade with consent of
  instructor prior to registration. Prerequisite: consent of department
  chair.
\item
  \textbf{SPA 446 Latin Am. \& Spanish Short Stories} Focuses on the
  rich and varied literary production of the short story in Latin
  America and/or Spain. Readings may include a wide range of short
  narratives with an emphasis on those of the 20th and 21st century.
  Prerequisite: Introduction to Hispanic Literature (SPA-345) or
  Hispanic Life and Culture (SPA-336 / SPA-338).
\item
  \textbf{SPA 451 Hispanic Cross-Over Literature} Focuses on literature
  that is designed for children or for both adults and children. Texts
  for this class will intersect with notions of memory, education,
  censorship, and the dialogue between image and written text. The class
  will emphasize the picture-book (álbum ilustrado), and it will hold a
  workshop where the students learn techniques to develop their own.
  Prerequisite: Introduction to Hispanic Literature (WE) (SPA-345) or
  Hispanic Life and Culture (SPA-336 / SPA-338).
\item
  \textbf{SPA 455 Spanish Drama} Same as Hispanic Drama (SPA-455) except
  focus of course is Latin American theatre. An introduction to Spanish
  theatre and the various techniques, themes, and images used to
  express, criticize, or romanticize Hispanic society and life.
  Prerequisite: Spanish Composition and Conversation (SPA-315) or
  Spanish Language and Literature (SPA-325).
\item
  \textbf{SPA 457 US LatinX Literature} Focuses on the major trends of
  LatinX literature that highlight the experiences of LatinX people
  within the US. Topics will include bilingualism, code-switching,
  identity, borders, immigration, and exile in LatinX groups such as
  Chicano/a, Nuyorican, Boricuas, Dominicans, etc. Their cultural
  productions will be analyzed within their social and political context
  and texts assigned will be in English, Spanish or mixed. Prerequisite:
  Introduction to Hispanic Literature (SPA-345) or Hispanic Life and
  Culture (SPA-336 / SPA-338).
\item
  \textbf{SPA 458 Travel Writing \& Transatlantic Lit} Explores the
  varying perceptions of life, history, culture, traditions, and customs
  in Spain and Latin America across time and space through the accounts
  of travel narratives through interdisciplinary perspectives.
  Prerequisite: Introduction to Hispanic Literature (SPA-345) or
  Hispanic Life and Culture (SPA-336 / SPA-338).
\item
  \textbf{SPA 475 Topics in Hispanic Literature} A course of narrowed
  focus that centers on a theme, region, time period, or genre of
  Hispanic literature. Approach and content varies from term to term as
  determined by the instructor. May be taken twice, provided the topics
  are substantially different. Prerequisite: Spanish Composition and
  Conversation (SPA-315) or Spanish Language and Literature (SPA-325).
\item
  \textbf{SPA 476 Topics in Hispanic Lit:Latin Amer} Same as Topics in
  Hispanic Literature (SPA-475, -485) except the focus of the course is
  Latin American culture.
\item
  \textbf{+++MISSING INFO: c.spa464.long +++} +++MISSING INFO:
  c.spa464.desc +++
\item
  \textbf{SPA 494 Internship in Spanish} Exploration of a career area
  related to Spanish. Application and supervision through the Internship
  Specialist. A minimum of 140 hours on-site experience is required. S/U
  basis only. This course does not satisfy any of the requirements for a
  major or minor in Spanish. Prerequisites: junior standing and consent
  of department chair.
\end{itemize}

\section{Statistics (Courses Only)}\label{statistics-courses-only}

Cross.

\subsection{Courses in Statistics}\label{courses-in-statistics}

\begin{itemize}
\tightlist
\item
  \textbf{STA 100 Statistical Reasoning I-Foundations} A hands-on
  introduction to the use of statistical techniques. Provides a
  foundation for statistical analysis and introduces the basic concepts
  involved in data collection and presentation. (0.5 course credit)
  (Offered first 7 weeks of Term)\\
\item
  \textbf{STA 105 Probability: A World of Chance} An introduction to
  probability and its applications in our world. This hands-on course
  examines how probability techniques can be used to understand topics
  in science, government, recreation, and communication. Bizarre events
  in everyday life are also discussed. This course is appropriate for a
  varied audience. Some ability in arithmetic and elementary algebra is
  assumed. This course does not satisfy any of the requirements for a
  major or minor in the mathematical sciences.
\item
  \textbf{STA 110 Stats IIA: Inferential Reasoning} A continuation of
  Statistical Reasoning I (STA-100), presenting a broad range of data
  analysis techniques. Topics covered include hypothesis testing,
  confidence intervals, Chi-square tests, and regression. Emphasis is on
  a project-based approach to analyzing data. Prerequisite: Statistical
  Reasoning I (STA-100) or consent of instructor. (0.5 course credit)
  (Offered second seven weeks of Term)
\item
  \textbf{STA 130 Stats IIB: Experimental Design} +++MISSING INFO:
  c.smt130.desc +++
\item
  \textbf{STA 315 Mathematical Probability} A calculus-based
  introduction to the mathematical theory of probability. Topics include
  enumeration techniques, Bayes' theorem, random variables, discrete and
  continuous distributions, expectation, moment-generating functions,
  sampling distribution theory, and simulation techniques.
  Prerequisites: Computational Linear Algebra (MTH-165) and Calculus II
  (MTH-145), or consent of instructor.
\item
  \textbf{STA 325 Mathematical Statistics} +++MISSING INFO:
  c.smt325.desc +++
\end{itemize}

\section{Theatre Arts}\label{theatre-arts}

D. Barnett, Charipar, Ganfield, Hahn, Rezabek, Schmidt, Steffens,
Wolverton (Chair)

The Theatre Arts program is designed to provide a balance between
academic and experiential learning for all interested students. The goal
is to offer opportunities for student participation in all aspects of
theatre, on stage and in the classroom, and to enrich the cultural and
academic life of the College.

The program accommodates both majors and minors through courses focused
on theatrical history, theory and literature, as well as through
training in acting, directing, design, and theatrical production. This
foundation prepares students for graduate study and for any field that
values high-level communication skills, as well as for careers in
theatre as teachers, artists, technicians, or managers.

As a means of helping students prepare for graduate study and/or a
professional career, all majors are required to present an audition
piece or a portfolio for annual review beginning in their second year.
The senior seminar provides a capstone experience, giving students an
opportunity to reflect on their development as theatre artists, and to
examine theatre as a collaborative art form.

For the campus community at large, as audience members or as occasional
participants, the program offers a wide range of drama selected both to
educate and to entertain.

\subsection{Theatre Arts Major}\label{sec-theatre-arts-major}

A major in Theatre Arts requires a cumulative 2.0 GPA in all courses
counted toward the major.

\textbf{Theatre Arts Core Courses} (required of all students majoring in
theatre arts):

\begin{enumerate}
\def\labelenumi{\arabic{enumi}.}
\item
  THE 102 Theatre Production Lab
\item
  THE 130 Technical Production I
\item
  THE 140 Design for the Stage
\item
  THE 150 Acting I
\item
  THE 228 History of Theatre and Drama I
\item
  THE 238 History of Theatre and Drama II
\item
  THE 290 Directing I
\item
  THE 464 Senior Seminar
\end{enumerate}

\subsection{Theatre Arts Minor}\label{theatre-arts-minor}

The minor in Theatre Arts consists of six course credits of Theatre
Arts.

Theatre Arts students select one of the following \textbf{emphases}:
general, acting, directing, musical theatre, and technical
theatre/design.

\textbf{General}

\begin{enumerate}
\def\labelenumi{\arabic{enumi}.}
\item
  \textbf{Eight} core courses (see Section~\ref{sec-theatre-arts-major}
  )
\item
  \textbf{Three} additional course credits of Theatre Arts, chosen with
  departmental approval.
\end{enumerate}

\textbf{Acting}

\begin{enumerate}
\def\labelenumi{\arabic{enumi}.}
\item
  \textbf{Eight} core courses (see Section~\ref{sec-theatre-arts-major}
  )
\item
  THE 170 Voice \& Diction
\item
  THE 250 Acting II
\item
  \textbf{One} of the following:

  \begin{itemize}
  \tightlist
  \item
    THE 160 Movement for the Stage
  \item
    THE 260 Acting for the Camera
  \item
    THE 270 Musical Theatre Acting
  \item
    THE 350 Advanced Acting:Shakespeare
  \end{itemize}
\end{enumerate}

\emph{Strongly recommended:} - Any dance course (DAN-101 through
DAN-152) - THE 162 Stage Make-Up - THE 452 Advanced Projects in Acting -
PHL 105 Introduction to Philosophy: - PSY 100 Introductory Psychology

\textbf{Directing}

\begin{enumerate}
\def\labelenumi{\arabic{enumi}.}
\item
  \textbf{Eight} core courses (see Section~\ref{sec-theatre-arts-major}
  )
\item
  THE 250 Acting II
\item
  THE 390 Directing II WE
\item
  \textbf{One} additional course in acting or design
\end{enumerate}

\emph{Strongly recommended:} - ARH 118 History of Western Architecture -
COM 237 Interpersonal Communication - THE 220 Tech Theatre Lab\\
- THE 242 Scene Design - THE 280 Costume Design - THE 350 Advanced
Acting:Shakespeare - THE 462 Advanced Projects in Directing - Any other
art history course

\textbf{Musical Theatre}

\begin{enumerate}
\def\labelenumi{\arabic{enumi}.}
\item
  \textbf{Eight} core courses (See Section~\ref{sec-theatre-arts-major}
  )
\item
  MU 109 Theory of Music I
\item
  THE 170 Voice \& Diction
\item
  THE 270 Musical Theatre Acting
\item
  \textbf{Four} terms of MUA 202V Voice (0.6 cc)
\item
  \textbf{One} additional course credit in practical musical theatre
  chosen from a combination of the following:

  \begin{itemize}
  \tightlist
  \item
    MUA 130V MusicalTheatreProductionExperience (0.5 cc)
  \item
    MUA 131V Song Interpretation Workshop (0.2 cc)
  \item
    A musical theatre internship approved by the Music or Theatre Arts
    department
  \end{itemize}
\item
  \textbf{Six} seven-week dance courses (DAN-101 through DAN-142) (0.2
  cc)
\end{enumerate}

\textbf{Technical Theatre/Design}

\begin{enumerate}
\def\labelenumi{\arabic{enumi}.}
\item
  \textbf{Eight} core courses (see Section~\ref{sec-theatre-arts-major}
  )
\item
  \textbf{Three}* of the following:

  \begin{itemize}
  \tightlist
  \item
    THE 220 Tech Theatre Lab
  \item
    THE 230 Technical Production II
  \item
    THE 232 Computer Aided Drafting and Design
  \item
    THE 242 Scene Design
  \item
    THE 280 Costume Design
  \item
    THE 288 History of Dress
  \end{itemize}
\end{enumerate}

\emph{Strongly recommended:} - ARH 118 History of Western Architecture -
Any other Art History course - ART 115 Drawing - ART 145 Digital Studio
- ART 364 The Human Form - PHY 155 Electronics and PHY 155L Electronics
Lab\\
- THE 442 Adv Proj-Design/Tech Production

\subsection{Courses in Theatre Arts}\label{courses-in-theatre-arts}

\begin{itemize}
\tightlist
\item
  \textbf{THE 100 Introduction to Theatre} A study of the art of
  theatre, emphasizing theatre's place among the humanities; its
  relationship to the other arts; and its cultural and social influences
  in our society. Students derive a foundation for discriminating
  theatregoing through analysis of dramatic form and of selected
  playtexts; consideration of the methods and techniques employed by
  theatre artists and crafts persons; and a brief survey of theatre and
  drama, both in their historical context and as they have been
  manifested through related media. As part of the study, students are
  required to attend some evening events. This course does not satisfy
  any of the requirements for a major in theatre arts.\\
\item
  \textbf{THE 102 Theatre Production Lab} Applied skills in one of the
  theatre production crews. Students assist with set construction,
  costuming, lighting, painting, publicity, sound, or run-crew positions
  for productions during a term. No previous experience is required. Lab
  meets four hours per week. Theatre arts majors are required to take
  four labs. (0.25 course credit)
\item
  \textbf{THE 112 Production Experience} Practical involvement in a
  mainstage production as an actor (in a leading or secondary role),
  stage manager, assistant director, dramaturge, technical crew head
  (property master, master electrician, wardrobe master, etc.), or in
  another capacity as approved by the faculty (excluding run-crew
  positions). May be taken more than once. Prerequisite: consent of
  department chair. (0.5 course credit)\\
\item
  \textbf{THE 118 Theatre \& Arts in Serbia} Explores the recent history
  and political realities of Belgrade, Serbia, through the lens of
  Theatre and other art forms, as well as visits to multiple cultural
  formations: museums, churches, monuments and schools. Offered May Term
  only. Prerequisite: consent of instructor.
\item
  \textbf{THE 130 Technical Production I} An introduction to the
  technical aspects of theatre production. Topics covered include
  safety, mechanical drawing, stage carpentry, craft techniques, stage
  lighting and electricity, costume construction, scene painting, and
  production organization. Class includes lecture-demonstration and
  practical application. Additional time outside of class is necessary
  to complete projects. (Offered Fall Term)\\
\item
  \textbf{THE 140 Design for the Stage} Examines the design process for
  all aspects of theatre design (costume, scenery, and lighting).
  Specific topics include the design elements, script analysis,
  research, basic drawing, basic drafting, and presentation techniques.
  A series of projects and readings introduce students to the basic
  language of visual story telling. Additional time outside of class is
  necessary to complete projects. (Offered Spring Term)
\item
  \textbf{THE 145 Viewpoints Ensemble Work} A course designed to provide
  students with a comprehensive understanding of Viewpoints Technique
  for actors. Viewpoints is a ``philosophy translated into a technique
  for training performers, building ensemble and creating movement for
  the stage.'' Students working within a group dynamic learn to access
  acutely their centers of awareness and intensify their ability to work
  in the ``here and now.'' Hands-on training is supplemented by
  readings, quizzes and journal writing. (Offered Fall Term, alternate
  years)\\
\item
  \textbf{THE 150 Acting I} Stanislavsky-based, comprehensive
  introduction to the elements of acting for the stage. Through theatre
  games, improvisations, exercises, and partnered work on scenes
  grounded in realism, students learn to identify and personalize a
  character's ``objectives'' and the ``obstacles'' that stand in the way
  of attaining them, and to engage themselves (via voice, body, mind,
  and spirit) in specific ``actions'' undertaken in pursuit of those
  identified goals. Emphasis is placed on ``interactive'' skills and on
  character-specific listening. Additional rehearsal time outside of
  class is required.
\item
  \textbf{THE 160 Movement for the Stage} Designed to help students
  learn the effective use of the body as a component of the acting
  process. Physical expression in movement and gesture is developed by
  way of in-class exercises and improvisations, leading to the solo and
  collaborative creation of movement pieces for performance. Skills of
  concentration, breath control, partner awareness, and physical
  characterization are also fostered in this work.\\
\item
  \textbf{THE 162 Stage Make-Up} The theory and practice of designing
  and creating make-up for the stage actor. Lecture/demonstration plus
  laboratory experience.
\item
  \textbf{THE 170 Voice \& Diction} Designed to help students learn how
  to use the voice as a component of the acting process. The mechanics
  of vocal production and of speech are examined, along with various
  approaches for their improvement. A number of performance projects
  supplement a wide range of vocal exercises and drills. Students are
  introduced to the International Phonetic Alphabet (IPA) as the basis
  for determining correct pronunciation, both in standard speech and in
  preparing dialects and accents. Additional rehearsal time outside of
  class is required.\\
\item
  \textbf{THE 185 Production and Performance} A ``theatre company''
  consisting of actors, designers, a stage manager, costume, set, and
  lighting crew (plus an accompanist and choreographer, if needed). The
  company has as its goal the mounting of the chosen production in a
  fully-collaborative atmosphere. In addition to fulfilling their
  various creative functions, company members may be asked to take on,
  under the supervision of a faculty member of the department,
  dramaturgical assignments designed to provide research and study on
  selected aspects of the play or musical. This material is presented to
  the Coe community in the form of public displays throughout the
  campus. Company members need to audition prior to spring registration.
  The basis for student evaluation is completion of assigned duties and
  quality of work. May be taken more than once for credit. A maximum of
  two course credits may be counted toward a major or minor in theatre
  arts. (Offered Spring Term)
\item
  \textbf{THE 220 Tech Theatre Lab} A study of the equipment, mechanics,
  and theories used by lighting designers to produce lighting for
  theatre, dance, concerts, and architecture. Specific topics include
  design research and conceptualization, color, angle, cueing, and
  methods of presentation (sketches, storyboards, light plots, and
  associated paperwork). Class includes lecture/demonstration and
  practical application. Additional time outside of class is necessary
  to complete projects. Prerequisites: Design for the Stage (THE-140)
  and Technical Production I (THE-130) or consent of instructor.
  (Offered Spring Term, alternate years)\\
\item
  \textbf{THE 228 History of Theatre and Drama I} A writing intensive
  course that provides students with an overview of World Theatre and
  Drama from cultures where oral traditions were the dominant forms of
  communication through and including those wherein the written word
  presented a new paradigm. The course provides a practical
  investigation of current critical discourses that examine dramatic
  literature and performance traditions from around the world. (Offered
  Fall Term in rotation with History of Theatre and Drama II (THE-238))
\item
  \textbf{THE 230 Technical Production II} A full-term study of the
  techniques and practical applications in a specific area of technical
  theatre. Possible course topics include stage management, drafting,
  sketching and rendering, scene painting, millinery, costume crafts, or
  fabric design. Prerequisite: Technical Production I (THE-130) or
  consent of instructor. (Offered by arrangement)\\
\item
  \textbf{THE 232 Computer Aided Drafting and Design} An introduction to
  the use of the computer as a drafting and design tool. Using
  theatre-related examples and projects, students create two- and
  three-dimensional drawings. Further work involves 3-D modeling and
  creating perspective images of virtual stage settings. Class includes
  lecture/demonstration and laboratory. Additional time outside of class
  is necessary to complete projects. (Offered Spring Term, alternate
  years)
\item
  \textbf{THE 238 History of Theatre and Drama II} A writing intensive
  course that provides students with an overview of World Theatre and
  Drama, focusing primarily on cultures and periods wherein new forms of
  technology, from the camera to the computer, have become determining
  factors in systems of communication. The course provides a practical
  investigation of current critical discourses that examine dramatic
  literature from around the world. (Offered Fall Term in rotation with
  History of Theatre and Drama I (THE-228))\\
\item
  \textbf{THE 242 Scene Design} A study of the aesthetic, historical,
  and technical aspects of stage design. Projects emphasize research,
  drawing, drafting, problem solving, model building, and rendering.
  Class includes lecture/demonstration and practical application.
  Additional time outside of class is necessary to complete projects.
  Prerequisites: Technical Production I (THE-130) and Design for the
  Stage (THE-140) or consent of instructor.
\item
  \textbf{THE 250 Acting II} A performance-based studio course designed
  to expand upon the work begun in Acting I by assisting students in the
  development of intermediate-level acting skills. Scenes and monologues
  progress beyond realism with an introduction to a number of other
  styles. Significant emphasis is placed on character development and on
  detailed analysis of dramatic action. Additional rehearsal time
  outside of class is required. Prerequisite: Acting I (THE-150).
  (Offered Spring Term)\\
\item
  \textbf{THE 255 Playwriting Workshop I} See CRW 255 Playwriting
  Workshop I , \textbf{?@sec-couses-in-creative-writing} Explores the
  basics of playwriting through the study of dramatic structure,
  creative exercises, and reading and analysis of existing play scripts,
  with emphasis on the one-act play. Students complete a one-act play
  script of their own creation.
\item
  \textbf{THE 260 Acting for the Camera} Development of basic acting
  techniques essential for work in film and television. The course is
  designed to help students become comfortable in front of the camera by
  way of breaking down their physical and internal inhibitions. The
  instructor helps the student find the most effective personal
  resources for the creation of truthful characters, given the unique
  pressures of a film or television shoot. The course also focuses on
  audition techniques and culminates in the shooting of a scripted
  scene. Additional rehearsal time outside of class is required.
  Prerequisite: Acting I (THE-150).\\
\item
  \textbf{THE 270 Musical Theatre Acting} See MU 270 Musical Theatre
  Acting , \textbf{?@sec-couses-in-music} A performance-based studio
  course focusing on the development of basic skills necessary for
  musical theatre performance. Students become familiar with the
  specialized requirements necessary for the merging of singing with
  dramatic action. Periodic performance projects (solos, duets, and
  ensemble numbers---some including dialogue) are supplemented by
  student research projects. The course is also designed to introduce
  students to a wide-ranging repertoire of available audition material.
  Additional rehearsal time outside of class is required.
\item
  \textbf{THE 280 Costume Design} A study of the aesthetic, historical,
  and technical aspects of costume design. Projects emphasize research,
  character analysis, figure drawing, textile selection, and rendering.
  Class includes lecture-demonstration and practical application.
  Additional time outside of class is necessary to complete projects.
  Prerequisites: Technical Production I (THE-130) and Design for the
  Stage (THE-140) or consent of instructor.\\
\item
  \textbf{THE 288 History of Dress} Traces the history of dress from
  ancient to modern times, with a special emphasis on dress as it
  relates to gender roles, social issues, cultural practices and
  beliefs, politics, and economic patterns within an historical context.
\item
  \textbf{THE 290 Directing I} Offers students an introduction to the
  fundamental tools of directing plays for the stage. Analysis of
  playtexts is undertaken to demonstrate how a director develops the
  vision of a play that serves as an interpretive guide throughout the
  production process. In-class exercises, improvisations, and staged
  ``image'' pieces focus on the acquisition of skills for communicating
  effectively with actors. Scene projects are rehearsed outside of class
  for in-class presentation, and comprehensive promptbooks are prepared
  in conjunction with each. Additional rehearsal time outside of class
  is required. Prerequisite: Acting I (THE-150). (Offered Spring Term)\\
\item
  \textbf{THE 350 Advanced Acting:Shakespeare} A performance-based
  studio course aimed at the development of advanced-level acting skills
  specifi- cally related to performing the playtexts of William
  Shakespeare. An eclectic approach to this complex material is offered,
  with emphasis divided between its verbal, physical, and psychological
  demands. Students prepare and perform several scenes and monologues
  throughout the term. Exercises and improvisations related to
  characterization and written character analyses are also components of
  this course. Additional rehearsal time outside of class is required.
  Prerequisite: Acting I (THE-150).
\item
  \textbf{THE 355 Playwriting Workshop 2} See also CRW 355 Playwriting
  Workshop 2 , \textbf{?@sec-couses-in-creative-writing} Focuses on the
  analysis and creation of play scripts of two acts or more. Emphasis is
  placed on the writing and marketing of the student's own creative
  work, culminating in the completion and public reading of a
  full-length script. Prerequisite: Beginning Playwriting
  (CRW/THE-255).\\
\item
  \textbf{THE 390 Directing II WE} An intermediate-level course or
  independent study expanding upon the work begun in Directing I by
  developing more complex analytical tools and deepening the work of
  conceptualization for production. Assigned readings guide the
  directing student to knowledge and understanding of a wide variety of
  20th-century and contemporary directing theories and methodologies.
  The course culminates in the analysis and preparation with actors of a
  short one-act play or an extended scene from a full-length play that
  allows the student director to work in a non-realist style.
  Prerequisite: Directing I (THE-290). (Offered by arrangement)
\item
  \textbf{THE 442 Adv Proj-Design/Tech Production} Individual work in a
  specific aspect of theatrical production: lighting design, costume
  design, scene design, sound design, technical direction, or stage
  management. A written proposal, conceptual statement, documentation of
  process, and self-evaluation are components of this upper-level
  course. May involve actual work for a departmental production. May be
  taken more than once, provided the emphasis varies. Prerequisite:
  consent of department chair. (Offered by arrangement)\\
\item
  \textbf{THE 444 Ind Study-Theatre} The faculty-supervised staging of
  an extended one-act or a full-length play as an advanced project. The
  production is staged in the Mills Experimental Theatre and may receive
  modest technical and design support, when appropriate. Comprehensive
  playtext analysis and documentation of process are components of this
  course. Prerequisite: Design for the Stage (THE-140), Directing II
  (THE-390), or consent of department chair. (Offered by arrangement)
\item
  \textbf{THE 452 Advanced Projects in Acting} A senior acting recital
  designed to demonstrate the proficiency level of majors with an acting
  emphasis. The recital may take the form of a one-person performance or
  a selection of scenes, monologues, and /or musical numbers
  demonstrating the student's range and versatility. The project
  requires conceptualization, organization and selection of performance
  material, and adherence to a pre-arranged rehearsal process. Written
  components include a detailed proposal, a comprehensive statement of
  concept, thorough documentation of process, and analytical
  self-evaluation. Prerequisites: senior standing and consent of
  department chair. (Offered by arrangement)\\
\item
  \textbf{THE 455 Playwriting Workshop 3} See also CRW 455 Playwriting
  Workshop 3 , \textbf{?@sec-couses-in-creative-writing} Continued
  advanced writing, with an emphasis on producing finished poems.
  Prerequisite: Poetry Workshop 2 (CRW-380).
\item
  \textbf{THE 462 Advanced Projects in Directing} The faculty-supervised
  staging of an extended one-act or a full-length play as an advanced
  project. The production is staged in the Mills Experimental Theatre
  and may receive modest technical and design support, when appropriate.
  Comprehensive playtext analysis and documentation of process are
  components of this course. Prerequisite: Design for the Stage
  (THE-140), Directing II (THE-390), or consent of department chair.
  (Offered by arrangement)\\
\item
  \textbf{THE 464 Senior Seminar} Capstone course for theatre arts
  majors that requires students to synthesize their study of dramatic
  theory and literature and their experiences in theatre performance and
  production. Assignments center on dramaturgy, dramatic theory and
  criticism, and on the current state of the art. Required of theatre
  arts majors in their senior year. (Offered Fall Term)
\item
  \textbf{THE 486 Spc Top Theatre or Film: NWP} None\\
\item
  \textbf{THE 488 Special Topics in THE/ FLM} Focuses on specific
  aspects of theatre or film. Possible topics include: Eastern European
  Theatre, Post-War Eastern European Cinema, Queer Cinema, Feminist
  Theatre. May be taken more than once, provided the topics are
  substantially different. With departmental approval, this course may
  be used to satisfy the requirements for a major in theatre arts.
\item
  \textbf{THE 494 Internship in Theatre Arts} An experience in
  professional or community theatre under the direction of an on-site
  supervisor in cooperation with a faculty member of the department and
  the Internship Specialist. A minimum of 140 hours on-site experience
  is required. S/U basis only. One internship credit may be used to
  satisfy the requirements for a major in theatre arts. Prerequisites:
  junior standing, declared major in theatre arts, and consent of
  department chair.
\end{itemize}

\section{Workshops (Courses Only)}\label{workshops-courses-only}

Hughes, Rogers.

Workshops (0.25 and 0.5 course credit) are designed to allow hands-on,
experiential learning in a practice-based context. WKS courses can be
interdisciplinary explorations, or offer opportunities to learn, refine,
or apply a skill set within a specific discipline.

\begin{itemize}
\tightlist
\item
  \textbf{WKS 201 Workshop: Studio Lighting/Portrait} An introduction to
  studio lighting for photography. Following a review of the history of
  the photographic portrait, students learn the basics of studio
  lighting for expressive portraits. (0.5 course credit).\\
\item
  \textbf{WKS 202 Workshop: Encaustic Painting} An introduction to
  encaustic painting. Encaustic, one of the oldest painting mediums, is
  pigment suspended in wax. Projects help students learn the technical
  process and explore the expressive potential of the medium. (0.5
  course credit).
\item
  \textbf{WKS 203 Workshop: Egg Tempera Painting} An introduction to egg
  tempera painting. Egg tempera was the most common painting medium
  before the introduction of oil paint. Projects help students learn the
  technical process and explore the expressive potential of the medium.
  (0.5 course credit).\\
\item
  \textbf{WKS 204 Workshop: Digital Toolbox} An introduction to Adobe
  Photoshop as an art-making tool through a series of self-directed
  creative projects (not tutorials). Students learn through hands on
  experience, guided with interactive demonstrations and assignments.
  (0.5 course credit).
\item
  \textbf{WKS 211 Workshop: Design Thinking Project} Learn to solve real
  world problems using design thinking processes. (0.5 course credit).\\
\item
  \textbf{WKS 212 Workshop: 3D Modeling \& Output} An introduction to 3D
  modeling software. Course culminates in outputting the final project
  by means of a 3D printer or other means appropriate to the project.
  (0.5 course credit).
\item
  \textbf{WKS 213 Workshop: Vector Graphics} An introduction to vector
  graphics programs such as Adobe Illustrator as art-making tools though
  a series of self-directed creative projects (not tutorials). Students
  learn through hands on experience, guided with interactive
  demonstrations and assignments. (0.5 course credit).\\
\item
  \textbf{WKS 214 Workshop: Stop Motion} An introduction to stop-motion
  animation as an art-making tool, through a series of self-directed
  creative projects (not tutorials). Students learn through hands on
  experience, guided with interactive demonstration and assignments.
  (0.5 course credit).
\item
  \textbf{WKS 221 Innovation Lab I} A workshop for students to engage in
  collaborative, innovative projects in the Center for Creativity.
  Students can work on smaller initiatives for 7 weeks (Innovation Lab
  I) or tackle more complex problems for an entire semester (Innovation
  Lab I and II)\\
\item
  \textbf{WKS 241 Workshop: Topics} Investigates specific topics and/or
  tools in art-making. (0.5 course credit).
\item
  \textbf{WKS 251 Workshop: Sound} An introduction to sound as an art
  medium, using recording devices, software and installation, through a
  series of self-directed creative projects (not tutorials). Students
  learn through hands on experience, guided with interactive
  demonstrations and assignments. Prerequisite: Illuminated Pixels
  (ART-145), or Movement (ART-170), or Narratives (ART-251), or
  SpaceTime (ART-203). (0.5 course credit).\\
\item
  \textbf{WKS 252 Workshop: Motion Graphics} An introduction to digital
  kinetic graphics programs, specifically Adobe AfterEffects among
  others, as art-making tools, through a series of self-directed
  creative projects (not tutorials). Students learn through hands on
  experience, guided with interactive demonstrations and assignments.
  Prerequisite: Illuminated Pixels (ART-145), or Movement (ART-170), or
  Narratives (ART-251), or SpaceTime (ART-203), or Vector Graphics
  (WKS-213). (0.5 course credit).
\item
  \textbf{WKS 253 Workshop: Competitions} This 7-week workshop is
  designed to prepare artists for submission, be that galleries, grants,
  festivals or graduate schools. Learn about what makes submissions
  attractive and improve your submission materials, including artist
  resumes, proposals, statements, etc. To be most successful, students
  should have a project in mind or already near completion. Prequisite:
  at least two ART- courses or consent of instructor.
\item
  \textbf{WKS 291 Workshop: Advanced Topics} This course will
  investigate specific topics and/or tools in art-making. Prerequisite:
  at least two ART- courses or consent of instructor. (0.5 course
  credit).
\end{itemize}

\section{Writing}\label{writing}

The writing major (within the Rhetoric department) is designed to help
students become skilled, reflective writers capable of composing texts
in a variety of genres, responding effectively to diverse rhetorical
situations.

\subsection{Writing Major}\label{writing-major}

A major in Writing requires a cumulative 2.0 GPA in all courses counted
toward the major.

\begin{enumerate}
\def\labelenumi{\arabic{enumi}.}
\item
  RHE 200 Rhetorical Theory and Practice
\item
  \textbf{One} of the following:

  \begin{itemize}
  \tightlist
  \item
    RHE 225 Journalism/Media Wtg Wksp
  \item
    RHE 255 The Essay
  \end{itemize}
\item
  \textbf{One} of the following:

  \begin{itemize}
  \tightlist
  \item
    RHE 377 Cultural Studies
  \item
    COM 382 ResearchMethods:Rhetorical/Critical \textbf{or} RHE 382
    Research Methods:Rhetorical/Critica
  \end{itemize}
\item
  \textbf{Six} credits, including any course with an RHE prefix not
  already fulfilling another requirement, or any of the courses from the
  list below. No more than two courses with the CRW or COM prefix may
  count toward the major.

  \begin{itemize}
  \tightlist
  \item
    COM 241 Intro to Multimedia Journalism
  \item
    COM 341 Digital Storytelling
  \item
    COM 361 Communication \& Social Change
  \item
    COM 362 U.S. Public Address
  \item
    COM 382 ResearchMethods:Rhetorical/Critical or RHE 382 Research
    Methods:Rhetorical/Critica (if not used to fulfill requirement \#3)
  \item
    +++MISSING INFO: c.com462.long +++
  \item
    +++MISSING INFO: c.com115.long +++
  \item
    +++MISSING INFO: c.com255.long +++
  \item
    +++MISSING INFO: c.com280.long +++
  \item
    +++MISSING INFO: c.com290.long +++
  \item
    +++MISSING INFO: c.com355.long +++
  \item
    +++MISSING INFO: c.com380.long +++
  \item
    +++MISSING INFO: c.com390.long +++
  \end{itemize}
\item
  \textbf{One} of the following:

  \begin{itemize}
  \tightlist
  \item
    RHE 415 How Writers Write (If not used to satisfy requirement \#4)
  \item
    RHE 425 AdvTop Writing \& Rhetorical Studies (if not used to satisfy
    requirement \#4)
  \item
    RHE 444 Independent Study in Writing (WE)
  \item
    An Honors Thesis
  \end{itemize}
\item
  A portfolio of works completed in the major (minimum 3, maximum 4),
  preceded by a preface that synthesizes major threads/themes of
  learning, and reflects on programmatic learning outcomes and on what
  takeaways from the major the student anticipates being most important
  as they prepare for their post-baccalaureate life. The portfolio
  should be submitted by the end of the term prior to the student's
  intended graduation date, unless circumstances suggest it should be
  submitted later (confer with advisor and department chair).
\end{enumerate}

\subsection{Writing Minor}\label{writing-minor}

A minor in Writing requires five credits, including RHE 200 Rhetorical
Theory and Practice and one RHE course numbered 300 or above. Any course
with an RHE prefix or chosen from the list below may count toward the
five credits; no more than two courses with CRW or COM prefix may count
toward the minor. A cumulative 2.0 GPA is required in all courses
counted toward the minor. COM 241 Intro to Multimedia Journalism COM 341
Digital Storytelling +++MISSING INFO: c.rhe361.long +++ +++MISSING INFO:
c.rhe362.long +++ RHE 382 Research Methods:Rhetorical/Critica
\textbf{or} RHE 382 Research Methods:Rhetorical/Critica +++MISSING INFO:
c.com462.long +++ CRW 115 Exploring Creative Writing CRW 255 Playwriting
Workshop I CRW 280 Poetry Workshop I CRW 290 Fiction Workshop I CRW 355
Playwriting Workshop 2 CRW 380 Poetry Workshop II CRW 390 Fiction
Workshop II

\subsection{Courses in Rhetoric}\label{courses-in-rhetoric}

\begin{itemize}
\tightlist
\item
  \textbf{RHE 100 Directed Summer Reading} Provides incoming students an
  opportunity to practice the academic reading and writing skills
  necessary for successful college work. Students read three books,
  prepare written responses to each book, and discuss the assignments
  with a faculty member during a conference in the first four weeks of
  the Fall Term. S/U basis only. (0.3 course credit)\\
\item
  \textbf{RHE 112 Intro to Writing Center Theory} Introduces foundations
  of writing center theory and pedagogy. Designed for Writing Center
  consultants who are concurrently beginning work at the writing center.
  Instruction takes place in weekly group meetings and individual
  conferences. Offered fall semesters only. (0.3 course credit)
\item
  \textbf{RHE 135 Writers Colony} An intensive writing workshop taught
  off campus; students engage in individual and collaborative writing
  projects. (Offered May Term only)\\
\item
  \textbf{RHE 137 Creative Nonfiction U.S.-Pluralism} Introduces the
  field of creative nonfiction and examines how authors have portrayed
  and interpreted U.S.-based economic, ethnic, racial, social,
  political, and cultural tensions through a diverse blend of nonfiction
  genres. Students compose their own texts in creative nonfiction genres
  such as journals, essays, short memoirs, literary journalism, and
  personal narratives.
\item
  \textbf{RHE 146 Creative Nonfiction:Global Perspect} Introduces
  students to the field of creative nonfiction and the use of personal
  narrative to explore and represent social and cultural issues. Course
  readings center international writers. Writing assignments encourage
  students to examine their own personal and cultural experiences.\\
\item
  \textbf{RHE 175 Writers Studio} A small-group workshop to help
  students develop basic writing, revising, and editing skills. May be
  taken for credit twice. Note: No more than 1.0 credit may be earned by
  enrolling in RHE-175 and RHE-375. (0.5 course credit)
\item
  \textbf{RHE 200 Rhetorical Theory and Practice} Explores the forms and
  functions of written and spoken language, including the study of
  classical rhetoric (Plato, Aristotle, Quintilian) and recent
  developments in rhetorical theory.\\
\item
  \textbf{RHE 210 Journalism Practicum} For Cosmos staff members.
  Introduction to college newspaper production, with discussions on
  professionalism, news gathering, ethics, advertising, layout, and
  computer skills. May be repeated each term a student serves on the
  Cosmos staff. Advanced students assist with instructing beginners. To
  receive credit, students complete a term of service to the Cosmos and
  participate in the workshops at the level agreed upon among the
  instructor, the editor-in-chief, and the staff member. S/U basis only.
  A maximum of one credit may be applied to a writing minor and no more
  than one credit may count toward graduation. Credit for Journalism
  Practicum is regarded as internship credit . (0.2 course credit)
\item
  \textbf{RHE 225 Journalism/Media Wtg Wksp} Introduces and analyzes
  several forms of writing for media in a digital age. In addition to
  print format, students create and analyze interactive and web-based
  texts while acquiring a systematic approach to compositions in a
  variety of media.\\
\item
  \textbf{RHE 230 Grammar \& Style Workshop} None
\item
  \textbf{RHE 255 The Essay} Practice in writing a variety of essay
  forms in non-fiction prose. Students read and discuss essayists chosen
  to represent a range of prose styles and subjects. Students also
  practice writing, workshopping, and revising nonfiction essays.\\
\item
  \textbf{RHE 257 Environmental Rhetoric} Examines how authors and
  organizations have attempted to define and influence the political,
  economic, social, and ethical debates on key environmental issues with
  particular attention to the birth and progress of global environmental
  movements. Students engage in close reading and analysis of a range of
  arguments and messaging pertaining to such movements.
\item
  \textbf{RHE 265 Professional Writing} Planning, drafting, revising,
  and presenting documents for business and professional audiences;
  focus on effective writing and document design. Students compose a
  variety of texts---resumés, memos, letters, manuals, public relations
  materials, and/or reports---working both independently and
  collaboratively; students also give at least one oral presentation,
  based on a major writing assignment.\\
\item
  \textbf{RHE 275 Advanced Writers Studio} A small group, multi-genre
  workshop for experienced writers. May be taken more than once for
  credit for a maximum of 1.0 credit. No more than 1.0 credit may be
  earned by enrolling in RHE-175 and RHE-275. (0.5 course credit)
\item
  \textbf{RHE 284 Topics in Writing and Rhetoric} Offers selected topics
  on specific concerns, problems, or trends in writing and rhetoric.
  Content varies as determined by instructor. May be taken more than
  once for credit, provided topics are distinct.\\
\item
  \textbf{RHE 285 Technical Writing \& Info Design} Introduction to the
  effective communication of scientific and technical information for
  both specialist and non-expert audiences. Instruction in audience
  analysis, writing processes, research strategies, integration of
  graphics and visual information, and the designing, composing,
  revising, editing, and assessment of technical documents.
\item
  \textbf{RHE 312 Writing Center Theory \& Practice} Explores current
  topics in writing center research. Students plan and complete a
  project informed by research and share with writing center peers.
  Offered fall and spring terms. May be taken more than once for credit
  for a maximum of 0.9 credits. Prerequisite: Introduction to Writing
  Center Theory \& Practice (RHE-112) (0.3 course credit)\\
\item
  \textbf{RHE 345 Writing Wilderness} An immersive writing workshop in
  which students write about their wilderness travel experiences in
  original prose and poetry. The focus of the student work will be the
  intersection of self and place. The class considers the idea of
  wilderness---as a concept, a place, a political designation, and a
  state of mind---through the lens of diverse environmental writers,
  theorists and poets across generations. Representative authors include
  Sigurd Olson, Annie Dillard, Terry Tempest Williams, Lorine Niedecker,
  Kimberly Blaeser, William Cronon, Robin Wall Kimmerer and Drew Lanham.
  Students in the class gain experience, confidence, and proficiency in
  wilderness travel and ethics. (Offered summers at the Wilderness Field
  Station)
\item
  \textbf{RHE 377 Cultural Studies} An exploration of American culture
  as a series of ``texts'' to be read, analyzed, and interpreted from a
  variety of rhetorical perspectives. Subjects for analysis may include
  media, art, architecture, lifestyles, entertainment, music, film,
  theatre, and a wide range of literary genres.\\
\item
  \textbf{RHE 382 Research Methods:Rhetorical/Critica} Explores
  rhetorical/critical research methodologies. Students gain experience
  conducting research in the discipline, honing their analytical skills,
  and improving their communication competencies. Prerequisite:
  Rhetorical Theory \& Practice (RHE-200).
\item
  \textbf{RHE 394 Drtd Learning in Writing \& Rhetoric} Completion of
  specialized study under the direction of a faculty member.
  Registration by consent of instructor and after submission of a
  written project proposal to the rhetoric department. May be taken more
  than once for credit with consent of department chair. Prerequisite:
  RHE-200 Rhetorical Theory and Practice, consent of instructor.
\item
  \textbf{RHE 415 How Writers Write} Focuses on established and emergent
  theories about writing from the disciplines of rhetoric and writing
  studies. Students will apply relevant theory and/or research in a
  self-study. Students will also undertake and independent writing
  project of their own design, to be included with the self-study and
  other documents in a portfolio of polished work. Prerequisites:
  RHE-200 Rhetorical Theory \& Practice, and either RHE-377 Cultural
  Studies or COM-382/RHE-382 Research Methods: Rhetorical/Critical\\
\item
  \textbf{RHE 425 AdvTop Writing \& Rhetorical Studies} Exploration of a
  topic in writing and rhetorical studies. May be taken more than once
  for credit with consent of department chair. Exploration of a topic in
  writing and rhetorical studies. May be taken more than once for credit
  with consent of department chair. Prerequisites: RHE-200 Rhetorical
  Theory \& Practice and either RHE-377 Cultural Studies or
  COM-382/RHE-382 Research Methods: Rhetorical/Critical
\item
  \textbf{RHE 444 Independent Study in Writing (WE)} Independent
  projects in writing and rhetoric, culminating in a substantial
  artifact such as a thesis or a work/collection of works suitable for
  publication. May be taken more than once for credit for a maximum of
  2.0 credits. A maximum of one course credit may count toward a writing
  major or minor. P/NP basis only. May be taken for an X status grade
  with consent of instructor prior to registration. Prerequisites:
  senior standing, written consent of instructor and departmental
  petition form. (0.5 or 1.0 course credit)\\
\item
  \textbf{RHE 490 Publications Practicum} A student may receive
  practicum credit while holding the position and performing the duties
  of the main editor of: The Cosmos, The Acorn, Colere, or The Pearl, or
  a comparable publication sponsored by the rhetoric department.
  Students must arrange with a supervising faculty member the amount of
  credit and a practicum agreement specifying skill development goals
  and the projects to be undertaken for the development of those skills.
  S/U basis only. No more than 2.0 course credits may be applied to
  graduation requirements. A maximum of one course credit may be applied
  to a writing major or minor. (0.5 or 1.0 course credit per term)
  ---SECONDARY EDUCATION (MINOR ONLY) See Education, p.~100 ---SOCIAL \&
  CRIMINAL JUSTICE L. Barnett (Administrative Coordinator) The Social \&
  Criminal Justice Program offers students an opportunity to immerse
  themselves in an interdisciplinary major that draws from Coe's rich
  tradition in the liberal arts, as well as the pre-professional
  opportunities at the college. Core courses in the major address
  multifaceted questions surrounding restorative, retributive,
  procedural, and distributive justice. While many programs addressing
  such issues are housed in the field of criminal justice studies, the
  SCJ faculty teach topics within the major from varying perspectives
  and disciplines, introducing students to the ways in which matters of
  justice are at work in --- and essential to --- many areas of study.
  The major is both local and global in nature, exhibiting the
  possibilities and limitations of social and criminal justice
  initiatives in the Cedar Rapids community, while also navigating
  matters of human rights and comparative justice systems that reveal
  our connections to the broader world. Through interdisciplinary
  engagement and a required practicum, the SCJ program demonstrates for
  students how a multiplicity of voices and viewpoints can help to shape
  new ideas about the impact of social justice initiatives upon the
  criminal justice system, and the foundations of individual and
  collective community engagement.
\item
  \textbf{RHE 494 Internship in Writing} An internship with a focus on
  writing supervised by a faculty member of the department. A minimum of
  140 hours on-site experience is required. S/U basis only. A maximum of
  one credit may count toward a major or minor in communication studies
  or writing with the consent of department chair. Prerequisites: junior
  standing and consent of department chair.
\end{itemize}

\chapter{RESERVE OFFICER TRAINING
CORPS}\label{reserve-officer-training-corps}

\section{Aerospace Studies}\label{aerospace-studies}

Clark, Spyker.

\subsection{Air Force ROTC Courses}\label{air-force-rotc-courses}

The Air Force Reserve Officers' Training Corps (AFROTC) program at Coe
College is administered through a cross-enrollment agreement with the
Department of Aerospace Studies at the University of Iowa in Iowa City.
Classes are held at the University of Iowa or at Coe College.
Information on the Air Force ROTC program is available by contacting the
Department of Aerospace Studies at 319-335-9222.

Air Force ROTC is typically a four-year program divided between the
General Military Course (first two years), field training, and the
Professional Officer Course (last two years). Enrollment in the General
Military Course is open to all students and carries no service
obligation. Students can join the program any time during their first or
freshman year. Students who complete the General Military Course attend
a paid two-week field training course. Normally, students attend the
camp between the sophomore and junior years of college. Successful
completion of field training and the Professional Officer Course
culminates in the student receiving a commission as an officer in the
United States Air Force. Opportunities are available in approximately
100 career fields.

Students are supplied all AFROTC books, uniforms, and necessary
materials free of charge. All students in the Professional Officer
Course receive a monthly stipend of either \$450 or \$500. Veterans
continue to draw both the AFROTC stipend plus any GI Bill benefits to
which they are entitled. General Military Course Students are eligible
to apply for AFROTC two- and three-year scholarships which provide
tuition, books, fees, and between \$300-\$500 tax-free monthly stipend.

\begin{itemize}
\tightlist
\item
  \textbf{MSA 110 Air Force Heritage and Values I} A survey course
  designed to introduce students to the United States Air Force (USAF)
  and Air Force Reserve Officer Training Corps (AFROTC). Featured topics
  include: structure of the U.S. Air Force, the Air Force's
  capabilities, career opportunities, benefits, Air Force installations,
  core values, leadership, teambuilding, and communication skills.
  Prerequisite: first-year or sophomore standing. (0.25 course credit)\\
\item
  \textbf{MSA 110L AFROTC Leadrshp Lab I} A progression of experiences
  designed to develop leadership ability; includes military customs and
  courtesies, drill and ceremonies, military professional development,
  and the life and work of a junior officer; leadership skills in a
  practical, supervised military lab setting. Corequisite: Foundations
  of the U.S. Air Force I (MSA-110); Prerequisite: first-year or
  sophomore standing. (0.25 course credit)
\item
  \textbf{MSA 120 Air Force Heritage and Values II} A survey course
  designed to introduce students to the United States Air Force (USAF)
  and Air Force Reserve Officer Training Corps (AFROTC). Spring semester
  featured topics include: Evolution of the U.S. Air Force/Air Force
  history, Principles of War/Tenets of Air Power, What the Air Force
  Brings to the Joint Fight and a look at the Department of the Air
  Force and Air Force Major Commands. It will also introduce several
  leadership concepts, to include ethical decision-making,
  communication, and professional speaking opportunities. Prerequisite:
  first-year or sophomore standing. (0.25 course credit)
\item
  \textbf{MSA 120L AFROTC Leadrshp Lab II} See MSA-110L. Corequisite:
  Air Force Heritage and Values II (MSA-120) (0.25 course credit)
\item
  \textbf{MSA 210 Team and Leadership Fundamentals I} Provide the
  foundation for both leadership and team building. The concepts will be
  applied in team building activities and class discussion to include
  demonstration of basic verbal and written communication. Featured
  topics include: listening, followership, and problem solving
  efficiently. (0.25 course credit)
\item
  \textbf{MSA 210L AFROTC Leadrshp Lab ASP I} See MSA-110L. Corequisite:
  Team and Leadership Fundamentals I (MSA-210). (0.25 course credit)
\item
  \textbf{MSA 220 Team and Leadership Fundamentals II}
\end{itemize}

\backmatter
\printindex

\end{document}
